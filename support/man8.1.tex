\documentclass[a4paper,12pt]{book}
\usepackage{latexsym,endnotes,adn,ibycus4,mauro}
\usepackage[italian]{babel}
\usepackage{url,graphicx,makeidx,multirow}

%-----------------------------------------------------------------------
\newcommand{\mtex}{$\cal{M}$\textit{auro}-{\TeX}}
\newcommand{\stelle}{\medskip\hfil\hbox to 3truein{ \leaders\hbox{
$\star$ } \hfil} \hfil\medskip}
\newenvironment{maurotex}{%
\renewcommand{\footnoterule}{\vspace*{-3.5mm}\rule{15mm}{.2mm}\vspace*{3.3mm}}%
\par\stelle\par\vspace{2.5mm}\begin{minipage}{.9\linewidth}%
}%
{\end{minipage}\vspace{2.5mm}\par\stelle\par}
\renewcommand{\thempfootnote}{\arabic{mpfootnote}}
\providecommand{\LEC}[1]{ \textrm{#1} }
\newcommand{\onlyhtml}[1]{}
\newcommand{\onlylatex}[1]{#1}
\newcommand{\new}{\marginpar{\protect\hspace*{1cm}$\bigodot$}}
\newcommand{\bs}{$\protect\backslash$}
\newcommand{\fg}[1]{\fontencoding{T1}\selectfont\symbol{19}\fontencoding{OT1}\selectfont\thinspace#1\thinspace\fontencoding{T1}\selectfont\symbol{20}\fontencoding{OT1}\selectfont}

\providecommand{\CON}{\"9}
\providecommand{\PER}{\"p}

%-----------------------------------------------------------------------
\makeindex

\title{Manuale \mtex}
\author{v. 8.1}
\date{\textit{update} Jean-Pierre Sutto}
%-----------------------------------------------------------------------
\begin{document}
\NomeTestimone{A}
\maketitle

\hyphenation{tras-cri-zione}

\tableofcontents

%-----------------------------------------------------------------------
\chapter*{Introduzione}

\label{ref-0}

\section{Scopi del manuale}

\label{ref-0.1}

Abbiamo elaborato le presenti indicazioni allo scopo di ottenere
un'edizione dei testi mauroliciani che risponda ai seguenti
criteri:

\begin{enumerate}

\item Fornire un testo critico basato sull'intera
tradizione manoscritta e a stampa che rispetti uno \textit{standard} accettabile di correttezza ed eleganza filologica.

\item Rendere possibile, attraverso l'uso di
un opportuno linguaggio di trascrizione, il recupero integrale del
testo dei vari testimoni su cui il testo critico {\`e} stato
costruito.

\item Costruire un sistema di trascrizione elettronica
semplice, che non costringa a memorizzare moltissimi comandi
o comandi particolarmente complessi da scrivere e che, anzi,
guidi in un certo senso  il trascrittore e l'editore.

\item Costruire un sistema che sia uniforme per tutta
l'edizione, in modo che le possibilit{\`a} di errore e
difformit{\`a} dovute al vasto numero di collaboratori e alle
diversit{\`a} delle loro esperienze possano essere ridotte
al minimo.

\end{enumerate}

Occorre tenere presente che caratteristica essenziale del
``Progetto Maurolico'' {\`e} quella di rappresentare un'edizione
\textit{in progress} dell'opera matematica dello studioso
messinese. Il che implica, molto terra terra, che i problemi
vengono risolti via via che si presentano: le soluzioni
tecniche, i livelli di accuratezza dell'edizione, vengono
trovate, vengono raggiunti per approssimazioni successive.

Cos{\'\i} {\`e} anche per questo manuale, che mettiamo a disposizione
di chi lavora all'edizione~---~e anche di chi a essa e alle sue
tecniche {\`e} semplicemente interessato~---~con l'esplicita
avvertenza che in molti punti esso (e soprattutto il
linguaggio che viene qui descritto, il \mtex)
rappresenta per l'appunto un'approssimazione di ci{\`o} che
speriamo di ottenere a conclusione del nostro lavoro.

Il manuale {\`e} diviso in vari capitoli, scritti in modo da
cercare di tener conto delle diverse fasi di lavorazione di un
testo. Il testo per prima cosa deve essere trascritto:
quindi, nel capitolo \ref{ref-1} si forniscono i criteri generali di
trascrizione che sono stati concordemente decisi, a
prescindere dalle tecniche elettroniche con cui essi
devono essere realizzati.
Similmente, nel capitolo \ref{ref-2}, vengono descritti in
generale i compiti dell'editore.

A partire dal capitolo \ref{ref-3} cominciamo a descrivere il
linguaggio con cui deve essere effettuato il lavoro di
trascrizione e di edizione. Qui vengono affrontati gli
aspetti grafici (come si trattano i titoli, come si
suddivide il testo in capoversi, ecc.) e degli aspetti
legati al recupero di informazioni quali le citazioni di
altri autori, la suddivisione del testo in proposizioni e in
paragrafi, ecc. Il capitolo \ref{ref-3} {\`e} dedicato essenzialmente alla
trascrizione.

Nel capitolo \ref{ref-4}, ``La collazione'', si descrive come trattare
le varianti  fra i vari testimoni del testo. Nel capitolo \ref{ref-5}
si descrive come dar conto delle correzioni, aggiunte ecc.
operate da Maurolico o da altre mani sui vari testimoni.
Si comincia
qui ad entrare nel lavoro di edizione propriamente detto.
Tuttavia {\`e} il capitolo \ref{ref-6}, ``Congetture'', ad essere dedicato
a compiti propri dell'editore, quali le correzioni, le
espunzioni, le integrazioni, stabilire l'esistenza di lacune su
basi congetturali.

Nel capitolo \ref{ref-7} si tratta un caso speciale delle varianti e
degli interventi congetturali dell'editore, quello in cui la
variante o la congettura in questione coinvolge un'ampia
porzione del testo. Il capitolo \ref{ref-8} affronta un caso
particolarmente complesso di variante: la trasposizione. Si
cerca qui di fornire alcuni strumenti per poterla trattare
nel modo pi{\'u} uniforme possibile.

A come trattare
situazioni particolarmente complesse che il presente \textit{Manuale} non ha previsto \textit{a priori} {\`e} dedicato il
capitolo \ref{ref-9}.

Nel capitolo \ref{ref-10} viene illustrato il \textit{Conspectus siglorum}
generale dell'edizione, ovvero l'elenco delle sigle che
verranno utilizzate per indicare i testi di Maurolico
(manoscritti e a stampa) e verranno anche fornite le
indicazioni necessarie su come esse dovranno venire
utilizzate.

Nel capitolo \ref{ref-11} si forniscono le indicazioni su come
i \textit{file} debbano essere preparati per poter essere immessi nel
sito del \textit{Progetto Maurolico}.

Forniamo infine un indice analitico e un indice dei comandi che possono
essere utilizzati nelle trascrizioni e nelle edizioni.

%-----------------------------------------------------------------------
\section{La tradizione di un testo e la sua edizione}

\label{ref-0.2}
\index{tradizione}

Per evitare confusione nel seguito sar{\`a} opportuno chiarire
fin d'ora alcuni termini che useremo assai frequentemente.

Un testo~---~cio{\`e} un'opera letteraria o scientifica come pure
una scrittura privata o documentaria~---~pu{\`o} essere giunto
sino a noi per \textit{tradizione diretta} o per \textit{tradizione indiretta}.

Si parla di tradizione diretta quando possediamo uno o pi{\'u}
manoscritti o libri a stampa espressamente destinati a
tramandare il testo in questione. Tali manoscritti e libri a
stampa sono dei veri e propri \textit{testimoni} che possono
riprodurre il testo in una forma pi{\'u} o meno vicina
all'\textit{originale} dell'autore.

La tradizione indiretta {\`e} costituita invece da tutte quelle
opere che del testo considerato riportino citazioni o
estratti.

Tradizione diretta e tradizione indiretta possono
coesistere o meno: di certe opere abbiamo solo testimoni
diretti; altre ci sono tramandate solo dalla tradizione
indiretta e in forma frammentaria; di altri testi, infine,
possediamo uno o pi{\'u} testimoni diretti insieme a saltuarie
citazioni di altri autori.

Se di un testo non si possegga l'originale nella forma
definitiva voluta dall'autore, tale originale deve essere
ricostruito sulla base della tradizione superstite (diretta
e indiretta). Poich{\'e} nessuna copia di una certa estensione
{\`e} esente da errori o \textit{corruttele testuali}, compito
dell'editore {\`e} quello di restituire un testo il pi{\'u} vicino
possibile all'originale (\textit{costituzione del testo}). A questo
scopo egli deve:

\begin{enumerate}

\item \textit{collazionare} (cio{\`e} leggere, trascrivere e
confrontare) i diversi testimoni e, sulla base delle loro
``coincidenze in errore'', stabilirne le relazioni reciproche,
provvedendo eventualmente ad escludere quelli che risultino
copie o \textit{apografi} di altri testimoni (\textit{antigrafi})
conservati, e procedendo a individuare la forma o le forme
pi{\'u} antiche di trasmissione del testo (\textit{recensio});

\item valutare le differenze testuali (\textit{varianti}) che in
questo o quel punto del testo i diversi testimoni presentano
e procedere alla scelta o \textit{selezione} delle \textit{lezioni} da
accogliere e da respingere;

\item \textit{esaminare} il testo tradito individuando eventuali
passi corrotti in tutta la tradizione e, ove possibile,
correggerli o ``emendarli'' per via congetturale.

\end{enumerate}

Poich{\'e} il lavoro di ricostruzione delle relazioni
fra i testimoni, di selezione delle varianti, e di
valutazione ed emendazione del testo tr{\`a}dito si basa sul
\textit{giudizio} dell'editore, egli deve dar conto delle sue
scelte al lettore, permettendogli di ripercorrere a ritroso
il cammino da lui compiuto. A questo scopo l'editore correda
il testo di un'\textit{introduzione} dedicata a presentare i
testimoni e le loro reciproche relazioni, nonch{\'e} di un
apparato di note (o \textit{apparato critico}) destinato a
registrare le varianti di tutti i testimoni che non
risultino copia di altri testimoni conservati.

L'edizione di un testo allestita secondo questi criteri
prende il nome di \textit{edizione critica}.

%-----------------------------------------------------------------------
\section{Trascrittori ed editori}

\label{ref-0.3}

A quanto brevemente accenato qui sopra, occorre aggiungere
una specificazione sul senso con cui sono usati in questo
\textit{Manuale} i termini che designano le due figure chiave
del nostro lavoro: \textit{trascrittore} e \textit{editore}.

Con ``trascrittore'' intendiamo chi materialmente copia il
testo dai testimoni in forma elettronica. Idealmente il
trascrittore non dovrebbe intervenire sul testo che ha di
fronte, rispettando puntigliosamente tutte le sue
particolarit{\`a}, seguendo alla lettera le indicazioni
fornite in questo manuale e evitando di introdurre nel testo
elettronico errori di copiatura.

Con ``editore'' intendiamo invece chi, sulla base delle sue
conoscenze specifiche, valutando tutte le varianti fra i
vari testimoni, propone e produce il testo critico
dell'opera mauroliciana affidata alle sue cure.

Va detto subito che trascrittore ed editore potranno anche
coincidere nella stessa persona; e che anche quando
cos{\'\i} non fosse, non intendiamo affatto che il
trascrittore abdichi alla sua intelligenza e umanit{\`a} per
ridurre il proprio lavoro a quello di una macchina. Ci{\`o}
che per{\`o} si deve in ogni modo evitare {\`e} la
contaminazione fra questi due aspetti del lavoro di
edizione: se il trascrittore (ove non coincida con l'editore)
ha osservazioni da fare, che le faccia, comunicandole a chi
gli ha commissionato il lavoro o annotandole con speciali
procedure previste allo scopo: saranno le benvenute! Ma non
provveda~---~per favore!~---~a correggere di testa sua,
senza lasciare indicazioni dei suoi interventi.
E se il trascrittore dovesse coincidere con il futuro
editore, questi eviti, mentre trascrive,  di operare scelte
che potr{\`a} compiere solo in una fase pi{\'u} avanzata del
suo lavoro.

Bisogna inoltre ricordare che il ``Progetto Maurolico'' {\`e}
un'impresa collettiva, anche se, ovviamente, ad ognuno
spetteranno gli onori e gli oneri degli impegni che si
assume. Pi{\'u} concretamente, questo significa che nel
partecipare a questa impresa occorre essere disponibili a
rimettere in discussione il proprio lavoro insieme agli
altri, consapevoli che si tratta di un lavoro lungo e che
pu{\`o} richiedere pi{\'u} di un intervento sui testi su cui si {\`e}
lavorato, e che occorre uniformarsi agli standard e ai
modelli che vengono decisi di comune accordo.

%-----------------------------------------------------------------------
\section{Ringraziamenti}

\label{ref-0.4}

Come dicevamo un attimo fa, questo progetto {\`e} un'impresa
collettiva, e il presente manuale non avrebbe potuto essere
realizzato senza l'attiva partecipazione, i consigli e la
collaborazione di tutti i membri del progetto.

Vogliamo per{\`o} in particolare ringraziare Dario
Besseghini~---~senza il cui forsennato entusiasmo iniziale
l'idea di preparare un linguaggio per la trascrizione dei
testi sarebbe forse morta sul nascere; Michela Cecchini e
Lorena Passalacqua che si sobbarcarono l'ingrato compito di
sperimentare i primi vagiti di questo linguaggio; Alessandra
La Spina che ci ha assistito in tante
fasi cruciali; Tito Tonietti e Gian~Paolo Pasquotto per la
pazienza dimostrata accettando di utilizzare le prime
versioni funzionanti del {\mtex} per le loro trascrizioni
e~---~soprattutto~---~per le decine e decine di problemi che
ci hanno posto, costringendoci cos{\'\i} a studiare
situazioni a cui non avevamo neppure pensato e a renderci
conto di molti difetti esistenti. Infine,
Jean-Pierre Sutto per le sue osservazioni e le sue critiche
sempre costruttive e stimolanti.

%-----------------------------------------------------------------------
\chapter[Indicazioni generali]{Indicazioni generali per la trascrizione dei
testi}

\label{ref-1}

\section{Fedelt{\`a} della trascrizione}

\label{ref-1.1}
\index{fedelt{\`a} della trascrizione}

Nel corso della trascrizione il testo deve essere reso nel
modo pi{\'u} possibile vicino all'originale. Primo compito del
trascrittore {\`e} quindi fornire una trascrizione fedele del
testo, depurata il pi{\'u} possibile da errori di trascrizione.
Va da s{\'e} che il testo che consegna all'editore deve essere
elettronicamente gestibile ed eseguito secondo i criteri
spiegati nel seguito del manuale.

Come criterio generale si {\`e} deciso per una
trascrizione parzialmente conformata all'uso moderno. Ci{\`o}
significa:

\begin{enumerate}

\item l'uso della \textit{u} e della \textit{v} viene
uniformato: dunque \textit{vero} e non \textit{uero}, \textit{Utinam}
e non \textit{Vtinam};

\item stesso discorso per l'uso della \textit{j}: si
trascriver{\`a} \textit{eius}, \textit{cuius} e non \textit{ejus}, \textit{cujus};

\item le abbreviazioni tipiche della scrittura cinquecentesca dovranno
essere tutte sciolte, comprese la \& che sar{\`a} resa con \textit{et} e le
%
\onlylatex{\textit{{\ae}}}\onlyhtml{\includegraphics{manicons/ae.gif}}, le
%
\onlylatex{\textit{{\oe}}}\onlyhtml{\includegraphics{manicons/oe.gif}}, le
%
\onlylatex{\textit{{\c e}}}\onlyhtml{\includegraphics{manicons/ce.gif}}
%
che saranno rese con \textit{ae}, \textit{oe}, \textit{ae\/};

\item la punteggiatura verr{\`a} rispettata, ma
conformata all'uso moderno. Si potr{\`a} per esempio sostituire
una virgola con un punto, seguito da maiuscola; eliminare
maiuscole per noi inutili; introdurne altre; sostituire i
due punti con una virgola: senza alcuna segnalazione. Non si
proceder{\`a} per{\`o} all'eliminazione di segni di punteggiatura,
n{\'e} alla loro introduzione \textit{ex novo} senza darne esplicitamente conto:
e questa operazione sar{\`a} fatta dall'editore nel caso
ritenesse assolutamente indispensabile trasformare la
punteggiatura originaria (cfr. \S\,\ref{ref-2.4});

\item lo stesso discorso vale per i capoversi.
Specie nel caso di manoscritti autografi, sar{\`a} opportuno
che, almeno a livello di trascrizione, si segua la scansione
originaria del testo. In particolare andranno riportate le
// con cui Maurolico scandisce il suo ragionamento
matematico (vedi \S\,\ref{ref-3.4.1});

\item Gli accenti dei testi latini non vengono
trascritti; nel caso dei testi italiani ci si conforma
all'uso moderno (``perch{\'e}'' e non ``perche''; ``a casa'' e
non ``{\`a} casa'').

\item Le lettere indicanti grandezze geometriche o
numeri (indicate o meno fra punti o con caratteri speciali
nell'originale:
``\textit{.ab.}'', ``ab'', ``\textit{A}b.'', ecc.) si trascrivono
tutte in corsivo (\textit{ab}, \textit{ab}, \textit{Ab}, ecc.)
rispettando il maiuscolo e il minuscolo. Se fosse
necessario introdurre segni di punteggiatura (ad esempio se
l'originale fosse: ``Sint .ab. .cd. .kl. rect\onlylatex{\textit{{\c e}}}\onlyhtml{\includegraphics{manicons/ce.gif}} line\onlylatex{\textit{{\c e}}}\onlyhtml{\includegraphics{manicons/ce.gif}}'') essi verranno di regola introdotti senza necessit{\`a} di
darne conto in nota (trascrivendo cio{\`e} ``Sint \textit{ab, cd,
kl} rectae lineae'').

\end{enumerate}

Il trascrittore dovr{\`a} inoltre dar conto del ``cambio
pagina'' dei fogli del manoscritto o dell'edizione  a stampa che va
trascrivendo, utilizzando i comandi indicati nel capitolo \ref{ref-3}
(\S\,\ref{ref-3.5.1}).

%-----------------------------------------------------------------------
\section{Trascrizione di particolari tipi di abbreviazioni}

\label{ref-1.2}
\index{abbreviazioni}

Maurolico utilizzava pesantemente un suo
peculiare sistema di abbreviazioni matematiche, che
caratterizza il suo modo di scrivere. Per scrivere
``triangula'', ad esempio, disegnava due triangolini
($\triangle\triangle$), accompagnati o meno dalla desinenza
``a'' ad esponente, e cos{\'\i} via. Tale particolarit{\`a}
{\`e} spesso filtrata, in vario modo,
nelle copie che sono state tratte dagli autografi o nelle edizioni a
stampa. Il trascrittore dovr{\`a} registrarle tutte,
utilizzando gli opportuni comandi illustrati nel capitolo \ref{ref-3}
(\S\,\ref{ref-3.6}).

Non dovr{\`a} inoltre sciogliere abbreviazioni   tipo ``per 13\textsuperscript{am} 2\textsuperscript{i} Elem.'' e  simili nell'indicazione dei libri e delle
proposizioni citate (cfr. \S\,\ref{ref-3.4.6} per come trascrivere
elettronicamente il testo ad esponente).

Scioglier{\`a} invece abbreviazioni del tipo ``q\textsuperscript{us}'' per ``quartus''. In caso di ambiguit{\`a} o di
incertezza, potr{\`a} utilizzare appositi comandi per
``marcare'' le abbreviazioni e, eventualmente, riprodurre
l'abbreviazione (\S\,\ref{ref-3.4.4}): in ogni caso sar{\`a} bene che
segnali la situazione con un suo commento.

Sar{\`a} poi compito dell'editore uniformare il sistema di
trascrizione delle abbreviazioni matematiche e altri tipi di
tachigrafie ai criteri generali della nostra edizione e
secondo le particolari esigenze del testo che sta
trattando.

%-----------------------------------------------------------------------
\section[Collazione dei testimoni]{Collazione dei testimoni e registrazione delle varianti.}

\label{ref-1.3}
\index{collazione}
\index{varianti}

Nel caso che un testo sia stato tr{\`a}dito da pi{\'u} testimoni, si
dovr{\`a} procedere alla loro collazione e alla registrazione
delle varianti fra di essi.

A livello di trascrizione si provveder{\`a} a segnalare
(secondo il sistema spiegato nel capitolo \ref{ref-4}) tutte le
varianti, anche quelle apparentemente di minor conto. Non si
segnaleranno tuttavia varianti puramente tachigrafiche,
dovute all'uso di abbreviazioni diverse (ad esempio se il
testimone A legge ``$\widetilde{coi}$'' e B
``co{\=m}uni'' tale variante non verr{\`a} di norma
registrata). Diverso sar{\`a}  il caso in cui A legge ``in
$\widetilde{coi}$ perspectiva'' e B ``in coni
perspectiva'', dato che $\widetilde{coi}$ non {\`e}
un'abbreviazione usuale per \textit{coni}, e soprattutto ove il
trascrittore avvertisse una qualche incongruenza. In ogni
caso, a livello di trascrizione: \textit{melius abundare} con
l'indicazione delle varianti (cfr. \S\,\ref{ref-3.4} e, in particolare,
\S\,\ref{ref-3.4.5}). Sar{\`a} poi l'editore a
decidere quali di esse conservare e quali far stampare in
apparato, secondo le modalit{\`a} illustrate nel capitolo \ref{ref-4}
(\S\,\ref{ref-4.2}).

%-----------------------------------------------------------------------
\section{Citazioni e date}

\label{ref-1.4}
\index{citazioni}
\index{date}

Frequentissimo {\`e} il caso che Maurolico citi testi di autori
classici o rimandi a proposizioni dell'opera che sta
scrivendo, utilizzando espressioni del tipo ``per 10.\textsuperscript{am} 5.\textsuperscript{i} Euclidis'', ``per praecedentis'',
``ut ait Ptolemaeus'' e simili. Tali citazioni vanno
identificate nel corso della trascrizione e ``marcate'' con
uno speciale comando che sar{\`a} illustrato nel capitolo \ref{ref-3}@
(\S\,\ref{ref-3.5.2}).
Sar{\`a} poi compito dell'editore valutare la
correttezza della
citazione, identificare il passo, ecc.

Tuttavia se il trascrittore individuasse lui stesso dei problemi
con una certa citazione, {\`e} caldamente invitato a segnalarli
(vedi \S\,\ref{ref-1.5}).

Un discorso completamente analogo vale  per le date che si
trovano con una certa frequenza nei testi. Il trascrittore
provveder{\`a} a codificarle utilizzando il comando descritto
nel \S\,\ref{ref-3.5.3}, e il compito dell'editore sar{\`a}  di controllarne
la correttezza, ecc.

%-----------------------------------------------------------------------
\section{Trascrizione degli autografi mauroliciani. Correzioni e altre
osservazioni del trascrittore}

\label{ref-1.5}
\index{correzioni del trascrittore}
\index{osservazioni del trascrittore}

Buona parte del \textit{corpus} dei testi mauroliciani ci {\`e}
pervenuto tramite un manoscritto autografo. In questo caso
gli interventi di Maurolico o di altre mani (cancellature,
aggiunte, interscrizioni, note marginali, ecc.) sul testo
devono essere scrupolosamente segnalate e trascritte secondo
il sistema illustrato nel capitolo \ref{ref-5}. Le aggiunte in margine
richiamate nel testo si inseriranno nel testo al loro posto;
per quelle non richiamate sar{\`a} opportuno che il trascrittore
dia conto in nota del motivo per cui le ha inserite in un
certo luogo.

Sar{\`a} anche opportuno che il trascrittore eviti di correggere il testo
se ritiene di trovarsi in presenza di errori, ma che annoti le sue
osservazioni\footnote{Per le annotazioni di questo tipo, che non dovranno
comparire nell'edizione finale, si utilizza il comando
\texttt{$\backslash$Comm\{\}}. Vedi \S\,\ref{ref-3.3}.}.

Lo stesso discorso vale per ci{\`o} che il trascrittore non
riesca a decifrare. Le sue osservazioni e congetture a riguardo possono
essere preziose: sar{\`a} pertanto opportuno che inserisca nel
testo l'apposito comando per i luoghi che non riesce a
decifrare, corredandolo di sue annotazioni.

Si osservi che tutto quanto {\`e} stato detto qui sopra vale, a
maggior ragione, nel caso che la persona fisica del
trascrittore coincida con quella dell'editore~---~che dovrebbe
lavorare (o almeno cos{\'\i} si suppone) sul testo con maggiore
cognizione di causa. Infatti se questi dovesse
per un motivo qualunque abbandonare il lavoro, le sue
osservazioni potrebbero rivelarsi preziose, se non
indispensabili per chi lo dovesse riprendere dopo di lui. E
anche nel caso che ci{\`o} non avvenisse, a distanza di mesi o
di anni potrebbe non risultare pi{\'u} chiaro all'editore per
quale motivo aveva proposto o fatto un certo tipo di
intervento sul testo.

Fare commenti e congetture {\`e} facile,
costa poca fatica, stimola la discussione e non inquina,
dato che in ogni caso i commenti  saranno gestiti e stampati
fuori dal testo.

%-----------------------------------------------------------------------
\chapter{Compiti dell'editore}

\label{ref-2}

\section{I rapporti tra i testimoni}

\label{ref-2.1}
\index{rapporti tra i testimoni}

Anche per l'opera di Maurolico~---~che pure {\`e}
costituita per la gran parte da autografi o da stampe di cui
non sono noti manoscritti~---~si danno casi in cui la tradizione
ci ha consegnato i suoi testi per mezzo di diversi
testimoni (manoscritti autografi, copie, edizioni a stampa).
Il lavoro proprio dell'editore inizia nel momento in cui
riceve dal trascrittore un testo completamente collazionato
e in cui siano riportate tutte le varianti fra i vari
testimoni.

Il primo compito dell'editore {\`e} quello di stabilire i
rapporti fra i vari testimoni, sia sulla base di quello che
sa della loro storia, sia attraverso lo studio delle
varianti testuali fra i testimoni stessi.  Questo lavoro
permetter{\`a} di stabilire, ove sia il caso, i rapporti di
dipendenza fra i vari testimoni e l'archetipo da cui essi
dipendono.

Prendiamo il \textit{Teodosio}, ad
esempio: disponiamo di un'edizione a stampa, di un
manoscritto probabilmente coevo, di vari manoscritti
seicenteschi. L'editore, confrontando questi testimoni pu{\`o} stabilire
se sono copiati l'uno dall'altro, e stabilire dei rapporti
di parentela (\textit{genetici}). In questo caso, sembrerebbe che
tutti i manoscritti siano stati copiati dall'edizione a
stampa, di cui riportano gli errori aggiungendovene altri.
In un caso simile i manoscritti possono essere eliminati
(\textit{eliminatio codicum descriptorum}) e l'editore del \textit{Teodosio} potr{\`a} allegramente procedere all'edizione solo
sulla base del testo a stampa che risulta essere l'\textit{archetipo} del'intera tradizione.

Casi pi{\'u} complessi sono quelli degli \textit{Arithmeticorum
libri}, dell'\textit{Ottica}, dell'\textit{Apollonio}; pi{\'u}
semplici quelli dell'\textit{Archimede} (solo una stampa) o della
gran parte dei testi euclidei (solo un autografo per
ognuno).

%-----------------------------------------------------------------------
\section{Il testo}

\label{ref-2.2}

Una volta definiti tali rapporti di dipendenza, l'editore
provvede a stabilire e a correggere il testo. Teoricamente
dovrebbe fornire il testo pi{\'u} vicino possibile
all'ultima volont{\`a} di Maurolico.

Se~---~eventualmente dopo
l'\textit{eliminatio}~---~dispone di un unico testimone, a stampa o
manoscritto, il suo compito {\`e} assai semplificato: dovr{\`a}
limitarsi a correggere il testo dando sistematicamente conto
dei suoi interventi. Se invece si trovasse ad aver a che
fare con due o pi{\'u} testimoni discendenti da un medesimo
archetipo e indipendenti l'uno dall'altro, dovr{\`a} stabilire
il testo sulla base delle varianti che questi offrono.

Se i
testimoni ad esempio fossero tre, $A, B, C$, e se $A$ recasse
la lezione ``a vertice trianguli'', mentre $B$ e $C$ ``a vertice
conico'', l'accordo fra $B$ e $C$ segnalerebbe con certezza
che nell'archetipo comune si trovava la lezione ``conico''.
E l'editore provvederebbe a registrare questa lezione nel
suo testo, indicando in apparato la variante di $A$.

Ma se i
testimoni a sua disposizione fossero solo due e $A$ leggesse
``cane'' contro il ``gatto'' di $B$, l'editore dovrebbe
decidere il testo in modo congetturale, secondo il suo
giudizio. Potr{\`a} darsi che entrambi le varianti siano
accettabili e che non ci sia nessun particolare motivo per
scegliere l'una o l'altra (\textit{varianti adiafore}). O potrebbe invece
decidere che $A$ e $B$ sbagliano entrambi e che la lezione
giusta doveva essere ``cono''. Naturalmente di tali
interventi e delle lezioni dei suoi due testimoni dovr{\`a} dare
conto nell'apparato o nell'introduzione della sua edizione.

%-----------------------------------------------------------------------
\section{Distinzione fra varianti sostanziali e varianti
di forma}

\label{ref-2.3}
\index{varianti sostanziali}
\index{varianti di forma}

Il trascrittore (o l'editore stesso in tale veste) avr{\`a}
provveduto a fornire all'editore un testo completamente
collazionato su tutti i testimoni  rilevanti.
Non tutte le varianti fra i testimoni dovranno per{\`o} essere
accolte nell'apparato critico. Se ad esempio uno dei
testimoni scrivesse ``matematica'' invece di ``mathematica''
l'editore si troverebbe in  presenza di una variante
ortografica di carattere puramente formale.  Tale variante
(salvo casi del tutto eccezionali) non verr{\`a} quindi stampata
nell'apparato critico, ma verr{\`a} comunque registrata dal
sistema di trascrizione elettronica.

Ci{\`o} per due motivi. Il
primo {\`e} che, in casi particolari,  le varianti formali potrebbero forse
rivelarsi importanti. Il secondo~---~e pi{\'u} importante~---~{\`e} che il
sistema di trascrizione e di edizione del ``Progetto
Maurolico'' vuole dare la possibilit{\`a} di recuperare i testi
dei vari testimoni cos{\'\i} come sono, con tutte le loro particolarit{\`a}.

Come si vedr{\`a} nel
\S\,\ref{ref-4.4.2}, ci{\`o} pu{\`o} essere ottenuto molto semplicemente,
cambiando nome ad un opportuno comando. Va da s{\'e} che
distinguere fra queste due tipologie di varianti pu{\`o} essere
fatto solo dall'editore e che si tratta di un suo compito
specifico.

%-----------------------------------------------------------------------
\section{Un testo graficamente leggibile}

\label{ref-2.4}
\index{leggibilit{\`a}}

Come {\`e} stato detto nel primo capitolo (\S\,\ref{ref-1.2}), il
trascrittore consegna all'editore un testo in cui dovrebbero
essere state sciolte le abbreviazioni, salvo quelle
``matematiche'', le citazioni di proposizioni e quelle in
cui il trascrittore si {\`e} trovato in dubbio o non {\`e} riuscito
a sciogliere.

Ovviamente, primo compito dell'editore {\`e} controllare la
correttezza della trascrizione effettuata. Dovr{\`a} quindi 
risolvere i dubbi del trascrittore e  le difficolt{\`a}
che questi gli ha trasmesso. Un aspetto pi{\'u}
delicato
sar{\`a} la valutazione di come trascrivere
le tachigrafie e i simboli matematici
utilizzati da Maurolico. In linea generale si pu{\`o} indicare
che \textit{non} andranno sciolte le espressioni che vengono a
costituire vere e proprie ``formule'' matematiche: cos{\'\i}
come, in un testo pi{\'u} moderno, non si scioglierebbe
un'espressione del tipo ``$A:B=C:D$'' scrivendo ``$A$ sta a
$B$ come $C$ sta a $D$''. Sembra invece opportuno sciogliere
un ``$\triangle$\textsuperscript{um}'' in ``triangulum'', in
modo da rendere il testo pi{\'u} leggibile, quando quest'uso
mauroliciano si configuri analogo all'uso di ``\&'' per
``et'' o di altre abbreviazioni e tachigrafie. {\`E} chiaro
per{\`o} che l'editore segnalar{\`a} nell'introduzione la presenza
di tali abbreviazioni peculiari e illustrer{\`a} il tipo di
scelte che ha adottato nei loro confronti.

Analogamente l'editore dovr{\`a} valutare  con attenzione la
punteggiatura del testo e l'uso delle maiuscole,
controllando il lavoro fatto dal trascrittore ed
eventualmente modificandola. Come gi{\`a} detto nel capitolo \ref{ref-1},
quando tale modifica comporti la soppressione di segni di
interpunzione esistenti nel testo o l'inserimento di segni
mancanti, tali interventi andranno registrati utilizzando
i mezzi offerti dal linguaggio che qui presentiamo. Tali
mezzi permettono, fra l'altro, di scegliere quali, fra le
varianti e gli interventi editoriali registrati verranno poi
stampati nell'apparato critico (cfr. \S\,\ref{ref-4.2.2}), e star{\`a} all'editore valutare
la situazione caso per caso. In linea generale {\`e}
raccomandabile un forte rispetto della punteggiatura degli
autografi; ci si potranno concedere maggiori libert{\`a} nel caso
di copie tarde o di stampati.

Ribadiamo comunque che, in ogni caso, il rispetto
dell'originale non dovrebbe compromettere la
leggibilit{\`a} del testo o dell'apparato. Di interventi
sistematici si potr{\`a} dar conto nell'introduzione
all'edizione.

%-----------------------------------------------------------------------
\section{Citazioni e date}

\label{ref-2.5}
\index{citazioni}
\index{date}

Come si {\`e} detto nel \S\,\ref{ref-1.4}., il trascrittore ``marca'' con
uno speciale comando le frasi del testo che costituiscono
una citazione.

L'editore dovr{\`a} in primo luogo controllare la completezza e
la correttezza di questo lavoro. In secondo luogo  dovr{\`a}
verificare se la citazione mauroliciana {\`e} corretta,
identificare il testo a cui si riferisce e assegnargli
un'etichetta abbreviata che lo individui in modo univoco
(ad esempio: 10.\textsuperscript{am} 5.\textsuperscript{i} Euclidis =
E.5.10.). La codificazione del sistema delle citazioni,
nonch{\'e} della trascrizione dei testi citati sar{\`a} in parte
trattata nel \S\,\ref{ref-3.5.2}, in parte in un manuale apposito.

Analogamente l'editore dovr{\`a} valutare i rimandi ad
altre parti dell'opera o ad altre opere mauroliciane diverse
dal testo di cui sta curando l'edizione, soprattutto nel
caso che questi rimandi possano servire a restaurare un
testo corrotto o mancante o a rendere pi{\'u} intelligibile un
testo frammentario.

Lo stesso discorso vale per le date: l'editore dovr{\`a}
controllare la corrispondenza delle datazioni fornite da
Maurolico (ad esempio se la data fosse ``gioved{\'\i} 31 maggio
1531'' bisogner{\`a} controllare se il 31 maggio 1531 era
effettivamente un gioved{\'\i}; se l'indizione indicata
corrisponde all'anno, e simili), tradurre la data secondo il
sistema attuale   ``giorno--mese--anno'' (cfr. \S\,\ref{ref-3.5.3}).

%-----------------------------------------------------------------------
\section{Suddivisione del testo in sottounit{\`a}. Assegnazione degli
argomenti}

\label{ref-2.6}
\index{suddivisione}
\index{argomenti}

Allo scopo di poter individuare con precisione un passo, i
testi mauroliciani dovranno essere suddivisi in unit{\`a} e
sottounit{\`a} dall'editore. I comandi da utilizzare a questo
scopo sono illustrati nel \S\,\ref{ref-3.7}.

Questa suddivisione {\`e} molto importante non solo per future
citazioni di un passo che vogliano fare riferimento alla
nostra edizione critica, ma anche per la costruzione di un
lessico mauroliciano e per avere la possibilit{\`a} di estrarre
dall'opera di Maurolico tutti i passi che trattino un
determinato argomento. A questo scopo l'editore dovr{\`a}
assegnare ad ogni unit{\`a} (ed eventualmente a certe
sottounit{\`a}) uno o pi{\'u} argomenti (ad esempio: ``coniche;
parabola; propriet{\`a} della sottotangente). Il sistema di
assegnazione degli argomenti sar{\`a} trattato (come per il caso
delle citazioni) nel \S\,\ref{ref-3.7}.

%-----------------------------------------------------------------------
\section{Standard per l'immissione in rete}

\label{ref-2.7}
\index{immissione in rete}

L'editore {\`e} responsabile del testo che consegna per
l'immissione in rete. Ne {\`e} responsabile scientificamente, {\`e}
ovvio, ma anche ``elettronicamente''. Potrebbe infatti
accadere che un testo stampato in modo ineccepibile venga
confezionato in modo tale da rendere impossibile la
sua pubblicazione nel sito, salvo la correzione sistematica
dell'uso deviante di alcuni comandi; o che non sia stata
rispettata la sintassi che permette l'estrazione del testo
dei vari testimoni; o che le figure non siano state
scannerizzate in modo utile; e molti altri inconvenienti.

L'editore dovr{\`a} quindi curare che i \textit{file che consegna
per la pubblicazione nel sito soddisfino i requisiti
necessari, ed essere disponibile a effettuare (o a trovar il
modo di far effettuare) eventuali modifiche che gli
venissero richieste}.

Infine, la ``pubblicazione'' del testo critico nel sito non
{\`e} l'equivalente di una pubblicazione in una
rivista. Corrisponde piuttosto alla consegna di una prima
bozza. Anche se il testo che egli aveva stampato sul suo
computer era graficamente ineccepibile, come in tutte le
operazioni di pubblicazione molte cose possono andare
storte: basti che salti un carattere di controllo perch{\'e}
intere pagine appaiano in corsivo o illeggibili o prive di
apparato critico sullo schermo di chi si collega col nostro
sito. E via dicendo. Va da s{\'e} che il \textit{webmaster}
effettua ed effettuer{\`a} un certo controllo prima della
pubblicazione nel sito per impedere scempi cos{\'\i}VV
macroscopici. Ma errori pi{\'u} sottili possono ovviamente
sfuggire. E come ogni rivista e ogni casa editrice, il
``Progetto Maurolico'' lascia ai curatori delle edizioni
critiche il diritto (e il dovere) di compiere tale revisione
fine.

Insomma: la responsabilit{\`a} dell'editore non si esaurisce
nella consegna di un dischetto contenente il suo lavoro, ma {\`e}
sottinteso che essa permane fino al momento in cui si
proceder{\`a} alla pubblicazione finale su \textit{compact
disk} e in volumi cartacei.

%-----------------------------------------------------------------------
\chapter[Gli elementi del {\mtex}]{Gli elementi del {\mtex}: i comandi di
base per la trascrizione}

\label{ref-3}

\section{Il linguaggio di base}

\label{ref-3.1}
\index{base del' linguaggio}

L'edizione e il trattamento dei testi di Maurolico avviene
secondo un processo a due tappe. La prima {\`e} quella della trascrizione e
dell'edizione, che devono essere effettuate utilizzando uno speciale
linguaggio di programmazione che permette (o che dovrebbe
permettere) di codificare tutta l'informazione che l'editore
vuole che venga conservata. Nella seconda fase, opportuni
programmi provvedono a trasformare i testi elaborati con
questo linguaggio in un testo stampabile, in un testo \textsc{html}
che pu{\`o} venir messo in rete con gli opportuni \textit{link}
ipertestuali, in un testo PDF che pu{\`o} venir messo in rete
con un'alta resa grafica. Altri programmi provvedono a
ricavare il testo dei vari testimoni su cui
{\`e} stata condotta l'edizione, ad elaborare il lessico
mauroliciano, ad estrarre le citazioni ecc.

Perch{\'e} la seconda fase possa aver luogo e i programmi
funzionare correttamente \textbf{{\`e} assolutamente necessario} che
l'editore si attenga con scrupolo alle regole
enunciate in questo manuale, evitando di trovare soluzioni
personali. (Per chiarire il punto, il presente manuale \textit{viola}
questa regola: {\`e} stato scritto senza
utilizzare il linguaggio qui descritto. Anche se il
risultato grafico {\`e} lo stesso, non potrebbe essere
sottoposto alla seconda fase di cui si diceva.)

Il linguaggio con cui effettuare le trascrizioni e le
edizioni {\`e} un'elaborazione del {\TeX} che (non senza una certa
megalomania) abbiamo battezzato {\mtex}\footnote{Si tratta, in effetti di un
elaborazione di una delle varianti del {\TeX}, il {\LaTeX}.
\textit{Intelligenti pauca}.}.

Il {\TeX} {\`e} un linguaggio che permette di fare moltissime
cose: proprio per questo l'abbiamo scelto. Ma {\`e}
assolutamente importante che il trascrittore e l'editore
\textbf{utilizzino solo i comandi elencati nell'indice allegato
al manuale} evitando iniziative personali che il {\TeX}
permetterebbe. Nel caso qualcuno si trovi in situazioni
particolari, che lo indurrebbero a usare un comando del {\TeX} che
non sia presente nell'indice~---~e, \textit{a fortiori}, nel caso che
inventi lui una macro apposita~---~deve provvedere a
discuterne con i responsabili dell'edizione. {\`E} chiaro che
non si vuole affatto limitare la creativit{\`a} {\TeX}nica di
chi collabora all'edizione: al contrario, essa {\`e} benvenuta.
Ma occorre, pena paralisi pi{\'u} o meno lunghe, che tale
creativit{\`a} si coordini con le regole che abbiamo adottato.

%-----------------------------------------------------------------------
\subsection{Che cosa non {\`e} il {\TeX} (e il {\mtex})}

\label{ref-3.1.1}

Per chi non conosca il {\TeX} va subito chiarito che \textit{non si tratta di un word-processor}. {\`E} invece un linguaggio
di programmazione (come il Basic o il C, ecc.) che serve a
dare informazioni a un opportuno programma per
trattare tipograficamente un testo. In pratica, per
scrivere un testo in {\TeX} occorre scrivere esplicitamente
tutto ci{\`o} che si vuole ottenere, cos{\'\i} come per scrivere un
programma in Basic che calcoli, per esempio, l'area di un
rettangolo bisogna fornire una lista di istruzioni del tipo
\verb"input a, input b, print a*b".

Il testo che si va scrivendo {\`e} quindi il listato di un
programma: per questo motivo {\`e} fortemente consigliabile
usare un \textit{text-editor} piuttosto che un \textit{word-processor}. Nel
listato finale che si fornisce al programma che dovr{\`a}
trattare il testo non devono infatti comparire che caratteri
compresi fra i primi 128 caratteri ASCII. Ad esempio, se si
vuole ottenere il seguente testo:

\begin{quote}
I {\it Promessi sposi} furono pubblicati 
da Alessandro {\bf Manzoni}
\end{quote}

\noindent bisogner{\`a} scrivere:

\begin{quote}
\begin{verbatim}
\begin{document}
I {\it Promessi sposi} furono pubblicati 
da Alessandro {\bf Manzoni}
\end{document}
\end{verbatim}
\end{quote}

\noindent ovvero dire esplicitamente al programma che inizia
un testo, di
stampare in corsivo \verb"Promessi sposi", in nero \verb"Manzoni" e
che il testo {\`e} finito.

Non possiamo certo fornire qui un manuale completo di
{\TeX}: rinviamo ai molti attualmente disponibili per chi ne
fosse completamente digiuno. Ci limiteremo a indicare le
cose fondamentali da sapere e, soprattutto, visto che non
tutti i comandi (detti anche macro) del {\TeX} possono
essere utilizzati all'interno del {\mtex} (pena problemi nel corso
della seconda fase del trattamento dei testi) a indicare
quali macro \textit{possono} essere inserite senza problemi.

%-----------------------------------------------------------------------
\section{Le macro fondamentali}

\label{ref-3.2}
\index{macro fondamentali}

La versione 2$\epsilon$ del {\LaTeX} \`e quella cui questo manuale fa riferimento.
Su come procurarvela e installarla consigliatevi con il vostro
esperto di {\TeX} preferito!

All'inizio del file dovrete scrivere:

\begin{quote}
\begin{verbatim}
\documentclass[12pt, a4paper]{article}
\usepackage{latexsym, endnotes, adn, mauro, babel}
\end{verbatim}
\end{quote}

Questa parte del vostro file vien detta ``preambolo''. 
Se intendete usare il greco (cfr. \S\,\ref{ref-3.4.2})
dovrete aggiungere \verb"ibycus4" all'interno di \verb"\usepackage"\index{\bs{}usepackage}: 
\begin{quote}
\begin{verbatim}
\usepackage{ibycus4, latexsym, endnotes, adn, mauro, babel}
\end{verbatim}
\end{quote}

Si suppone che un documento {\mtex} sia sempre contenuto
nell'ambiente \verb"document" (vedi l'esempio dei \textit{Promessi
sposi}). Quindi subito dopo dovrete scrivere:

\begin{quote}
\begin{verbatim}
\begin{document}
\htmlcut
\end{verbatim}
\end{quote}

\noindent e, alla fine del file:

\begin{quote}
\begin{verbatim}
\end{document}
\end{verbatim}
\end{quote}

La commande \verb"\htmlcut"\index{\bs{}htmlcut}{\new} est devenue
absolument obligatoire depuis les versions 2004 du {\mtex}: voir la section
\ref{ref-11.2.2}.

I comandi del {\TeX} sono sempre introdotti da uno \textit{slash}
rovesciato (\verb"\"). Dovrete quindi fare attenzione a non scordare di
battere lo \verb"\" altrimenti il comando non verr{\`a} interpretato in
fase di stampa e non verr{\`a} nemmeno segnalato un errore. I comandi sono
poi in genere delimitati da parentesi graffe (\verb"{"~\verb"}"). Ad
esempio per scrivere in corsivo si possono usare i due comandi \verb"\it"\index{\bs{}it} o
\verb"\em"\index{\bs{}em}. Se quindi vogliamo scrivere {\it Arithmeticorum libri duo} si
dovr{\`a} delimitare il comando \verb"\it"\index{\bs{}it}. Infatti se scrivessimo

\begin{quote}
\begin{verbatim}
Gli \it Aritmeticorum libri duo furono pubblicati nel 1575.
\end{verbatim}
\end{quote}

\noindent otterremmo

\begin{quote}
Gli \textit{Aritmeticorum libri duo furono pubblicati nel 1575}.
\end{quote}
% ici on triche.

\noindent dove, come si vede, tutto il testo dopo \verb"Gli"
viene stampato in corsivo. Bisogner{\`a} invece scrivere:

\begin{quote}
\begin{verbatim}
Gli {\it Aritmeticorum libri duo} furono pubblicati nel 1575.
\end{verbatim}
\end{quote}

\noindent delimitando il comando \verb"\it"\index{\bs{}it} e ottenendo,
correttamente\footnote{Un modo alternativo per ottenere il corsivo {\`e} il
comando \texttt{$\backslash$textit\{ \}}. Con questo comando per ottenere
\textit{Arimeticorum libri} occorre scrivere
\texttt{$\backslash$textit\{Arihmeticorum libri\}}.
Simili comandi valgono anche
per il neretto, il tondo inclinato, ecc. Per maggiori dettagli, 
cfr. \S\,\ref{ref-3.4.2}.}:

\begin{quote}
Gli {\it Aritmeticorum libri duo} furono pubblicati nel 1575.
\end{quote}

Bisogner{\`a} fare molta attenzione perch{\'e} ad ogni
parentesi graffa aperta ne corrisponda una chiusa, e nel punto
giusto, altrimenti si otterranno risultati bizzarri e, di
norma, messaggi di errore nel corso della compilazione.

{\`E} molto opportuno sapere che un comando {\TeX} {\`e} costituito
da una stringa di caratteri alfabetici maiuscoli o minuscoli
(le 52 lettere dell'alfabeto inglese). Un tale comando
{\`e} introdotto dallo \verb"\" ed {\`e} terminato \textit{dal primo
carattere non alfabetico che segue}. Lo spazio bianco, in
particolare, {\`e} un carattere non alfabetico. Il che significa
che in certe situazioni si potrebbero ottenere delle cose
bizzarre e non capire perch{\'e}. Ad esempio il comando per far
stampare il logo {\TeX} {\`e} \verb"\TeX"\index{\bs{}TeX} (si notino le maiuscole e
le minuscole: il {\TeX} distingue fra le due). Se scrivete:

\begin{quote}
\begin{verbatim}
Questo \TeX mi sembra una cosa ...
\end{verbatim}
\end{quote}

\noindent otterrete:

\begin{quote}
Questo {\TeX}mi sembra una cosa ...
\end{quote}
% ici on triche

\noindent senza alcuno spazio bianco fra ``{\TeX}'' e ``mi''. Questo
perch{\'e} lo spazio bianco (ma io l'ho battuto,
maledizione!) viene ignorato, in quanto considerato come
``terminatore'' del comando. Per ottenere che la frase
venga scritta correttamente dovrete scrivere (ad esempio):

\begin{quote}
\begin{verbatim}
Questo {\TeX} mi sembra una cosa ...
\end{verbatim}
\end{quote}

\noindent e questa volta otterrete:

\begin{quote}
Questo {\TeX} mi sembra una cosa ...
\end{quote}

\noindent in quanto la parentesi graffa chiusa (carattere
non alfabetico) terminer{\`a} lei il comando \verb"\TeX"\index{\bs{}TeX} e lo spazio
bianco successivo verr{\`a} interpetato correttamente.

%-----------------------------------------------------------------------
\subsection{Caratteri speciali}

\label{ref-3.2.1}
\index{caratteri speciali}

Il {\TeX} e i suoi dialetti utilizzano, come si {\`e} appena
visto, alcuni caratteri ($\backslash$, \{ \}, \%, \&, \$, \~~ , [, ]) come
caratteri che servono a fornire istruzioni: occorre
conoscerli per evitarne un uso improprio che genererebbe
messaggi di errore (e anche errori senza messaggi,
purtroppo!) a non finire.

Il carattere \% (segno di percentuale) indica al programma
di ignorare ci{\`o} che {\`e} scritto dopo di lui nella riga che lo
contiene. Se battete, ad esempio:

\begin{quote}
\begin{verbatim}
Gli {\it Aritmeticorum libri %duo} furono pubblicati
nel 1575. Sono un testo molto interesssante.
\end{verbatim}
\end{quote}

\noindent otterrete solo

\begin{quote}
Gli {\it Aritmeticorum libri %duo} furono pubblicati
nel 1575. Sono un testo molto interesssante.}
\end{quote}

E non solo vi mancher{\`a} parte del testo; non
solo, siccome la parentesi \verb"}" viene ignorata, tutto il resto del
testo che avete scritto dopo \verb"duo" risulta in corsivo;
ma oltre a questi e ad altri possibili errori, il
{\TeX}, non trovando la parentesi graffa che chiude quella
aperta mander{\`a} lamentele a non finire in fase di
compilazione.

Il carattere \texttt{\%} {\`e} utile per ``commentare'' certe parti del
testo che state scrivendo e che non volete che vengano stampate.

Il carattere \texttt{\&} serve per costruire tabelle. Rinviamo a un manuale
di {\TeX} per il suo uso: qui avvertiamo solo che se volete scrivere ``\&''
dovrete digitare \texttt{\&} inserendo uno \verb"\" prima di \texttt{\&}.

Il carattere \texttt{\$} serve a delimitare l'ambiente matematico: se
volete ad esempio scrivere ``sar{\`a} dunque $a+b=7x-y$ e quindi'' dovrete
battere

\begin{quote}
\begin{verbatim}
sar{\`a} dunque $a+b=7x-y$ e quindi
\end{verbatim}
\end{quote}

All'interno dell'ambiente matematico le lettere sono
automaticamente stampate in corsivo; se volete che esse
risultino in tondo occorre segnalarlo esplicitamente (vedi
pi{\'u} avanti). Il {\mtex}, tuttavia, preferisce
utilizzare per aprire e chiudere l'ambiente matematico la
coppia \verb"\(" e \verb"\)". In altre parole se scrivete
\verb"\(a+b=7x-y\)" otterrete lo stesso risultato illustrato qui
sopra. Tuttavia, sull'ambiente matematico si veda pi{\'u}
avanti il \S\,\ref{ref-3.6}\footnote{Si tratta comunque di una
\textit{preferenza} e non di un obbligo. Il motivo
principale di questa indicazione a favore della
coppia \texttt{$\backslash$(...$\backslash$)} {\`e} che l'uso del \texttt{\$}
rende difficile
scoprire se si {\`e} dimenticato di batterne uno, e dove.
Dato che i testi delle trascrizioni sono gi{\`a} piuttosto
complessi, consigliamo quello della coppia \texttt{$\backslash$( $\backslash$)} che
permette di stabilire pi{\'u} facilmente dove inizia e dove
termina l'ambiente matematico e di scoprire il punto dove si
fosse dimenticato il delimitatore. Tuttavia anche questa
soluzione ha il suo prezzo, dato che bisogna battere due
caratteri invece di uno solo. Potete usare quello con cui vi
trovate pi{\'u} a vostro agio.}.

Il carattere \verb"~" serve a indicare uno spazio codificato fra
due parole. Il {\TeX} infatti nell'impaginare allarga o
restringe gli spazi fra le parole a seconda delle sue
necessit{\`a}. Se volete scrivere ``Pier~Paolo'' e non volete che
queste due parole vengano separate (per esempio che ``Pier''
si trovi alla fine di una riga e ``Paolo'' all'inizio della
successiva) dovrete scrivere:

\begin{quote}
\begin{verbatim}
Pier~Paolo
\end{verbatim}
\end{quote}

Il che {\`e} particolarmente utile quando non volete
separare il numero di pagina o di nota dall'espressione
``pp.'' o ``n.'': basta scrivere \verb"pp.~123--48" o \verb"vedi n.~1568".

%-----------------------------------------------------------------------
\subsection{Lettere accentate}

\label{ref-3.2.2}
\index{lettere accentate}
\index{accenti}

Come si {\`e} detto il {\TeX} riconosce solo i primi 128 caratteri
ASCII. Il che implica che le lettere accentate, le cediglie,
ecc. generabili da tastiera non possono essere utilizzate (in effetti
{\`e} possibile configurare il {\TeX} in modo che tali caratteri possano
essere riconosciuti: ma lo riteniamo altamene sconsigliabile).
Per le lettere accentate si consiglia di usare la seguente
tabella:

\begin{center}
\begin{tabular}{rcl}

{\`a} & = & \verb"{\`a}" \\
{\`e} & = & \verb"{\`e}" \\
{\'e} & = & \verb"{\'e}" \\
{\'\i} & = & \verb"{\'\i}" \\
{\`o} & = & \verb"{\`o}" \\
{\'u} & = & \verb"{\'u}"

\end{tabular}
\end{center}

Si noti che gli accenti acuto e grave sono generati
rispettivamente dal comando \verb"\'" e \verb"\`" (per chi dispone
solo di
una tastiera italiana pu{\`o} ottenere quest'ultimo (in ambiente
MSDOS) tenendo
premuto il tasto ALT e digitando sulla tastiera numerica
96). Non sarebbe strettamente necessario delimitarli fra \verb"{}",
ma in questo modo si ottiene un testo molto pi{\'u} pulito e su
cui si pu{\`o} pi{\'u} facilmente intervenire.

Il {\TeX} pu{\`o} per{\`o} generare una quasi infinita variet{\`a} di
accenti: ad esempio se volete scrivere ``ni\~nos'' in
spagnolo baster{\`a} scrivere \verb"ni\~nos": il carattere \verb"~"
(ALT 126)
preceduto dallo \verb"\" genera una tilde sulla lettera seguente.
Per maggiori particolari consultate un manuale di {\TeX}.

%-----------------------------------------------------------------------
\subsection{Spazi, fine riga e capoversi. Testi centrati.}

\label{ref-3.2.3}
\index{spazi}
\index{fine riga}
\index{capoversi}
\index{centrare}

Per il {\TeX} un qualsiasi numero di spazi bianchi equivale
sempre e solo a uno spazio. Cos{\'\i} il codice ASCII che indica
la fine di una riga viene trattato come se fosse uno spazio
bianco. In questo caso per{\`o} occorre maggiore attenzione: se
lasciate una riga vuota (due codici di fine riga
consecutivi) il {\TeX} andr{\`a} a capo in fase di stampa.
Questo sistema per andare a capo {\`e} per{\`o} assolutamente da
evitare nel {\mtex}: {\`e} molto meglio
ricorrere al comando \verb"\par"\index{\bs{}par}. Con questo comando il {\TeX}
andr{\`a} a capo, rientrando il capoverso. Scrivendo:

\begin{quote}
\begin{verbatim}
... stanchissimo per quella trascrizione infernale,
and{\`o} a letto e si addorment{\`o} subito. \par La
mattina dopo, ...
\end{verbatim}
\end{quote}

\noindent otterrete:

\begin{quote}
... stanchissimo per quella trascrizione infernale, and{\`o} a letto e si
addorment{\`o} subito. \par La mattina dopo, ...
\end{quote}

Se non volete che venga effettuato il rientro (o \textit{indent}) andando a
capoverso, baster{\`a} scrivere \verb"\noindent"\index{\bs{}noindent} dopo \verb"\par"\index{\bs{}par}.

Se volete ottenere un testo centrato, dovrete utilizzare
l'ambiente {\LaTeX} ``center''. Per ottenere ad esempio:

\begin{quote}
\begin{center}
{\bf Manuale delle Giovani Maurmotte}
\end{center}

In questo manuale vi insegneremo come fare una
trascrizione utilizzando pietre, legnetti e ossicini, 
pelli di daino conciate e iscrizioni rupestri.
\end{quote}

\noindent dovrete scrivere:

\begin{quote}
\begin{verbatim}
\begin{center}
{\bf Manuale delle Giovani Maurmotte}
\end{center}

In questo manuale vi insegneremo come fare una
trascrizione utilizzando pietre, legnetti e ossicini, 
pelli di daino conciate e iscrizioni rupestri.
\end{verbatim}
\end{quote}

Abbiamo fin qui descritto per sommi capi le caratteristiche
fondamentali del {\LaTeX}. Cominciamo ora a vedere alcuni
dei comandi propri del {\mtex}

%-----------------------------------------------------------------------
\section{Commenti}

\label{ref-3.3}
\index{commenti}

\new Come abbiamo sottolineato nel \S\,\ref{ref-1.5} {\`e} importante che il
trascrittore abbia la possibilit{\`a} di fare annotazioni su ci{\`o}
che viene trascrivendo senza per{\`o} che tali annotazioni
entrino a far parte del testo. Per far ci{\`o} si usa la macro\index{\bs{}Comm}

\begin{quote}
\begin{verbatim}
\Comm{}
\end{verbatim}
\end{quote}

\noindent e nelle \verb"{}" si inserisce il testo del
commento. Il passo commentato verr{\`a} evidenziato nel testo con un
$\bullet$ posto in esponente seguito da un numero; il testo del commento
verr{\`a} stampato nel punto del testo scelto dal trascrittore. Si
consiglia di farli stampare (di norma) alla fine del testo: allora
occorrer{\`a} scrivere, subito prima di \verb"\end{document}"\index{\bs{}end\{document\}}:

\begin{quote}
\begin{verbatim}
\Commenti
\end{verbatim}
\end{quote}

Pi{\'u} in generale i commenti verranno stampati nel punto in
cui si scriver{\`a} \verb"\Commenti"\index{\bs{}Commenti}. Se questo comando non viene
inserito in nessun posto, non verr{\`a} stampato nessun
commento, ma verranno stampati i $\bullet$ in esponente con il loro
relativo numero.

%Se non si vuole che siano stampati neanche questi occorre inserire %all'inizio del file, dopo %\verb"\begin{document}"\index{\bs{}begin\{document\}}, il %comando\index{\bs{}Nocomment}
%
%\begin{quote}
%\begin{verbatim}
%\Nocomment
%\end{verbatim}
%\end{quote}
%
%In altre parole si pu{\`o} ottenere sempre una
%stampa depurata dai commenti del trascrittore.

%-----------------------------------------------------------------------
\subsection{Quando il trascrittore non sa leggere}

\label{ref-3.3.1}
\index{lacune soggettive}

Un caso particolarmente importante in cui il trascrittore
dovr{\`a} utilizzare \verb"\Comm"\index{\bs{}Comm} {\`e} quello in cui, \textit{a causa di
difficolt{\`a} ``soggettive''}, non gli riuscisse di leggere il
testo del testimone che sta trascrivendo. Con ``difficolt{\`a}
soggettive'' intendiamo situazioni per cui il trascrittore
non riesca a sciogliere
un'abbreviazione, la parola che dovrebbe trascrivere si
trova nascosta dalla rilegatura del codice e non pu{\`o} essere
letta utilizzando un microfilm, il microfilm di cui dispone
rende difficoltosa la decifrazione della testo e simili;
situazioni cio{\`e} in cui la difficolt{\`a} risiede o nelle
capacit{\`a} del trascrittore o nella scarsezza dei mezzi di cui
questi dispone. Provveder{\`a} a segnalare la cosa all'editore
inserendo al posto della lezione per lui indecifrabile la macro 

\begin{quote}
\begin{verbatim}
\LACs
\end{verbatim}
\end{quote}

(lacuna \textit{soggettiva}) che stampa tre
asterischi (***) nel testo trascritto, facendola seguire immediatamente
da un commento in cui si spieghi la natura del problema.

Ad esempio, se il testimone (A) da cui trascrive recasse il
seguente testo:

\begin{quote}
A: et erit $\bullet$ aequalis quadrato
\end{quote}

dove il $\bullet$ indica che il trascrittore si trova di fronte
ad una delle situazioni elencate qui sopra, dar{\`a} conto del
problema nel seguente modo:

\begin{quote}
\begin{verbatim}
et erit \LACs\Comm{In A c'{\`e} un'abbreviazione
che non riesco a sciogliere} aequalis quadrato
\end{verbatim}
\end{quote}

Star{\`a} poi all'editore cercare di capire l'abbreviazione,
recarsi a Vladivostok (dove {\`e} conservato il testimone A) per
leggere ci{\`o} che {\`e} nascosto dalla rilegatura del codice, ordinare alla
biblioteca di Vladivostok un nuovo microfilm, consultare un
paleografo di fama: insomma fare
tutto quanto {\`e} in suo potere per riuscire a leggere il
luogo. Se poi, nonostante tutte queste fatiche, il luogo
rimanesse indecifrabile, rimandiamo lo sfortunato editore al
capitolo \ref{ref-9} (\S\,\ref{ref-9.1}) in cui si parla dei modi con
cui affrontare i casi eccezionali.

Diverso (e molto!) {\`e} il caso in cui il trascrittore non
riesca a leggere per motivi oggettivi (ad esempio perch{\'e} c'{\`e}
un foro nella carta). Per come regolarsi di fronte a questo
genere di problematica si veda il capitolo \ref{ref-4} (\S\,\ref{ref-4.4}).

%-----------------------------------------------------------------------
\subsection{I commenti nella storia}

\label{ref-3.3.2}
\index{commenti}

La macro \verb"\Comm"\index{\bs{}Comm} sar{\`a} da usare con generosit{\`a} e senza
timori, come abbiamo gi{\`a} osservato. Sar{\`a} anzi
opportuno che l'editore nei suoi interventi successivi
sul testo, piuttosto che procedere brutalmente ad
$\widetilde{elimare}$ i commenti precedenti cancellandoli, li
mantenga aggiungendovene dei nuovi. Supponiamo ad
esempio che il trascrittore TT abbia scritto un
commento di questo tipo:

\begin{quote}
\begin{verbatim}
\Comm{Secondo me qui il testo {\`e} sbagliato, ci
dovrebbe essere ``recta \(ac\)'' e non ``recta
\(cd\)''TT, 18.4.2000}
\end{verbatim}
\end{quote}

L'editore EE studia il passo e, in un primo tempo,
decide che a suo avviso ``recta $cd$'' {\`e} accettabile.
Per{\`o}, invece che cancellare il commento del trascrittore,
l'arricchisce:

\begin{quote}
\begin{verbatim}
\Comm{Secondo me qui il testo {\`e} sbagliato, ci
dovrebbe essere ``recta \(ac\)'' e non ``recta
\(cd\)''TT, 18.4.2000. NO! mi sembra che \(cd\) possa
andare bene. Peraltro non riesco a capire perch{\'e}
\(ab\) sarebbe meglio. EE, 2.8.2000.}
\end{verbatim}
\end{quote}

Mesi dopo, rivedendo la sua edizione, EE legge nuovamente il
commento e finalmente capisce la motivazione del
suggerimento del trascrittore, e si rende conto che aveva
ragione. Quindi corregge il testo (vedi
capitolo \ref{ref-6}) e al
tempo stesso integra il commento:

\begin{quote}
\begin{verbatim}
\Comm{Secondo me qui il testo {\`e} sbagliato, ci
dovrebbe essere ``recta \(ac\)'' e non ``recta
\(cd\)''TT, 18.4.2000. NO! mi sembra che \(cd\) possa
andare bene. Peraltro non riesco a capire perch{\'e}
\(ab\) sarebbe meglio. EE, 2.8.2000. Invece andava
BENE \(ab\)! Infatti {\`e} il lato retto della parabola e
in questa proposizione Maurolico con diameter intende
il lato retto. Quindi deve essere \(ab\) e non \(ac\)
che {\`e} il diametro in senso stretto. EE, 5.2.2001.}
\end{verbatim}
\end{quote}

In questa maniera i commenti verranno a costituire
una specie di storia dell'edizione che potrebbe
rivelarsi preziosa, soprattutto nel caso che il
trascrittore o l'editore debbano passare ad altri il
proprio lavoro non ancora terminato.

%-----------------------------------------------------------------------
\section{Macro grafiche del {\mtex}}

\label{ref-3.4}
\index{macro grafiche}

In questo paragrafo introduciamo alcune macro specifiche del
{\mtex} e che servono a impaginare il testo (centrature, caratteri
pi{\'u} grandi, corsivi, greco, caratteri speciali, ecc.).

%-----------------------------------------------------------------------
\subsection{Ambienti grafici}

\label{ref-3.4.1}
\index{ambienti grafici}

Per i titoli, i titoletti e tutto il materiale che si voglia
far comparire centrato e maiuscolo si deve utilizzare
l'ambiente ``Enunciatio''. Cos{\'\i} ad esempio l'intestazione
della lettera ad Eudemo all'inizio delle \textit{Coniche} verr{\`a}
battuta cos{\'\i}:

\begin{quote}
\begin{verbatim}
\begin{Enunciatio}
Apollonius Eudemo salutem
\end{Enunciatio}
Si corpore bene vales ...
\end{verbatim}
\end{quote}

Gli enunciati delle proposizioni, definizioni, scolii,
corollari e tutto il materiale che si voglia far comparire
in caratteri leggermente pi{\'u} grandi del resto del testo
verrano inseriti invece all'interno dell'ambiente
``Protasis''. Per cui scriveremo:

\begin{quote}
\begin{verbatim}
\begin{Enunciatio}
Propositio XVII
\end{Enunciatio}
\begin{Protasis}
Sphaerarum superficies sunt quadratis diametrorum
proportionales.
\end{Protasis}
\par
Sint duae sphaerae ...
\end{verbatim}
\end{quote}

Si noti il \verb"\par"\index{\bs{}par} dopo \verb"\end{Protasis}"\index{\bs{}end\{Protasis\}}, che fa s{\'\i} che il testo della protasi venga separato
con un capoverso dal testo dell'ectesi. Come si {\`e} detto, \textit{per
andare a capo si deve usare esclusivamente} \verb"\par"\index{\bs{}par}
\textit{e non lasciare una riga bianca}.

Maurolico fa nei suoi autografi un uso frequentissimo di una doppia
barretta (\verb"\\") o di una barretta (\verb"\"). Come avvertito nel
capitolo \ref{ref-1} tali simboli gli servono per scandire il suo ragionamento ed
{\`e} quindi opportuno riportarli almeno nel corso della prima
trascrizione, lasciando poi all'editore la decisione se fornirli o meno
nell'edizione. Essi si traducono molto semplicemente nelle macro

\begin{quote}
\begin{verbatim}
{\DB}
{\SB}
\end{verbatim}
\end{quote}

\noindent ovvero ``\textbf{D}oppia'' e ``\textbf{S}emplice \textbf{B}arretta''.
Per motivi analoghi a quelli pe cui {\`e} pi{\'u} opportuno racchiudere le
lettere accentate fra parentesi graffe, {\`e} assai consigliabile fare
altrettanto con \verb"\DB"\index{\bs{}DB} e \verb"\SB"\index{\bs{}SB}. Si rischia altrimenti che lo spazio che
segue venga ignorato e che la barretta venga appiccicata alla parola
seguente. Racchiudendo questi comandi fra \verb"{"\,\verb"}" la spaziatura sar{\`a}
regolare.

%-----------------------------------------------------------------------
\subsection{Corsivi, neretti e greco}

\label{ref-3.4.2}
\index{corsivi}
\index{neretti}
\index{greco}

Come si {\`e} gi{\`a} visto, per ottenere che un testo venga scritto
in corsivo o in nero, basta racchiuderlo fra parentesi graffe,
facendolo precedere dal comando \verb"\it"\index{\bs{}it}
o \verb"\em"\index{\bs{}em} per il corsivo, dal comando \verb"\bf"\index{\bs{}bf} per il nero.
Importante {\`e} anche il comando \verb"\sl"\index{\bs{}sl} che produce una specie
di ``tondo inclinato'' e il comando \verb"\rm"\index{\bs{}rm} che si usa quando
in un testo che non {\`e} in tondo si vogliono far apparire
alcuni caratteri in tondo. Le seguenti frasi, ad esempio:

\begin{quote}
{\it Questo {\`e} il corsivo.}
\par
\textsl{E questo il tondo inclinato.}
\par
{\bf Questo {\`e} il nero.}
\par
\textit{E questo \textrm{{\`e} un po'} corsivo, un po'
\textsl{tondo inclinato} e \textrm{un po' tondo normale}.}
\end{quote}
% ici on triche

\noindent si ottengono cos{\'\i}:

\begin{quote}
\begin{verbatim}
{\it Questo {\`e} il corsivo.}
\par
{\sl E questo il tondo inclinato.}
\par
{\bf Questo {\`e} il nero.}
\par
{\it E questo \rm{{\`e} un po'} corsivo, un po'
{\sl tondo inclinato} e {\rm un po' tondo normale}.}
\end{verbatim}
\end{quote}

Un modo alternativo (e pi{\'u} omogeneo al resto dello stile
della sintassi) per ottenere lo stesso risultato {\`e} quello di
usare i comandi \verb"\textit{}"\index{\bs{}textit}, \verb"\textsl{}"\index{\bs{}textsl}, \verb"\textbf{}"\index{\bs{}textbf}:

\begin{quote}
\begin{verbatim}
\textit{Questo {\`e} il corsivo.}
\par
\textsl{E questo il tondo inclinato.}
\par
\textbf{Questo {\`e} il nero.}
\par
\textit{E questo \textrm{{\`e} un po'} corsivo, un po'
\textsl{tondo inclinato} e \textrm{un po' tondo normale}.}
\end{verbatim}
\end{quote}

Il carattere di \textit{default} {\`e} il tondo (\verb"\rm"\index{\bs{}rm} o
\verb"\textrm"\index{\bs{}textrm}), quindi non {\`e} necessario dichiararlo a meno che
non ci si trovi all'interno di un altro 
carattere. I due
caratteri pi{\'u} importanti sono lo \verb"\sl"\index{\bs{}sl} o \verb"\textsl"\index{\bs{}textsl} e il
\verb"\it"\index{\bs{}it} o \verb"\textit"\index{\bs{}textit} (questi ultimi due comandi sono
essenzialmente equivalenti a \verb"\em"\index{\bs{}em} o, rispettivamente,
\verb"\textem"\index{\bs{}textem}). Infatti nell'apparato testuale le lezioni
riportate in nota vanno in tondo; i commenti editoriali in
\textsl{tondo inclinato} (\verb"\sl"\index{\bs{}sl}) e le lettere denotanti
grandezze matematiche o astronomiche in \textit{corsivo}
(\verb"\it"\index{\bs{}it}) (vedi per{\`o} pi{\'u} avanti, \S\,\ref{ref-3.6.1}). In \textit{corsivo}
andranno anche le citazioni di opere che, occasionalmente,
ci si trover{\`a} a dover fare nell'apparato (cfr. 
%\S\,\ref{ref-4.4.4}%
\S\,\ref{ref-3.5.4}%
% correction jps le 25-01-2005
). Di
norma, {\`e} il {\mtex} a gestire tutti questi cambi di
caratteri; si possono per{\`o} dare situazioni eccezionali in
cui l'editore si trover{\`a} costretto a intervenire manualmente
(cfr. \S\,\ref{ref-6.1.3} e \S\,\ref{ref-9.2}).

Per quanto riguarda il greco si deve invece usare il
comando \verb"\GG{}"\index{\bs{}GG}. Le lettere greche vengono trascritte secondo
la seguente tabella:

\begin{center}
\begin{tabular}{lll}

$\alpha$ = a & $\beta$ = b & $\gamma$ = g \\
$\delta$ = d & $\epsilon$ = e & $\zeta$ = z \\
$\eta$ = h & $\theta$ = q & $\iota$ = i \\
$\kappa$ = k & $\lambda$ = l & $\mu$ = m \\
$\nu$	 = n & $\xi$	 = c & o 	 = o \\
$\pi$	 = p & $\rho$ = r & $\sigma$ = s \\
$\tau$ = t & $\upsilon$ = u & $\phi$ = f \\
$\chi$ = x & $\psi$ = y & $\omega$ = w

\end{tabular}
\end{center}

\noindent per le minuscole e per le maiuscole (per scrivere $\Theta$,
ad esempio, baster{\'a} quindi battere \verb"Q"). Il programma
provvede da s{\'e} a distinguere fra il sigma interno alla parola
($\sigma$) e il sigma finale ($\varsigma$). Per altri
caratteri meno usati la codifica {\`e} la seguente:

\begin{quote}
v\hphantom{+} = digamma
\par
k+ = koppa
\par
s+ = sampi (solo minuscola)
\end{quote}

Per gli accenti, spiriti, e dieresi, la codifica {\`e}:

\begin{quote}
spiriti: \verb")" e \verb"("
\par
accento acuto: \verb"'"
\par
accento grave: \verb"`"
\par
accento circonflesso: \verb"="
\par
iota sottoscritto: \verb"|"
\par
dieresis dopo $\upsilon$ o $\iota$: \verb"+" 
\end{quote}

Si osservi che, quando si utilizza il greco, gli accenti e
gli spiriti vanno collocati \textbf{dopo} la lettera cui si riferiscono,
e non prima come si fa normalmente in {\TeX}. Quindi per ottenere
$\Lambda\acute{o}\gamma{o}\varsigma$ si dovr{\`a} battere

\begin{quote}
\begin{verbatim}
\GG{Lo'gos}
\end{verbatim}
\end{quote}

\index{\bs{}GG}
\begin{maurotex}
\GG{Lo'gos}
\end{maurotex}

%-----------------------------------------------------------------------
\subsection{Simboli astronomici}

\label{ref-3.4.3}
\index{simboli astronomici}
\index{caratteri speciali}

Il {\mtex} permette di introdurre inuntesto i
simboli astronomici e altri caratteri. La codificazione {\`e} la
seguente.

Per la Terra e i sette pianeti (Luna, Mercurio, Venere,
Sole, Marte, Giove\new, Saturno):

\index{\bs{}TER}
\index{\bs{}LUN}
\index{\bs{}MER}
\index{\bs{}VEN}
\index{\bs{}SOL}
\index{\bs{}MAR}
\index{\bs{}GIO}
\index{\bs{}SAT}

\begin{quote}
\begin{verbatim}
\TER, \LUN, \MER, \VEN, \SOL, \MAR, \GIO, \SAT
\end{verbatim}
\end{quote}
% JUP corrig\'e en GIO jps le 28-01-05

\begin{maurotex}
\TER, \LUN, \MER, \VEN, \SOL, \MAR, \GIO, \SAT
\end{maurotex}

Per le costellazioni dello Zodiaco (Ariete, Toro,
Gemelli, Cancro, Leone, Vergine, Bilancia, Scorpione,
Sagittario, Capricorno, Acquario\new e Pesci):

\index{\bs{}ARS}
\index{\bs{}TRS}
\index{\bs{}GMN}
\index{\bs{}CNC}
\index{\bs{}LEO}
\index{\bs{}VRG}
\index{\bs{}LBR}
\index{\bs{}SCR}
\index{\bs{}SGT}
\index{\bs{}CPR}
\index{\bs{}AQR}
\index{\bs{}PSC}

\begin{quote}
\begin{verbatim}
\ARS, \TRS, \GMN, \CNC, \LEO, \VRG,
\LBR, \SCR, \SGT, \CPR, \AQR, \PSC
\end{verbatim}
\end{quote}

\begin{maurotex}
\ARS, \TRS, \GMN, \CNC, \LEO, \VRG,
\LBR, \SCR, \SGT, \CPR, \AQR, \PSC
\end{maurotex}
% ACQ corrig\'e en \AQR. jps le 28-01-05

Pour les conjonctions et les oppositions\new:

\index{\bs{}CNJ}
\index{\bs{}OPP}

\begin{quote}
\begin{verbatim}
\CNJ, \OPP
\end{verbatim}
\end{quote}

\begin{maurotex}
\CNJ, \OPP
\end{maurotex}

Inoltre per il segno di ``radix'' (cio{\`e} la R con la gamba allungata e
tagliata da un segno trasversale), maiuscola e minuscola, et pour les
signes alg\'ebriques du ``plus'' et du ``moins''\new:

\index{\bs{}RDX}
\index{\bs{}rdx}
\index{\bs{}ptilde}
\index{\bs{}mtilde}

\begin{quote}
\begin{verbatim}
\RDX, \rdx, \ptilde, \mtilde
\end{verbatim}
\end{quote}

\begin{maurotex}
\RDX, \rdx, \ptilde, \mtilde
\end{maurotex}

{\new} On prendra garde cependant \`a transcrire les signes ``r.'' par de simples ``r.'' dans le texte.

Questa sezione di caratteri speciali verr{\`a} ampliata via via
che se ne presenter{\`a} la necessit{\`a}.

Segnaliamo inoltre
che potr{\`a} essere utile, nell'apparato o nelle introduzioni,
riferirsi a un certo paragrafo di un certo testo. Il comando
{\TeX} da usare {\`e} \verb"\S"\index{\bs{}S}, che genera il segno \S.

%-----------------------------------------------------------------------
\subsection{Abbreviazioni comuni}

\label{ref-3.4.4}
\index{abbreviazioni}
\index{abbreviazioni comuni}

Potr{\`a} avvenire che l'editore avverta la necessit{\`a} di
rendere in apparato una particolare abbreviazione (cfr. pi{\'u}
avanti \S\,\ref{ref-4.2.1}). Come {\`e} noto moltissime abbreviazioni sono
indicate da un segno di contrazione: ad esempio $\widetilde{aia}$\index{\bs{}CONTR}, per anima. Per ottenere questo
risultato dovrete scrivere:

\begin{quote}
\begin{verbatim}
\CONTR{aia}
\end{verbatim}
\end{quote}

Pi{\'u} semplice {\`e} il caso in cui si tratti di riportare
l'usuale abbreviazione per la nasale ``m'' o ``n''. Se si
vuole avere ``co{\=m}un{\=e}'' per ``communem'', baster{\`a}
scrivere:

\begin{quote}
\begin{verbatim}
co{\=m}un{\=e}
\end{verbatim}
\end{quote}

\noindent dove il comando \verb"\="\index{\bs{}=} provvede a collocare il
segno abbreviativo sopra la lettera a lui seguente, esattamente come per gli
accenti.

%{\new} Il est aussi possible d'utiliser les commandes
%\verb"\CON"\index{\bs{}CON} et \verb"\PER"\index{\bs{}PER} pour marquer les
%abbr\'eviations {\CON} et {\PER}.
%XXX Ces commandes ne donnent pas le m�me r�sultat en html et ps; strike pour html; \" pour le ps.

Forniamo una breve lista di come rendere abbreviazioni
comuni, utilizzando alcuni comandi {\TeX}:

\begin{center}
\begin{tabular}{rcll}

\onlylatex{{\ae}}\onlyhtml{\includegraphics{manicons/ae.gif}} & = & \verb"{\ae}" \\

\onlylatex{{\oe}}\onlyhtml{\includegraphics{manicons/oe.gif}} & = & \verb"{\oe}" \\

\onlylatex{{\AE}}\onlyhtml{\includegraphics{manicons/AE.gif}} & = & \verb"{\AE}" \\

\onlylatex{{\OE}}\onlyhtml{\includegraphics{manicons/OE.gif}} & = & \verb"{\OE}" \\

\onlylatex{{\c e}}\onlyhtml{\includegraphics{manicons/ce.gif}} & = & \verb"{\c e}" & (la \verb"{\c}" appone una cediglia sotto \\

& & & la lettera seguente)\\

\onlylatex{{\.o}}\onlyhtml{\includegraphics{manicons/do.gif}} & = & \verb"{\.o}" \\

\onlylatex{{\d o}}\onlyhtml{\includegraphics{manicons/po.gif}} & = & \verb"{\d o}" \\

\onlylatex{{\v o}}\onlyhtml{\includegraphics{manicons/vo.gif}} & = & \verb"{\v o}" & (segno di vocale breve)\\

{\"u} & = & \verb+{\"u}+ \\

{\&} & = & \verb"{\&}" \\

\onlylatex{{\it\&}}\onlyhtml{\includegraphics{manicons/itet.gif}} & = & \verb"{\it\&}" \\

{\ss} & = &\verb"{\ss}"

\end{tabular}
\end{center}

Le parentesi graffe servono per far s{\'\i} che il
segno che scrivete si comporti a tutti gli effetti come un
solo carattere, senza essere influenzato da spazi bianchi.
Ad esempio, battendo:

\begin{quote}
\begin{verbatim}
\aequalis
\end{verbatim}
\end{quote}

\noindent ottereste un messaggio di errore in fase di
compilazione che vi direbbe che ``aequalis'' {\`e} una
``undefined control sequence''. Cosa che non vi accadr{\`a} se
scriverete:

\begin{quote}
\begin{verbatim}
{\ae}qualis
\end{verbatim}
\end{quote}

\noindent ottenendo invece, come volevate, ``{\ae}qualis''.

Notate poi che gli spazi bianchi in \verb"{\c e}", \verb"{\v o}", \verb"{\d o}"
sono essenziali, pena ritrovarvi con il messaggio di una
``undefined control sequence''!

%-----------------------------------------------------------------------
\subsection{Codificazione di abbreviazioni}

\label{ref-3.4.5}
\index{abbreviazioni}

Si deve tuttavia ricordare che \textit{le abbreviazioni}, di norma,
\textit{vengono tutte sciolte} (cfr. \S\,\ref{ref-1.1}, 3) e che si
riportano solo \textit{in casi eccezionali} (cfr., di nuovo, l'esempio del
\S\,\ref{ref-4.2.1}). Il trascrittore potr{\`a} per{\`o} trovarsi a volte
in situazioni ambigue, in cui potrebbe non riuscire a sciogliere
l'abbreviazione o esitare fra vari scioglimenti possibili. Sar{\`a}
opportuno che tali casi, piuttosto che venire risolti graficamente, vengano
codificati per mezzo della macro \verb"\ABBR{}"\index{\bs{}ABBR}.

Ci si potrebbe ad esempio trovare ad esitare fra leggere
``propositio'' e ``proportio''. In un caso del genere, una
volta operata la scelta (diciamo ``proportio'') si scriver{\`a}

\begin{quote}
\begin{verbatim}
\ABBR{proportio}\Comm{L'abbbreviazione potrebbe
essere sciolta anche come `propositio'}
\end{verbatim}
\end{quote}

Il risultato sar{\`a} che nel testo verr{\`a} stampato
solo \textit{proportio}, tuttavia l'ambiguit{\`a} rimarr{\`a}
codificata (e quindi, per esempio, si potr{\`a} far s{\'\i}
che si possa vedere, volendo, una riproduzione fotografica
dell'abbreviazione stessa) e segnalata nel commento. Come
abbiamo gi{\`a} pi{\'u} volte osservato, \textit{melius abundare} con
indicazioni di questo tipo, almeno in fase di trascrizione
iniziale.

%-----------------------------------------------------------------------
\subsection{Le abbreviazioni che non si devono sciogliere: esponenti e
deponenti}

\label{ref-3.4.6}
\index{abbreviazioni}
\index{esponenti}
\index{deponenti}

Come abbiamo gi{\`a} osservato nelle indicazioni per il
trascrittore (\S\,\ref{ref-1.2}) le abbreviazioni indicanti proposizioni
del tipo ``per 11\textsuperscript{am}~6\textsuperscript{i}'' devono invece
essere conservate. Analogamente, si conseveranno formule del
tipo ``Ill.\textsuperscript{mo} ac Rev.\textsuperscript{mo} Domino'' e simili,
consacrate dall'uso. Per rendere queste abbreviazioni che
prevedono di dover scrivere del testo ad esponente in
carattere minore, si utilizzer{\`a} la macro \verb"\Sup{}"\index{\bs{}Sup{}}, nel
seguente modo. Per ottenere, ad esempio

\begin{maurotex}
11\Sup{am} 6\Sup{i}
\end{maurotex}

\noindent si scriver{\`a}:

\begin{quote}
\begin{verbatim}
11\Sup{am} 6\Sup{i}
\end{verbatim}
\end{quote}

\noindent e per ottenere

\begin{maurotex}
Ill.\Sup{mo} ac Rev.\Sup{mo} Domino
\end{maurotex}

\noindent si scriver{\`a}

\begin{quote}
\begin{verbatim}
Ill.\Sup{mo} ac Rev.\Sup{mo} Domino
\end{verbatim}
\end{quote}

Si noti che, se per qualche strano motivo, si volessero ottenere gli
esponenti in nero (Ill.\textsuperscript{\textbf{mo}} ac Rev. \textsuperscript{\textbf{mo}} Domino),
basterebbe scrivere:

\begin{quote}
\begin{verbatim}
Ill.\Sup{\bf mo} ac Rev. \Sup{\bf mo} Domino
\end{verbatim}
\end{quote}

Del tutto simile a \verb"\Sup{}"\index{\bs{}Sup{}} {\`e} la macro
\verb"\Sub{}"\index{\bs{}Sub{}} che permette di scrivere a deponente nel caso
fosse necessario in particolari situazioni. Infine {\`e} disponibile la
macro \verb"\SupSub{}{}"\index{\bs{}SupSub} che permette di realizzare un esponente e un
deponente allineati: se si volesse ottenere ad esempio

\begin{maurotex}
M\SupSub{auro}{lico}
\end{maurotex}

\noindent basterebbe scrivere
 
\begin{quote}
\begin{verbatim}
M\SupSub{auro}{lico}
\end{verbatim}
\end{quote}

%-----------------------------------------------------------------------
\section[Macro per codificazione]{Macro per la codificazione di
informazioni  riguardanti il testo}

\label{ref-3.5}

%-----------------------------------------------------------------------
\subsection{Folium}

\label{ref-3.5.1}
\index{folium}

Il testimone che viene trascritto ha una sua paginazione che
dovr{\`a} essere riportata. Ci{\`o} si ottiene con la macro

\begin{quote}
\begin{verbatim}
\Folium{}
\end{verbatim}
\end{quote}

All'interno delle \verb"{}" andr{\`a} scritta il \textit{siglum} che
contraddistingue il testimone (A, C o S: per maggiori dettagli
sull'utilizzo dei \textit{sigla} si veda il capitolo \ref{ref-10}\footnote{Nonostante
che, per le ragioni illustrate nel capitolo \ref{ref-10}, nell'ambito del
\textit{Progetto Maurolico} si utilizzino solo le lettere A, C e S, nel
corso della nostra esposizione abbiamo liberamente utilizzato tutte le
lettere dell'alfabeto latino per motivi di semplicit{\`a}.}), seguita da
`\textbf{:}' e dall'indicazione del numero di folio che inizia. Ad esempio
se dopo le parole ``tangat recta'' finisce il folio 34v del testimone A e
inizia il folio 35r con la parola ``circulum iam descriptum'', si
scriver{\`a}:

\begin{quote}
\begin{verbatim}
tangat recta \Folium{A:35r} circulum iam descriptum
\end{verbatim}
\end{quote}

\noindent producendo un testo di questo tipo :

%\begin{quote}
%tangat recta \Folium{A:35r} circulum iam descriptum
%\end{quote}

\begin{quote}
tangat recta $\vert$ circulum iam descriptum \hspace{\stretch{1}}
\textsc{a:35}r
\end{quote}
% ici on triche

Si pu{\`o} scegliere dove si vuole che venga apposta l'annotazione
``A:35r''. Se si vuole che tali annotazioni vengano poste in margine, come
nell'esempio, si deve scrivere nel preambolo all'inizio del file, prima
cio\`e di
\verb"\begin{document}"\index{\bs{}begin\{document\}},\index{\bs{}FoliumInMargine}

\begin{quote}
\begin{verbatim}
\FoliumInMargine
\end{verbatim}
\end{quote}

Se si vuole invece che essa venga posta nel testo, fra parentesi quadre (e
in tal caso non verr{\'a} stampata la $\vert$), nel preambolo del file
occorre scrivere\index{\bs{}FoliumInTesto}

\begin{quote}
\begin{verbatim}
\FoliumInTesto
\end{verbatim}
\end{quote}

\noindent e si otterr{\`a} quindi

\begin{quote}
tangat recta [\textsc{a:35}r] circulum iam descriptum
\end{quote}
% ici on triche

Questa seconda opzione pu{\`o} essere utile nel caso
si disponga di pi{\'u} di un testimone e che l'annotazione in
margine risulti confusa, perch{\'e} i due testimoni finiscono
pi{\'u} o meno nello stesso punto molto spesso.

Infine, se non si vuole che vengano indicate nella stampa la
divisione in fogli del testimone, basta non scrivere nulla
all'inizio: in altre parole, la macro \verb"\Folium"\index{\bs{}Folium} per \textit{default} non produce nulla.

%-----------------------------------------------------------------------
\subsection{Citazioni}

\label{ref-3.5.2}
\index{citazioni}

La trascrizione dovr{\`a} anche provvedere a codificare le citazioni di
altri testi. Ad esempio, se ---~come avviene assai spesso~--- Maurolico
scrive ``erit, per primam sexti, ut triangulus~{\dots}'' occorrer{\`a}
codificare in modo opportuno tale informazione. La macro da utilizzare
{\`e} \verb"\Cit"\index{\bs{}Cit}. Ne vedremo fra un attimo la sintassi, ma
prima {\`e} meglio chiarire alcuni punti.

In primo luogo si codifica solo ci{\`o} che Maurolico afferma
espressamente (pena una sorta di regresso agli assiomi di
Euclide).

Ma, anche cos{\'\i}, ci si pu{\`o} trovare di fronte a tre diversi tipi di
situazione:

\begin{enumerate}

\item la citazione mauroliciana {\`e} chiara e
univoca, come nell'esempio sopra riportato;

\item la citazione non {\`e} cos{\'\i} esplicita, ma
l'editore riesce ad individuare un passo che si riferisce
univocamente ad essa. Ad esempio, Maurolico potrebbe
scrivere, in termini vaghi, ``per doctrinam Euclidis'', ma
la sagacia dell'editore lo porterebbe a scoprire che il
riferimento a Euclide non pu{\`o} essere altro che al teorema
di Pitagora (\textit{Elementi}, I.47);

\item infine, la citazione potrebbe essere
equivoca. Se ad esempio Maurolico scrivesse, parlando delle
terre abitate, ``per doctrinam Ptolemaei'' l'editore
potrebbe rimanere in dubbio se si vuole riferire a un passo
dell'\textit{Almagesto} o della \textit{Geographia}, dato che
Tolomeo parla della cosa in entrambe le opere.

\end{enumerate}

Come si vede il lavoro del trascrittore e quello
dell'editore sono qui molto diversi, in quanto il primo deve
provvedere solo a codificare il testo, il secondo deve
specificare e arricchire tale codifica. La sintassi di
\verb"\Cit"\index{\bs{}Cit} cerca di tener conto di questa casistica e delle
differenti esigenze di lavoro del trascrittore e
dell'editore.

Per la trascrizione la sintassi di \verb"\Cit"\index{\bs{}Cit} {\`e} molto semplice:

\begin{quote}
\begin{verbatim}
erit, \Cit{
          {per primam sexti}
          }, ut triangulus
\end{verbatim}
\end{quote}

\noindent provvedendo cos{\'\i} a codificare il fatto che ``per primam
sexti'' non {\`e} testo normale, ma una citazione. Per l'editore le cose si
complicano un po'. Se si tratta di una citazione univoca come questa,
dovr{\`a} solo provvedere ad aggiungere un'etichetta che identifichi il
passo in modo univoco, rispetto all'edizione critica moderna o, in
mancanza, caso, a un'edizione che dovr{\`a} servire di riferimento
assoluto:

% \`a partir d'ici les changements de Veronica.
{\new}

\begin{verbatim}
erit, \Cit{
          {per primam sexti}{EUC/ELE/VI/1}
          }, ut triangulus
\end{verbatim}

\noindent dove \texttt{EUC/ELE/VI/1} sta per ``Euclide, \textit{Elementi},
edizione di Heiberg, libro VI, proposizione 1''. Si noti che questa
situazione di ``univocit\`a'' \`e quella che riguarda la grandissima
maggioranza dei casi.

Se ci si trova invece nel secondo caso, sar\`a bene specificare nella
codifica che l'identificazione della citazione ha comportato un intervento
non banale dell'editore. Invece di scrivere semplicemente
\verb"\Cit"\index{\bs{}Cit}, si scriver\`a
\verb"\Cit[imp]"\index{\bs{}Cit[imp]} e la codifica completa sar\`a:

\begin{verbatim}
erit, \Cit[imp]{
{per doctrinam Euclidis}{EUC/ELE/I/47}
}, quadratus praedictus aequalis duobus quadratis
\end{verbatim}

\noindent e, volendo, l'editore potr\`a lasciare traccia del suo lavoro in
questo modo:

\begin{verbatim}
erit, \Cit[imp]{
{per doctrinam Euclidis}{EUC/ELE/I/47}
{{\`e} senza dubbio il teorema di Pitagora}
}, quadratus praedictus aequalis duobus quadratis
\end{verbatim}

La sintassi completa delle macro \verb"\Cit"\index{\bs{}Cit} e di
\verb"\Cit[imp]"\index{\bs{}Cit[imp]} prevede quattro sottocampi:

\begin{enumerate}

\item nel primo va inserito il testo originale della citazione. Questo
campo viene sempre riempito, dato che contiene il testo originale;

\item il secondo contiene il riferimento completo alla proposizione citata:
``sigla autore/sigla opera/libro/proposizione'' oppure ``sigla autore/sigla
opera/proposizione'', come ad esempio:

\begin{verbatim}
    EUC/ELE/libro/proposizione
    ARC/DIM/proposizione
    TOL/ALM/riferimento
\end{verbatim}

Nel primo caso si stanno citando gli \textit{Elementi} di Euclide, nel
secondo il \textit{De dimensione circuli} di Archimede e nel terzo
l'\textit{Almagesto} di Tolomeo. Nel terzo caso, il termine 'riferimento'
sta ad indentificare qualsiasi cosa che serva ad identificare il passo
dell'\textit{Almagesto} (numero della carta, linee del testo, ecc...).
Questo campo deve essere obbligatoriamente riempito, ma eventualmente in un
secondo momento. Parleremo fra breve della questione delle sigle da
indicare in questo sottocampo.

\item il terzo campo, \`e riservato ad eventuali commenti dell'editore che
non compaiono in nota. Questo campo potrebbe anche non venire mai riempito.
Se l'editore non ritiene di doversi annotare qualcosa, non ha bisogno di
questo campo, come si vede dagli esempi precedenti.

\item il quarto campo, che \`e opzionale, pu� contenere qualsiasi cosa che
si vuole far comparire in nota dopo la decodifica del secondo campo
(eventualmente, si lascia vuoto).

\end{enumerate}

Resta infine da considerare il caso della citazione equivoca. L'editore
specificher\`a la situazione utilizzando
\verb"\Cit[eqv]"\index{\bs{}Cit[eqv]}, in questo modo:

%%% QUESTA CODIFICA \`e VECCHIA E SAREBBE TUTTA DA VERIFICARE!!!
%%% ho dei seri dubbi che funzioni!!!

\begin{verbatim}
tales gentes, \Cit[eqv]{
                       {ut vult Ptolemaeus}
                       {TOL/ALM/1.7/TOL/ALM/3.5}
                       }, vivunt in germanicis sylvis
\end{verbatim}

dove le etichette per i due luoghi vengono inserite nel terzo campo,
separandole con una barra /.

La costruzione dell'apparato delle fonti pone diversi problemi. Il caso
pi\`u semplice \`e quello in cui viene identificata chiaramente la
proposizione citata nel testo, come nel caso seguente:

\begin{quote}
\par
\stelle
\par
Quamvis ergo per Archimedem\textsuperscript{[I]} ostensum sit, rationem
periferiae ad diametrum minorem quidem esse, quam triplam sesquiseptimam,
maiorem vero quam triplam ac decem septuagesimas primas superpartientem ...
\\
\rule{1.5cm}{1pt}
\onlylatex{\\ }
\begin{footnotesize}
\textsuperscript{[I]} \textsl{\textsf{Archim. Dim. Circ., 3}} Cuiusvis
sphaerae perimetrus diametro triplo maior est, et praeterea excedit spatio
minore, quam septima pars diametri est, maiore autem quam $\frac{10}{71}$.
\end{footnotesize}
\par
\stelle
\par
\end{quote}
% ici on triche doublement

In questo caso, non c'\`e alcun dubbio che l'autore del
testo si sta riferendo alla proposizione 3 del \textit{De
dimensione circuli} di Archimede. Quindi, una volta scelta
l'edizione archimedea di riferimento, nell'apparato delle fonti
compare -- nel modo che vedremo in seguito -- il testo della
proposizione citata.

Talvolta le cose non sono cos� semplici. Nel caso dell'opera di
Maurolico, per esempio, quando il matematico cita una proposizione
euclidea, non \`e sempre chiaro se stia citando dagli
\textit{Elementa} nell'edizione di Campano o di Zamberti o se si
stia addirittura riferendo all'edizione euclidea ``ex traditione
Maurolyci''.

Supponiamo che l'editore voglia costruire un apparato delle fonti
in cui la citazione euclidea si possa riferire ad alcune oppure a
tutte queste possibili edizioni, compresa l'edizione critica
moderna di Heiberg. Nell'esempio che segue, l'editore ritiene che
Maurolico si stia riferendo alla proposizione VI.16 secondo gli
\textit{Elementa}  di Campano, pur tuttavia vuole evidenziare che
la proposizione VI.16 ``ex traditione Campani'' corrisponde alla
proposizione VI.17 dell'edizione di Zamberti e di Heiberg. In
questo caso, non ci sono ri\-fe\-ri\-men\-ti all'edizione degli
\textit{Elementa} di Maurolico, perch\'e non ci \`e pervenuto il sesto
libro. L'editore vuole dunque ottenere un apparato delle fonti di
questo tipo:

\begin{quote}
\par
\stelle
\par
Nam cum per 16\textsuperscript{am} Sexti\textsuperscript{[I]}
rectangulum\textsuperscript{1} quod sub extremis, aequale sit ei, quod a
media quadrato. Iam ex prima vel 3\textsuperscript{a} huius absolvitur
problema.
\\
\rule{1.5cm}{1pt}
\onlylatex{\\ }
\begin{footnotesize}
\textsuperscript{[I]} \textsl{\textsf{Eucl. Elemen. VI.16 Camp.}}
(\textit{VI.17 Zamb.,  VI.17 Heib.}) Si fuerint tres lineae proportionales,
quod sub prima et tertia rectangulum continetur, aequum erit ei quod a
secunda quadrato describitur. Si vero quod sub prima et tertia continetur
aequum ei quadrato quod a secunda producitur, ipsae tres lineae
proportionales erunt.
\end{footnotesize}
\\
\rule{1.5cm}{1pt}
\onlylatex{\\ }
\begin{footnotesize}
\textsuperscript{1} rectangulum \textit{conieci} rectangulo \textit{A}
\end{footnotesize}
\par
\stelle
\par
\end{quote}

Vediamo come si pu� costruire un siffatto apparato
delle fonti.

Primo compito dell'editore \`e quello di associare ad ogni coppia
(autore citato, opera citata) un'opportuna ed univoca
``etichetta'', cio\`e una coppia di sigle formate da caratteri
alfanumerici che identificano rispettivamente l'autore e l'opera.

Vediamo qualche esempio:

EUC
-- ELE = Edizione critica degli \textit{Elementa} di Euclide
curata da Heiberg

CAM -- ELE = Edizione degli \textit{Elementa} stampata a Venezia
nel 1482 ed attribuita a Campano da Novara

ZAM -- ELE = Edizione degli \textit{Elementa} stampata a Venezia
nel 1505 e curata da Bartolomeo Zamberti

ARC -- DIM = Edizione del \textit{De dimensione circuli} di
Archimede curata da Heiberg

MAU -- DIM = Edizione del \textit{De dimensione circuli} di
Archimede curata da Maurolico

BAS -- DIM = Edizione del \textit{De dimensione circuli} di
Archimede pubblicata a Basilea nel 1544

TOL -- ALM = Edizione critica dell'\textit{Almagesto} curata da
Heiberg

Dopo aver stabilito le sigle, bisogna costruire un \textit{file}
di testo (\texttt{.txt}) in cui viene riportato il testo delle
citazioni che l'editore intende utilizzare. Si tenga presente che
bisogna costruire un \textit{file} per \textit{ogni} opera citata.
Vediamo, ad esempio il \textit{file} \texttt{campano.txt} che
contiene alcune proposizioni dell'\textit{Euclide} di Campano. Ad
ogni riga deve corrispondere un'unica proposizione (cosa che
nell'esempio successivo non \`e possibile fare per ragioni evidenti
di spazio), mentre lo spazio fra la numerazione ed il vero e
proprio enunciato \`e uno spazio fisso e corrisponde ad una
tabulazione:

\begin{verbatim}
I.22    Propositis tribus lineis rectis, quarum duae quaelibet
simul iunctae reliqua sint longiores, de tribus aliis lineis illis
aequalibus triangulum constituere.
II.5    Si linea recta per duo aequalia duoque inaequalia secetur,
quod sub inaequalibus totius sectionis rectangulum continetur cum
eo quadrato quod ab ea quae inter utrasque est sectiones
describitur, aequum est ei quadrato quod a dimidio totius lineae
in se ducto describitur.
VI.15   Si fuerint quatuor lineae proportionales, quod sub prima
et ultima rectangulum continetur, aequum erit ei quod sub duabus
reliquis. Si vero quod sub prima et ultima continetur, aequum
fuerit ei quod duabus reliquis continetur rectangulum, quatuor
lineas proportionales esse convenit.
VI.16 Si fuerint tres lineae proportionales, quod sub prima et
tertia rectangulum continetur, aequum erit ei quod a secunda
quadrato describitur. Si vero quod sub prima et tertia continetur
aequum ei quadrato quod a secunda producitur, ipsae tres lineae
proportionales erunt.
\end{verbatim}

Se si trova di fronte ad un caso come quello delle citazioni
euclidee, l'editore dovr\`a anche costruirsi una tabella di
corrispondenze. Nel caso delle citazioni degli \textit{Elementa}
in Maurolico, abbiamo quattro edizioni in gioco: le edizioni di
Campano, Zamberti e Maurolico e l'edizione critica di Heiberg.

La tabella di corrispondenza \`e un \textit{file} di testo con
estensione \texttt{.tab} (ad es. \texttt{elementi.tab}) che
contiene tante righe quante sono le proposizioni da mettere in
relazione. In ogni riga si trovano ordinatamente indicate le
proposizioni corrispondenti nelle quattro edizioni, separate da
una tabulazione. Le proposizioni devono essere espresse da un
numero romano, che indica il libro degli \textit{Elementa} e da un
numero arabo, che indica la proposizione, separati da un punto:

\begin{verbatim}
    I.2    I.3    I.2   I.2
           I.4    I.3   I.4
    I.3    I.5    I.4
    VI.17  VI.16  VI.17
\end{verbatim}

Le proposizioni della prima, seconda, terza e quarta colonna si
riferiscono rispettivamente all'edizione di Heiberg, Campano,
Zamberti e Maurolico. Le prime due righe di questa tabella
indicano cos� che la proposizione I.2 di Heiberg corrisponde alla
I.3 di Campano e di Maurolico ed alla I.2 di Zamberti, mentre la
proposizione I.4 di Campano e Maurolico non \`e accolta in Heiberg e
corrisponde alla I.3 di Zamberti.

A questo punto l'editore dovr\`a costruirsi un proprio \textit{file}
di configurazione. Si tratta di un \textit{file} di testo con
estensione \texttt{.cnf} nel quale vanno messe in relazione le
opere (in forma abbreviata) con i \textit{file} delle citazioni.
Il \textit{file} \texttt{.cnf} pu� avere un nome qualsiasi: per
\textit{default} viene utizzato il nome del \textit{file} di input
che contiene il testo critico. Il \textit{file} contiene le
informazioni necessarie a ritrovare le proposizioni citate. Ad
ogni riga corrisponde un'unica opera. In ogni riga deve essere
indicata una coppia ``sigla autore, sigla opera'' (nell'esempio,
CAM ELE) a cui sono associati rispettivamente il nome dell'autore
e dell'opera cos� come si vuole che vengano prodotti in apparato
(Camp.\,Elem.), il nome del \textit{file} che contiene le
citazioni da far figurare in apparato (\texttt{campano.txt}), il
nome del \textit{file} HTML corrispondente da generare
(\texttt{campano.htm}) ed un'eventuale tabella di corrispondenza
(\texttt{elementi.tab}):

\begin{verbatim}
CAM ELE  Camp. Elem.   campano.txt    campano.htm  elementi.tab
ZAM ELE  Zamb. Elem.   zamberti.txt   zamberti.htm  elementi.tab
EUC ELE  Eucl. Elem.   euclide.txt    euclide.htm   elementi.tab
EUC DAT Eucl. Data  data.txt    data.htm
MAU TES Maur. Sphaer.Sferica.txt Sferica.htm
ARC DIM Archim. Dim. Circ. Dimensio.txt Dimensio.htm
ARC DSC Archim. Sp. et Cyl. Sferacilindro.txt Sferacilindro.htm
\end{verbatim}

La costruzione effettiva dell'apparato delle fonti interviene
quando si chiama in causa il preprocessore \texttt{m2lv}.
L'editore dovr\`a digitare nella riga di comando:

\begin{verbatim}
m2lv data.tex -f data.cnf
\end{verbatim}

Se tutto va bene, compare il seguente messaggio:

\begin{verbatim}
analisi del file di configurazione (data.cnf)
opere da citare:
CAM ELE  Camp. Elem.   campano.txt    campano.htm
ZAM ELE  Zamb. Elem. zamberti.txt   zamberti.htm
EUC ELE  Eucl. Elem.   euclide.txt euclide.htm
EUC DAT Eucl. Data data.txt data.htm
MAU TES Maur. Sphaer.Sferica.txt Sferica.htm
ARC DIM Archim. Dim. Circ. Dimensio.txt Dimensio.htm
ARC DSC Archim. Sp. et Cyl. Sferacilindro.txt Sferacilindro.htm
analisi terminata
\end{verbatim}

Il preprocessore crea il \textit{file} \texttt{data.m.tex} che si
pu� compilare normalmente (con il comando \texttt{latex
data.m.tex} oppure con il preprocessore \texttt{m2hv}) e produce
un'edizione con due fasce d'apparato: l'apparato critico, con note
numerate secondo numeri arabi e l'apparato delle fonti, con note
numerate secondo numeri romani, come abbiamo visto negli esempi
precedenti.

\paragraph{Ripetizioni}

Il problema delle citazioni ripetute che appaiono troppo
ravvicinate \`e chiaramente un problema legato alla versione
cartacea dell'edizione e non a quella elettronica (dove appare un
link per ogni citazione).

Quando, nella stessa pagina, compaiono almeno due citazioni uguali
sarebbe auspicabile riportare per esteso solamente la prima
limitandosi poi ad indicare la seconda, come si vede nell'esempio:

\begin{quote}
\par
\stelle
\par
Datur ergo quadratum $bd$ datum et quod ex $bc$, $ca$, scilicet per
5$^{am}$ Secundi \textit{Elementorum}\textsuperscript{[I]} rectangulum
$bc~ca$ una cum quadrato $cd$ \ldots\, Sed tale quadratum cum rectangulo
$bc~ca$ iam dato conflat quadratum $bd$ per per 5$^{am}$ Secundi
\textit{Elementorum}\textsuperscript{[II]}. Datur ergo quadratum $bd$ et
ipsa $bd$, de qua si auferatur $cd$ iampridem data superest $bc$ data ...
\\
\rule{1.5cm}{1pt}
\onlylatex{\\ }
\begin{footnotesize}
\textsuperscript{[I]} \textsl{\textsf{Camp. Elem. II.5}} (\textit{Zamb.
Elem. II.5,  Eucl. Elem. II.5}) Si linea recta per duo aequalia duoque
inaequalia secetur, quod sub inaequalibus totius sectionis rectangulum
continetur cum eo quadrato quod ab ea quae inter utrasque est sectiones
describitur, aequum est ei quadrato quod a dimidio totius lineae in se
ducto describitur.
\\
\textsuperscript{[II]} \textsl{\textsf{Camp. Elem. II.5}} (\textit{Zamb.
Elem. II.5,  Eucl. Elem. II.5})
\end{footnotesize}
\par
\stelle
\par
\end{quote}

Il risultato si ottiene mediante un marcatore in grado di inibire la stampa
della citazione. Questo marcatore \`e la macro
\verb"\Cit*"\index{\bs{}Cit*}. La macro \verb"\Cit*" ha la stessa sintassi
di \verb"\Cit"\index{\bs{}Cit}, ma inibisce la stampa del testo della
citazione, limitandosi a produrre nell'apparato solo il riferimento
all'opera citata e alle eventuali corrispondenze.

%-----------------------------------------------------------------------
\subsection{Date}

\label{ref-3.5.3}
\index{date}

Maurolico usava spesso inserire date all'interno dei
suoi scritti. Esso vengono codificate utilizzando la macro
\verb"\Date"\index{\bs{}Date}.

Ad esempio se il trascrittore si imbatte in un testo del
tipo ``Completum Messanae in Freto siculo Dominicae
Incarnationis 1554 Indictione XIII'' trascriver{\`a} il brano in
questo modo:

\begin{quote}
\begin{verbatim}
\Date{
     {Completum Messanae in Freto siculo Dominicae
     Incarnationis 1554 Indictione XIII}
     }
\end{verbatim}
\end{quote}

L'editore dovr{\`a} provvedere a identificare tale
data secondo il sistema attuale (giorno/mese/anno) e
aggiunger{\`a} un secondo sottocampo:

\begin{quote}
\begin{verbatim}
\Date{
     {Completum Messanae in Freto siculo Dominicae
     Incarnationis 1554 Indictione XIII}
     {15.06.1554}
     }
\end{verbatim}
\end{quote}

Se poi avesse da notare qualcosa (ad esempio che
l'indizione del 1554 non era la XIII) lo potr{\`a} fare
aggiungendo un terzo sottocampo:

\begin{quote}
\begin{verbatim}
\Date{
     {Completum Messanae in Freto siculo Dominicae
     Incarnationis 1554 Indictione XIII}
     {15.06.1554}
     {L'indizione del 1554 non era la XIII. O {\`e}
     sbagliato l'anno o {\`e} sbagliata l'indizione}
     }
\end{verbatim}
\end{quote}

Queste informazioni verranno stampate in un file a parte che
terr{\`a} un registro delle date presenti nei testi
mauroliciani.

%-----------------------------------------------------------------------
\subsection{Codificazione di titoli}

\label{ref-3.5.4}
\index{titoli}

Il trascrittore dovr{\`a} inoltre provvedere a marcare con una macro
apposita i titoli delle opere che compaiono nel testo. Ad esempio per
trascrivere la frase ``ut ait Ptolemaeus in Almagesto'' dovr{\`a}
segnalare che ``Almagesto'' {\`e} il titolo di un libro usando la macro
\verb"\Tit{}"\index{\bs{}Tit}. Scriver{\`a} quindi, in questo caso:

\begin{quote}
\begin{verbatim}
ut ait Ptolemaeus in \Cit[imp]{
                              {\Tit{Almagesto}}
                              }
\end{verbatim}
\end{quote}

pour obtenir:

\begin{maurotex}
ut ait Ptolemaeus in \Cit[imp]{
                              {\Tit{Almagesto}}
                              }
\end{maurotex}

Lo scopo di questa marcatura {\`e} quello di far s{\'\i} che i titoli delle
opere citate compaiano in corsivo. L'uso di \verb"\Tit"\index{\bs{}Tit} non si limita
all'interno di \verb"\Cit"\index{\bs{}Cit} (si veda ad esempio \S\,\ref{ref-6.1.3}).

%-----------------------------------------------------------------------
\section[``formule'' mauroliciane]{Macro per la resa delle ``formule'' mauroliciane}

\label{ref-3.6}

\subsection{L'ambiente matematico}

\label{ref-3.6.1}
\index{ambiente matematico}

Come abbiamo gi{\`a} osservato, Maurolico va pazzo per la
simbolizzazione e certi suoi testi contengono quasi pi{\'u}
simboli che parole. Tale caratteristica andr{\`a}
rispettata, fatto salvo ci{\`o} che si osservava nel \S\,\ref{ref-1.2} e nel
\S\,\ref{ref-2.4}.

La trascrizione della matematica mauroliciana avviene
ponendosi all'interno dell'ambiente matematico, cosa
che si pu{\`o} fare semplicemente scrivendo il testo
all'interno di una coppia di parentesi tonde precedute dal
backslash: \verb"\( \)"\footnote{Come si {\`e} gi{\`a} osservato (\S\,\ref{ref-3.2.1}, nota 4), si pu{\`o}
utilizzare anche il carattere \texttt{\$} per delimitare l'inizio e la fine
dell'ambiente matematico. Si osservi, inoltre, che l'ambiente matematico
permette anche di scrivere ad esponente. Per esempio,
``per 22\textsuperscript{am} 5\textsuperscript{i}''
si pu{\`o} ottenere scrivendo:

\begin{quote}
\texttt{
per 22$\backslash$($\wedge$\{$\backslash$rm am\}$\backslash$) 5$\backslash$($\wedge$\{$\backslash$rm i\}$\backslash$)
}
\end{quote}
%} pour r\'ecup\'erer les couleurs de l'\'editeur !

\noindent dove il carattere~\texttt{$\wedge$}~(accento circonflesso) posto fra \texttt{$\backslash$( $\backslash$)}
serve appunto ad indicare che ci{\`o} che segue deve essere
collocato ad esponente. Si noti anche il comando \texttt{$\backslash$rm} che
ordina al \TeX\ di scrivere i caratteri alfabetici ad
esponente in tondo (nell'ambiente matematico tutte i
caratteri alfabetici vengono automaticamente stampati in
corsivo).

{\`E} tuttavia consigliabile utilizzare la macro
\texttt{$\backslash$Sup\{\}} che non richiede l'apertura dell'ambiente
matematico e permette molta pi{\'u} elasticit{\`a} e pulizia nella
trascrizione.}.

L'esempio pi{\'u} banale {\`e} la trascrizione di
lettere indicanti grandezze geometriche o numeri o altro
che, come si {\`e} detto, devono risultare in corsivo. Per
trascrivere il testo ``Rectae \(ab\), \(cd\) erunt aequales'',
baster{\`a} scrivere:

\begin{quote}
\begin{verbatim}
Rectae \(ab\), \(cd\) erunt aequales
\end{verbatim}
\end{quote}

Si noti che bisogna fare molta attenzione a chiudere
l'ambiente matematico una volta che esso sia stato aperto,
pena un diluvio di messaggi di errore in fase di
compilazione del vostro testo (o peggio: nessun messaggio ma
un testo tutto sbagliato, che il {\TeX} non riesce a
compilare) e un risultato a dir poco
strano. 

Occorre anche tenere presente che \textit{lo spazio in ambiente matematico
non viene interpretato dal {\TeX}}. Di conseguenza, se battete:

\begin{quote}
\begin{verbatim}
Rectae \(ab cd\) erunt aequales
\end{verbatim}
\end{quote}

\noindent otterrete come risultato

\begin{quote}
Rectae \(ab cd\) erunt aequales
\end{quote}

Per evitare questo inconveniente potrete o fare come {\`e} stato 
indicato qui sopra, aprendo e chiudendo l'ambiente matematico due volte 
di seguito, oppure battere al posto di un normale spazio biancco uno
\textit{spazio bianco codificato} (\verb"~", cfr. \S\,\ref{ref-3.2.1}):

\begin{quote}
\begin{verbatim}
Rectae \(ab~cd\) erunt aequales
\end{verbatim}
\end{quote}

Spesso nei testi mauroliciani si incontrano delle frazioni: esse poranno
essere tratate con la macro \verb"\frac{}{}"\index{\bs{}frac}{\new} che ha
come primo argomento il numeratore della frazione e come secondo il
denominatore. Ad esempio volendo ottenere:

\begin{maurotex}
erit parabola \(\frac{4}{3}\) trianguli \(abc\) inscripti
\end{maurotex}
% ici j'ai corrig\'e \over en \frac partout: jps le 25-01-05

\noindent si dovr{\`a} battere

\begin{quote}
\begin{verbatim}
erit parabola \(\frac{4}{3}\) trianguli \(abc\) inscripti
\end{verbatim}
\end{quote}

Per maggiori dettagli sull'ambiente matematico si
faccia riferimento a un manuale di {\LaTeX}.

%-----------------------------------------------------------------------
\subsection{Formule}

\label{ref-3.6.1b}
\index{formule}

{\new}
Le {\mtex} permet maintenant d'int\'egrer des formules math\'ematiques. Attention toutefois:

\begin{itemize}

\item \textbf{Il ne faut pas utiliser cette possibilit\'e dans le cadre du Projet Maurolico !} Ses math\'ematiques ne n\'ecessitent pas cette fonctionnalit\'e.

\item Cette fonctionnalit\'e est \textbf{incompatible} ---~pour le moment en tout cas~--- avec la sortie \textsc{html}.

\end{itemize}

En fonction de quoi, si vous continuez \`a lire ces lignes, c'est que vous ne travaillez pas \`a un texte du Projet Maurolico \textbf{et} que vous ne souhaitez qu'une sortie \textsc{postscript} ou \textsc{pdf}.

La syntaxe pour les formules est strictement celle de {\LaTeX} et on se
reportera \`a un manuel g\'en\'eral de {\LaTeX} pour la conna\^itre.
L'important ici est de savoir qu'il est possible d'utiliser les macros du
{\mtex} (en particulier pour la collation, \verb"\VV", etc.; voir les
chapitres \ref{ref-4} et suivants) dans les formules math\'ematiques:

\begin{quote}
\begin{verbatim}
$$\left(\frac{n}{\int_{a}^{b}{\VV{
{A:\frac{1}{x^{2}}}
{B:\frac{2}{x^{3}}}
} \VV{{A:dx}{B:dy}}
}}\right)^{2}$$
\end{verbatim}
\end{quote}

\onlylatex{
\begin{quote}
\par
\stelle
\par
$$\left(
\frac{n}{
\int_{a}^{b} \frac{1}{x^{2}}^{[1]} dx^{[2]}
}
\right)^{2}$$
\rule{1.5cm}{1pt} \\
\begin{footnotesize}
$^1$ $\frac{1}{x^2}$ $A$ $\frac{2}{x^2}$ $B$ \\
$^2$ $dx$ $A$ $dy$ $B$
\end{footnotesize}
\par
\stelle
\par
\end{quote}
}
% ici on triche

\onlyhtml{

\begin{quote}
Rien...
\end{quote}
\par

On vient de vous dire que cette fonctionnalit\'e est incompatible avec le
\textsc{html}: que vous attendiez-vous \`a voir dans cette version
\textsc{html} du manuel ? Reportez \`a la version \textsc{pdf} !

}

%-----------------------------------------------------------------------
\subsection{Quadratini, triangoli \& Co.}

\label{ref-3.6.2}
\index{quadratini}
\index{triangoli}
\index{symboles math\'ematiques}

{\`E} all'interno dell'ambiente matematico che andranno usate
le macro del {\mtex} per ottenere i vari simboli utilizzati da
Maurolico. Ne diamo un elenco:

\begin{quote}
parallelogramma: \verb"\PRL"\index{\bs{}PRL} \\
triangolo: \verb"\TRN"\index{\bs{}TRN} \\
quadrato: \verb"\QDR"\index{\bs{}QDR} \\
rettangolo: \verb"\RTT"\index{\bs{}RTT} \\
trapezio: \verb"\TRP"\index{\bs{}TRP}{\new} \\
cube: \verb"\CUB"\index{\bs{}CUB}{\new} \\
parall\'el\'epip\`ede: \verb"\PPD"\index{\bs{}PPD}{\new} \\
pentagone: \verb"\PEN"\index{\bs{}PEN}{\new} \\
hexagone: \verb"\HEX"\index{\bs{}HEX}{\new} \\
hexagone central: \verb"\HEXC"\index{\bs{}HEXC}{\new} \\
double rectangle: \verb"\DRTT"\index{\bs{}DRTT}{\new} \\
pyramide: \verb"\PYR"\index{\bs{}PYR}{\new}
\end{quote}

\begin{maurotex}
\PRL, \TRN, \QDR, \RTT, \TRP, \CUB, \PPD, \PEN, \HEX, \HEXC, \DRTT, \PYR
\end{maurotex}


%-----------------------------------------------------------------------
\subsection{Proporzioni e proporzioni in schemi}

\label{ref-3.6.3}
\index{proporzioni}
\index{schemi}

Maurolico usa spesso una notazione simbolica anche per le
proporzioni. Ad esempio invece che ``erit $a$ ad $b$ ita $c$
ad $d$'' scrive

\begin{maurotex}
erit \(a\)---\(b\) ita \(c\)---\(d\) 
\end{maurotex}

Per
ottenere questo risultato grafico baster{\`a} scrivere:

\begin{quote}
\begin{verbatim}
erit \(a\)---\(b\) ita \(c\)---\(d\) 
\end{verbatim}
\end{quote}

\noindent rendendo il trattino mauroliciano con tre trattini
battuti uno di seguito all'altro \verb"---" senza spazi.

Spesso la simbolizzazione diventa ancora pi{\'u} pesante, perch{\'e}
una serie di possibilit{\`a} vengono racchiuse in una schema
unico.

Per trascrivere una successione di scelte, indicata
graficamente con un
fascio di lineette che si dirigono verso destra, uscendo da
un punto, si usa la macro

\begin{quote}
\begin{verbatim}
\Rcases{{scelta1}{scelta2}...{sceltaN}}
\end{verbatim}
\end{quote}

\begin{maurotex}
\Rcases{{scelta1}{scelta2}{scelta3}}
\end{maurotex}

\noindent dove \verb"scelta1" ecc. indica il testo che si riferisce
alla prima lineetta ecc. (\verb"R" sta ovviamente per
``right'', destra.)

Se le linee escono a sinistra, invece, si user{\`a} (``L'' per
``left''):

\begin{quote}
\begin{verbatim}
\Lcases{{scelta1}{scelta2}...{sceltaN}}
\end{verbatim}
\end{quote}

\begin{maurotex}
\Lcases{{scelta1}{scelta2}{scelta3}}
\end{maurotex}

Se le linee escono sia a destra che a sinistra:

\begin{quote}
\begin{verbatim}
\RLcases{{scelta1}{scelta2}...{sceltaN}}
\end{verbatim}
\end{quote}

\begin{maurotex}
\RLcases{{scelta1}{scelta2}{scelta3}}
\end{maurotex}

La stessa situazione si riscontra anche con l'uso di una sorta di
parentesi graffe. In tal caso si useranno, con la stessa
sintassi, le macro \verb"\Rbracecases"\index{\bs{}Rbracecases}, \verb"\Lbracecases"\index{\bs{}Lbracecases},
\verb"\RLbracecases"\index{\bs{}RLbracecases}. Infine a volte Maurolico propone una serie
di scelte ma senza utilizzare lineette o parentesi graffe.
In tal caso si user{\`a}

\begin{quote}
\begin{verbatim}
\Voidcases{{scelta1}{scelta2}...{sceltaN}}
\end{verbatim}
\end{quote}

\begin{maurotex}
\Voidcases{{scelta1}{scelta2}{scelta3}}
\end{maurotex}

Le macro appena descritte possono essere annidate una dentro
l'altra: se Maurolico apre una serie di 4 lineette a destra, e
da ciascuna di esse si dipartono 3 lineette sempre a destra,
si potr{\`a} scrivere:

\begin{quote}
\begin{verbatim}
\Rcases{
       {\Rcases{{scelta1.1}{scelta1.2}{scelta1.3}}}
       {\Rcases{{scelta2.1}{scelta2.2}{scelta2.3}}}
       {\Rcases{{scelta3.1}{scelta3.2}{scelta3.3}}}
       {\Rcases{{scelta4.1}{scelta4.2}{scelta4.3}}}
       }
\end{verbatim}
\end{quote}

\begin{maurotex}
\Rcases{
       {\Rcases{{scelta1.1}{scelta1.2}{scelta1.3}}}
       {\Rcases{{scelta2.1}{scelta2.2}{scelta2.3}}}
       {\Rcases{{scelta3.1}{scelta3.2}{scelta3.3}}}
       {\Rcases{{scelta4.1}{scelta4.2}{scelta4.3}}}
       }
\end{maurotex}

Se la formula con cui vi trovate a che fare {\`e}
per{\`o} troppo
complessa per essere trascritta con i mezzi qui sopra
indicati, usate \verb"\Formula{}"\index{\bs{}Formula}. Se scrivete qualcosa fra le
sue \verb"{}" esso verr{\`a} stampato. Ad esempio, se scrivete

\begin{quote}
\begin{verbatim}
\Formula{di c. 37v}
\end{verbatim}
\end{quote}

\noindent apparir{\`a} nel testo, centrato e con caratteri un
po' pi{\'u} grandi:

\begin{quote}
{\large [Formula di c. 37v] }
\end{quote}
% ici on triche. jps le 25-1-2005

%-----------------------------------------------------------------------
\subsection{Come trattare figure}

\label{ref-3.6.4}
\index{figure}

In modo simile alle formule troppo complesse si trattano le figure,
utilizzando la macro \verb"\Figskip{}"\index{\bs{}Figskip}. Sar{\`a}
opportuno che il trascrittore provveda a numerare tutte le figure del testo
che sta trascrivendo, o comunque ad assegnare loro un riferimento univoco
(ad esempio: ``la prima delle tre figure di carta 23v''). Ci{\`o} fatto,
arrivando a trascrivere la carta 23v misurer{\`a} in centimetri le figure.
Supponendo che la prima misuri 3~cm. di altezza, scriver{\`a}, nel punto in
cui vuole che venga lasciato uno spazio bianco:

\begin{quote}
\begin{verbatim}
\Figskip{3cm}\Comm{qui va la figura 17}
\end{verbatim}
\end{quote}

\noindent se le ha numerate progressivamente, oppure:

\begin{quote}
\begin{verbatim}
\Figskip{3cm}\Comm{qui va la prima figura di c. 23v}
\end{verbatim}
\end{quote}

Il commento permette di poter recuperare pi{\'u}
facilmente quale figura deve andare in quale posto.

{\new} Il est aussi possible d'ajouter une legende \`a une figure. Cela se fait dans une option (entre crochets \verb"[]") \`a ins\'erer entre la commande et son argument:

\begin{quote}
\begin{verbatim}
\Figskip[didascalia]{3cm}\Comm{qui va la prima figura di c. 23v}
\end{verbatim}
\end{quote}

Cette l\'egende sera imprim\'ee juste sous la figure.

Le figure e gli schemi in margine devono essere indicati con la macro
\verb"\FigMarg{}"\index{\bs{}FigMarg{}}. Qui, a priori, non c'{\`e} bisogno
di commentare perch{\'e} nelle parentesi si potr{\`a} inserire un breve
testo che spieghi di quale figura si tratti.

La macro \verb"\Figura{}"\index{\bs{}Figura}{\new} ins\`ere une image
dans le texte. L'argument est le nom du fichier de l'image. La macro ne
reconna\^it que deux formats: le \textsc{postscript} pour la sortie
\textsc{postscript} ou \textsc{pdf} et le format \textsc{jpg} pour la
sortie \textsc{html}. Les fichiers images doivent absolument avoir les
extensions respectives \texttt{.ps} et \texttt{.jpg}, mais ces extensions
ne doivent pas appara\^itre dans l'argument de la macro\verb"\Figura{}".

Exemple: si l'on a une image \textsc{postscript} ``fig4.ps''; pour ins\'erer
l'image dans le texte, il suffit d'\'ecrire:

\begin{quote}
\begin{verbatim}
\Figura{fig4}
\end{verbatim}
\end{quote}

Si de plus, on a une image \textsc{jpg} ``fig4.jpg'', elle figurera dans la
sortie \textsc{html} sans qu'il soit n\'ecessaire de changer quoi que ce soit dans le code source.

%-----------------------------------------------------------------------
\section{Tavole et schemi}

\label{ref-3.6.b}

{\new} Le {\mtex} permet maintenant\footnote{Octobre 2002} de g\'erer les
tableaux et certains types de sch\'emas \fg{simples}. Il utilise une syntaxe
h\'erit\'ee de {\LaTeX} mais largement adapt\'ee aux besoins de notre
\'edition. Dans les tableaux d\'ecrits ci-apr\`es, il est bien entendu
possible d'utiliser la plupart des commandes du {\mtex}, en particulier la
commande \verb"\VV" (voir le chapitre \ref{ref-4}). Les explications qui
suivent s'appuient surtout sur des exemples de cas usuels.

%--------------------------------------------------------------------
\subsection{Tableaux}

\label{ref-3.6.b.1}

\index{tabelle}
\index{tavole}
\index{tableau}

La commande principale pour cr\'eer un tableau est l'environnement
\texttt{tabula}\index{tabula}{\new}. Elle n\'ecessite un argument: la
justification de chaque cellule, colonne par colonne. Ainsi:

\begin{quote}
\begin{verbatim}
\begin{tabula}{lcr}
...
\end{tabula}
\end{verbatim}
\end{quote}

\noindent indique un tableau de trois colonnes, dont la premi\`ere a ses
\'el\'ements justifi\'es \`a gauche dans leur cellule (l = left), la
deuxi\`eme ses \'el\'ements centr\'es dans leur cellule (c = center) et la
troisi\`eme justifi\'es \`a droite (r = right).

\`A l'int\'erieur de cet environnement, l'\'ecriture du tableau se fait
ligne par ligne, de la gauche vers la droite. Chaque cellule d'une m\^eme
ligne est s\'epar\'ee par un \fg{et commercial} (\verb"&"); chaque ligne
est s\'epar\'ee par une double barre invers\'ee (\verb"\\"). Comme
d'habitude, les espaces et les retours chariot ne sont g\'en\'eralement pas
significatifs en {\LaTeX} et en {\mtex}.

Ainsi le tableau:

\onlylatex{
\par
\stelle
\par
\begin{tabular}{lcr}
cellule 1.1 & cellule 1.2 & cellule 1.3 \\
2.1 & 2.2 & 2.3
\end{tabular}
\par
\stelle
\par
}

\onlyhtml{
\begin{maurotex}
\begin{tabula}{lcr}
cellule 1.1 & cellule 1.2 & cellule 1.3 \\
2.1 & 2.2 & 2.3
\end{tabula}
\end{maurotex}
}

\noindent est-il cod\'e de la fa\c{c}on suivante:

\begin{quote}
\begin{verbatim}
\begin{tabula}{lcr}
cellule 1.1 & cellule 1.2 & cellule 1.3 \\
2.1 & 2.2 & 2.3
\end{tabula}
\end{verbatim}
\end{quote}

Toutes les cellules d'un tableau peuvent \^etre encadr\'ees: on encadre
pour cela chaque \'el\'ement de l'argument de l'environnement
\texttt{tabula}\index{tabula} avec un \verb"|" et on ins\`ere une ligne
\verb"\linea"\index{\bs{}linea}{\new} pour chaque ligne.

Ainsi le tableau:

\onlylatex{
\par
\stelle
\par
\begin{tabular}{|l|c|r|}
\hline
cellule 1.1 & cellule 1.2 & cellule 1.3 \\
\hline
2.1 & 2.2 & 2.3 \\
\hline
\end{tabular}
\par
\stelle
\par
}

\onlyhtml{
\begin{maurotex}
\begin{tabula}{|l|c|r|}
\linea \\
cellule 1.1 & cellule 1.2 & cellule 1.3 \\
\linea \\
2.1 & 2.2 & 2.3 \\
\linea 
\end{tabula}
\end{maurotex}
}

\noindent identique au pr\'ec\'edent, mais avec les cellules encadr\'ees,
est-il cod\'e de la fa\c{c}on suivante:

\begin{quote}
\begin{verbatim}
\begin{tabula}{|l|c|r|}
\linea \\
cellule 1.1 & cellule 1.2 & cellule 1.3 \\
\linea \\
2.1 & 2.2 & 2.3 \\
\linea 
\end{tabula}
\end{verbatim}
\end{quote}

Il peut \^etre n\'ecessaire qu'une cellule puisse \^etre \'etendue sur
plusieurs lignes ou plusieurs colonnes. On utilise pour cela dans les
cellules concern\'ees les commandes \verb"\mrow"\index{\bs{}mrow}{\new} et
\verb"\mcol"\index{\bs{}mcol}{\new} munies d'un argument indiquant le nombre
de lignes, respectivement de colonnes, qui seront occup\'ees par la cellule
concern\'ee. Ainsi si l'on souhaite pouvoir construire le tableau suivant
dans lequel la cellule contenant $A$ tient sur deux lignes et la cellule
contenant $C$ sur deux colonnes:

\onlylatex{
\par
\stelle
\par
\begin{tabular}{|c|c|c|c|c|}
\hline
\multirow{2}{*}{A} & B & \multicolumn{2}{c}{C} & D \\
\hline
& E & F & G & H \\
\hline
\end{tabular}
\par
\stelle
\par
}
% aie: le A est ici coup\'e par un trait horizontal

\onlyhtml{
\begin{maurotex}
\begin{tabula}{|c|c|c|c|c|}
\linea \\
\mrow{2} A & B & \mcol{2} C & D \\
\linea \\
E & F & G & H \\
\linea
\end{tabula}
\end{maurotex}
}

\noindent on \'ecrira:

\begin{quote}
\begin{verbatim}
\begin{tabula}{|c|c|c|c|c|}
\linea \\
\mrow{2} A & B & \mcol{2} C & D \\
\linea \\
E & F & G & H \\
\linea
\end{tabula}
\end{verbatim}
\end{quote}

On remarque que le $E$ qui appara\^it en premier sur la deuxi\`eme ligne du
code ci-dessus, n'occupe dans le tableau que la deuxi\`eme colonne. De
fait, le langage tient automatiquement compte du fait que $A$ occupe deux
lignes. $E$ occupe donc bien la premi\`ere \fg{vraie} cellule de la
deuxi\`eme ligne.

Il est enfin possible de remplir une cellule d'un trait horizontal avec la
commande \verb"\regula"\index{\bs{}regula}{\new}. Exemple:

\onlylatex{
\par
\stelle
\par
\begin{tabular}{|c|c|c|c|}
\hline
A & \multicolumn{2}{c}{\rule{1cm}{.5pt}} & B \\
\hline
C & D & E & F \\
\hline
\end{tabular}
\par
\stelle
\par
}

\onlyhtml{
\begin{maurotex}
\begin{tabula}{|c|c|c|c|}
\linea \\
A & \mcol{2} \regula & B \\
\linea \\
C & D & E & F \\
\linea
\end{tabula}
\end{maurotex}
}

\noindent que l'on obtient avec le code suivant:

\begin{quote}
\begin{verbatim}
\begin{tabula}{|c|c|c|c|}
\linea \\
A & \mcol{2} \regula & B \\
\linea \\
C & D & E & F \\
\linea
\end{tabula}
\end{verbatim}
\end{quote}

La commande \verb"\regula" doit toujours \^etre seule dans sa cellule.

%--------------------------------------------------------------------
\subsection{Sch\'emas}

\index{schemi}

La construction des tableaux permet de g\'erer un certain nombre de
sch\'emas\footnote{Ces sch\'emas \'etaient g\'er\'es jusqu'\`a
aujourd'hui par les commandes \texttt{$\backslash$bracecases},
\texttt{$\backslash$rcases}, etc. Le rendu html n'\'etait pas \`a la
hauteur du rendu PostScript, la nouvelle gestion des tableaux et sch\'emas
permet de l'uniformiser.}. Elle est aid\'ee par l'ajout de quatre commandes
g\'erant les accolades et les crochets:

\begin{itemize}

\item \verb"\lgra"\index{\bs{}lgra}{\new} pour \emph{\textbf{l}eft \textbf{gra}phical parenthesis}: \{

\item \verb"\rgra"\index{\bs{}rgra}{\new} pour \emph{\textbf{r}ight \textbf{gra}phical parenthesis}: \}

\item \verb"\lang"\index{\bs{}lang}{\new} pour \emph{\textbf{l}eft \textbf{ang}ular bracket}: $<$

\item \verb"\rang"\index{\bs{}rang}{\new} pour \emph{\textbf{r}ight \textbf{ang}ular bracket}: $>$

\end{itemize}

La hauteur des accolades et des crochets est g\'er\'ee automatiquement par
le {\mtex} selon le nombre de colonnes que ceux-ci doivent embrass\'es.
Exemple:

\onlylatex{
\par
\stelle
\par
A
$\left\{
\begin{tabular}{c}
B \\
D
\end{tabular}
\right\}$
C
\par
\stelle
\par
}

\onlyhtml{
\begin{maurotex}
\begin{tabula}{ccccc}
\mrow{2} A & \mrow{2} \lgra & B & \mrow{2} \rgra & \mrow{2} C \\
D
\end{tabula}
\end{maurotex}
}

\noindent est obtenu avec le code:

\begin{quote}
\begin{verbatim}
\begin{tabula}{ccccc}
\mrow{2} A & \mrow{2} \lgra & B & \mrow{2} \rgra & \mrow{2} C \\
D
\end{tabula}
\end{verbatim}
\end{quote}

Les commandes d'accolades et de crochets doivent \^etre seules dans leur
cellule et il est n\'ecessaire de tenir compte de leur nombre dans
l'argument de \verb"\begin{tabula}".

%--------------------------------------------------------------------
\subsection{Remarques}

{\new} D'un point de vue pratique, une bonne proc\'edure peut \^etre
d'\'ecrire les \'el\'ements en partant du premier en haut \`a gauche et
proc\'eder ligne par ligne vers le dernier \'el\'ement le plus en bas en
droite. Il peut \^etre int\'eressant dans les cas complexes de construire
une grille sur un papier et de remplir les cases de la grille avec les
\'el\'ements du tableau. Enfin, pour des raisons de pr\'esentation, il peut
\^etre n\'ecessaire de construire des cellules vides. Le plus simple moyen
de le faire est d'utiliser \fg{$\sim$} \`a l'int\'erieur d'une cellule.

%--------------------------------------------------------------------
\subsection{Liste compl\`etes des commandes des tableaux}

{\new}
\begin{quote}
\begin{verbatim}
\begin{tabula}
\end{tabula}
\mrow
\mcol
\lgra
\rgra
\lang
\rang
\regula
\linea
\end{verbatim}
\end{quote}

%--------------------------------------------------------------------
\subsection{Exemples}

\subsubsection{Long tableau}

{\new}
\onlyhtml{
\begin{maurotex}
\begin{tabula}{|l|c|c|c|c|c|c|c|c|c|c|}

\linea \\
\mcol{11} FORMAE NUMERARIAE PRIMI GENERIS \\
\linea \\
Radices & 1 & 2 & 3 & 4 & 5 & 6 & 7 & 8 & 9 & 10 \\
\linea \\
Pares & 0 & 2 & 4 & 6 & 8 & 10 & 12 & 14 & 16 & 18 \\
\linea \\
Impares & 1 & 3 & 5 & 7 & 9 & 11 & 13 & 15 & 17 & 19 \\
\linea \\
Trianguli & 1 & 3 & 6 & 10 & 15 & 21 & 28 & 36 & 45 & 55 \\
\linea \\
Quadrati & 1 & 4 & 9 & 16 & 25 & 36 & 49 & 64 & 81 & 100 \\
\linea \\
Parte altera longiores &
0 & 2 & 6 & 12 & 20 & 30 & 42 & 56 & 72 & 90 \\
\linea \\
Pentagoni &
1 & 5 & 12 & 22 & 35 & 51 & 70 & 92 & 117 & 145 \\
\linea \\
Hexagoni primi &
1 & 6 & 15 & 28 & 45 & 66 & 91 & 120 & 153 & 190 \\
\linea \\
Pyramides \VB{{C:triangulae}{S:\TRN\Sup{lae}}} &
1 & 4 & 10 & 20 & 35 & 56 & 84 & 120 & 165 & 220 \\
\linea \\
Pyramides quadratae. Pyramides pentagonae &
1 & 5 & 14 & 30 & 55 & 91 & 140 & 204 & 285 & 385 \\
\linea \\
Columnae \VB{{C:triangulae}{S:\TRN\Sup{lae}}} &
1 & 6 & 18 & 40 & 75 & 126 & 196 & 288 & 405 & 550 \\
\linea \\
Pyramides hexagonae primae. Cubi. Columnae quadratae &
1 & 7 & 22 & 50 & 95 & 161 & 252 & 372 & 525 & 715 \\
\linea \\
Pyramides hexagonae aequiangulae &
1 & 8 & 27 & 64 & 125 & 216 & 343 & 512& 729 & 1000 \\
\linea \\
Columnae pentagonae &
1 & 10 & 36 & 88 & 175 & 306 & 490 & 736 & 1053 & 1450 \\
\linea \\
Columnae hexagonae primae &
1 & 12 & 45 & 112 & 225 & 396 & 637 & 960 & 1377 & 1900 \\
\linea \\
Hexagoni aequianguli &
1 & 7 & 19 & 37 & 61 & 91 & 127 & 169 & 217 & 271\\
\linea \\
Columnae hexagonae aquiangulae &
1 & 14 & 57 & 148 & 305 & 546 & 889 &  1352 & 1953 & 2710 \\
\linea \\

\end{tabula}
\end{maurotex}
}

\onlylatex{\'Etant donn\'e un bug de \texttt{m2lv}, il est encore impossible de montrer le r\'esultat de l'exemple suivant.}
% XXX

\begin{quote}
\begin{verbatim}
\begin{tabula}{|l|c|c|c|c|c|c|c|c|c|c|}

\linea \\
\mcol{11} FORMAE NUMERARIAE PRIMI GENERIS \\
\linea \\
Radices & 1 & 2 & 3 & 4 & 5 & 6 & 7 & 8 & 9 & 10 \\
\linea \\
Pares & 0 & 2 & 4 & 6 & 8 & 10 & 12 & 14 & 16 & 18 \\
\linea \\
Impares & 1 & 3 & 5 & 7 & 9 & 11 & 13 & 15 & 17 & 19 \\
\linea \\
Trianguli & 1 & 3 & 6 & 10 & 15 & 21 & 28 & 36 & 45 & 55 \\
\linea \\
Quadrati & 1 & 4 & 9 & 16 & 25 & 36 & 49 & 64 & 81 & 100 \\
\linea \\
Parte altera longiores &
0 & 2 & 6 & 12 & 20 & 30 & 42 & 56 & 72 & 90 \\
\linea \\
Pentagoni &
1 & 5 & 12 & 22 & 35 & 51 & 70 & 92 & 117 & 145 \\
\linea \\
Hexagoni primi &
1 & 6 & 15 & 28 & 45 & 66 & 91 & 120 & 153 & 190 \\
\linea \\
Pyramides \VB{{C:triangulae}{S:\TRN\Sup{lae}}} &
1 & 4 & 10 & 20 & 35 & 56 & 84 & 120 & 165 & 220 \\
\linea \\
Pyramides quadratae. Pyramides pentagonae &
1 & 5 & 14 & 30 & 55 & 91 & 140 & 204 & 285 & 385 \\
\linea \\
Columnae \VB{{C:triangulae}{S:\TRN\Sup{lae}}} &
1 & 6 & 18 & 40 & 75 & 126 & 196 & 288 & 405 & 550 \\
\linea \\
Pyramides hexagonae primae. Cubi. Columnae quadratae &
1 & 7 & 22 & 50 & 95 & 161 & 252 & 372 & 525 & 715 \\
\linea \\
Pyramides hexagonae aequiangulae &
1 & 8 & 27 & 64 & 125 & 216 & 343 & 512& 729 & 1000 \\
\linea \\
Columnae pentagonae &
1 & 10 & 36 & 88 & 175 & 306 & 490 & 736 & 1053 & 1450 \\
\linea \\
Columnae hexagonae primae &
1 & 12 & 45 & 112 & 225 & 396 & 637 & 960 & 1377 & 1900 \\
\linea \\
Hexagoni aequianguli &
1 & 7 & 19 & 37 & 61 & 91 & 127 & 169 & 217 & 271\\
\linea \\
Columnae hexagonae aquiangulae &
1 & 14 & 57 & 148 & 305 & 546 & 889 &  1352 & 1953 & 2710 \\
\linea \\

\end{tabula}
\end{verbatim}
\end{quote}

%--------------------------------------------------------------------
\subsubsection{Sch\'ema simple avec $\backslash$lang}

{\new}
\onlylatex{\'Etant donn\'e un bug de \texttt{m2lv}, il est encore impossible de montrer le r\'esultat de l'exemple suivant.}
% XXX

\onlyhtml{
\begin{maurotex}
\begin{tabula}{llllll}

\mrow{2} \QDR{} $bac$ &
\mrow{2} \lang &
\RTT{} area $d$ & -----~ & per praemissam \TRN{} $abc$ &
\mrow{2} \rang &
\mrow{2} per antepraemissam \\

\QDR{} $bc$ & \regula & per hypothesim \RTT{} $bac$

\end{tabula}
\end{maurotex}
}

\begin{quote}
\begin{verbatim}
\begin{tabula}{llllll}

\mrow{2} \QDR{} $bac$ &
\mrow{2} \lang &
\RTT{} area $d$ & -----~ & per praemissam \TRN{} $abc$ &
\mrow{2} \rang &
\mrow{2} per antepraemissam \\

\QDR{} $bc$ & \regula & per hypothesim \RTT{} $bac$

\end{tabula}
\end{verbatim}
\end{quote}

%--------------------------------------------------------------------
\subsubsection{Sch\'ema avec symboles astronomiques}

{\new}
\onlylatex{\'Etant donn\'e un bug de \texttt{m2lv}, il est encore impossible de montrer le r\'esultat de l'exemple suivant.}
% XXX

\onlyhtml{
\begin{maurotex}
\begin{tabula}{lllllll}

\mrow{6} fulsio &
\mrow{6} \lgra &
\mrow{3} prima &
\mrow{3} \lgra &
\mrow{2} matutina &
\mrow{2} \lgra &
fixarum et 3 superiorum in auge epicycli \\

\VEN, \MER{} in opposito augis epicycli \\

vespertina &
\regula &
\LUN,\VEN, \MER{} in auge epicycli \\

\mrow{3} postrema &
\mrow{3} \lgra &
matutina &
\regula &
\LUN,\VEN, \MER{} in auge epicycli \\

\mrow{2} vespertina &
\mrow{2} \lgra &
fixarum et 3 superiorum in auge epicycli \\

\VEN, \MER{} in opposito augis epicycli

\end{tabula}
\end{maurotex}
}

\begin{quote}
\begin{verbatim}
\begin{tabula}{lllllll}

\mrow{6} fulsio &
\mrow{6} \lgra &
\mrow{3} prima &
\mrow{3} \lgra &
\mrow{2} matutina &
\mrow{2} \lgra &
fixarum et 3 superiorum in auge epicycli \\

\VEN, \MER{} in opposito augis epicycli \\

vespertina &
\regula &
\LUN,\VEN, \MER{} in auge epicycli \\

\mrow{3} postrema &
\mrow{3} \lgra &
matutina &
\regula &
\LUN,\VEN, \MER{} in auge epicycli \\

\mrow{2} vespertina &
\mrow{2} \lgra &
fixarum et 3 superiorum in auge epicycli \\

\VEN, \MER{} in opposito augis epicycli

\end{tabula}
\end{verbatim}
\end{quote}

%--------------------------------------------------------------------
\subsubsection{Sch\'ema arithm\'etique complexe}

{\new}
\onlylatex{\'Etant donn\'e un bug de \texttt{m2lv}, il est encore impossible de montrer le r\'esultat de l'exemple suivant.}
% XXX

\onlyhtml{
\begin{maurotex}
\begin{tabula}{llllll}

\mrow{4} $+$ \QDR{} $ab$ & \mrow{4} per 4\Sup{am} &
\mrow{4} \lgra & \QDR{} $ac$ $+$ \\

\QDR{} $bc$ & \mrow{2} \rgra & \mrow{2} per 3\Sup{am}
\RTT{} $ab${} $bc${} $+$ \\

\RTT{} $ac${} $cb$ \\

\RTT{} $ac${} $cb$ & \mrow{2} \rgra & \mrow{2} per 3\Sup{am}
\RTT{} $ab${} $bc${} $+$ \\

$+$ \QDR{} $bc$ & \mcol{2} \regula & \QDR{} $bc$

\end{tabula}
\end{maurotex}
}

\begin{quote}
\begin{verbatim}
\begin{tabula}{llllll}

\mrow{4} $+$ \QDR{} $ab$ & \mrow{4} per 4\Sup{am} &
\mrow{4} \lgra & \QDR{} $ac$ $+$ \\

\QDR{} $bc$ & \mrow{2} \rgra & \mrow{2} per 3\Sup{am}
\RTT{} $ab${} $bc${} $+$ \\

\RTT{} $ac${} $cb$ \\

\RTT{} $ac${} $cb$ & \mrow{2} \rgra & \mrow{2} per 3\Sup{am}
\RTT{} $ab${} $bc${} $+$ \\

$+$ \QDR{} $bc$ & \mcol{2} \regula & \QDR{} $bc$

\end{tabula}
\end{verbatim}
\end{quote}

Pour mieux voir la construction de ce tableau, donnons le m\^eme tableau en
s\'eparant les colonnes et les lignes:

\onlyhtml{
\begin{maurotex}
\begin{tabula}{|l|l|l|l|l|l|}

\linea \\

\mrow{4} $+$ \QDR{} $ab$ & \mrow{4} per 4\Sup{am} &
\mrow{4} \lgra & \QDR{} $ac$ $+$ \\

\linea \\

\QDR{} $bc$ & \mrow{2} \rgra & \mrow{2} per 3\Sup{am}
\RTT{} $ab${} $bc${} $+$ \\

\linea \\

\RTT{} $ac${} $cb$ \\

\linea \\

\RTT{} $ac${} $cb$ & \mrow{2} \rgra & \mrow{2} per 3\Sup{am}
\RTT{} $ab${} $bc${} $+$ \\

\linea \\

$+$ \QDR{} $bc$ & \mcol{2} \regula & \QDR{} $bc$ \\

\linea \\

\end{tabula}
\end{maurotex}
}

\begin{quote}
\begin{verbatim}
\begin{tabula}{|l|l|l|l|l|l|}

\linea \\

\mrow{4} $+$ \QDR{} $ab$ & \mrow{4} per 4\Sup{am} &
\mrow{4} \lgra & \QDR{} $ac$ $+$ \\

\linea \\

\QDR{} $bc$ & \mrow{2} \rgra & \mrow{2} per 3\Sup{am}
\RTT{} $ab${} $bc${} $+$ \\

\linea \\

\RTT{} $ac${} $cb$ \\

\linea \\

\RTT{} $ac${} $cb$ & \mrow{2} \rgra & \mrow{2} per 3\Sup{am}
\RTT{} $ab${} $bc${} $+$ \\

\linea \\

$+$ \QDR{} $bc$ & \mcol{2} \regula & \QDR{} $bc$ \\

\linea \\

\end{tabula}
\end{verbatim}
\end{quote}

%-----------------------------------------------------------------------
\subsection{Tableaux impossibles}

Dans le cas o\`u il serait vraiment impossible de faire un tableau ou un
sch\'ema avec les commandes ci-dessus, on utilisera la commande
\verb"\Tav{}"\index{\bs{}Tav} qui ne fait que marquer l'emplacement d'une
table par un ``Tavola'' en le centrant et en l'\'ecrivant dans un corps un
peu plus grand\footnote{Tr\`es probablement, il faudra ensuite utiliser un
autre logiciel pour faire le tableau et l'ins\'erer en tant qu'image.}. Si
de plus on ins\`ere entre les accolades de la macro un nombre, par exemple
\verb"\Tav{5}", on obtiendra un ``Tavola 5''.

%-----------------------------------------------------------------------
\section[Scansione e descrizione del testo]{Macro per la scansione e la descrizione del testo}

\label{ref-3.7}

\subsection{Suddivisione in ``proposizioni''}

\label{ref-3.7.1}
\index{scansione del testo}
\index{suddivisione in proposizioni}

Come si {\`e} detto, per il futuro ricupero dell'informazione
contenuta nell'edizione {\`e} essenziale che il testo venga
scandito in unit{\`a} e sottounit{\`a}. Le unit{\`a} le chiameremo
convenzionalmente ``proposizioni''; le sottounit{\`a},
``paragrafi''.

Diciamo convenzionalmente, per due tipi di motivi: in primo
luogo non tutti i testi mauroliciani sono divisi in
proposizioni vere e proprie; inoltre anche i testi
matematici cos{\'\i} suddivisi recano spessissimo
``proposizioni'' non numerate (lemmi, corollari, scolii,
agggiunte). L'editore provveder{\`a} quindi a
spezzare il testo in ``proposizioni'' fittizie (non
c'{\`e} da preoccuparsi: non apparir{\`a} nulla di
stampato)\footnote{Per maggiori indicazioni su come si
devono operare queste suddivisioni, cfr. \S\,\ref{ref-11.2}.}. Per
suddividere in proposizioni si utilizza la macro:

\begin{quote}
\begin{verbatim}
\Prop{}
\end{verbatim}
\end{quote}

Come si {\`e} appena detto essa non fa nulla, salvo tenere un registro
delle proposizioni introdotte. Le \verb"{}" servono ad ``appiccicare'' alla
``proposizione'' uno o pi{\'u} argomenti. L'attribuzione degli argomenti
{\`e} compito dell'editore: vi torneremo fra un attimo
(\S\,\ref{ref-3.7.3}). L'editore (o anche il trascrittore: l'editore
potr{\`a} in seguito affinare il suo lavoro) \textit{prima dell'inizio di
una ``proposizione''} scriver{\`a} dunque \verb"\Prop{}"\index{\bs{}Prop} e
in seguito attribuir{\`a} degli argomenti a quel passo del testo.

%-----------------------------------------------------------------------
\subsection{Suddivisione in paragrafi}

\label{ref-3.7.2}
\index{suddivisione}
\index{suddivisione in paragrafi}

Molto simile a quello di \verb"\Prop{}"\index{\bs{}Prop} {\`e} l'uso di \verb"\Unit"\index{\bs{}Unit},
che suddivide il testo in ``paragrafi''\footnote{Si noti
che mentre \texttt{$\backslash$Prop\{~\}} ha un argomento che pu{\`o}
(temporaneamente) essere
lasciato vuoto, \texttt{$\backslash$Unit} non ne ha.}. Con un'importante
differenza: il numero di paragrafo verr{\`a} stampato nel
testo, in neretto.

Un'opera mauroliciana verr{\`a}
quindi suddivisa in paragrafi dall'inizio alla fine: i
paragrafi saranno
numerati automaticamente dal {\mtex}. Ci{\`o} {\`e}
fondamentale per vari motivi. In primo luogo un domani si
potr{\`a} citare un'opera mauroliciana cos{\'\i}: ``\textit{Mom. aeq.}, 1.177'', dando cos{\'\i} modo di reperire in
modo assolutamente certo il passo (il \S\,177) del primo
libro del \textit{De momentis aequalibus} che si sta citando.
Inoltre servir{\`a} di riferimento ai programmi che dovranno
gestire i \textit{data-base} delle citazioni, il lessico, ecc.
Come si vedr{\`a} la scansione in paragrafi servir{\`a} nel
corso stesso del lavoro di trascrizione per identificare
l'inizio e la fine di brani che si devono citare
nell'apparato testuale.

Per tutti questi motivi {\`e} opportuno che l'editore apponga
una certa cura alla suddivisione in paragrafi. Star{\`a} alla
sua sensibilit{\`a} e alla sua conoscenza del testo effettuare
questa divisione: come indicazione generale un paragrafo
dovrebbe avere circa la lunghezza di una o due frasi.
Essendo poi la
gestione della numerazione completamente automatica, potr{\`a}
cancellare o aggiungere quanti paragrafi voglia senza alcun
problema. Ad esempio, l'inizio degli \textit{Apollonii Conica
Elementa} (la ``Lettera a Eudemo'') potrebbe essere trattato
cos{\'\i}:

\begin{maurotex}
\Unit Apollonius Eudemo Salutem. Si corpore bene vales et
alia secundum mentem tibi sunt, bene habetur; mediocriter
valemus et nos. \Unit Tempore quo eramus tecum Pergami,
cognovi te cupientem participem fieri Conicorum a nobis
compositorum. \Unit ...
\end{maurotex}

\noindent il che si otterr{\`a} scrivendo

\begin{quote}
\begin{verbatim}
\Unit Apollonius Eudemo Salutem. Si corpore bene vales et
alia secundum mentem tibi sunt, bene habetur; mediocriter
valemus et nos. \Unit Tempore quo eramus tecum Pergami,
cognovi te cupientem participem fieri Conicorum a nobis
compositorum. \Unit ...
\end{verbatim}
\end{quote}

%-----------------------------------------------------------------------
\subsection{Descrizione del testo: assegnazione di argomenti}

\label{ref-3.7.3}
\index{argomenti}

Allo scopo di poter pi{\'u} facilmente interrogare il testo, una
volta che esso sia stato edito, occorre assegnare uno o pi{\'u}
``argomenti'' alle
opere, ai libri di cui esse sono composte, alle loro
``proposizioni'' (nel senso specificato qui sopra) e,
eventualmente, ad alcuni gruppi di paragrafi.
Le macro da utilizzare a questo scopo sono:

\begin{quote}
\begin{verbatim}
\Opera{}
\Liber{}
\Prop{}
\Arg{}
\end{verbatim}
\end{quote}

secondo un'ordine gerarchico. Se l'opera non {\`e}
divisa in libri, la macro \verb"\Liber{}"\index{\bs{}Liber} non verr{\`a}
utilizzata o il suo campo verr{\`a} lasciato vuoto.

Consideriamo ad esempio il primo libro degli \textit{Arithmeticorum libri duo}. All'inizio del file (subito dopo
il \verb"\begin{document}"\index{\bs{}begin\{document\}}) l'editore scriver{\`a}:

\begin{quote}
\begin{verbatim}
\Opera{aritmetica}
\Liber{numeri figurati}
..........
\Prop{numeri quadrati}
\Prop{numeri quadrati}
...........
\Prop{numeri triangolari}
.........
\end{verbatim}
\end{quote}

L'argomento di \verb"\Prop{}"\index{\bs{}Prop} viene
automaticamente chiuso non appena viene aperta un'altra
\verb"\Prop{}"\index{\bs{}Prop}.

Supponiamo adesso che alcuni paragrafi della seconda
proposizione che parla di numeri quadrati parlino anche
della serie dei numeri dispari e del fatto che essa
rappresenta anche la serie delle differenze dei numeri
quadrati. L'editore vorr{\`a} registrare questo fatto, e dovr{\`a}
quindi assegnare ai paragrafi in questione l'argomento
``numeri dispari''. La situazione verr{\`a} trattata in
questo modo:

\begin{quote}
\begin{verbatim}
\Opera{aritmetica}
\Liber{numeri figurati}
..........
\Prop{numeri quadrati}
\Prop{numeri quadrati}
...........
\Arg{numeri dispari} \Unit Notandum est quod numeri
impares ab unitate ... \Arg{} \Unit Numeri quadrati ...
\Prop{numeri triangolari}
.........
\end{verbatim}
\end{quote}

Ogni \verb"\Arg"\index{\bs{}Arg} annulla il precedente. Per chiudere
un argomento aggiuntivo (i numeri dispari, nel caso
precedente) basta quindi inserire un \verb"\Arg"\index{\bs{}Arg} lasciando il
campo vuoto, come nell'esempio. Si noti che, in ogni caso,
quando finisce la proposizione e si inserisce una nuova
\verb"\Prop{}"\index{\bs{}Prop} tutti gli \verb"\Arg"\index{\bs{}Arg} della proposizione precedente
vengono automaticamente chiusi. Se l'editore nell'esempio
precedente si fosse dimenticato di aggiungere \verb"\Arg{}"\index{\bs{}Arg}, il
danno si sarebbe quindi limitato alla proposizione in cui era
contenuto \verb"\Arg{numeri dispari}"\index{\bs{}Arg}.

La macro \verb"\Arg{}"\index{\bs{}Arg} (a differenza di \verb"\Opera{}"\index{\bs{}Opera},
\verb"\Prop{}"\index{\bs{}Prop} e \verb"\Liber{}"\index{\bs{}Liber}) eredita l'argomento del livello
immediatamente precedente. Rispetto all'esempio fatto qui
sopra ci{\`o} significa che i paragrafi in cui parla della serie
dei numeri dispari vengono automaticamente etichettati come
paragrafi in cui parla \textit{anche} dei numeri quadrati.

Che fare per{\`o} se ad un certo punto Maurolico, all'interno di
una proposizione che parla dei numeri quadrati inserisse una
digressione sulle propriet{\`a} della parabola lunga una
cinquantina di paragrafi? In questo caso si vuole che tale
digressione non venga etichettata anche con ``numeri quadrati''.

La sintassi da utilizzare {\`e} la seguente:

Se non si vuole che
\verb"\Arg{}"\index{\bs{}Arg} erediti gli argomenti della \verb"\Prop{}"\index{\bs{}Prop} in cui viene
collocato, l'argomento che si assegna ad \verb"\Arg{}"\index{\bs{}Arg}
deve essere preceduto da un \verb"*" senza spazi bianchi.

Ad esempio:

\begin{quote}
\begin{verbatim}
\Opera{aritmetica}
\Liber{numeri figurati}
..........
\Prop{numeri quadrati}
\Prop{numeri quadrati}
...........
\Arg{*parabola} \Unit Conica sectio, quae parabola
vocetur ... \Unit ... \Arg{} \Unit Numeri
quadrati ...
\Prop{numeri triangolari}
.........
\end{verbatim}
\end{quote}

e fra la descrizione che si inserisce nel campo
di \verb"\Arg"\index{\bs{}Arg} e l'\verb"*" non devono esserci spazi.

Diverso {\`e} il caso in cui in un testo di cosmografia si trovi
un passo (cio{\`e} una ``proposizione'') sulle coniche:

\begin{quote}
\begin{verbatim}
\Opera{cosmografia}
..........
\Prop{coniche}
.........
\end{verbatim}
\end{quote}

Infatti l'argomento di \verb"\Prop{}"\index{\bs{}Prop} non eredita
automaticamente gli argomenti di livello superiore.

Si possono, ovviamente, attribuire pi{\'u} argomenti ad
un'opera, a un libro, a una proposizione, a un gruppo di
paragrafi. Baster{\`a} separarli da una virgola. Ad esempio:

\begin{quote}
\begin{verbatim}
\Prop{numeri triangolari, numeri quadrati, 
 numeri esagonali}
\end{verbatim}
\end{quote}

Ci{\`o} pone il problema di cosa fare nel caso si
voglia che \verb"\Arg{}"\index{\bs{}Arg} erediti solo alcuni degli
argomenti della sua \verb"\Prop{}"\index{\bs{}Prop}. Supponiamo ad esempio che
una proposizione parli di numeri triangolari, quadrati ed esagonali. Ma
che un gruppo di paragrafi di questa proposizione parli per
inciso
dei numeri triangolari e dei numeri ottagonali, \textit{ma non}
di quelli quadrati ed esagonali. La situazione verr{\`a} cos{\'\i}
codificata:

\begin{quote}
\begin{verbatim}
\Prop{numeri triangolari, numeri quadrati, 
 numeri esagonali}
\Unit ... \Arg{-numeri quadrati, -numeri esagonali, 
 numeri ottagonali} \Unit Octogonales 
numeri ... \Arg{} \Unit Numeri quadrati 
.........
\Prop{numeri pentagoni}
.........
\end{verbatim}
\end{quote}

Si riportano cio{\`e} nella sottodivisione gli
argomenti che \textit{non} si vogliono ereditare preceduti da
un segno meno (\verb"-"), senza lasciare spazi bianchi fra il
segno meno e il nome dell'argomento.

Il motivo di questa regola di ereditariet{\`a} parziale {\`e} il
seguente. Se \textit{tutte} le sottodivisioni ereditassero gli
argomenti dei livelli pi{\'u} alti, l'editore sarebbe spesso
costretto ad utilizzare le regole di esclusione. Si
consideri ad esempio il secondo libro delle coniche. Gli
argomenti verranno assegnati, ad esempio, in questo modo:

\begin{quote}
\begin{verbatim}
\Opera{coniche}
\Liber{asintoti, tangenti}
\end{verbatim}
\end{quote}

Se \verb"\Prop{}"\index{\bs{}Prop} ereditasse da \verb"\Liber{}"\index{\bs{}Liber}, tutte le proposizioni
che parlano di tangenti ma non di asintoti dovrebbero essere
scritte in questo modo:

\begin{quote}
\begin{verbatim}
\Prop{-asintoti, tangenti}
\end{verbatim}
\end{quote}

costringendo l'editore a battere un sacco di cose
essenzialmente inutili. Diverso {\`e} il caso del rapporto fra
\verb"\Prop{}"\index{\bs{}Prop} e \verb"\Arg{}"\index{\bs{}Arg} perch{\'e} si pu{\`o} supporre che nella
maggior parte dei casi l'etichetta che si assegna ad
\verb"\Arg{}"\index{\bs{}Arg} sia una specificazione di quella di \verb"\Prop{}"\index{\bs{}Prop} e che
quindi sia pi{\'u} opportuno far s{\'\i} che gli argomenti del
livello superiore passino automaticamente a quello inferiore.

Le parole chiave, o etichette che l'editore inserisce nelle
macro che abbiamo ora descritte devono essere parole che
descrivano \textit{l'oggetto del discorso}. Ad esempio, si
scriver{\`a}

\begin{quote}
\begin{verbatim}
\Prop{Teorema di Pitagora}
\end{verbatim}
\end{quote}

solo nel caso che la proposizone in questione
tratti del teorema di Pitagora, non se questo viene
utilizzato o citato.

Non si {\`e} ritenuto opportuno stabilire \textit{a priori} un
elenco di parole chiave. Come nel caso di \verb"\Cit"\index{\bs{}Cit}, star{\`a}
all'editore stabilirle per il suo testo, osservando le
seguenti regole:

\begin{enumerate}

\item Le parole chiave (o etichette) devono essere usate
in modo \textit{coerente}. Se si usa il termine mauroliciano per
asintoti \textit{nontangentes}, si dovr{\`a} utilizzarlo sempre, e non
ogni tanto ``asintoti'' e ogni tanto ``nontangentes''. Cos{\'\i},
se si adottano abbreviazioni, tali abbreviazioni dovranno essere
usate in modo uniforme.

\item L'editore dovr{\`a} fornire una lista delle parole
chiave che ha introdotto.

\item Tale lista dovr{\`a} essere strutturata
gerarchicamente. Ad esempio, siccome le parabole, le
iperboli e le ellissi sono tutte coniche, potr{\`a} scrivere:

\begin{quote}
\begin{verbatim}
Coniche = {parabola, ellisse, iperbole}
\end{verbatim}
\end{quote}

\noindent o in qualsiasi altro modo risulti chiaro
la dipendenza di un concetto da un altro.

\end{enumerate}

%-----------------------------------------------------------------------
\section{Annotazioni dell'editore}

\label{ref-3.8}
\index{annotazioni}

L'apparato di annotazione finale consister{\`a} di tre
parti. Un apparato puramente testuale, in cui si d{\`a}
conto delle operazioni compiute dall'editore sul testo (di
esso diremo nei due prossimi capitoli); un apparato di fonti
(gestito essenzialmente dalla macro \verb"\Cit"\index{\bs{}Cit}); e infine un
apparato di annotazioni dell'editore. Tali annotazioni
dovranno essere il pi{\'u} possibile fattuali, in nessun
caso interpretative del testo. Si dovr{\`a} ricorrere a tale
tipo di annotazione unicamente quando essa risulti \textit{indispensabile} per la comprensione del testo. Si potr{\`a}
trattare, ad esempio, del rinvio ad un altro passo
dell'opera mauroliciana, a quella di un altro autore non
citato nel testo, al chiarimento di un termine
incomprensibile utilizzando preferibilmente un rinvio a un
lessico.

Per quest'ultimo tipo di annotazione l'editore
dovr{\`a} ricorrere alla macro

\begin{quote}
\begin{verbatim}
\Adnotatio{}
\end{verbatim}
\end{quote}

Fra le parentesi andr{\`a} ovviamente inserito il testo
dell'annotazione.

Occorrer{\`a} far attenzione a non confondere i vari tipi di note
che possono essere prodotti dal {\mtex}. L'editore e il
trascrittore hanno infatti a disposizione \verb"\Comm"\index{\bs{}Comm} (per
commenti che in ultima analisi non dovranno pi{\'u} comparire);
l'editore dispone inoltre di \verb"\Adnotatio"\index{\bs{}Adnotatio} che, a differenza
di \verb"\Comm"\index{\bs{}Comm}
produrr{\`a} un testo che entrer{\`a} a far parte dell'edizione
critica. {\`E} ovvio che \verb"\Adnotatio"\index{\bs{}Adnotatio} {\`e} una macro
riservata all'editore; e anche nel caso che editore e
trascrittore coincidano nella stessa persona fisica dovr{\`a}
essere usata con molta prudenza. In prima istanza, fin quando
non si {\`e} assolutamente sicuri di ci{\`o} che si fa, bisogna
usare sempre \verb"\Comm"\index{\bs{}Comm}.

Nel capitolo \ref{ref-9}
 si vedr{\`a} un'altro tipo di nota, \verb"\footnote"\index{\bs{}footnote},
riservata ai casi disperati in cui le risorse normali
dell'{\mtex} sembrano non poter risolvere la
situazione.

%-----------------------------------------------------------------------
\chapter{La collazione}

\label{ref-4}

\section*{Premessa}

\label{ref-4}
\index{collazione}

L'ambito di discussione del capitolo precedente (\ref{ref-3}) riguardava
principalmente la trascrizione di un singolo testimone e il
modo con cui codificarne le particolarit{\`a} grafiche o
codificare certe informazioni (ad esempio le citazioni delle
fonti.)

In questo capitolo affronteremo invece ci{\`o} che costituisce
la caratteristica principale del {\mtex}: il sistema di
trascrizione delle varianti fra i vari testimoni in un unico
file di testo.

%-----------------------------------------------------------------------
\section{Tipo di apparato e trattamento delle varianti testuali}

\label{ref-4.1}
\index{tipo di apparato}
\index{varianti testuali}
\index{apparato critico positivo}

Sulla base di varie considerazioni, in special modo quella
che i testi mauroliciani possiedono in genere una tradizione
limitata al massimo a 3 o 4 testimoni, e che almeno in
circa la met{\`a} dei casi si tratta di testimoni unici, si
{\`e} ritenuto opportuno adottare un apparato critico \textit{positivo} (come
Heiberg, ad esempio). Ci{\`o}
significa, in concreto, che  nell'apparato verranno
esplicitamente segnalate le lezioni di tutti i testimoni e
non solo quelle dei testimoni che divergono dal testo
critico accolto.

D'ora in poi, qui e nel seguito, TC sta per testo critico, le lettere
maiuscole A, B, C, ecc. distinguono i vari testimoni. Le parti fra
``stelle'' indicano esempi di testo che si vuole ottenere.

%-----------------------------------------------------------------------
\section{$\protect\backslash$VV, la madre di tutte le macro}

\label{ref-4.2}

\subsection{La regola e l'eccezione}

\label{ref-4.2.1}
\index{varianti}

Per maggiore chiarezza, supponiamo di avere la seguente situazione:

\begin{quote}
A ha la lezione ``Sit data ratio, sit datus cubus.'' \\
B ha la lezione ``Sit data gratia, sit datus cubus'' \\
C ha la lezione ``Sit data latio, sit datus cubus''
\end{quote}

TC segue A e legge ``Sit data ratio, sit datus
cubus.''.

Avremo allora:

\begin{maurotex}
Sit data \VV{
            {A:ratio}
            {B:gratia}
            {C:latio}
            }, sit datus cubus.
\end{maurotex}

Il formato della macro che produce questa nota {\`e}

\begin{quote}
\begin{verbatim}
Sit data \VV{
            {A:ratio}
            {B:gratia}
            {C:latio}
            }, sit datus cubus.
\end{verbatim}
\end{quote}

\noindent con la \textbf{regola generale} che nel TC \textit{verr{\`a}
sempre riportato il testo del primo campo} (in questo caso quello di A). Si
noti anche il formato di \verb"\VV"\index{\bs{}VV}: ad essa segue una coppia di parentesi
graffe, dentro la quale si inseriscono tante coppie di parentesi graffe
quanti sono i testimoni di cui si riportano varianti. I testimoni
all'interno delle parentesi graffe devono essere indicati con `\verb"A:"',
ovvero con una lettera maiuscola, seguita da due punti seguiti dalla
lezione, senza lasciare nessuno spazio\footnote{Quanto viene detto qui e
nei prossimi due capitoli {\`e} riferito essenzialmente al trattamento di
varianti ``puntuali'', tali cio{\`e} da coinvolgere una o poche parole del
testo. Al trattamento delle varianti ``lunghe'' {\`e} dedicato il capitolo
7.}.

Come ogni regola che si rispetti anche questa ha la sua
\textbf{eccezione}. Si potrebbe infatti dare il caso che la lezione
del testimone contenesse, come segno di interpunzione, il
segno `\textbf{:}'. In un caso del genere il {\mtex} sarebbe
indotto a far confusione, e a non capire pi{\'u} quali sono i due
punti che indicano la separazione fra \textit{siglum} indicante
il testimone e i due punti che compaiono nella lezione come
segno di interpunzione. Ad esempio:

\begin{quote}
A legga ``Manifestum ergo est: triangulum
aequale quadrato $ab$''

B legga ``Patet igitur quadratum $ab$ aequalem esse
trigono $mno$''
\end{quote}

\noindent il testo critico segue A e si vuole ottenere:

%\stelle
%Manifestum ergo est: triangulum aequale quadrato $ab^1$
%\myrule
%\textsuperscript{1}\qquad {\rmnot Manifestum ergo est: triangulum
%aequale quadrato $ab$} \textsc{A} {\rmnot
%Patet igitur quadratum {\itnot ab} aequalem esse
%trigono {\itnot mno}} \textsc{B}
%\stelle

\begin{maurotex}
\VV{
   {A:{Manifestum ergo est: triangulum
   aequale quadrato \(ab\)}}
   {B:Patet igitur quadratum \(ab\) aequalem esse
   trigono \(mno\)}
   }
\end{maurotex}

Per ovviare al problema qui sopra descritto, la
lezione contenente i `\textbf{:}', cio{\`e} quella di A, andr{\`a} racchiusa
fra parentesi graffe \verb"{}", in questo modo:

\begin{quote}
\begin{verbatim}
\VV{
   {A:{Manifestum ergo est: triangulum
   aequale quadrato \(ab\)}}
   {B:Patet igitur quadratum \(ab\) aequalem esse
   trigono \(mno\)}
   }
\end{verbatim}
\end{quote}

In questo modo si avverte il {\mtex} che
tutta la frase ``Manifestum \dots $ab$'' deve essere
considerata una sola unit{\`a} e che i `\verb":"' che vi si trovano non
costituiscono un separatore.

%-----------------------------------------------------------------------
\subsection{Possibilit{\`a} di passare a un apparato misto}

\label{ref-4.2.2}
\index{apparato critico misto}

Possono verificarsi casi in cui l'editore ritenga opportuno
passare ad un apparato misto, ovverossia non dichiarare in
apparato le lezioni di tutti i testimoni. O anche, pi{\'u}
banalmente, che abbia la necessit{\`a} di riportare in apparato
la lezione di TC senza che essa sia seguita immediatamente
dall'indicazione di un testimone. Si consideri il seguente
esempio. La tradizione sia costituita da due testimoni A e
B, discendenti in modo indipendente da un archetipo comune
$\alpha$; A legga ``$\widetilde{mul.nis}$''
(un'abbreviazione); B legga ``multiplicationis''. L'editore,
grazie alla sua conoscenza del testo giunge alla conclusione
che il copista di B ha sciolto male l'abbreviazione presente
in $\alpha$ e che la lezione corretta debba essere
``multitudinis''. Vorr{\`a} quindi produrre una nota del tipo

\onlylatex{
\begin{maurotex}
\VV{
   {*:multitudinis}
   {A:\CONTR{mul.nis}}
   {B:multiplicationis}
   }
\end{maurotex}
}

\onlyhtml{
multitudinis\textsuperscript{1}
forhtml.myrule
\textsuperscript{1} multitudinis\textsl{:}
\includegraphics{manicons/mulnis.gif} \textsc{A} multiplicationis \textsc{B}
\par
}
% ici on triche

\noindent dando cos{\'\i} modo al lettore di capire la motivazione della
sua scelta. Per ottenere questo risultato occorre scrivere

\begin{quote}
\begin{verbatim}
\VV{
   {*:multitudinis}
   {A:\CONTR{mul.nis}}
   {B:multiplicationis}
   }
\end{verbatim}
\end{quote}


\noindent dove, come si vede, basta inserire una \verb"*" nel
primo campo (quello destinato al testo critico) al posto
dell'indicazione di un testimone.

Allo stesso modo, se si volesse trattare l'esempio
iniziale (``Sit data ratio, sit datus
cubus.'') con un apparato negativo (che registri cio{\`e} solo
le lezioni dei testimoni che differiscono da quella accolta
in TC), basterebbe scrivere

\begin{quote}
\begin{verbatim}
Sit data \VV{
            {*:ratio}
            {B:gratia}
            {C:latio}
            }
\end{verbatim}
\end{quote}

\noindent ottenendo

%\stelle
%Sit data ratio\textsuperscript{1}, sit datus cubus.
%\myrule
%\textsuperscript{1}\qquad ratio\textsc{:} gratia \textsc{B} latio \textsc{C}
%\stelle

\begin{maurotex}
Sit data \VV{
            {*:ratio}
            {B:gratia}
            {C:latio}
            }
\end{maurotex}

Vale dunque la seguente \textbf{regola generale}:

\begin{itemize}
\item Qualora, per un qualunque motivo, non venisse indicato in
uno dei campi di \verb"\VV"\index{\bs{}VV} la sigla del testimone, si dovr{\`a}
inserire al suo posto una \verb"*" seguita da \verb":".
\end{itemize}

%-----------------------------------------------------------------------
\subsection{Dove mettere la punteggiatura?}

\label{ref-4.2.3}
\index{punteggiatura}

Nonostante occupi parecchio posto nel file che scrivete, \verb"\VV{}"\index{\bs{}VV}
produrr{\`a} nel vostro TC solo le parole che mettete nel sottocampo
destinato alla lezione. Al tempo stesso, ci{\`o} che si trova in tale
sottocampo verr{\`a} riportato in apparato. Dove mettere dunque la
punteggiatura: dentro o fuori \verb"\VV{}"\index{\bs{}VV}? La regola da seguire,
\textbf{senza eccezioni}, {\`e} che i  segni di punteggiatura finali dovranno sempre
trovarsi fuori da \verb"\VV{}"\index{\bs{}VV}. Per esempio, il vostro testo critico
dovrebbe essere simile al seguente:

%\stelle
%Sit data ratio\textsuperscript{1}, sit datus cubus$^2$.
%\myrule
%\textsuperscript{1}\qquad ratio \textsc{A} gratia \textsc{B} latio \textsc{C}
%$^2$\qquad cubus \textsc{C} tronus \textsc{A} rufus \textsc{B}
%\stelle

\begin{maurotex}
Sit data \VV{
           {A:ratio}
           {B:gratia}
           {C:latio}
           }, sit datus \VV{
                           {C:cubus}
                           {A:tronus}
                           {B:rufus}
                           }.
\end{maurotex}

\noindent dove la virgola dopo \textit{ratio} e il punto dopo \textit{cubus}
non sono riportati in apparato. Ci{\`o} si otterr{\`a} in questo modo:

\begin{quote}
\begin{verbatim}
Sit data \VV{
           {A:ratio}
           {B:gratia}
           {C:latio}
           }, sit datus \VV{
                           {C:cubus}
                           {A:tronus}
                           {B:rufus}
                           }.
\end{verbatim}
\end{quote}

\noindent battendo la \verb"," e il \verb"." \textbf{dopo} l'ultima parentesi
graffa di \verb"\VV"\index{\bs{}VV}.

%-----------------------------------------------------------------------
\subsection{$\protect\backslash$VB: come eliminare dalla stampa le
varianti indesiderabili}

\label{ref-4.2.4}
\index{varianti di forma}
\index{varianti banale}

Come si {\`e} detto nel \S\,\ref{ref-2.3}, l'editore deve provvedere a
distinguere fra varianti sostanziali e varianti di forma. La
macro \verb"\VV"\index{\bs{}VV} {\`e} riservata alle varianti di sostanza, ed esiste
una macro \verb"\VB"\index{\bs{}VB} per le varianti di forma. \verb"\VB"\index{\bs{}VB} (variante
``banale'') ha la stessa sintassi e produce lo stesso TC di
\verb"\VV"\index{\bs{}VV}, ma non produce la nota. Tiene comunque un registro
del-le varianti fra i vari testimoni, che in caso di
necessit{\`a} potranno sempre venire stampate.

Si tenga per{\`o} presente che, nella prima fase, il
trascrittore dovr{\`a} considerare tutte le varianti come
varianti di sostanza: sar{\`a} compito dell'editore stabilire
successivamente attraverso un loro attento studio quali di
esse debbano essere declassate a varianti di forma.

Per ottenere ci{\`o}, \textit{all'editore baster{\`a} cambiare
nell'espressione} \verb"\VV"\index{\bs{}VV} \textit{la seconda} \verb"V" \textit{con una}
\verb"B". Nel seguito, tuttavia, parleremo sempre di \verb"\VV"\index{\bs{}VV} per evitare
noiose ripetizioni.

Riteniamo opportuno e ragionevole
stabilire \textit{a priori} che le varianti nel caso di pi{\'u} di due
codici o siano tutte di forma o debbano essere riportate
tutte, considerandole quindi di sostanza. L'idea sar{\`a}
chiarita dall'esempio seguente:

\begin{quote}
A ha la lezione ``aequales'' \\
B ha la lezione ``equites'' \\
C ha la lezione ``aequates'', \\
TC segue A e scrive ``aequales''
\end{quote}

\noindent e dunque

%\stelle
%aequales\textsuperscript{1}
%\myrule
%\textsuperscript{1}\qquad aequales \textsc{A} aequates
%\textsc{C} equites \textsc{B}
%\stelle

\begin{maurotex}
\VV{{A:aequales}{C:aequates}{B:equites}}
\end{maurotex}

Come si vede, pur essendo \textit{aequates} una variante
di forma di \textit{aequales}, {\`e} importante riportarla: non
solo perch{\'e} rafforza la scelta che TC fa di \textit{aequales} contro \textit{equites} di B, ma perch{\'e} altrimenti, data la scelta
dell'apparato positivo, s'indurrebbe a credere che C porti
anch'egli la lezione \textit{aequales}.

%-----------------------------------------------------------------------
\subsection{Ordinamento delle varianti in apparato}

\label{ref-4.2.5}
\index{ordinamento delle varianti in apparato}

L'esempio precedente mostra anche come l'editore pu{\`o}
procedere ad ordinare le varianti in apparato. Per esempio,
potrebbe voler disporle in ordine di allontanamento crescente
dal testo critico. La cosa andrebbe dunque scritta in
questo modo:

\begin{quote}
\begin{verbatim}
\VV{
   {A: aequales}
   {C: aequates}
   {B: equites}
   }
\end{verbatim}
\end{quote}

\noindent inserendo C nel secondo campo e non nel terzo. 
Se invece decidesse (dichiarandolo opportunamente
nell'introduzione alla sua edizione) di dare sempre le
varianti secondo un ordine fisso dei testimoni (ad esempio
ABC) gli basterebbe invertire l'ordine dei due ultimi campi.

Riassumendo:

\begin{quote}
il primo campo di \verb"\VV"\index{\bs{}VV} riporta sempre
la lezione di TC,  che in esso compaia una \verb"*" o  il \textit{siglum} di un testimone; i campi
successivi riportano le lezioni degli altri testimoni e
saranno stampate in
nota secondo l'ordine di inserimento (e non secondo
un'ordine precostituito dal {\mtex}).
\end{quote}

%-----------------------------------------------------------------------
\subsection{Caso in cui due o pi{\'u} testimoni hanno la stessa lezione}

\label{ref-4.2.6}

Potr{\`a} ovviamente darsi il caso che (ad esempio) due testimoni riportino
la stessa lezione. Ad esempio

\begin{quote}
A ha la lezione ``aequales'' \\
B ha la lezione ``aequales'' \\
C ha la lezione ``equites'' \\
D ha la lezione ``equites''; \\
TC segue A e B e scrive aequales
\end{quote}

\noindent in questo caso si vuole ottenere:

%\stelle
%aequales\textsuperscript{1}
%\myrule
%\textsuperscript{1}\qquad aequales \textsc{A B} equites
%\textsc{C D}
%\stelle

\begin{maurotex}
\VV{
   {A/B:aequales}
   {C/D:equites}
   }
\end{maurotex}

\noindent e si allora si dovr{\`a} scrivere:

\begin{quote}
\begin{verbatim}
\VV{
   {A/B:aequales}
   {C/D:equites}
   }
\end{verbatim}
\end{quote}

Si noti che in questo caso TC accoglie la lezione di A e di B,
perci{\`o} nel primo campo i due testimoni vengono raggruppati,
scrivendo \verb"A/B:", ovvero le lettere maiuscole che indicano i
testimoni, separate da una \verb"/", senza spazi fra di loro
e seguite da un `\textbf{:}', che separa la lista dei testimoni dalla
lezione \textit{aequales}.
La stessa cosa avviene nel secondo campo per i testimoni di cui
si rifiuta la lezione.

%-----------------------------------------------------------------------
\section{Omissioni e omissioni in lacuna}

\label{ref-4.3}

\subsection{Omissioni}
\index{omissioni}

\label{ref-4.3.1}

Pu{\`o} accadere evidentemente che una certa lezione sia stata
omessa da uno o pi{\'u} testimoni. Esempio:

\begin{quote}
A\hphantom{T}: a vertice coni demittitur \\
B\hphantom{T}: a vertice canis demittitur \\
C\hphantom{T}: a vertice demittitur \\
TC: segue A e scrive ``a vertice coni demittitur''.
\end{quote}

In questo caso si dovrebbe ottenere:

%\stelle
%a vertice coni\textsuperscript{1} demittitur
%\myrule
%\textsuperscript{1}\qquad coni \textsc{A}~~canis
%\textsc{B}~~\textsc{om. C}
%\stelle

\begin{maurotex}
\VV{
   {A:coni}
   {B:canis}
   {C:\OM}
   }
\end{maurotex}

\noindent dove ``om. C'' sta per ``omisit C'': ``C ha
omesso''. La relativa macro {\`e}:

\begin{quote}
\begin{verbatim}
\VV{
   {A:coni}
   {B:canis}
   {C:\OM}
   }
\end{verbatim}
\end{quote}

\noindent dove occorre notare che nel terzo campo invece di \verb"C:"
seguito da una lezione si scrive invece \verb"C:\OM" (\verb"\OM"\index{\bs{}OM} {\`e} una
macro che provvede a scrivere in nota ``om.''). Naturalmente se fosse stato
B ad omettere si sarebbe scritto \verb"B:\OM", ecc.

%-----------------------------------------------------------------------
\subsection{Omissioni in lacuna}

\label{ref-4.3.2}
\index{omissioni in lacuna}

Per ``omissioni in lacuna'' intendiamo il caso in cui un
testimone omette una o pi{\`u} parole, lasciando per{\`o} uno spazio vuoto.
Potrebbe darsi che nell'esempio di prima C avesse s{\'\i} omesso la parola
\textit{coni}, ma lasciando uno spazio vuoto. Il risultato dovrebbe
allora essere:

%\stelle
%a vertice coni\textsuperscript{1} demittitur
%\myrule
%\textsuperscript{1}\qquad coni \textsc{A}~~canis
%\textsc{B}~~\textsc{spatio relicto om. C}
%\stelle

\begin{maurotex}
a vertice \VV{
             {A:coni}
             {B:canis}
             {C:\OMLAC}
             }
\end{maurotex}

\noindent dove ``spatio relicto om. C''= ``spatio relicto omisit C'',
ovvero ``C ha omesso lasciando uno spazio [bianco]''. La
macro relativa {\`e}:

\begin{quote}
\begin{verbatim}
a vertice \VV{
             {A:coni}
             {B:canis}
             {C:\OMLAC}
             }
\end{verbatim}
\end{quote}

\noindent cio{\`e} nel terzo campo si scrive \verb"C:\OMLAC", dove
\verb"\OMLAC"\index{\bs{}OMLAC} {\`e} una macro che produce ``spatio relicto om.'' nel luogo
e con i caratteri opportuni. Naturalmente se fosse stato B ad omettere in
lacuna si sarebbe scritto \verb"B:\OMLAC" ecc.

Si noti che in questo modo si potr{\`a} recuperare il testo dei vari
testimoni, ma solo in forma ``diplomatica'': vale a dire se C non ha
scritto \textit{coni}, \textit{coni} non si trover{\`a} scritto nemmeno nel
testo di C che si potr{\`a} produrre dalla nostra edizione. Non ci sembra
che ci{\`o} presenti particolari svantaggi: semmai il contrario, perch{\'e}
restituir{\`a} il testo del testimone nella sua veste ``storica'' cio{\`e}
per come esso {\`e} stato utilizzato storicamente (questo vale soprattutto
per i testimoni a stampa).

Se per{\`o} C omette in lacuna sar{\`a} opportuno che nel testo di C venga
segnalata la presenza dello spazio lasciato bianco: allora  la macro
\verb"\OMLAC"\index{\bs{}OMLAC} provveder{\`a} a stampare tre asterischi (***) al posto della
parola mancante. Si tenga conto, inoltre, che il lettore interessato
avr{\`a} sempre e comunque a disposizione il testo critico.

%-----------------------------------------------------------------------
\section{Lacune materiali}

\label{ref-4.4}

\subsection{Parole non pi{\'u} leggibili, macchie d'inchiostro, fori nella
carta e simili piacevolezze in una parte della tradizione}

\label{ref-4.4.1}
\index{lacune materiali}
\index{parole non pi{\'u} leggibili}
\index{macchie d'inchiostro}
\index{fori nella carta}

Succede spesso che
trascrivendo  un testimone ci si imbatta in una delle
situazioni descritte nel titolo di questo paragrafo: in
altre parole che un testimone presenti una \textit{lacuna
materiale}, provocata cio{\`e} da una situazione
oggettiva\footnote{Sottolineiamo che, per parlare di
lacuna materiale l'impossibilit{\`a} della lettura deve essere
oggettiva: per altri tipi di situazione si veda il \S\,\ref{ref-3.3.1}}. Se,
tuttavia, il testo {\`e} comunque tr{\`a}dito dal resto della tradizione,
la cosa non sar{\`a} irrimediabile, ma dovr{\`a}
essere segnalata nel  modo illustrato dal seguente
esempio. I testimoni rechino questo testo:

\begin{quote}
A: et erit $\bullet$ aequalis quadrato \\
B: et erit triangulum aequalis quadrato
\end{quote}

\noindent dove il $\bullet$ indica che ci trova di fronte
ad una delle situazioni elencate qui sopra. Se ne dar{\`a} conto
in questo modo:

%\stelle
%et erit triangulum\textsuperscript{1} aequalis quadrato
%\myrule
%\textsuperscript{1}\qquad triangulum \textsc{B}
%\textsc{non legitur A}
%\stelle

\begin{maurotex}
et erit \VV{
           {B:triangulum}
           {A:\NL}
           }\Comm{In A al posto di ``triangulum''
                  c'{\`e} una macchia d'inchiostro}
           aequalis quadrato
\end{maurotex}

\noindent ovvero: ``B legge \textsl{triangulum}, A non si
legge''.

La lacuna verr{\`a} registrata utilizzando la
macro \verb"\NL"\index{\bs{}NL}, che produce l'espressione ``non legitur''; e
non
sar{\`a} male, inoltre, che si provveda a illustrare la situazione
con un commento (utilizzando la macro \verb"\Comm"\index{\bs{}Comm}, vedi
\S\,\ref{ref-3.3}). Si scriver{\`a} dunque:

\begin{quote}
\begin{verbatim}
et erit \VV{
           {B:triangulum}
           {A:\NL}
           }\Comm{In A al posto di ``triangulum''
                  c'{\`e} una macchia d'inchiostro}
           aequalis quadrato
\end{verbatim}
\end{quote}

\verb"\NL"\index{\bs{}NL} provvede poi a stampare i tre asterischi
(***) nel testo di A.

Se per{\`o} le parole illeggibili fossero pi{\'u} di due o tre
(ovverossia ci si trova di fronte a una lacuna materiale
\textit{lunga}) si proceder{\`a} come indicato nel \S\,\ref{ref-7.1.1.5}.

%-----------------------------------------------------------------------
\subsection{Lacune presenti nell'intera tradizione}

\label{ref-4.4.2}
\index{lacune presenti nell'intera tradizione}
\index{lacune}

Potr{\`a} per{\`o} anche accadere che una porzione di testo sia
caduta nell'intera tradizione e risulti perci{\`o}
irimediabilmente perduta:  ad esempio perch{\'e} si ha  a che fare con un
testimone unico in cui o manchino dei fogli, o una o
pi{\'u} parole o addirittura una o pi{\'u} righe non siano pi{\'u}
leggibili per una qualunque ragione (macchia d'inchiostro);
oppure
perch{\'e}, pur disponendo di pi{\'u} di un testimone, il passo era
andato perduto gi{\`a} nell'archetipo, sicch{\'e} le copie che ne
sono derivate lasciano al pi{\'u} uno spazio bianco.
Di questa situazione bisogner{\`a} dar conto in
apparato e lo si far{\`a} ricorrendo a una delle seguenti
formule, a seconda dei casi:

\begin{quote}
\textit{%
spatium aliquot literarum rel. A \\
spatium aliquot verborum rel. A \\
spatium unius versus rel. A \\
spatium duorum (trium, quattuor, \dots) versuum rel. A \\
aliquot literae legi nequeunt in  A \\
aliquot verba legi nequeunt in A \\
unus versus legi nequit in A \\
duo (tres, quattuor, {\dots}) versus legi nequeunt in A \\
unum folium deest in A \\
folium 54 deest in A \\
duo (tria, quattuor, aliquot) folia desunt in A \\
folia 54--55 desunt in A
}
\end{quote}

\noindent che significano, rispettivamente

\begin{quote}
\textit{%
A ha lasciato lo spazio per alcune lettere \\
A ha lasciato lo spazio per alcune parole \\
A ha lasciato una riga bianca \\
A ha lasciato due (o tre, quattro, ecc.) righe bianche \\
in A non si leggono alcune lettere \\
in A non si leggono alcune parole \\
in A non si riesce a leggere una riga \\
in A non si riescono a leggere due (o tre, quattro, ecc.) righe \\
in A manca una carta \\
in A manca la carta 54 \\
in A mancano due (o tre, o alcune, se non si pu{\`o} dire quante) carte \\
in A mancano le carte 54 e 55
}
\end{quote}

Le macro da usare, poi, sono la macro \verb"\LACm"\index{\bs{}LACm} (lacuna
\textit{materiale}) e la macro \verb"\DES{}"\index{\bs{}DES}, la cui
sintassi pu{\`o} essere chiarita dall'esempio seguente.
Supponiamo che nella tradizione, costituita da quattro
testimoni

\begin{itemize}
\item[A] abbia due righe completamente illeggibili
\item[B] non lasci spazi bianchi (ma naturalmente {\`e} comunque
privo del testo)
\item[C] lasci in bianco mezza riga
\item[D] lasci in bianco due righe
\end{itemize}

Nel testo critico vogliamo che la lacuna sia
segnalata da tre asterischi e che in apparato venga
riassunta la situazione ora descritta, cio{\`e}:

%\stelle
%(TC) ***\textsuperscript{1} (TC)
%\myrule
%\textsuperscript{1}\qquad {\slnot duo
%versus legi nequeunt in A; spatium duorum versuum rel. D; spatium
%aliquot verborum rel. C}
%\stelle

\begin{maurotex}
(TC) \VV{
        {A:\DES{duo versus legi nequeunt in}:\LACm}
        {D:\DES{spatium duorum versuum rel.}:}
        {C:\DES{spatium aliquot verborum rel.}:}
        } (TC)
\end{maurotex}

Ci{\`o} si potr{\`a} ottenere scrivendo

\begin{quote}
\begin{verbatim}
(TC) \VV{
        {A:\DES{duo versus legi nequeunt in}:\LACm}
        {D:\DES{spatium duorum versuum rel.}:}
        {C:\DES{spatium aliquot verborum rel.}:}
        } (TC)
\end{verbatim}
\end{quote}

La macro \verb"\DES{}"\index{\bs{}DES} serve per \textbf{des}scrivere in nota
quale sia la situazione, inserendo come argomento una
formula del tipo di quelle elencate qui sopra. \verb"DES" sta per
\textit{descrizione} e questa macro {\`e} appunto riservata alle
note il cui testo il trascrittore deve inserire manualmente
e non {\`e} fornito automaticamente dal {\mtex}.
Infatti nell'argomento di \verb"\DES{}"\index{\bs{}DES} deve essere scritta, a
seconda del caso, una delle formule elencate sopra,
omettendo per{\`o} l'indicazione del testimone che {\`e} gi{\`a}
specificata.

\verb"\LACm"\index{\bs{}LACm} provvede invece a stampare, sia nel testo critico,
che in quello dei testimoni i tre *** che denotano
l'esistenza di una lacuna materiale. A questo proposito
occorre notare che basta apporre \verb"\LACm"\index{\bs{}LACm} solo nel primo
campo di \verb"\VV"\index{\bs{}VV}, dato che il {\mtex} provveder{\`a}
automaticamente a collocare i *** anche nel testo degli
altri testimoni citati all'interno di \verb"\VV"\index{\bs{}VV}. Nell'esempio,
tuttavia, il testimone B non lascia spazi bianchi, pur essendo
privo del testo. Di conseguenza non compare fra i testimoni
elencati nei vari campi (compaiono infatti solo A, C e D).
Di conseguenza il {\mtex} terr{\`a} conto di questa
situazione e \textit{non} apporr{\`a} gli *** nel testo di B, come
deve avvenire.

%-----------------------------------------------------------------------
\section{La struttura segreta di $\protect\backslash$VV}

\label{ref-4.5}

Nell'ultimo esempio occorre notare molto
attentamente un'altra particolarit{\`a}. Nel primo campo
troviamo \verb"A:", seguito da \verb"\DES{}:"\index{\bs{}DES}, seguito da \verb"\LACm"\index{\bs{}LACm}.
Si tratta di un esempio di una \textbf{regola generale}
relativa alla sintassi della macro \verb"\VV"\index{\bs{}VV}. Ogni campo di
\verb"\VV"\index{\bs{}VV} possiede  infatti una sua strutturazione interna in
tre sottocampi separati da `\textbf{:}' e la sintassi {\`e}:

\begin{quote}
\begin{verbatim}
{sigla:eventuali informazioni:lezione}
\end{verbatim}
\end{quote}

\noindent ovvero il primo dei tre sottocampi {\`e} riservato
alle sigle che contraddistinguono i testimoni o a una \verb"*"
nel caso che le sigle non si debbano o non si vogliano
indicare; il secondo a
informazioni ``speciali'' da gestire con opportune macro (in
questo caso  a indicare l'esistenza di una lacuna nel
testimone A, grazie alla macro \verb"\DES{}"\index{\bs{}DES}); il terzo alla lezione di TC o
del testimone. In un caso come questo, dato che non vi {\`e}
testo in nessuno dei testimoni, la lezione {\`e} una sorta di
``pseudolezione'': la macro \verb"\LACm"\index{\bs{}LACm} che provvede a gestire
l'inserimento degli asterischi nel modo appena descritto.

Si noti che \verb"\LACm"\index{\bs{}LACm} non {\`e} l'unico caso di pseudolezione che
conosciamo. Anche \verb"\OM"\index{\bs{}OM}, \verb"\OMLAC"\index{\bs{}OMLAC}, e \verb"\NL"\index{\bs{}NL} giocano un ruolo
simile e la differenza di nome {\`e} dovuta al fatto che servono
a descrivere e a codificare situazioni diverse. Occorre
anche notare che una macro si trova o meno nel secondo o nel
terzo sottocampo (cio{\`e} in quello destinato alle informazioni
aggiuntive o in quello destinato alle lezioni) a seconda
della presenza o meno dei `\textbf{:}'.
Il secondo sottocampo, essendo riservato ad informazioni
speciali, {\`e} infatti
``opzionale''; ci{\`o} che invece non {\`e} opzionale sono i `\textbf{:}' che
separano un sottocampo dall'altro, e che devono venir sempre
scritti esplicitamente, \textbf{anche quando il terzo campo}
\textbf{{\`e} vuoto}.  Nell'esempio precedente, infatti, nel primo campo
tutti e tre i sottocampi erano occupati, ma nei due seguenti no:

\begin{quote}
\begin{verbatim}
    {A:\DES{duo versus legi nequeunt in}:\LACm}
    {D:\DES{spatium duorum versuum rel.}:}
    {C:\DES{spatium aliquot verborum rel.}:}
\end{verbatim}
\end{quote}

Vedremo molti esempi di terzo sottocampo vuoto nei prossimi capitoli, in
cui questa struttura ``segreta'' di \verb"\VV"\index{\bs{}VV} verr{\`a} sfruttata fino in
fondo\footnote{Questa struttura interna di \texttt{$\backslash$VV} dovrebbe
anche chiarire perch{\'e}, nel caso la lezione contenga il segno di
interpunzione `\textbf{:}' deve essere posta fra \texttt{\{\}} come abbiamo
spiegato nel \S\,\ref{ref-4.2.1}. Ponendo la lezione che contiene i `\textbf{:}' fra
parentesi graffe, si evita al {\mtex} di considerare quei
`\textbf{:}' come un separatore di campo. Altrimenti il programma li
interpreterebbe in tal modo e produrrebbe note sottilmente bizzarre ed
erronee e difficili poi da scovare in sede di correzione di bozze.}.

Si noti che il primo
campo non pu{\`o} mai essere vuoto, perch{\'e} deve contenere
necessariamente o la sigla di un testimone o l'asterisco
\verb"*"; il secondo pu{\`o} esserlo; e anche il terzo. Ci{\`o} dovrebbe
far capire perch{\'e} diciamo che \verb"\OM"\index{\bs{}OM} si trova nel terzo
sottocampo: si scrive infatti

\begin{quote}
\begin{verbatim}
\VV{
   {A:lezione}
   {B:\OM}
   }
\end{verbatim}
\end{quote}

e \verb"\OM"\index{\bs{}OM} \textbf{non {\`e} seguita dai} `\textbf{:}'. Ci{\`o},
sintatticamente, significa che si trova nel terzo sottocampo
e che il secondo (che {\`e} opzionale, come abbiamo detto), non
{\`e} stato utilizzato. Ci{\`o} dipende dal fatto che, come gi{\`a}
detto, \verb"\OM"\index{\bs{}OM}, \verb"\OMLAC"\index{\bs{}OMLAC}, \verb"\NL"\index{\bs{}NL} e \verb"\LACm"\index{\bs{}LACm}
rapppresentano una sorta di lezione \textit{in absentia}, e non di informazione
aggiuntiva o speciale.

%-----------------------------------------------------------------------
\section{Ripetizioni e trasposizioni}

\label{ref-4.6}
\index{ripetizioni}
\index{trasposizioni}

Un altro esempio di questa struttura a tre sottocampi {\`e}
fornito dal modo di trattare un particolare tipo di
varianti, la ripetizione. Supponiamo
che la tradizione sia costituita da A e B che leggano:

\begin{quote}
A: erunt igitur quatuor triangula maius quam dimidio portionum \\
B: erunt igitur quatuor triangula triangula
maius quam dimidio portionum
\end{quote}

Si dar{\`a} conto di questa situazione in questo modo:

%\stelle
%erunt igitur quatuor triangula\textsuperscript{1} maius quam dimidio portionum
%\myrule
%\textsuperscript{1}\qquad triangula{\slnot: bis B}
%\stelle

\begin{maurotex}
erunt igitur quatuor \VV{
                        {*:triangula}
                        {B:\BIS:}
                        } maius quam dimidio portionum.
\end{maurotex}

Come si vede, in questo caso {\`e} preferibile
passare all'apparato negativo. Per ottenere questo, si
utilizzer{\`a} una nuova macro, la macro \verb"\BIS"\index{\bs{}BIS}, e si
scriver{\`a} in {\mtex}:

\begin{quote}
\begin{verbatim}
erunt igitur quatuor \VV{
                        {*:triangula}
                        {B:\BIS:}
                        } maius quam dimidio portionum.
\end{verbatim}
\end{quote}

Come \verb"\LACm"\index{\bs{}LACm}, \verb"\BIS"\index{\bs{}BIS} va collocata nel secondo
sottocampo, ma in questo caso, dato che si {\`e} passati
all'apparato negativo, il terzo sottocampo resta vuoto.

Potrebbe per{\`o} avvenire che A abbia la lezione corretta
(``triangula''), mentre B legga ``quadrata'', ripetendolo
due volte, cio{\`e}:

\begin{quote}
B: erunt igitur quatuor quadrata quadrata
maius quam dimidio portionum
\end{quote}

In un caso del genere sar{\`a} meglio non utilizzare
\verb"\BIS"\index{\bs{}BIS}, ma scrivere la ripetizione per esteso, utilizzando
un apparato positivo:

%\stelle
%erunt igitur quatuor triangula\textsuperscript{1} maius quam dimidio portionum
%\myrule
%\textsuperscript{1}\qquad triangula \textsc{A} quadrata quadrata \textsc{B}
%\stelle

\begin{maurotex}
erunt quatuor \VV{
                 {A:triangula}
                 {B:quadrata quadrata}
                 } maius quam dimidio portionum.
\end{maurotex}

\noindent il che si otterr{\`a}, ovviamente, usando \verb"\VV"\index{\bs{}VV} nella sua
forma di base:

\begin{quote}
\begin{verbatim}
erunt quatuor \VV{
                 {A:triangula}
                 {B:quadrata quadrata}
                 } maius quam dimidio portionum.
\end{verbatim}
\end{quote}

Se per{\`o} la ripetizione coinvolgesse una porzione di testo
pi{\'u} lunga di due o tre parole, si proceder{\`a} come indicato
nel \S\,\ref{ref-7.1.3}.

Al capitolo \ref{ref-8}
rinviamo invece per quanto riguarda il trattamento delle
trasposizioni, soprattutto per quelle che superino le due o tre parole, 
limitandoci
qui ad anticipare che le trasposizioni ``puntuali'' (``il
povero cieco''al posto di ``il cieco povero'') dovranno essere trattate
utilizzando invece di \verb"\VV"\index{\bs{}VV} una macro a lei del tutto
simile, \verb"\TV{}"\index{\bs{}TV}:

\begin{quote}
\begin{verbatim}
\TV{
   {A:il povero cieco}
   {B:il cieco povero}
   }
\end{verbatim}
\end{quote}

%-----------------------------------------------------------------------
\section{Varianti attestate della tradizione indiretta}

\label{ref-4.7}
\index{varianti della tradizione indiretta}
\index{tradizione indiretta}

Pu{\`o} accadere che di un'opera di Maurolico si abbia anche una
tradizione indiretta, e cio{\`e} che uno o pi{\'u} brani siano
citati in un'altra opera da Maurolico stesso o da un autore
contemporaneo o di poco successivo (vedi ``Premessa'',
\S\,\ref{ref-2}). {\`E} il caso di alcuni passi dei \textit{Photismi} che, prima della loro
pubblicazione, erano gi{\`a} citati in una lettera del gesuita
G.G.~Staserio a Clavio. Tale lettera di Staserio  {\`e} allora a
tutti gli effetti, sia pure limitatamente a quei passi, un
testimone indiretto dei \textit{Photismi} e l'editore dovr{\`a} tener
conto di eventuali varianti testuali da lui attestate, che
potrebbero risalire a un manoscritto perduto.

All'occasione perci{\`o} il testimone indiretto sar{\`a} trattato
come gli altri testimoni:  al solito,
nel campo di \verb"\VV"\index{\bs{}VV} prescelto, la sua lezione
occuper{\`a} l'ultimo sottocampo e il primo sottocampo sar{\`a}
occupato dall'indicazione del testimone, che qui 
non sar{\`a} per{\`o} costituito da una sigla, ma dal nome
dell'autore che cita il passo o dal titolo dell'opera in cui
si trova la citazione. L'editore, ovviamente, provveder{\`a} a
spiegare nell'introduzione la situazione in dettaglio.

Ad esempio: siano A e B i testimoni di un'opera
mauroliciana citata occasionalmente anche da Clavio:

\begin{quote}
A:\hphantom{\textit{lavius}} primus et secundus \\
B:\hphantom{\textit{lavius}} primus vel secundus \\
Clavius: primus secundusque.
\end{quote}

L'editore segue Clavio e scrive:

%\stelle
%primus secundusque\textsuperscript{1}
%\myrule
%\textsuperscript{1}\qquad secundusque \textsc{Clavius}
%{\rmnot et
%secundus} \textsc{A} vel secundus \textsc{B}
%\stelle

\begin{maurotex}
primus \VV{
          {Clavius:secundusque}
          {A:et secundus}
          {B:vel secundus}
          }
\end{maurotex}

\noindent cio{\`e}, in {\mtex}:

\begin{quote}
\begin{verbatim}
primus \VV{
          {Clavius:secundusque}
          {A:et secundus}
          {B:vel secundus}
          }
\end{verbatim}
\end{quote}

Naturalmente si  potrebbe  rifiutare
il testo offerto da Clavio e scrivere:

\begin{quote}
\begin{verbatim}
primus \VV{
          {A:et secundus}
          {Clavius:secundusque}
          {B:vel secundus}
          }
\end{verbatim}
\end{quote}

Se poi Clavio avesse omesso le parole ``et secundus'', si
sarebbe scritto:

\begin{quote}
\begin{verbatim}
primus \VV{
          {A:et secundus}
          {B:vel secundus}
          {Clavius:\OM}
          }
\end{verbatim}
\end{quote}

\verb"Clavius" viene dunque trattato a tutti gli
effetti come se fosse il \textit{siglum} di un testimone e
tutte le situazioni descritte in questo capitolo e
nei prossimi si applicano anche al caso della tradizione
indiretta.

%-----------------------------------------------------------------------
\chapter[Integrazioni, aggiunte e correzioni]{Integrazioni, aggiunte e correzioni del copista o di altre mani}

\label{ref-5}

\section{Come trattare le varie mani e i vari interventi subitida un
testimone}

\label{ref-5.1}
\index{mani}

La filosofia di questa sezione del manuale si riassume in
quest'idea: trattare le varie mani, aggiunte, ecc. presenti
in un testimone \textit{come~se} fossero altrettanti testimoni
diversi. Ci{\`o} permette di utilizzare la struttura di
\verb"\VV"\index{\bs{}VV} definita nel
capitolo precedente (\ref{ref-4}), con un  minimo di adattamenti, e di
ottenere risultati ragionevolmente soddisfacenti.

Il primo passo per poter procedere in questa direzione
consiste nel  distinguere all'interno di ogni singolo
testimone i vari livelli di intervento  che esso ha subito,
specificando i risultati dell'analisi in un opportuno \textit{conspectus siglorum}\footnote{Forniamo nel capitolo \ref{ref-10}
i \textit{sigla} che si {\`e} convenuto di utilizzare per i
manoscritti e gli stampati che porteranno alla costituzione
dell'edizione complessiva delle opere mauroliciane.
L'editore dovr{\`a} comunque illustrare in un opportuno \textit{conspectus} quelli che utilizzer{\`a} per la sua particolare
edizione critica.}. Per fare un esempio, supponiamo di avere
a che fare con un testimone (A) che contiene un testo
autografo, che ha sub{\'\i}to rimaneggiamenti da parte di
un'altra mano non meglio nota, e ulteriori rimaneggiamenti
autografi di Maurolico. In tal caso il \textit{conspectus
siglorum} risulterebbe il seguente:

\begin{itemize}

\item[A] Paris, Biblioth{\`e}que Nationale
de France, Par.~Lat. 7465, cc.~17r--30v.

\item[A\textsuperscript{m}] interventi autografi di Maurolico
in A

\item[A\textsuperscript{2}] interventi di mano ignota in A, in questo caso
precedenti agli interventi di A\textsuperscript{m}.

\end{itemize}

Pi{\'u} precisamente, dato il testimone denotato dal
\textit{siglum} A, intenderemo con
``copista'' la mano (eventualmente quella di Maurolico
stesso) che l'ha steso. Con ``correttore/i'', intenderemo la/e mano/i che
hanno provveduto ad effettuare modifiche ad A (ivi compreso
Maurolico, di nuovo). E, in generale, indicheremo con i
\textit{sigla}:

\begin{itemize}

\item[A\textsuperscript{1}] gli interventi del copista di A sul suo
stesso testo per correggere ci{\`o} che aveva scritto;

\item[A\textsuperscript{m}] gli interventi autografi di
Maurolico sul testimone A (nel caso che il copista di A sia
stato Maurolico stesso si user{\`a} per{\`o} A\textsuperscript{1} e
non A\textsuperscript{m});

\item[A\textsuperscript{2}] interventi su A da parte di un correttore
diverso dal copista e da Maurolico (nell'introduzione si
specificher{\`a}, se possibile,  di chi si
tratti, o almeno a quale epoca
risalgano);

\item[A\textsuperscript{3}] interventi su A da parte di un correttore
diverso dal copista, da Maurolico e da A\textsuperscript{2}, ecc.

\end{itemize}

Naturalmente lo stesso varr{\`a} anche per le sigle
relative a B: B\textsuperscript{1}, B\textsuperscript{m}, B\textsuperscript{2}  ecc.

Per introdurre nelle macro questi \textit{sigla} modificati per
dar conto delle varie mani, baster{\`a} battere \verb"A1", \verb"Am",
\verb"A2", \verb"A3", ecc. e penser{\`a} poi il {\mtex} a
sistemare i numeri in esponente. Per maggiori dettagli
riguardo al modo di inserire i \textit{sigla} e per le
convenzioni adottate in questa edizione rinviamo al capitolo
\ref{ref-10},
in cui viene presentato anche il \textit{conspectus
siglorum} generale dell'edizione.

Qui ci limitiamo a segnalare un punto relativo a come
trattare le correzioni e le aggiunte del copista, che esso
coincida o meno con Maurolico stesso.  L'uso di \verb"A1" va
effettuato con una certa parsimonia. Come si vedr{\`a} fra breve, una serie
di macro permettono di dar conto di
aggiunte interlineari, aggiunte in margine, correzioni ecc.,
senza ricorrere al \textit{siglum} A\textsuperscript{1}. Non
avrebbe molto senso dire in apparato
che  la tale o altra  lezione {\`e} stata aggiunta nell'interlinea
o nel margine di A e che tale aggiunta l'ha fatta 
A\textsuperscript{1}: 
chi diavolo dovrebbe averla fatta, senn{\`o}, visto che il
significato di A\textsuperscript{1} {\`e} quello di indicare gli
interventi del copista di A che corregge s{\'e} stesso?

Per contro torner{\`a} molto comodo usare il \textit{siglum} A\textsuperscript{1} in tutti quei casi in cui si ha un contrasto fra una
primitiva lezione ed una successiva: se il copista si
corregge, cambiando ad esempio ``23 maii'' in ``22 aprilis''
si potr{\`a} dire in apparato: ``22 aprilis
\textsl{A}\textsuperscript{1}
23
maii \textsl{A}''. Nel corso di questo capitolo si troveranno
abbondanti esempi di utilizzo del \textit{siglum} A e del \textit{siglum} A\textsuperscript{1} che speriamo possano rendere chiara la
situazione.

Sar{\`a} per{\`o} bene (soprattutto ai fini di studi
successivi) fare in modo che l'informazione a proposito
degli interventi del copista sia  codificata in
modo uniforme. Pertanto gli interventi del copista, di
qualunque tipo essi siano, verranno ``marcati'' aggiungendo
un \verb"+" subito dopo il \textit{siglum} del testimone: \verb"A+". Tale \verb"+"
non verr{\`a} interpretato ai fini della costruzione del TC e
dell'apparato, ma lo potr{\`a} essere nel caso si vogliano
studiare gli interventi del copista stesso, confrontarli con
quelli di altre mani, ecc. Tuttavia quando si utilizzi il
siglum A\textsuperscript{1}, non sar{\`a} necessario scrivere \verb"A1+", ma potr{\`a}
bastare, ai fini del \textit{mark-up} del testo scrivere solo
\verb"A1". L'{\mtex} tuttavia non segnaler{\`a} errore nel caso si aggiunga
il \verb"+" ad \verb"A1".

Stabilito questo sistema di sigle, passiamo a vedere come
si potr{\`a} dar conto dei vari interventi subiti dalla
tradizione, utilizzando la macro \verb"\VV"\index{\bs{}VV}. Avvertiamo prima
per{\`o} che, come criterio generale, sar{\`a} opportuno non
separare in apparato le varie mani di un testimone (reale),
ma  raggrupparle per testimone (tutti gli interventi
delle varie mani di A insieme, non prima A\textsuperscript{2}, poi B, poi
C, poi A\textsuperscript{m} e infine C\textsuperscript{3}).

%-----------------------------------------------------------------------
\section[Integrazioni interlineari, marginali e in lacuna]{Integrazioni
interlineari, marginali e in lacuna del copista o di un'altra mano}

\label{ref-5.2}
\index{integrazioni}
\index{integrazioni interlineari}
\index{integrazioni in margine}
\index{integrazioni in lacuna}

In questo paragrafo ci riferiamo ad integrazioni apposte dal
copista o dal correttore, \textit{miranti a colmare
un'omissione precedente}. Sono invece escluse:

\begin{enumerate}
\item le correzioni alla lezione originaria del
testimone (che tratteremo nel \S\,\ref{ref-5.3});

\item le aggiunte di Maurolico, autografe o meno, che
costituiscono a giudizio dell'editore una seconda redazione
del testo (\S\,\ref{ref-5.3});

\item note marginali che non costituiscono un'aggiunta al
testo  ma glosse o commenti;  aggiunte non risalenti a
Maurolico e che costituiscono un rifacimento del
testo effettuato da altri (\S\,\ref{ref-5.4}).

\end{enumerate}

%-----------------------------------------------------------------------
\subsection{Integrazioni interlineari}

\label{ref-5.2.1}
\index{integrazioni interlineari}

Cominciamo dal caso pi{\'u} semplice. La
tradizione consista di un solo testimone, non autografo, A. Il copista 
di A, dovendo scrivere ``primum et secundum'', 
ha  saltato le parole ``et secundum'', accorgendosene
poi e integrandole in interlinea. Il trascrittore
dovr{\`a} ottenere:

%\stelle
%primum et secundum\textsuperscript{1}
%\hrule width 1cm
%\textsuperscript{1}\qquad {\rmnot et secundum} {\slnot in interl. A}
%\stelle

\begin{maurotex}
primum \VV{
          {A+:\INTERL:et secundum}
          }
\end{maurotex}

Non si tratta di un intervento di altra
mano e l'intervento del copista non si riferisce ad un
testo alternativo:
non sar{\`a} dunque necessario nominare nessun altro
\textit{siglum} (n{\'e} A\textsuperscript{1}, n{\'e} A\textsuperscript{2}, n{\'e} A\textsuperscript{m}). Il formato della macro che produce questa nota {\`e} il
seguente:

\begin{quote}
\begin{verbatim}
primum \VV{
          {A+:\INTERL:et secundum}
          }
\end{verbatim}
\end{quote}

\noindent e si noti che la \verb"\INTERL"\index{\bs{}INTERL} viene inserita nel
secondo sottocampo del campo relativo ad A. Si tratta di un
esempio della ``struttura segreta di \verb"\VV"\index{\bs{}VV}'' (cfr. \S\,\ref{ref-4.5}):
nel primo sottocampo si indica il testimone (o la \verb"*", nel
caso si utilizzi l'apparato negativo o non sia opportuno
specificarlo); nel secondo indicazioni ``speciali''; nel
terzo la lezione del testimone. I `\textbf{:}' servono a separare i
tre campi e non vanno lasciati spazi bianchi prima o dopo di
essi.

Si noti anche che nel primo sottocampo compare non \verb"A", ma
\verb"A+", in quanto si tratta di un intervento del copista sul
testo che sta scrivendo.

Vediamo ora un caso leggermente pi{\'u} complesso. Se  B avesse
anch'esso omesso ``et secundum'' ma, a differenza di
A non l'avesse integrato neppure in interlinea, dovremmo
ottenere

%\stelle
%primum et secundum\textsuperscript{1}
%\hrule width 1cm
%\textsuperscript{1}\qquad {\rmnot et secundum} {\slnot in interl. A; om. B}
%\stelle

\begin{maurotex}
primum \VV{
          {A+:\INTERL:et secundum}
          {B:\OM}
          }
\end{maurotex}

La macro relativa sar{\`a}

\begin{quote}
\begin{verbatim}
primum \VV{
          {A+:\INTERL:et secundum}
          {B:\OM}
          }
\end{verbatim}
\end{quote}

\noindent e se B avesse omesso lasciando uno spazio bianco,
scriveremmo invece

%\stelle
%primum et secundum\textsuperscript{1}
%\hrule width 1cm
%\textsuperscript{1}\qquad {\rmnot et secundum} {\slnot in interl. A; spatio
%relicto om. B}
%\stelle

\begin{maurotex}
primum \VV{
          {A+:\INTERL:et secundum}
          {B:\OMLAC}
          }
\end{maurotex}

\noindent e per ottenere questo batteremo

\begin{quote}
\begin{verbatim}
primum \VV{
          {A+:\INTERL:et secundum}
          {B:\OMLAC}
          }
\end{verbatim}
\end{quote}

Potrebbe poi darsi che il copista di A si sia
accorto di non aver scritto ``et secundum'' solo in un
secondo tempo (cosa dimostrata dall'uso
di un inchiostro diverso) e solo in un secondo tempo
abbia integrato queste parole. Questo esempio mostra
l'importanza di codificare gli interventi del copista
mediante il \verb"+", soprattutto nel
caso che  essi siano datati o databili, perch{\'e}
ci{\`o} potrebbe aiutare a collocare cronologicamente B. Segnaleremo
quindi che B ha un testo che A omette, ma che il suo copista
ha poi agggiunto. Supponendo che la lezione di B sia
``primum et secundum'' avremo allora in apparato:

%\stelle
%primum et secundum\textsuperscript{1}
%\hrule width 1cm
%\textsuperscript{1}\qquad {\rmnot et secundum} {\slnot B in
%interl. diverso atramento A}
%\stelle

\begin{maurotex}
primum \VV{
          {B: et secundum}
          {A+:\INTERL\DES{diverso atramento}:}
          }
\end{maurotex}

Utilizzeremo la macro \verb"\DES{}"\index{\bs{}DES} per dar conto del
cambio di inchiostro:

\begin{quote}
\begin{verbatim}
primum \VV{
          {B: et secundum}
          {A+:\INTERL\DES{diverso atramento}:}
          }
\end{verbatim}
\end{quote}

Si osservi che in questo modo non solo viene
codificato l'intervento del copista per mezzo di \verb"A+", ma vengono fornite in
apparato anche gli elementi per stabilire possibili datazioni
relative di A, di B e degli interventi fatti dal copista di
A.

{\`E} opportuno anche osservare in quest'ultimo esempio che l'ultimo
campo dopo \verb"\INTERL"\index{\bs{}INTERL} e \verb"\DES{}"\index{\bs{}DES} {\`e} lasciato vuoto. La cosa {\`e}
\textbf{indispensabile}: non si potrebbe scrivere
\verb"{A+:\INTERL\DES{diverso atramento}}", senza i `\textbf{:}'
finali. Scrivendo in questo modo si sta infatti dicendo alla macro
\verb"\VV"\index{\bs{}VV} che A ha la stessa lezione del testo critico e che l'editore
vuole solo dare un'informazione aggiuntiva, fornita da \verb"\INTERL"\index{\bs{}INTERL} e da
\verb"\DES{}"\index{\bs{}DES} (si ricordi che le informazioni aggiuntive devono essere
collocate nel secondo sottocampo: cfr. \S\,\ref{ref-4.5}). I `\textbf{:}' dopo
\verb"\INTERL"\index{\bs{}INTERL} servono appunto per lasciare la lezione di A vuota, in modo
che nella nota risulti che {\`e} la stessa di quella di B.

Per lo stesso motivo non bisogna
assolutamente battere

\begin{quote}
\begin{verbatim}
{A+:\INTERL\DES{diverso atramento}:et secundum}
\end{verbatim}
\end{quote}

\noindent che
produrrebbe una nota orribile:

%\stelle
%primum et secundum\textsuperscript{1}
%\hrule width 1cm
%\textsuperscript{1}\qquad {\rmnot et secundum} {\slnot B} {\rmnot
%et secundum} {\slnot in interl. diverso atramento A}
%\stelle

\begin{maurotex}
primum \VV{
          {B:et secundum}
          {A+:\INTERL\DES{diverso atramento}:et secundum}
          }
\end{maurotex}

\noindent scrivendo due volte ``et secundum'' in
nota\footnote{Diverso {\`e} il caso di quando si scrive
\texttt{\{A:$\backslash$OM\}}\thinspace: le macro \texttt{$\backslash$OM}, \texttt{$\backslash$OMLAC}, \texttt{$\backslash$NL} e
\texttt{$\backslash$LACm}
infatti sono state costruite in modo da costituire una sorta
di lezione e quindi vanno collocate nel terzo sottocampo dei
campi interni di \texttt{$\backslash$VV} e non si devono mettere  i \texttt{:} dopo
di loro. Cfr. \S\,\ref{ref-4.5}.}.

Potrebbe darsi poi che il copista di A (un gran distratto),
non abbia effettuata l'integrazione nemmeno in un secondo
tempo e che essa sia stata poi eseguita da un'altra mano
(mettiamo A\textsuperscript{2}). Vorremo allora ottenere

%\stelle
%primum et secundum\textsuperscript{1}
%\hrule width 1cm
%\textsuperscript{1}\qquad {\rmnot et secundum} {\slnot B om. A in
%interl. add.
%A$^{\hbox{\slnotmm 2}}$}
%\stelle

\begin{maurotex}
primum \VV{
          {B:et secundum}
          {A:\OM}
          {A2:\INTERL\DES{add.}:}
          }
\end{maurotex}

\noindent dove ``add.'' sta per ``addidit'': A\textsuperscript{2} ha
aggiunto ``et secundum'' nell'interlinea. Scriveremo:

\begin{quote}
\begin{verbatim}
primum \VV{
          {B:et secundum}
          {A:\OM}
          {A2:\INTERL\DES{add.}:}
          }
\end{verbatim}
\end{quote}

Se invece l'integrazione interlineare fatta
da A\textsuperscript{2} fosse ``et tertium'' invece della lezione
``et secundum'' di B che noi vogliamo accogliere in TC,
l'apparato dovr{\`a} essere:

%\stelle
%primum et secundum\textsuperscript{1}
%\hrule width 1cm
%\textsuperscript{1}\qquad {\rmnot et secundum} {\slnot B om. A}
%{\rmnot et tertium} {\slnot in interl. A$^{\hbox{\slnotmm 2}}$}
%\stelle

\begin{maurotex}
primum \VV{
          {B:et secundum}
          {A:\OM}
          {A2:\INTERL:et tertium}
          }
\end{maurotex}

\noindent e scriveremo

\begin{quote}
\begin{verbatim}
primum \VV{
          {B:et secundum}
          {A:\OM}
          {A2:\INTERL:et tertium}
          }
\end{verbatim}
\end{quote}

\noindent (si noti che in questo caso dopo
\verb"\INTERL"\index{\bs{}INTERL} segue la lezione, dato che diverge da
quella del TC!)

Infine, a fronte dell'integrazione di A\textsuperscript{2}, B potrebbe
avere omesso ``et secundum'' senza integrarlo in nessun modo: dovremmo
allora avere:

%\stelle
%primum et secundum\textsuperscript{1}
%\hrule width 1cm
%\textsuperscript{1}\qquad {\rmnot et secundum} {\slnot om. AB in interl. add.
%A$^{\slnot 2}$}
%\stelle

\begin{maurotex}
primum et secundum \VV{
                      {A/B:\OM}
                      {A2:\INTERL\DES{add.}:et secundum}
                      }
\end{maurotex}

\noindent scrivendo

\begin{quote}
\begin{verbatim}
primum et secundum \VV{
                      {A/B:\OM}
                      {A2:\INTERL\DES{add.}:et secundum}
                      }
\end{verbatim}
\end{quote}
% qui donne un om. dans le texte
% ajouter une note ?

{\new} Attention: ce dernier exemple est un cas tr\`es particulier,
puisqu'il produit un \textit{om.} dans le texte critique. La pr\'esence de
ce type de \textit{om.} dans le langage {\mtex} est encore en
discussion\footnote{F\'evrier 2005.}. Avant d'utiliser cette ressource, il
est pr\'ef\'erable d'en faire part au directeur et au comit\'e technique du
projet.

%-----------------------------------------------------------------------
\subsection{Integrazioni in margine}

\label{ref-5.2.2}
\index{integrazioni in margine}

Per integrazioni in margine intendiamo esattamente la
stessa cosa delle integrazioni interlineari, salvo il fatto
che invece di trovarsi in interlinea si trovano scritte per
l'appunto in margine. Eviteremo quindi di ripetere gli
esempi fatti sopra, sottintendendo che devono essere tenuti presenti
anche per questa situazione.

C'{\`e} per{\`o} una differenza importante: l'integrazione in
margine pu{\`o} trovarsi nel testimone \textbf{con o senza un
opportuno segno di richiamo}. I due casi andranno
evidentemente distinti, e si useranno anche due macro
diverse come si vede dai due esempi seguenti.

%-----------------------------------------------------------------------
\subsubsection{Senza uno segno di richiamo}

\label{ref-5.2.2.1}
\index{integrazioni in margine}
\index{integrazioni in margine senza segno di richiamo}

Supponiamo che la frase ``per 47 primi'' si trovi nel
margine di A, di mano del copista di A e con lo stesso
inchiostro; che nella carta \textit{non
compaia alcun segno di richiamo}, ma che l'editore ritenga
che non possa che riferirsi alla frase del testo ``quadratum
$ab$ aequale erit duobus quadratis $ac$ et $bc$'' e
che quindi voglia ottenere il seguente testo:

%\stelle
%quadratum $ab$ aequale erit duobus quadratis $ac$ et $bc$ per 47
%primi\textsuperscript{1}
%\hrule width 1cm
%\textsuperscript{1}\qquad {\rmnot per 47 primi} {\slnot in marg. A}
%\stelle

\begin{maurotex}
quadratum $ab$ aequale erit duobus quadratis
\(ab\) et \(bc\) \VV{
                    {A+:\MARG:per 47 primi}
                    }
\end{maurotex}

Per ottenere questo risultato si dovr{\`a} scrivere:

\begin{quote}
\begin{verbatim}
quadratum $ab$ aequale erit duobus quadratis
\(ab\) et \(bc\) \VV{
                    {A+:\MARG:per 47 primi}
                    }
\end{verbatim}
\end{quote}

\noindent dove si utilizza la nuova macro \verb"\MARG"\index{\bs{}MARG} invece di
\verb"\INTERL"\index{\bs{}INTERL} ma con la stessa identica sintassi. Tutti i casi
previsti dal \S\,\ref{ref-5.2.1} possono essere ripresi \textit{verbatim},
sostituendo \verb"\INTERL"\index{\bs{}INTERL} con \verb"\MARG"\index{\bs{}MARG}. Si potranno dare
ovviamente anche casi misti: per esempio ``per 47 primi''
potrebbe essere di mano di A\textsuperscript{2} e trovarsi nel margine di
A, mentre il testimone B la d{\`a} in interlinea. In questo
caso vorremo ottenere:

%\stelle
%quadratum $ab$ aequale erit duobus quadratis $ac$ et $bc$ per 47
%primi\textsuperscript{1}
%\hrule width 1cm
%\textsuperscript{1}\qquad {\rmnot per 47 primi} {\slnot om AB  in interl. add.
%B$^{\slnot 1}$ in marg. A$^{\slnot 2}$}
%\stelle

\begin{maurotex}
quadratum $ab$ aequale erit duobus quadratis
$ac$ et $bc$ \VV{
                {*:per 47 primi}
                {A/B:\OM}
                {B1:\INTERL\DES{add.}:}
                {A2:\MARG:}
                }
\end{maurotex}

Per ottenere questo risultato si dovr{\`a} battere\new:

\begin{quote}
\begin{verbatim}
quadratum $ab$ aequale erit duobus quadratis
$ac$ et $bc$ \VV{
                {*:per 47 primi}
                {A/B:\OM}
                {B1:\INTERL\DES{add.}:}
                {A2:\MARG:}
                }
\end{verbatim}
\end{quote}
% J'ai corrig\'e: il y avait une \'etoile apr\`es le B1

\noindent dove si pu{\`o} notare che in una situazione
simile anche \verb"\MARG"\index{\bs{}MARG} richiede che il terzo sottocampo venga
lasciato vuoto.

%-----------------------------------------------------------------------
\subsubsection{Con uno segno di richiamo}

\label{ref-5.2.2.2}
\index{integrazioni in margine}
\index{integrazioni in margine con segno di richiamo}

Se il copista di A avesse per{\`o} apposto un richiamo accanto
alla parola ``erit'' la situazione sarebbe
leggermente diversa e andrebbe segnalata: nel caso
precedente infatti l'editore colloca la frase a suo
arbitrio, guidato dalla sua conoscenza del modo di scrivere
di Maurolico e dal senso del discorso che si viene
sviluppando; qui invece {\`e} il testimone a dirgli dove deve
essere collocata l'integrazione marginale. Scriver{\`a} quindi:

%\stelle
%quadratum $ab$ aequale erit, per 47 primi\textsuperscript{1}, duobus
%quadratis $ac$ et $bc$
%\hrule width 1cm
%\textsuperscript{1}\qquad {\rmnot per 47 primi} {\slnot signo posito in marg. A}
%\stelle

\begin{maurotex}
quadratum $ab$
aequale erit, \VV{
                 {A+:\MARGSIGN:per 47 primi}
                 }, duobus
quadratis $ac$ et $bc$
\end{maurotex}

\noindent dove l'espressione ``signo posito in marg. A''  sta per ``in
A {\`e} in margine con un segno di richiamo''. Per ottenere
questo risultato baster{\`a} scrivere:

\begin{quote}
\begin{verbatim}
quadratum $ab$
aequale erit, \VV{
                 {A+:\MARGSIGN:per 47 primi}
                 }, duobus
quadratis $ac$ et $bc$
\end{verbatim}
\end{quote}

\verb"\MARGSIGN"\index{\bs{}MARGSIGN} si comporta esattamente come \verb"\MARG"\index{\bs{}MARG}
o \verb"\INTERL"\index{\bs{}INTERL} e tutti gli esempi precedenti si applicano anche
a essa.

Occorre notare che l'editore, nel momento in cui riceve il
testo dal trascrittore far{\`a} bene a ricercare tutte le
occorenze di \verb"\MARG"\index{\bs{}MARG} per controllare se le scelte del
trascrittore (eventualmente di lui stesso in questa veste)
soddisfano pienamente la sua sensibilit{\`a} critica.

%-----------------------------------------------------------------------
\subsection{Che fare quando le integrazioni sono lunghe?}

\label{ref-5.2.3}
\index{integrazioni lunghe}

Spesso le aggiunte del copista (specialmente nel caso che
esso sia Maurolico in persona) sono molto lunghe, a volte
mezze pagine intere. Si pone quindi il problema cui abbiamo
accennato nel \S\,\ref{ref-4.2.1}, nota~9: il sistema qui descritto
riporterebbe in nota tutta la mezza pagina che si trova gi{\`a}
nel testo critico. Per il caso particolare delle
integrazioni lunghe del copista o di altre mani, rinviamo quindi
il lettore che vi si imbattesse al \S\,\ref{ref-7.1.2}.

%-----------------------------------------------------------------------
\section[Correzioni del copista]{Correzioni del copista e correzioni dovute
ad  altre mani}

\label{ref-5.3}

\subsection{Correzioni del copista}
\index{correzioni del copista}

\label{ref-5.3.1}

In questo paragrafo 5.3.1 introdurremo quattro nuove macro,
che servono principalemente a trattare le correzioni \textit{in
scribendo} del copista, fatte cio{\`e}~---~presumibilmente~---~dal
copista nel corso della stesura del manoscritto. La mano da
indicare quindi {\`e} sempre A (non A\textsuperscript{1}). Questo tipo
di interventi {\`e} concettualmente distinto da interventi pi{\'u}
pesanti (non necessariamente come estensione) che saranno
affrontati nel \S\,\ref{ref-5.3.2}.

%-----------------------------------------------------------------------
\subsubsection{$\backslash$ANTEDEL; $\backslash$POSTDEL}

\label{ref-5.3.1.1}

Il caso pi{\'u} semplice {\`e}
quello in cui il copista di A abbia in un primo momento
scritto ``triangulum primum'', lo abbia poi cancellato e
abbia
scritto a seguire ``secundum erit''. A si presenta dunque in
questo stato:

\begin{itemize}

\item[A:] triangulum $\overline{\underline{\hbox{\rm primum}}}$ secundum erit

\end{itemize}

\noindent dove con $\overline{\underline{\hbox{\rm primum}}}$ vogliamo appunto indicare che ``primum'' {\`e} stato
concellato con due tratti di penna ma resta sempre ben
leggibile. In un caso del genere vorremo avere:

%\stelle
%\hrule width 1cm
%\textsuperscript{1}\qquad {\slnot ante} {\rmnot secundum} {\slnot
%del.} {\rmnot primum} {\slnot A}
%\stelle

\begin{maurotex}
triangulum \VV{
              {A+:\ANTEDEL{primum}:secundum}
              } erit
\end{maurotex}

\noindent dove ``del.'' sta per ``delevit'' e il  tutto
significa: ``[il copista di] A ha cancellato `primum' subito
prima di `secundum'\thinspace''. Per ottenere questa scrittura
bisogner{\`a} scrivere:

\begin{quote}
\begin{verbatim}
triangulum \VV{
              {A+:\ANTEDEL{primum}:secundum}
              } erit
\end{verbatim}
\end{quote}

Si noti che \verb"\ANTEDEL{}"\index{\bs{}ANTEDEL} non solo {\`e} una macro
``informativa''~---~da collocare quindi nel secondo sottocampo
come \verb"\INTERL"\index{\bs{}INTERL} o \verb"\MARG"\index{\bs{}MARG}~---~ma ha anche  un
argomento: fra le sue \verb"{}" si deve inserire la lezione
cancellata (``primum''), mentre nel
terzo sottocampo deve andare la lezione che il testimone riporta
(``secundum'').

Potrebbe per{\`o} darsi il caso che in certe situazioni
l'editore preferisca, per motivi di chiarezza, procedere
invece cos{\'\i}\footnote{In certi casi vi pu{\`o} essere
praticamente obbligato:  per un esempio, si veda il
\S\,\ref{ref-6.1.1}.}:

%\stelle
%triangulum\textsuperscript{1} secundum erit
%\hrule width 1cm
%\textsuperscript{1}\qquad {\slnot post} {\rmnot triangulum} {\slnot
%del.} {\rmnot primum} {\slnot A}
%\stelle

\begin{maurotex}
\VV{
   {A+:\POSTDEL{primum}:triangulum}
   } secundum erit
\end{maurotex}

L'esempio {\`e} lo stesso di prima, ma qui per qualche
ragione si preferisce mettere la nota dopo ``triangulum''
invece che dopo ``secundum'' e, naturalmente, scrivere \textit{post} invece che \textit{ante}. Si user{\`a} allora la macro
\verb"\POSTDEL"\index{\bs{}POSTDEL}:

\begin{quote}
\begin{verbatim}
\VV{
   {A+:\POSTDEL{primum}:triangulum}
   } secundum erit
\end{verbatim}
\end{quote}

La sintassi di \verb"\POSTDEL{}"\index{\bs{}POSTDEL} {\`e} la stessa di \verb"\ANTEDEL{}"\index{\bs{}ANTEDEL}:
viene collocata nel secondo sottocampo di \verb"\VV"\index{\bs{}VV}, nel suo
argomento va la lezione cancellata, nel terzo sottocampo si
scrive la lezione cui
si deve riferire la nota.

Potrebbe succedere che che la lezione depennata dal copista
non risulti pi{\'u} leggibile; nel campo di \verb"\POSTDEL"\index{\bs{}POSTDEL} o
di \verb"\ANTEDEL"\index{\bs{}ANTEDEL} si inserir{\`a} allora una breve descrizione della
situazione, usando una delle seguenti formule:

\begin{enumerate}

\item \textit{aliquot literas}: se sono state depennate
alcune lettere o~---~presumibilmente~---~non pi{\'u} di una parola.
\item \textit{aliquot verba} se sono state depennate alcune
parole
\item \textit{unum versum} se {\`e} stata cancellata un'intera
riga.
\item \textit{duos versus}, oppure \textit{tres versus}, ecc.
se sono state cancellate due, tre ecc. righe.

\end{enumerate}

Se ad esempio il copista di A avesse scritto ``Propositio
13'', ma se dopo  ``Propositio'' e prima di ``13'',
comparisse una cancellatura che indicava il numero
primitivamente assegnato a questa proposizione,  ormai non
pi{\'u} leggibile vorremo avere:

%\stelle
%Propositio\textsuperscript{1} 13
%\hrule width 1cm
%\textsuperscript{1}\qquad {\slnot post} {\rmnot Propositio} {\slnot
%del. aliquot literas A}
%\stelle

\begin{maurotex}
\VV{
   {A+:\POSTDEL{\PL{aliquot literas}}:Propositio}
   } 13
\end{maurotex}
% ici je corrige le \DES en \PL sinon cela ne passe pas m2lv

\noindent e si scriver{\`a} in {\mtex}\new:

\begin{quote}
\begin{verbatim}
\VV{
   {A+:\POSTDEL{PL{aliquot literas}}:Propositio}
   } 13
\end{verbatim}
\end{quote}

\noindent dove, come si vede, si utilizza la
nuova
% ajout\'e par jps le 31-01-05
macro \verb"\PL"\index{\bs{}PL}
%(cfr. \S\,\ref{ref-4.4.2})
per codificare l'intervento del trascrittore e ottenere che \texttt{aliquot
literas} compaia in tondo inclinato. Cette macro est n\'ecessaire car la
le\c{c}on n'est pas v\'eritablement une le\c{c}on au sens strict, mais une
``Pseudo-Le\c{c}on''\footnote{\`A ce propos, on ne connait pas \`a ce jour
(janvier 2005) d'autres cas que celui de l'exemple, d'utilisation de cette
macro.}.

Si osservi che sia nel caso di \verb"\POSTDEL"\index{\bs{}POSTDEL} che in quello di
\verb"\ANTEDEL"\index{\bs{}ANTEDEL} sembra preferibile usare A piuttosto che A\textsuperscript{1}, dato che {\`e} chiaro, senza bisogno di ulteriori
specificazioni, che si tratta di una
correzione del copista del codice. La stessa osservazione
vale per le macro che ora introdurremo: \verb"\EX{}"\index{\bs{}EX} e \verb"\PC"\index{\bs{}PC}.

%-----------------------------------------------------------------------
\subsubsection{$\backslash$EX; $\backslash$PC}

\label{ref-5.3.1.2}

Si trover{\`a} spesso il caso in cui
il copista, subito dopo aver scritto, ad esempio, ``primum''
si accorga di aver sbagliato e, scrivendo sopra la parola,
la corregga
in ``pristinum''. In tal caso scriveremo:

%\stelle
%pristinum\textsuperscript{1}
%\hrule width 1cm
%\textsuperscript{1}\qquad {\rmnot pristinum} {\slnot ex} {\rmnot
%primum} {\slnot A}
%\stelle

\begin{maurotex}
\VV{
   {A+:\EX{primum}:pristinum}
   }
\end{maurotex}

\noindent dove ``\textsl{ex} primum \textsl{A}'' significa ``A [ha
corretto] da \textit{primum}''. Per ottenere questa scrittura
useremo la nuova macro \verb"\EX{}"\index{\bs{}EX{}}:

\begin{quote}
\begin{verbatim}
\VV{
   {A+:\EX{primum}:pristinum}
   }
\end{verbatim}
\end{quote}

\noindent che ha come argomento la lezione cancellata.
La lezione corretta si inserisce nel terzo sottocampo, come al
solito.

Potrebbe per{\`o} darsi che la lezione originaria non sia pi{\'u}
leggibile; per segnalarlo scriveremo allora

%\stelle
%pristinum\textsuperscript{1}
%\hrule width 1cm
%\textsuperscript{1}\qquad {\rmnot pristinum} {\slnot post corr. A}
%\stelle

\begin{maurotex}
\VV{
   {A+:\PC:pristinum}
   }
\end{maurotex}

\noindent dove ``post corr.'' sta per ``post correctionem''
e significa ``dopo aver corretto'', dove {\`e} sottinteso ``da
qualcosa non pi{\'u} leggibile''. Questa scrittura si otterr{\`a}
con la macro \verb"\PC"\index{\bs{}PC}

\begin{quote}
\begin{verbatim}
\VV{
   {A+:\PC:pristinum}
   }
\end{verbatim}
\end{quote}

Si noti che \verb"\PC"\index{\bs{}PC} \textbf{non ha argomenti}, a
differenza di \verb"\ANTEDEL"\index{\bs{}ANTEDEL}, \verb"\POSTDEL"\index{\bs{}POSTDEL} e di \verb"\EX"\index{\bs{}EX}: la lezione
corretta dal copista viene inserita nel terzo sottocampo e non
c'{\`e} bisogno di fare altro, dato che la lezione eliminata non
risulta pi{\'u} leggibile. Naturalmente, in presenza di pi{\`u}
testimoni, le macro \verb"\EX"\index{\bs{}EX} e \verb"\PC"\index{\bs{}PC} potranno essere combinate
con altre oppure
potranno presentare un sottocampo vuoto. Per esempio

\begin{quote}
\begin{verbatim}
\VV{
   {A+:\EX{primum}:pristinum}
   {B/C:\INTERL:priorem}
   }
\end{verbatim}
\end{quote}

oppure:

\begin{quote}
\begin{verbatim}
\VV{
   {C:pristinum}
   {A+:\PC:}
   {B:primum}
   }
\end{verbatim}
\end{quote}

%-----------------------------------------------------------------------
\subsubsection{Correzioni \textit{in scribendo}}

\label{ref-5.3.1.3}
\index{correzioni in scribendo}

Le quattro macro introdotte in questo
\S\,\ref{ref-5.3.1} devono essere valutate con criterio dall'editore.
Spesso questo tipo di correzioni \textit{in scribendo} non sono
significative, perch{\'e} dovute a banali errori di copiatura o
\textit{lapsus} del copista (che, ripetiamo, pu{\`o} essere il
medesimo Maurolico). A livello di prima
trascrizione andranno riportate tutte nella forma qui
esposta, utilizzando cio{\`e} \verb"\VV"\index{\bs{}VV}. L'editore dovr{\`a}
decidere se accoglierle o meno nell'apparato di TC, valutando
se tali correzioni \textit{in scribendo} tradiscono
un'effettiva indecisione sul testo: valutando cio{\`e} se la lezione
originaria darebbe un senso accettabile, se corrisponde alla
presenza di varianti in altri testimoni, se~---~essendo
autografa e
piuttosto lunga~---~indica o meno un ripensamento di
Maurolico, ecc. Le correzioni che non passino questo esame e
vengano considerate da respingere per la costruzione del
testo critico saranno eliminate dalla stampa di TC
cambiando, come al solito, \verb"\VV"\index{\bs{}VV} in \verb"\VB"\index{\bs{}VB} (cfr. \S\,\ref{ref-4.2.2}).

%-----------------------------------------------------------------------
\subsection{Correzioni e varianti
interlineari e marginali di A\textsuperscript{1}, A\textsuperscript{m}, A\textsuperscript{2},
ecc.}

\label{ref-5.3.2}
\index{correzioni e varianti interlineari e marginali}
\index{correzioni interlineari}
\index{varianti interlineari}
\index{correzioni marginali}
\index{varianti marginali}

Le correzioni e le varianti interlineari e marginali
apportate da una mano diversa da quella di A (A\textsuperscript{1}
compresa) possono essere trattate tutte allo stesso modo. In
pratica si tratta di immaginare A\textsuperscript{1}, A\textsuperscript{m}, A\textsuperscript{2}
\textit{come se} fossero testimoni distinti da A stesso e
trattare quindi le loro  lezioni secondo i
criteri stabiliti nella prima parte di questo manuale di
trascrizione. Chiariamo il concetto con alcuni esempi.

%-----------------------------------------------------------------------
\subsubsection{Caso uno}

\label{ref-5.3.2.1}

Immaginiamo che il copista di A abbia scritto 
``primum'', ma
successivamente  sia intervenuto a correggere il testo (ma
lo stesso discorso varrebbe se fosse intervenuto a
correggere A\textsuperscript{m}, A\textsuperscript{2}, A\textsuperscript{3}\dots) .
Nell'interlinea, sopra la parola ``primum'' ha scritto
infatti la variante alternativa ``secundum''. TC si
presenter{\`a} allora come segue:

%\stelle
%secundum\textsuperscript{1}
%\hrule width 1cm
%\textsuperscript{1}\qquad {\rmnot secundum} {\slnot in interl. A$^{\hbox{\slnotmm
%1}}$} {\rmnot primum} {\slnot A}
%\stelle

\begin{maurotex}
\VV{
   {A1:\INTERL:secundum}
   {A:primum}
   }
\end{maurotex}

\noindent usando \verb"\VV"\index{\bs{}VV} nella forma consueta:

\begin{quote}
\begin{verbatim}
\VV{
   {A1:\INTERL:secundum}
   {A:primum}
   }
\end{verbatim}
\end{quote}

\noindent con tutte le solite norme che regolano l'uso di
\verb"\VV"\index{\bs{}VV}. Per esempio, se volessimo accogliere in TC la lezione
di A invece che quella di A\textsuperscript{1}, basterebbe scrivere
\verb"\VV{{A:primum}{A1:\INTERL:secundum}}"\index{\bs{}VV}\index{\bs{}INTERL}, scambiando l'ordine dei due
campi.

Si osservi che qui, a differenza dei casi precedenti, l'uso
di A\textsuperscript{1} {\`e} essenziale. Il copista di A, infatti non solo
interviene sul testo, non solo interviene \textit{per
correggere} s{\'e} stesso, ma la sua correzione viene a
costituire un vero e proprio testo alternativo a quello di
A: quasi come se si trattasse di un nuovo testimone.
Perci{\`o}, se sar{\`a} il caso, utilizzando la macro \verb"\DES{}"\index{\bs{}DES} o le
macro \verb"\MARG"\index{\bs{}MARG}, \verb"\MARGSIGN"\index{\bs{}MARGSIGN} e \verb"\INTERL"\index{\bs{}INTERL} si pu{\`o} dar conto
precisamente di cosa sia effettivamente successo: se la
correzione si trova in margine, se {\`e} scritta in interlinea,
se {\`e} scritta con inchiostro diverso, o altro ancora.

%-----------------------------------------------------------------------

\subsubsection{Caso due}

\label{ref-5.3.2.2}

Consideriamo un caso apparentemente
pi{\'u} complesso: la tradizione {\`e} costituita da tre testimoni
A, B, C; B\textsuperscript{2} ha depennato dal testo di B la lezione ``et
secundae'' che seguiva immediatamente ``est aequalis
primae''. A e C portano invece la sola lezione ``primae''
senza interventi del copista o di altre mani. In TC noi
accogliamo ``est aequalis primae'', respingendo la lezione
``est aequalis primae et secundae''. Vorremo allora
scrivere:

%\stelle
%est aequalis primae\textsuperscript{1}
%\hrule width 1cm
%\textsuperscript{1}\qquad {\rmnot primae} {\slnot A B$^{\slnot
%2}$ C} {\rmnot primae et secundae} {\slnot B}
%\stelle

\begin{maurotex}
est aequalis \VV{
                {A/B2/C:primae}
                {B:primae et secundae}
                }
\end{maurotex}

\noindent e useremo \verb"\VV"\index{\bs{}VV} come se avessimo a che fare con
quattro testimoni diversi:

\begin{quote}
\begin{verbatim}
est aequalis \VV{
                {A/B2/C:primae}
                {B:primae et secundae}
                }
\end{verbatim}
\end{quote}

%-----------------------------------------------------------------------
\subsubsection{Caso tre}

\label{ref-5.3.2.3}

Consideriamo lo stesso caso di prima, 
solo che questa volta
B\textsuperscript{2}  non solo ha
depennato dal testo di B la lezione ``et secundae'' che
seguiva immediatamente ``primae'', ma ha anche aggiunto in
margine, apponendo un segno di richiamo, la lezione ``per 5
quinti Euclidis''. A e C portano invece la sola lezione
``primae'' senza interventi del copista o di altre mani.
Dopo un attento esame, l'editore stabilisce che il
correttore cui appartiene la mano B\textsuperscript{2} ha
ripristinato il
testo originale di Maurolico modificato invece da ABC.
Scriver{\`a} allora

%\stelle
%est aequalis primae\textsuperscript{1}, per 5$^{\rm am}$ 2$^{\rm di}$ %Euclidis\textsuperscript{2},
%ergo~\dots
%\hrule width 1cm
%\textsuperscript{1}\qquad {\rmnot primae} {\slnot A
%B$^{\hbox{\slnotmm 2}}$ C}
%{\rmnot primae et secundae} {\slnot B}
%\textsuperscript{2}\qquad {\rmnot per 5$^{\rm am}$ 2$^{\rm di}$ Euclidis}
%{\slnot om.
%A B C signo posito in marg. add. B$^{\hbox{\slnotmm 2}}$} 
%\stelle

\begin{maurotex}
est aequalis \VV{
                {A/B2/C:primae}
                {B:primae et secundae}
                }, \VV{
                      {*:per 5\Sup{am} 2\Sup{di} Euclidis}
                      {A/B/C:\OM}
                      {B2:\MARGSIGN\DES{add.}:}
                      }, ergo ...
\end{maurotex}

\noindent usando \verb"\VV"\index{\bs{}VV} e \verb"\MARGSIGN"\index{\bs{}MARGSIGN}:

\begin{quote}
\begin{verbatim}
est aequalis \VV{
                {A/B2/C:primae}
                {B:primae et secundae}
                }, \VV{
                      {*:per 5\Sup{am} 2\Sup{di} Euclidis}
                      {A/B/C:\OM}
                      {B2:\MARGSIGN\DES{add.}:}
                      }, ergo ...
\end{verbatim}
\end{quote}

\noindent e naturalmente la stessa cosa si potr{\`a} fare in
caso di varianti interlineari (usando \verb"\INTERL"\index{\bs{}INTERL}) o di aggiunte
marginali senza segno di richiamo (\verb"\MARG"\index{\bs{}MARG}).

%-----------------------------------------------------------------------

\subsubsection{Caso 4}

\label{ref-5.3.2.4} Potrebbe accadere
che dopo la correzione di un'altra mano la lezione
originaria non risulti pi{\'u} decifrabile. Supponiamo ad
esempio di avere due testimoni, A e B. Il primo porta la
lezione ``Archimedes''; in quel luogo B risulta illeggibile
perch{\'e} il correttore (B\textsuperscript{2}) ha cancellato la lezione originaria di B
scrivendo ``Euclides''. L'editore sceglie la correzione di
B\textsuperscript{2} e scrive:

%\stelle
%EuclideS\textsuperscript{1}
%\hrule width 1cm
%\textsuperscript{1}\qquad {\rmnot Euclides} {\slnot B$^{\hbox{\slnotmm
%2}}$;} {\rmnot Archimedes} {\slnot A non legitur B} 
%\stelle

\begin{maurotex}
\VV{
   {B2:Euclides}
   {A:Archimedes}
   {B:\NL}
   }
\end{maurotex}

\noindent usando \verb"\VV"\index{\bs{}VV} e la macro \verb"\NL"\index{\bs{}NL} che
abbiamo introdotto nel \S\,\ref{ref-4.4.1} proprio per dar conto di
piccole lacune materiali del testimone:

\begin{quote}
\begin{verbatim}
\VV{
   {B2:Euclides}
   {A:Archimedes}
   {B:\NL}
   }
\end{verbatim}
\end{quote}

%-----------------------------------------------------------------------
\subsection{E se la correzione {\`e} molto lunga? E se la
mano 2 effettua una trasposizione?}

\label{ref-5.3.3}

Anche per le correzioni di cui si {\`e} trattato nel \S\,\ref{ref-5.3.2},
come per le integrazioni del copista, potr{\`a} darsi il caso
che abbiano una notevole estensione e sia sconsigliabile far
riportare in nota l'intero passo. Ne trattiamo, come gi{\`a}
avvertito, nel capitolo \ref{ref-7}. L'editore o il trascrittore che
si trovassero ad avere a che fare con loro consultino in
particolare il \S\,\ref{ref-7.2.2}.

Fra le correzioni del copista o di altre mani ci si pu{\`o} ovviamente
imbattere nel caso in cui sia stato trasposta una pare di testo. Le
trasposizioni, tuttavia,  sono un'argomento (spinoso) cui fin qui abbiamo
appena accennato. Chi avesse la sfortuna di doverci avere a
che fare pu{\`o} leggere il
capitolo \ref{ref-8}.

%-----------------------------------------------------------------------
\section{Marginalia}

\label{ref-5.4}

\subsection{$\protect\backslash$NOTAMARG: Parente, non figlia}
\index{marginalia}

\label{ref-5.4.1}

Col nome generico \textit{marginalia} intendiamo note del tipo
``vide quod dicit Martianus de astrologia'' o ``pulcherrima
est haec contemplatio'' che non mirano a
correggere o integrare il testo, ma solo a spiegarlo o
commentarlo. Esse
non costituiscono propriamente delle varianti testuali, e
quindi non possono venire accolte in TC. Devono essere
per{\`o} segnalate. Potranno trovarsi in margine, o scritte in
inchiostro di diverso colore, o (nel caso di un testimone a
stampa) in caratteri diversi da quello del testo, per
esempio in corsivo\footnote{Lo stesso discorso vale per
gli eventuali interventi riportati sul testo di un testimone
da personaggi diversi da Maurolico stesso con il fine di
correggerne errori concettuali, riempire silenzi, commentare
le procedure mauroliciane, indicare riferimenti ad altre
opere ecc. Si tratta di riscritture dell'opera che, dal
punto di vista filologico, nulla hanno a che vedere con il
testo da pubblicare e che andranno segnalate solo in apparato
o, eventualmente, in apposite appendici all'edizione. Essi
vengono trattati con lo stesso sistema descritto qui,
ovviamente indicando con un \textit{siglum} opportuno la mano
che li ha eseguiti (A\textsuperscript{2}, A\textsuperscript{3}, ecc.).}.

Ad
esempio, supponendo che Maurolico, parlando
dell'importanza
della trigonometria sferica annoti in margine ``Vide etiam
quod dicit Martianus de astrologia'' nel testimone A, mentre
B e C non recano tale annotazione, si agir{\`a} in questo modo:

%\stelle
%Et talis est praestantia sphaericorum doctrinA\textsuperscript{1}. Sed
%ad definitiones accedamus.
%\hrule width 1cm
%\textsuperscript{1}\qquad{\rmnot Vide etiam
%quod dicit Martianus de astrologia} {\slnot in marg.}
%{\slnot A$^{\hbox{\slnotmm m}}$}
%\stelle

\begin{maurotex}
Et talis est praestantia sphaericorum
doctrina\NOTAMARG{
                 {Am:\MARG:Vide etiam quod dicit
                  Martianus de astrologia.}
                 }. Sed ad definitiones accedamus.
\end{maurotex}

\noindent utilizzando la macro \verb"\NOTAMARG"\index{\bs{}NOTAMARG}:

\begin{quote}
\begin{verbatim}
Et talis est praestantia sphaericorum
doctrina\NOTAMARG{
                 {Am:\MARG:Vide etiam quod dicit
                  Martianus de astrologia.}
                 }. Sed ad definitiones accedamus.
\end{verbatim}
\end{quote}

Come si vede, \verb"\NOTAMARG"\index{\bs{}NOTAMARG} ha una sintassi molto simile a
quella  di \verb"\VV"\index{\bs{}VV}:
all'interno del suo campo~---~suddiviso come quelli di \verb"\VV"\index{\bs{}VV} in
tre sottocampi~---~si inserisce il \textit{siglum} del testimone
(primo sottocampo);
eventuali informazioni, ad esempio su dove si trovi la glossa (secondo
sottocampo); la lezione (terzo
sottocampo). I tre sottocampi sono come al solito separati
dai \verb":". Nel
secondo sottocampo si inseriranno le indicazioni del caso: se
la glossa si trova in margine, in interlinea etc. si
potranno utilizzare le macro introdotte in questo capitolo;
per altre situazioni si ha comunque a disposizione la macro
\verb"\DES{}"\index{\bs{}DES} che permette di descrivere ci{\`o} che si vuole. Cos{\'\i},
se la glossa dell'esempio precedente fosse stata scritta in
inchiostro rosso si potr{\`a} battere:

\begin{quote}
\begin{verbatim}
... doctrina\NOTAMARG{
                     {Am:\DES{rubro atramento}:Vide etiam
                      quod dicit Martianus de astrologia.}
                     }. Sed ad definitiones accedamus.
\end{verbatim}
\end{quote}

Si tratter{\`a} poi di scegliere, a seconda dei casi (numero
delle glosse, loro importanza, ecc.) se dare le note
marginali in apparato o in un'apposita appendice. A questo
scopo occorrer{\`a} marcare il testo che si riferisce a una o
a un gruppo di note marginali in modo che possa essere
isolato insieme alle sue glosse e trattato poi in modo
opportuno. A ci{\`o} provvedono le macro \verb"\BeginNM"\index{\bs{}BeginNM} e \verb"\EndNM"\index{\bs{}EndNM}.
Nell'esempio di prima si dovr{\`a} quindi scrivere:

\begin{quote}
\begin{verbatim}
\BeginNM
Et talis est praestantia sphaericorum
doctrina\NOTAMARG{
                 {Am:\MARG:Vide etiam quod dicit
                  Martianus de astrologia.}
                 }.
\EndNM
Sed ad definitiones accedamus.
\end{verbatim}
\end{quote}

Si potr{\`a} poi decidere se si vuole che le note marginali
vengano stampate in appendice o in apparato scrivendo,
nel preambolo del file (prima cio{\`e} di \verb"\begin{document}"\index{\bs{}begin\{document\}}) una
di queste due macro:

\begin{quote}
\begin{verbatim}
\MarginaliaInNota
\MarginaliaInAppendice
\end{verbatim}
\end{quote}

%-----------------------------------------------------------------------
\subsection{Un'osservazione sulla punteggiatura}

\label{ref-5.4.2}
\index{punteggiatura}

Il nostro sagace lettore si sar{\`a} forse accorto che negli
esempi qui sopra riportati la macro \verb"\NOTAMARG"\index{\bs{}NOTAMARG} {\`e} stata
attaccata all'ultima parola del testo e prima del segno di
punteggiatura:

\begin{quote}
\begin{verbatim}
... doctrina\NOTAMARG{ ...  .}.
\end{verbatim}
\end{quote}

Nel caso di \verb"\VV"\index{\bs{}VV}, invece viene lasciato uno spazio bianco
fra l'ultima parola non coinvolta nella variante e l'inzio
della macro. Ci{\`o} dipende dal fatto che \verb"\VV"\index{\bs{}VV} automaticamente
provvede ad apporre l'esponente di nota attaccato all'ultima
parola del testo che compare nel terzo sottocampo del suo
primo campo, testo che~---~seconde le regole descritte nel
\S\,\ref{ref-4.2.2}
sar{\`a} quello stampato in TC. Il testo di \verb"\NOTAMARG"\index{\bs{}NOTAMARG} invece, o
va in nota o va a finire in un'appendice.

Morale: fate attenzione a dove mettete le vostre macro,
controllate frequentemente l'\textit{output}. E cercate di
evitare errori nella punteggiatura. {\`E} difficile scovarli
tutti, {\`e} difficile (se non impossibile) scrivere programmi
che possano correggere automaticamente. E un testo con la
punteggiatura sbagliata fa una pessima impressione. Per cui
cercate di immaginare cosa succeder{\`a} alla punteggiatura
quando introducete una macro nel testo e gli spazi bianchi
dentro  i suoi campi e nelle
immediate vicinanze.

%-----------------------------------------------------------------------
\subsection{Marginalia con varianti}

\label{ref-5.4.3}
\index{marginalia con varianti}

Con le glosse marginali ci si potrebbe trovare di
fronte a situazioni pi{\'u} complesse. Potrebbe infatti avvenire
che tali glosse siano state riportate non solo da uno, ma da
pi{\'u} testimoni. In tal caso occorrer{\`a} dar conto di eventuali
varianti fra i testimoni. Supponiamo ad esempio che la
situazione sia la seguente

\begin{itemize}
\item[A:] Vide etiam quod dicit Martianus de astrologia
(\textit{in margine, di mano di Maurolico})

\item[B:] Vide quod dicit Capella de astrologia (\textit{in
margine, di  mano diversa da quella di Maurolico})

\end{itemize}

L'editore naturalmente sceglie come TC dell'annotazione
marginale il testo autografo, ma vuole comunque dar conto
delle varianti di B. Tale situazione andr{\`a} cos{\'\i} descritta al {\mtex}:

\begin{quote}
\begin{verbatim}
\BeginNM
Et talis est praestantia sphaericorum
doctrina\NOTAMARG{
                 {Am/B:\MARG:Vide \VV{
                                     {*:etiam}
                                     {B:\OM}
                                     } quod dicit
                   \VV{
                      {Am:Martianus}
                      {B:Capella}
                      } de astrologia.}
                 }.
\EndNM
Sed ad definitiones accedamus.
\end{verbatim}
\end{quote}

Come si vede dall'esempio, {\`e} possibile innestare \verb"\VV"\index{\bs{}VV}
all'interno di \verb"\NOTAMARG"\index{\bs{}NOTAMARG} e dar conto delle varianti. Nel
caso si scelga di dare i \textit{marginalia} in apparato il
risultato sar{\`a} il seguente:

%\stelle
%Et talis est praestantia sphaericorum doctrinA\textsuperscript{1}. Sed
%ad definitiones accedamus.
%\hrule width 1cm
%\textsuperscript{1}\qquad{\rmnot Vide etiam {\slnot(}etiam {\slnot
%om. B)}
%quod dicit Martianus {\slnot(}Capella {\slnot B)} de astrologia} {\slnot in %marg.}
%{\slnot A$^{\hbox{\slnotmm m}}$ B}
%\stelle

\begin{maurotex}
\BeginNM
Et talis est praestantia sphaericorum
doctrina\NOTAMARG{
                 {Am/B:\MARG:Vide \VV{
                                     {*:etiam}
                                     {B:\OM}
                                     } quod dicit
                   \VV{
                      {Am:Martianus}
                      {B:Capella}
                      } de astrologia.}
                 }.
\EndNM
Sed ad definitiones accedamus.
\end{maurotex}

Si noti che nell'esempio sopra riportato la prima
\verb"\VV"\index{\bs{}VV} {\`e} un'omissione e quindi, per chiarezza, {\`e} bene riportare
fra parentesi la lezione che B omette. La seconda invece
{\`e} una variante  testuale ``vera'' e si {\`e} ritenuto
sufficientemente chiaro riportare fra parentesi solo la
lezione di B (apparato negativo). Corrispondentemente la
codifica delle due \verb"\VV"\index{\bs{}VV} {\`e} stata:

\begin{quote}
\begin{verbatim}
\VV{
   {*:etiam}
   {B:\OM}
   }
\end{verbatim}
\end{quote}

\noindent nel primo caso, mentre nel secondo:

\begin{quote}
\begin{verbatim}
\VV{
   {Am:Martianus}
   {B:Capella}
   }
\end{verbatim}
\end{quote}

La regola per codificare questi due casi {\`e} la
seguente:

\begin{itemize}

\item Se si vuole che la lezione principale venga
riportata anche tra parentesi (primo caso) occorre non
riportare nel primo campo di \verb"\VV"\index{\bs{}VV} il \textit{siglum} ma
inserire un asterisco (\verb"*"); se si vuole che la lezione
principale \textit{non} venga riportata (secondo caso) occorre
inserire il \textit{siglum} del testimone nel primo campo.

\end{itemize}

Come si vedr{\`a} nel capitolo \ref{ref-7} (\S\,\ref{ref-7.1.2.3}) questa regola per
la costruzione di un ``sub-apparato'' (inserito all'interno di quello
principale fra  parentesi tonde) verr{\`a} utilizzata anche per trattare le
varianti lunghe.

Se invece si decide di dare il testo del \textit{marginale} in appendice, la codificazione per mezzo di \verb"\VV"\index{\bs{}VV}
annidata dentro \verb"\NOTAMARG"\index{\bs{}NOTAMARG} produrr{\`a} un appparato normale,
come se si stesse facendo l'edizione di un altro testo.

%-----------------------------------------------------------------------
\subsection{Un esempio interessante (in cui si impara
anche come trattare gli \textsl{errata corrige})}

\label{ref-5.4.4}
\index{errata corrige}

Si osservi che la regola sopra enunciata pu{\`o} essere utilizzata anche
nel caso che il \textit{marginale}, invece di trovarsi in diversi
testimoni, sia stato corretto o integrato da mani diverse dello stesso
testimone. Si consideri ad esempio questo caso. Il testo di Maurolico,
tr{\`a}dito da un unico testimone a stampa (S), ha:

\begin{itemize}

\item[S:] Sed hi radii densiores plures sunt quam radii sub angulo
$BGD$ comprehensi.

\end{itemize}

In S si trova tuttavia un'annotazione marginale di Clavio,
collocata subito dopo ``densiores'' che spiega perch{\'e} tali raggi
dovrebbero essere pi{\'u} densi:

\begin{quote}
quia minus spatium \textit{nuncupant}
\end{quote}

Tale annotazione claviana contiene un errore, debitamente
corretto dalla tabella degli \textit{errata} di S (che indicheremo con
S\textsuperscript{1}, dato che {\`e} come se si trattasse del copista (in questo caso il
tipografo) che corregge s{\'e} stesso):

\begin{itemize}

\item[S\textsuperscript{1}:] nuncupant; \textit{occupant} 

\end{itemize}

Si vorr{\`a} allora ottenere un apparato di questo genere:

%\stelle
%Sed hi radii densioreS\textsuperscript{1} plures sunt quam radii sub angulo
%$BGD$ comprehensi.
%\hrule width 1cm
%\textsuperscript{1}\qquad{\rmnot quia minus spatium occupant
%{\slnot (}occupant{\slnot:} occupant {\slnot S$^{\hbox{\slnotmm 1}}$}
%nuncupant {\slnot S) Clavius in S}}
%\stelle

\begin{maurotex}
\BeginNM 
Sed hi radii
densiores\NOTAMARG{
                  {S:\DES{Clavius in}:quia 
                   minus spatium \VV{
                                    {*:occupant}
                                    {S1:occupant}
                                    {S:nuncupant}
                                    }
                  }
                  } plures sunt quam 
radii sub angulo $BGD$ comprehensi.
\EndNM
\end{maurotex}

Si osservi che in un caso come questo {\`e} necessario avere un
subapparato positivo, con ripetizione della lezione di S e di S\textsuperscript{1}
fra parentesi. Sfruttando la regola enunciata qui sopra, si potr{\`a}
ottenere questo risultato scrivendo:

\begin{quote}
\begin{verbatim}
Sed hi radii
densiores\NOTAMARG{
                  {S:\DES{Clavius in}:quia 
                   minus spatium \VV{
                                    {*:occupant}
                                    {S1:occupant}
                                    {S:nuncupant}
                                    }
                  }
                  } plures sunt quam 
radii sub angulo $BGD$ comprehensi.
\end{verbatim}
\end{quote}

%-----------------------------------------------------------------------
\section[Macros de simplification]{Macros de simplification dans le cas d'un t\'emoin unique}

Lorsque le transcripteur n'a \`a faire qu'\`a \textbf{un t\'emoin unique},
il peut trouver fastidieux d'\'ecrire constamment des
\verb"\VV{{}}"\index{\bs{}VV{{}}}. Pour \'eviter cette fatigue, le {\mtex}
inclut quelques macros qui rendent sa t\^ache plus simple. Prenons
l'exemple d'une \emph{marginalia}, tel l'exemple canonique de
\S\,\ref{ref-5.2.2.1}. Si l'on voulait obtenir:

\begin{maurotex}
quadratum $ab$ aequale erit duobus quadratis
\(ab\) et \(bc\) \VV{
                    {A+:\MARG:per 47 primi}
                    }
\end{maurotex}

on utilisait en {\mtex} la macro \verb"\VV"\index{\bs{}VV} et sa commande \verb"\MARG"\index{\bs{}MARG}:

\begin{quote}
\begin{verbatim}
quadratum $ab$ aequale erit duobus quadratis
\(ab\) et \(bc\) \VV{
                    {A+:\MARG:per 47 primi}
                    }
\end{verbatim}
\end{quote}

Une macro de simplification ---~dans le cas d'un \textbf{t\'emoin unique},
rappelons-le ~--- donnera \textbf{exactement} le m\^eme r\'esultat, tout en
\'etant plus \'economique. Elle s'appelle \verb"\Smarg"\index{\bs{}Smarg} et
s'utilise de la mani\`ere suivante:

\begin{quote}
\begin{verbatim}
quadratum $ab$ aequale erit duobus quadratis
\(ab\) et \(bc\) \Smarg{per 47 primi}
\end{verbatim}
\end{quote}

Le transcripteur a donc \'economis\'e 8 signes (!) et surtout gagn\'e un
code source plus simple et plus clair.

Les macros de simplification  n\'ecessitent une information
suppl\'ementaire: le nom du t\'emoin sur lequel le transcripteur travaille.
Cette information n'est toutefois fournie qu'une seule fois pour tout le
document. On utilise pour cela la macro \verb"\NomeTestimone"\index{\bs{}NomeTestimone} dont l'argument est le nom du t\'emoin et que l'on place au
d\'ebut du document. Par exemple, si le t\'emoin est nomm\'e \texttt{A}, le
d\'ebut du document ressemblera \`a:

\begin{quote}
\begin{verbatim}
\begin{document}
\htmlcut
\NomeTestimone{A}
...
\end{verbatim}
\end{quote}

Une fois cette formalit\'e accomplie, on dispose d'une s\'erie de macros
suppl\'ementaire. Toutes commencent par un "S" majuscule et reprennent
pour la plupart la commande dont elles sont cousines en minuscules:

\begin{quote}
\begin{verbatim}
\Smarg
\Smargsign
\Sinterl
\Spc
\Sbis
\Santedel
\Spostdel
\Sex
\Smargcorr
\Smargsigncorr
\Sintcorr
\end{verbatim}
\end{quote}

Voici les codes sources et les r\'esultats obtenus. Les cinq premi\`eres
macros ne prennent qu'un seul argument: la le\c{c}on du texte critique.

%--------------------------------------------------------------------
\begin{itemize}

\item \verb"\Smarg"\index{\bs{}Smarg} pour \verb"\MARG"\index{\bs{}MARG}:

\begin{quote}
\begin{verbatim}
quadratum $ab$ aequale erit duobus quadratis
\(ab\) et \(bc\) \Smarg{per 47 primi}
\end{verbatim}
\end{quote}

\begin{maurotex}
quadratum $ab$ aequale erit duobus quadratis
\(ab\) et \(bc\) \Smarg{per 47 primi}
\end{maurotex}

%--------------------------------------------------------------------
\item \verb"\Smargsign"\index{\bs{}Smargsign} pour \verb"\MARGSIGN"\index{\bs{}MARGSIGN}:

%\begin{maurotex}
%quadratum $ab$
%aequale erit, \VV{
%                 {A+:\MARGSIGN:per 47 primi}
%                 }, duobus
%quadratis $ac$ et $bc$
%\end{maurotex}

\begin{quote}
\begin{verbatim}
quadratum $ab$
aequale erit, \Smargsign{per 47 primi}, duobus
quadratis $ac$ et $bc$
\end{verbatim}
\end{quote}

\begin{maurotex}
quadratum $ab$
aequale erit, \Smargsign{per 47 primi}, duobus
quadratis $ac$ et $bc$
\end{maurotex}

%--------------------------------------------------------------------
\item \verb"\Sinterl"\index{\bs{}Sinterl} pour \verb"\INTERL"\index{\bs{}INTERL}:

%\begin{maurotex}
%primum \VV{
%          {A+:\INTERL:et secundum}
%          }
%\end{maurotex}

\begin{quote}
\begin{verbatim}
primum \Sinterl{et secundum}
\end{verbatim}
\end{quote}

\begin{maurotex}
primum \Sinterl{et secundum}
\end{maurotex}

%--------------------------------------------------------------------
\item \verb"\Spc"\index{\bs{}Spc} pour \verb"\PC"\index{\bs{}PC}:

%\begin{maurotex}
%\VV{
%   {A+:\PC:pristinum}
%   }
%\end{maurotex}

\begin{quote}
\begin{verbatim}
\Spc{pristinum}
\end{verbatim}
\end{quote}

\begin{maurotex}
\Spc{pristinum}
\end{maurotex}

%--------------------------------------------------------------------
\item \verb"\Sbis"\index{\bs{}Sbis} pour \verb"\BIS"\index{\bs{}BIS}:

%\begin{maurotex}
%erunt igitur quatuor \VV{
%                        {A:\BIS:triangula}
%                        } maius quam dimidio portionum.
%\end{maurotex}

\begin{quote}
\begin{verbatim}
erunt igitur quatuor \Sbis{triangula}
maius quam dimidio portionum.
\end{verbatim}
\end{quote}

\begin{maurotex}
erunt igitur quatuor \Sbis{triangula}
maius quam dimidio portionum.
\end{maurotex}

\end{itemize}

Les trois suivantes prennent deux arguments: le deuxi\`eme est le texte
critique qui corrige le premier.

%--------------------------------------------------------------------
\begin{itemize}

\item \verb"\Santedel"\index{\bs{}Santedel} pour \verb"\ANTEDEL"\index{\bs{}ANTEDEL}:

%\begin{maurotex}
%triangulum \VV{
%              {A+:\ANTEDEL{primum}:secundum}
%              } erit
%
%\end{maurotex}

\begin{quote}
\begin{verbatim}
triangulum \Santedel{primum}{secundum} erit
\end{verbatim}
\end{quote}

\begin{maurotex}
triangulum \Santedel{primum}{secundum} erit
\end{maurotex}

%--------------------------------------------------------------------
\item \verb"\Spostdel"\index{\bs{}Spostdel} pour \verb"\POSTDEL"\index{\bs{}POSTDEL}:

%\begin{maurotex}
%\VV{
%   {A+:\POSTDEL{primum}:triangulum}
%   } secundum erit
%\end{maurotex}

\begin{quote}
\begin{verbatim}
\Spostdel{primum}{triangulum} secundum erit
\end{verbatim}
\end{quote}

\begin{maurotex}
\Spostdel{primum}{triangulum} secundum erit
\end{maurotex}

%--------------------------------------------------------------------
\item \verb"\Sex"\index{\bs{}Sex} pour \verb"\EX"\index{\bs{}EX}:

%\begin{maurotex}
%\VV{
%   {A+:\EX{primum}:pristinum}
%   }
%\end{maurotex}

\begin{quote}
\begin{verbatim}
\Sex{primum}{pristinum}
\end{verbatim}
\end{quote}

\begin{maurotex}
\Sex{primum}{pristinum}
\end{maurotex}

\end{itemize}

%--------------------------------------------------------------------

Enfin, on ajoute les trois commandes suivantes, qui cr\'eent simplement des
notes plus longues en indiquant une correction anterieure. Le premier
argument est le mot corrig\'e par le second, en interligne, en marge, en
marge avec un signe de rappel.

%--------------------------------------------------------------------
\begin{itemize}

\item \verb"\Sintcorr"\index{\bs{}Sintcorr}:

\begin{quote}
\begin{verbatim}
Tota $ab$ media. Item $a$ media. Erit $b$ 
\Sintcorr{media}{irrationalis}.
\end{verbatim}
\end{quote}

\begin{maurotex}
Tota $ab$ media. Item $a$ media. Erit $b$ 
\Sintcorr{media}{irrationalis}.
\end{maurotex}

%--------------------------------------------------------------------
\item \verb"\Smargcorr"\index{\bs{}Smargcorr}:

\begin{quote}
\begin{verbatim}
Tota $ab$ media. Item $a$ media. Erit $b$ 
\Smargcorr{media}{irrationalis}.
\end{verbatim}
\end{quote}

\begin{maurotex}
Tota $ab$ media. Item $a$ media. Erit $b$ 
\Smargcorr{media}{irrationalis}.
\end{maurotex}

%--------------------------------------------------------------------
\item \verb"\Smargsigncorr"\index{\bs{}Smargsigncorr}:

\begin{quote}
\begin{verbatim}
Tota $ab$ media. Item $a$ media. Erit $b$ 
\Smargsigncorr{media}{irrationalis}.
\end{verbatim}
\end{quote}

\begin{maurotex}
Tota $ab$ media. Item $a$ media. Erit $b$ 
\Smargsigncorr{media}{irrationalis}.
\end{maurotex}

\end{itemize}

%-----------------------------------------------------------------------
\chapter{Congetture}

\label{ref-6}

\section*{Premessa: una macro per l'editore}
\index{congetture}

\label{ref-6}
Nei due capitoli precedenti si {\`e} discusso di come trattare
con l'{\mtex} le diverse lezioni dei testimoni e le aggiunte,
integrazioni, correzioni riscontrabili in un testimone.

Questo capitolo {\`e} riservato invece a presentare il modo
con cui l'editore potr{\`a} intervenire \textit{congetturalmente}. Potr{\`a} infatti avvenire che egli debba
correggere lezioni evidentemente errate fornite
concordemente dalla tradizione del testo, dar conto
dell'esistenza di luoghi del tutto corrotti, integrare
parole o frasi o stabilire sulla base di considerazioni
testuali l'esistenza di lacune.

Tutti questi compiti spettano ovviamente solo all'editore, e
nelle ultime fasi del suo lavoro: quando la sua conoscenza
della tradizione, del contenuto dei testi che sta trattando
e la sua sensibilit{\`a} critica gli permetteranno di effettuare
interventi di questa natura. A tali scopi {\`e} riservata la
macro:

\begin{quote}
\begin{verbatim}
\ED{}
\end{verbatim}
\end{quote}

\noindent la cui sintassi e i possibili usi sono
illustrati in questo capitolo.

%-----------------------------------------------------------------------
\section{Correzioni}

\label{ref-6.1}
\index{correzioni}

L'ambito logico in cui ci poniamo in questo paragrafo {\`e} una
situazione in cui si suppone che tutta la tradizione del
testo~---~cio{\`e} tutti i suoi testimoni~---~si presenti
corrotta per un
qualche motivo e che quindi divergano dal testo critico che
viene accolto. Va inoltre tenuto presente che il testo
critico potr{\`a} essere restituito congetturalmente dal nostro
editore oppure accogliere correzioni,
integrazioni, ecc. fatte in edizioni moderne (ad esempio
Napoli o Clagett) o proposte da altri studiosi in loro
opere, carteggi ecc. (ad esempio Clavio, Borelli, ecc.)

%-----------------------------------------------------------------------
\subsection{Correzioni congetturali dell'editore}

\label{ref-6.1.1}
\index{correzioni congetturali}

Per ``correzione congetturale'' si deve intendere un
intervento dell'editore che colloca in TC una lezione
diversa da quelle tr{\`a}dite da tutti i testimoni, che vengono
ritenute erronee.

Supponiamo, per cominciare, che il passo in questione non
sia mai stato corretto da nessuno in precedenza e che la
situazione sia la seguente: 

\begin{itemize}

\item[A:] a vertice canis demittitur

\item[B:] a vertice cunni demittitur

\item[C:] a vertice comico demittitur

\end{itemize}

Nessuna delle tre lezioni {\`e}  accettabile, e quindi

\begin{quote}
TC congettura ``a vertice coni demittitur''
\end{quote}

In questo caso si vuole ottenere:

%\stelle
%a vertice coni\textsuperscript{1} demittitur
%\hrule width 1cm
%\noindent\textsuperscript{1}\qquad {\rmnot coni} {\slnot  conieci}~~{\rmnot comico} %{\slnot
%C}~~{\rmnot cunni} {\slnot B}~~{\rmnot canis} {\slnot A}
%\stelle

\begin{maurotex}
a vertice \VV{
             {*:\ED{conieci}:coni}
             {C:comico}
             {B:cunni}
             {A:canis}
             }
\end{maurotex}

\noindent dove con ``coni \textsl{conieci\/}'' si intende ``ho interpretato (che si debba leggere) \textsl{coni\/}''. La
relativa  macro {\`e}:

\begin{quote}
\begin{verbatim}
a vertice \VV{
             {*:\ED{conieci}:coni}
             {C:comico}
             {B:cunni}
             {A:canis}
             }
\end{verbatim}
\end{quote}

Si noti che pur essendo i testimoni solo
tre, {\`e} stato necessario introdurre un quarto campo
per inserire il testo critico. Al solito, vale
la regola generale per cui la lezione inserita nel terzo
sottocampo del primo campo {\`e} quella che viene accolta in TC.
Il primo sottocampo, dato che non contiene il \textit{siglum}
di un testimone, {\`e} riempito dalla \verb"*", mentre nel secondo,
destinato alle informazioni ``speciali'', si inserisce la
nuova macro \verb"\ED{"\thinspace\verb"}"\index{\bs{}ED}. Fra le parentesi graffe che seguono
\verb"\ED"\index{\bs{}ED} si inserisce in questo caso l'espressione \verb"conieci";
\verb"\ED"\index{\bs{}ED} {\`e} per{\`o} molto versatile e si adatta a
tutta una serie di situazioni diverse, semplicemente
cambiando l'argomento che si scrive fra le sue parentesi
graffe.

Un esempio ovvio di questa versatilit{\`a} {\`e} quello in cui
l'editore corregge s{\'\i}, ma con un grado di certezza da far
sembrare ridicola l'espressione ``conieci''. Se ad esempio
ci disponesse di un solo testimone che legga ``$2+2=5$''
l'editore scriver{\`a} in TC ``$2+2=4$'' e in apparato sar{\`a} pi{\'u}
opportuno che dica ``$2+2=4$ \textsl{correxi} $2+2=5$ \textit{A}''. Per ottenere ci{\`o} baster{\`a} scrivere:

\begin{quote}
\begin{verbatim}
\VV{
   {*:\ED{correxi}:\(2+2=4\)}
   {A:\(2+2=5\)}
   }
\end{verbatim}
\end{quote}

Nell'argomento di \verb"\ED"\index{\bs{}ED} potranno essere inserite anche
altre informazioni, come vedremo fra breve. Notiamo
intanto che nel caso che la cura di un testo sia affidata a
pi{\'u} di un editore come argomento di \verb"\ED"\index{\bs{}ED} si scriver{\`a}
ovviamente \verb"coniecimus", \verb"correximus". Questa
osservazione
vale anche per tutti gli altri esempi che presenteremo nel
resto di questo capitolo.

Prima di passare a vedere come comportarsi quando il passo
sia gi{\`a} stato corretto da altri, aggiungiamo un esempio che
illustri come la correzione congetturale possa combinarsi
con la registrazione di correzioni del copista o di altre
mani.

Supponiamo, per esempio, che il copista di A abbia scritto
``primum et secundum sunt triangula'', cancellando poi le
parole ``et secundum'' ma lasciando i plurali. In questo
caso, risulter{\`a} forse pi{\'u} chiaro e agevole utilizzare la
macro \verb"\POSTDEL"\index{\bs{}POSTDEL} piuttosto che \verb"\ANTEDEL"\index{\bs{}ANTEDEL} (cfr. \S\,\ref{ref-5.3.1}) e
si potr{\`a} procedere in questo modo:

%\stelle
%primum\textsuperscript{1} est triangulum$^2$
%\hrule width 1cm
%\textsuperscript{1}\qquad {\slnot post} {\rmnot primum} {\slnot
%del.} {\rmnot et secundum} {\slnot A}
%$^2$\qquad {\rmnot est triangulum} {\slnot
%correxi}~~{\rmnot sunt triangula} {\slnot A}
%\stelle

\begin{maurotex}
\VV{
   {A:\POSTDEL{et secundum}:primum}
   } \VV{
        {*:\ED{correxi}:est triangulum}
        {A:sunt triangula}
        }
\end{maurotex}

\noindent dicendo all'{\mtex}:

\begin{quote}
\begin{verbatim}
\VV{
   {A:\POSTDEL{et secundum}:primum}
   } \VV{
        {*:\ED{correxi}:est triangulum}
        {A:sunt triangula}
        }
\end{verbatim}
\end{quote}

%-----------------------------------------------------------------------
\subsection{Quando il testo {\`e} gi{\`a} stato trattato da
altri}

\label{ref-6.1.2}

Se il passo in questione ha subito correzioni o {\`e} stato
fatto oggetto di congetture ad opera di un altro editore o
studioso, bisogner{\`a} darne conto (ma si rifletta sul \S\,\ref{ref-6.1.2}.e qui sotto!).

Supponiamo che i testimoni (nell'esempio due soli) offrano
le seguenti lezioni:

\begin{itemize}

\item[A:] 123

\item[B:] 112

\end{itemize}

\noindent ma che inoltre il passo in questione sia stato
edito da F.~Napoli e M.~Clagett:

\begin{itemize}

\item Napoli ha la lezione ``123'' e non dice
altro.

\item Clagett ha la lezione ``132'' e \textit{afferma esplicitamente} che 123 e 112 sono erronee

\end{itemize}

Possono ovviamente darsi tutta una serie di casi: si accetta
A come Napoli, si congettura ``132'' accettando Clagett e
via di seguito. Cercheremo di chiarire la cosa con qualche
esempio.

%-----------------------------------------------------------------------
\subsubsection{Caso a}

\label{ref-6.1.2.1}

Si accetta Clagett e si ritengono erronei A e B (e quindi
anche Napoli):

%\stelle
%132\textsuperscript{1}
%\hrule width 1cm
%\textsuperscript{1}\qquad {\rmnot 132} {\slnot Clagett}~~{\rmnot 123} %{\slnot
%A}~~{\rmnot 112} {\slnot B}
%\stelle

\begin{maurotex}
\VV{
   {*:\ED{Clagett}:132}
   {A:123}
   {B:112}
   }
\end{maurotex}

\noindent dove ``132 \textsl{Clagett}'' significa ``Clagett ha
congetturato che debba leggersi 132''. E si noti che non si
riporta Napoli perch{\'e} accetta A. La relativa scrittura
{\`e}

\begin{quote}
\begin{verbatim}
\VV{
   {*:\ED{Clagett}:132}
   {A:123}
   {B:112}
   }
\end{verbatim}
\end{quote}

Si noti che nelle parentesi
graffe che seguono \verb"\ED{}"\index{\bs{}ED} si {\`e} inserita in questo caso
l'espressione \verb"Clagett" invece di \verb"conieci", ma che per il resto
la situazione {\`e} formalmente del tutto analoga al caso di una
correzione dell'editore\footnote{Volendo
sottolineare l'atteggiamento di Napoli nei confronti di
questo punto del testo, si potebbe aggiungere
l'espressione \textsl{prob. Napoli}, cio{\`e} \textsl{probante
Napoli}, in questo modo:

\begin{quote}\texttt{%
$\backslash$VV\{ \\
\{*:$\backslash$ED\{Clagett\}:132\} \\
\{A:123\} \\
\{*:$\backslash$ED\{prob. Napoli\}:\} \\
\{B:112\} \\
\} \\
}\end{quote}
}.

%\textbf{6.1.2.b.}
%-----------------------------------------------------------------------
\subsubsection{Caso b}

\label{ref-6.1.2.2}

\begin{itemize}

\item A ha la lezione ``123'';
\item B ha la lezione ``112'';
\item Napoli corregge con ``125'' e rifiuta
esplicitamente 123 e 112;
\item Clagett ha la lezione ``132'' e rifiuta
esplicitamente 123, 112 e il 125 di Napoli.

\end{itemize}

Come prima,  si accoglie Clagett e non si accettano  
le lezioni di A e di B; ma in questo caso occorrer{\`a} rifiutare 
anche la correzione di Napoli, scrivendo in apparato:

%\stelle
%132\textsuperscript{1}
%\hrule width 1cm
%\textsuperscript{1}\qquad {\rmnot 132} {\slnot Clagett}~~{\rmnot 123} %{\slnot
%A}~~{\rmnot 112} {\slnot B}~~{\rmnot 125} {\slnot Napoli}
%\stelle

\begin{maurotex}
\VV{
   {*:\ED{Clagett}:132}
   {A:123}
   {B:112}
   {*:\ED{Napoli}:125}
   }
\end{maurotex}

\noindent dove sono riportati prima la correzione e chi l'ha
fatta, poi i codici secondo la regola stabilita, poi le
eventuali altre proposte di correzione che non vengono
accolte. In {\mtex} scriveremo:

\begin{quote}
\begin{verbatim}
\VV{
   {*:\ED{Clagett}:132}
   {A:123}
   {B:112}
   {*:\ED{Napoli}:125}
   }
\end{verbatim}
\end{quote}

Si noti che in questo caso
{\`e} stato aggiunto un quarto campo (oltre a quello del TC e dei due
testimoni) per dar conto della correzione di Napoli, che viene
trattato anch'esso grazie alla macro \verb"\ED"\index{\bs{}ED}.

%\textbf{6.1.2.c.}
%-----------------------------------------------------------------------
\subsubsection{Caso c}

\label{ref-6.1.2.3}

\begin{itemize}

\item A ha la lezione ``123''
\item B ha la lezione ``112'';
\item Napoli congettura esplicitamente ``125'';
\item Clagett congettura esplicitamente ``113'';

\end{itemize}

In questo caso l'editore congettura che la
lezione debba essere 132 e rifiuta le lezioni di A e di
B, e le correzioni dei due editori precedenti:

%\stelle
%132\textsuperscript{1}
%\hrule width 1cm
%\textsuperscript{1}\qquad {\rmnot 132} {\slnot conieci}~~{\rmnot 123} %{\slnot
%A}~~{\rmnot 112} {\slnot B}~~{\rmnot 125} {\slnot Napoli}~~{\rmnot 113}
%{\slnot Clagett}
%\stelle

\begin{maurotex}
\VV{
   {*:\ED{conieci}:132}
   {A:123}
   {B:112}
   {*:\ED{Napoli}:125}
   {*:\ED{Clagett}:113}
   }
\end{maurotex}

La relativa scrittura sarebbe

\begin{quote}
\begin{verbatim}
\VV{
   {*:\ED{conieci}:132}
   {A:123}
   {B:112}
   {*:\ED{Napoli}:125}
   {*:\ED{Clagett}:113}
   }
\end{verbatim}
\end{quote}

%\textbf{6.1.2.d.}
%-----------------------------------------------------------------------
\subsubsection{Caso d}

\label{ref-6.1.2.4}

\begin{itemize}
\item A ha la lezione ``132'';
\item B ha la lezione ``112'';
\item Napoli ha la lezione ``132''
\item Clagett congettura ``518''
\end{itemize}

Qui l'editore accoglie la lezione di A, e
respinge la correzione fatta da Clagett (chi sa per quale
diavolo di motivo: questi americani!) e quindi TC scrive:

%\stelle
%132\textsuperscript{1}
%\hrule width 1cm
%\textsuperscript{1}\qquad {\rmnot 132} {\slnot A}~~{\rmnot 112}
%{\slnot B}~~{\rmnot 518} {\slnot Clagett}
%\stelle

\begin{maurotex}
\VV{
   {A:132}
   {B:112}
   {*:\ED{Clagett}:518}
   }
\end{maurotex}

\noindent (come al solito non si riporta Napoli perch{\'e}
accoglie A). La relativa scrittura sarebbe

\begin{quote}
\begin{verbatim}
\VV{
   {A:132}
   {B:112}
   {*:\ED{Clagett}:518}
   }
\end{verbatim}
\end{quote}

%\textbf{6.1.2.e.}
%-----------------------------------------------------------------------
\subsubsection{Caso e}

\label{ref-6.1.2.5}

\begin{itemize}
\item{} A ha la lezione ``sancta'';
\item{} B ha la lezione ``secta'';
\item{} Napoli ha la lezione ``samba'', ma non dice
nulla su tale lezione;
\item{} Clagett congettura ``recta''
\end{itemize}

Questa volta l'editore accoglie la correzione di Clagett;
ma non riporta quella di Napoli. La lezione ``samba'' pu{\`o}
essere infatti il frutto di una distrazione di Napoli, di un
tipografo troppo amante dei ritmi latino-americani, di milioni
di altri motivi e non pu{\`o}
essere considerata una correzione o una congettura voluta.
Compito dell'editore non {\`e} ovviamente quello di lapidare i
suoi predecessori indicandone tutti gli errori (e chi non
sbaglia?), ma solo di indicare gli interventi espliciti da
questi fatti e di dire se intende accoglierli o meno nel
suo testo. 

Il TC sar{\`a} quindi:

%\stelle
%recta\textsuperscript{1}
%\hrule width 1cm
%\textsuperscript{1}\qquad {\rmnot recta} {\slnot Clagett}~~{\rmnot secta}
%{\slnot B}~~{\rmnot sancta} {\slnot A}
%\stelle

\begin{maurotex}
\VV{
   {*:\ED{Clagett}:recta}
   {B:secta}
   {A:sancta}
   }
\end{maurotex}

Il tutto, come al solito, si otterr{\`a} scrivendo

\begin{quote}
\begin{verbatim}
\VV{
   {*:\ED{Clagett}:recta}
   {B:secta}
   {A:sancta}
   }
\end{verbatim}
\end{quote}

Pensiamo che gli esempi sin qui trattati possano bastare a
chiarire la cosa. Ne facciamo solo un ultimo per far vedere come
il problema della correzione si pu{\`o} combinare con quello
dell'omissione o dell'omissione in lacuna.

%\textbf{6.1.2.f.}
%-----------------------------------------------------------------------
\subsubsection{Caso f}

\label{ref-6.1.2.6}

\begin{quote}
A ha la lezione ``a vertice cani demittitur'' \\
B ha la lezione ``a vertice demittitur'' \\
C ha la lezione ``a vertice comico demittitur''; \\
Clagett congettura ``a vertice conico demittitur''.
\end{quote}

L'editore per suoi motivi ritiene da respingere la correzione
di Clagett (per esempio perch{\'e} Clagett conosceva solo B e C
ma non A) quindi scrive

%\stelle
%a vertice coni\textsuperscript{1} demittitur
%\hrule width 1cm
%\textsuperscript{1}\qquad {\rmnot coni} {\slnot conieci}~~{\rmnot cani} %{\slnot
%A}~~{\rmnot comico} {\slnot C}~~{\slnot om. B}~~{\rmnot
%conico} {\slnot Clagett}
%\stelle

\begin{maurotex}
a vertice \VV{
             {*:\ED{conieci}:coni}
             {A:cani}
             {C:comico}
             {B:\OM}
             {*:\ED{Clagett}:conico}
             } demittitur
\end{maurotex}

\noindent battendo

\begin{quote}
\begin{verbatim}
a vertice \VV{
             {*:\ED{conieci}:coni}
             {A:cani}
             {C:comico}
             {B:\OM}
             {*:\ED{Clagett}:conico}
             } demittitur
\end{verbatim}
\end{quote}

Si noti che questo esempio mostra che se un testimone presenta
omissione o omissione in lacuna laddove gli altri testimoni
offrono lezioni non accolte dall'editore, esso va
considerato il pi{\'u} lontano dal TC e quindi {\`e} consigliabile
indicarlo per ultimo in apparato, dopo tutti gli altri
testimoni (ma comunque sempre prima delle congetture di
altri studiosi rifiutate dall'editore di TC).

%-----------------------------------------------------------------------
\subsection{Interventi congetturali dell'editore basati
su testi paralleli}

\label{ref-6.1.3}
\index{interventi congetturali}
\index{testi paralleli}

Potrebbe accadere che un certo luogo di un testo risulti
corrotto, o per errori dei copisti, o per una svista di
Maurolico stesso; e che, ci{\`o} nonostante, esso sia emendabile
ricorrendo a un altro testo di Maurolico in cui il nostro
autore dice la stessa cosa.

L'esempio pi{\'u} evidente {\`e} costituito dalla \textit{Cosmographia}
latina e dai \textit{Dialoghi tre della Cosmographia}. Si tratta di due
testi senz'altro diversi: composti in date diverse, con
larghe porzioni che differiscono e scritti per di pi{\'u}
uno in
latino, l'altro in italiano. Tuttavia in molti punti sono
grosso modo sovrapponibili. In situazioni di questo genere
si pu{\`o} dire che questi due testi costituiscono un
testimone ``improprio'' uno dell'altro.

Se dunque l'editore della \textit{Cosmografia} (A= copia
settecentesca dei \textit{Dialoghi tre della Cosmographia}) s'imbatte in un
passo corrotto e privo di senso:

\begin{itemize}

\item[A:] La \textit{torre} {\`e} immobile al centro del mondo

\end{itemize}

\noindent potr{\`a} consultare il testo della \textit{Cosmographia}
pubblicato a Venezia nel 1543 (Z), dove trover{\`a}:

\begin{itemize}

\item[Z:] \textit{Terra} stat in medio mundi.

\end{itemize}

Converr{\`a} allora che la sua correzione si appoggi
sull'autorit{\`a} del parallelo testo latino e che scriva:

%\stelle
%La terra\textsuperscript{1} {\`e} immobile al centro del mondo
%\hrule width 1cm
%\textsuperscript{1}\qquad {\rmnot terra} {\slnot conieci coll.
%{\itnot Cosmographia 1543} I f.~17v 34}~~{\rmnot torre}
%{\slnot A}
%\stelle

\begin{maurotex}
La \VV{
      {*:\ED{
             conieci coll. \Tit{Cosmographia 1543}, 
             I, f. 17v, 34}:terra}
      {A:torre}
      } {\`e} immobile al centro del mondo
\end{maurotex}

\noindent dove il \textit{coll.} sta per \textit{collata}, ovvero:
``ho corretto io in `terra' il `torre' di A, ma sulla base
di ci{\`o} che {\`e} scritto nell'edizione di Venezia del 1543,
Dialogo primo, carta~17v, riga~34''. Si noti poi che si cita
l'opera esplicitamente (la \textit{Cosmographia}) e non il
testimone (Z) dato che ci{\`o} che qui interessa {\`e} il \textit{testo} parallelo su cui si basa la congettura. Quando poi
sar{\`a} disponibile l'edizione suddivisa in paragrafi della
\textit{Cosmographia} latina, la nota dovr{\`a} diventare:

%\stelle
%La terra\textsuperscript{1} {\`e} immobile al centro del mondo
%\hrule width 1cm
%\textsuperscript{1}\qquad {\rmnot terra} {\slnot conieci coll.
%{\itnot Cosm.} I, 101}~~{\rmnot torre}
%{\slnot A}
%\stelle

\begin{quote}
\begin{verbatim}
\VV{
   {*:\ED{conieci coll. \Tit{Cosm.}, I, 101}:terra}
   {A:torre}
   }
\end{verbatim}
\end{quote}

\noindent rimandando cio{\`e} alla nostra edizione di questo
testo (indicata con la sigla \textit{Cosm.}), dialogo I,
paragrafo 101.

Per ottenere questo baster{\`a} scrivere:

\begin{quote}
\begin{verbatim}
\VV{
   {*:\ED{conieci coll. \Tit{Cosm.}, I, 101}:terra}
   {A:torre}
   }
\end{verbatim}
\end{quote}

\noindent dove si noti il cambiamento di carattere dal tondo
inclinato (che viene prodotto automaticamente) al corsivo per
scrivere \textit{Cosm.} (\verb"\textit{Cosm.}"\index{\bs{}textit}). Ovviamente, per
ottenere il primo dei due esempi si sarebbe dovuto scrivere:

\begin{quote}
\begin{verbatim}
La \VV{
      {*:\ED{
             conieci coll. \Tit{Cosmographia 1543}, 
             I, f. 17v, 34}:terra}
      {A:torre}
      } {\`e} immobile al centro del mondo
\end{verbatim}
\end{quote}

\noindent (Si noti l'uso della macro \verb"\Tit"\index{\bs{}Tit}  per ottenere che
il titolo dell'opera compaia in corsivo: cfr. \S\,\ref{ref-3.5.4}.)

Nel caso dell'esistenza di testi paralleli, si potranno
trattare in modo simile anche i casi delle lacune pi{\'u} o meno
estese (cfr. \S\,\ref{ref-6.3.2})

%-----------------------------------------------------------------------
\subsection{Proposte di correzione avanzate in apparato e non in
TC}

\label{ref-6.1.4}
\index{correzione avanzate in apparato}

Pu{\`o} accadere che ci si trovi di fronte al dubbio
irrisolvibile su cosa Maurolico intendesse e l'editore non
sia in grado di correggere il testo con sicurezza, soprattutto se
autografo. Ad esempio supponiamo di disporre dell'autografo
(A) e che esso rechi la lezione ``per 21\textsuperscript{am} sexti
Euclidis''. Il riferimento per{\`o} non risulta coerente
e la proposizione che l'editore
riterrebbe opportuno citare, mettiamo la 15 del V libro
degli \textit{Elementi} non ha niente a che vedere con
quella apparentemente citata da Maurolico. (Diverso sarebbe
il caso se la proposizione che si ritiene dovesse essere
citata fosse ad esempio la 12.VI o la 21.V: in questo caso
si potrebbe pensare a un \textit{lapsus} di Maurolico e procedere
alla correzione in TC.) In casi del genere si dovrebbe
procedere nel seguente modo:

%\stelle
%per 21$^{\rm am}$ sexti Euclidis\textsuperscript{1}
%\hrule width 1cm
%\textsuperscript{1}\qquad {\rmnot per 21$^{\hbox{\rmnot am}}$ sexti
%Euclidis} {\slnot A}~~{\rmnot per 15$^{\hbox{\rmnot am}}$ quinti
%Euclidis} {\slnot intelligendum est}
%\stelle

\begin{maurotex}
\VV{
   {A:\Cit{
          {per 21\Sup{am} sexti Euclidis}
          {E.6.21}
          {Qui avrebbe dovuto citare E.5.15}
          }
    }
    {*:\ED{intelligendum est}:per 15\Sup{am} quinti Euclidis}
    }
\end{maurotex}

Ovvero: ``Si deve intendere la proposizione 15 del quinto
di Euclide, anche se il testo parla della 21 del sesto''. 

 In {\mtex} si scriver{\`a}:

\begin{quote}
\begin{verbatim}
\VV{
   {A:\Cit{
          {per 21\Sup{am} sexti Euclidis}
          {E.6.21}
          {Qui avrebbe dovuto citare E.5.15}
          }
    }
    {*:\ED{intelligendum est}:per 15\Sup{am} quinti Euclidis}
    }
\end{verbatim}
\end{quote}

Si noti che in questo caso la correzione deve andare nel
secondo campo e non nel primo, dato che TC {\`e} costretto ad
accettare la lezione di A, che infatti viene inserita nel
primo campo. Si noti anche l'uso di \verb"\Cit"\index{\bs{}Cit} (cfr. \S\,\ref{ref-3.5.2}),
con la codificazione della citazione nel secondo campo e la
nota nel terzo.

%-----------------------------------------------------------------------
\section{Cruces e espunzioni}

\label{ref-6.2}

\subsection{Croci senza delizie}
\index{croci}

\label{ref-6.2.1}
Potr{\`a}  darsi il caso che l'editore sia in grado di leggere
il testo dei vari testimoni, ma che esso non abbia alcun
senso. E che per di pi{\'u}, a differenza della situazione
descritta nei \S\S\,\ref{ref-6.1.1}--4,

l'editore non riesca ad elaborare una
congettura credibile. Si parla, in casi, del genere, di \textit{loci desperati}. Se ad esempio la tradizione fosse
unanime nel leggere

\begin{quote}
``et erit $ab$ parallelus $cd$ per doctrinam de piscibus siculis''
\end{quote}

\noindent si scriver{\`a} semplicemente

%\stelle
%per doctrinam \dag de piscibus siculis\dag
%\stelle

\begin{maurotex}
per doctrinam \CRUX{de piscibus siculis}
\end{maurotex}

\noindent senza alcuna nota, racchiudendo il brano privo di senso
fra \textit{cruces desperationis} (\dag). E si scriver{\`a} in {\mtex}

\begin{quote}
\begin{verbatim}
per doctrinam \CRUX{de piscibus siculis}
\end{verbatim}
\end{quote}

\noindent dove la macro \verb"\CRUX{}"\index{\bs{}CRUX} provvede appunto a racchiudere
fra \textit{cruces} il testo contenuto nelle sue parentesi
graffe. In una situazione simile non c'{\`e} bisogno di annotare
alcunch{\'e} in apparato, dato che l'editore si limita a
indicare, segnalandola con le \dag, una situazione
oggettiva. Si noti inoltre che \verb"\CRUX"\index{\bs{}CRUX} {\`e} una macro
``grafica'' che produce solo le {\dag} all'inizio e alla fine
di una stringa di testo

Ci si potrebbe per{\`o} trovare di fronte a casi pi{\'u} complessi.
Per esempio, supponiamo che il testo sia tr{\`a}dito da due
testimoni, A e B, che leggano:

\begin{itemize}

\item[A:] et erit $ab$ parallelus $cd$ per doctrinam de
piscibus siculis

\item[B:] et erit $ab$ parallelus $cd$ per doctrinam de
arcanis antiquis

\end{itemize}

L'editore non riesce a elaborare una sua
congettura che possa restituire un senso al testo chiarendo perch{\'e} $ab$
dovrebbe essere parallelo a $cd$; decide tuttavia di
inserire il testo di A in TC come soluzione del meno peggio,
segnalando per{\`o} la corruzione a suo avviso irrimediabile.
In tal caso si dovr{\`a} dar conto della situazione in questo
modo:

%\stelle 
%et erit $ab$ parallelus $cd$ per doctrinam \dag de
%piscibus siculis\textsuperscript{1}\thinspace\dag
%\hrule width 1cm
%\textsuperscript{1}\qquad {\rmnot de
%piscibus siculis} {\slnot A}~~{\rmnot de
%arcanis antiquis} {\slnot B~~~locum valde corruptum}
%\stelle

\begin{maurotex}
et erit \(ab\) parallelus \(cd\) per doctrinam
\CRUX{
      \VV{
         {A:de piscibus siculis}
         {B:de arcanis antiquis}
         {*:\ED{locum valde corruptum}:}
         }
     }
\end{maurotex}

Per ottenere tutto questo si scriver{\`a} in {\mtex}:

\begin{quote}
\begin{verbatim}
et erit \(ab\) parallelus \(cd\) per doctrinam
\CRUX{
      \VV{
         {A:de piscibus siculis}
         {B:de arcanis antiquis}
         {*:\ED{locum valde corruptum}:}
         }
     }
\end{verbatim}
\end{quote}

Si noti che:

\begin{itemize}

\item \verb"\CRUX"\index{\bs{}CRUX} abbraccia l'intera variante, dato
che l'{\mtex} deve provvedere ad apporre le \textit{cruces} in TC;

\item nel primo campo di \verb"\VV"\index{\bs{}VV} si inserisce la lezione di
A che si vuole, \textit{faute de mieux}, inserire in TC;

\item nel secondo si indica solo il testimone B e
la sua lezione;

\item nel terzo, infine si utilizza, come al solito la macro
\verb"\ED{}"\index{\bs{}ED}, che produce il commento (\textsl{locum valde corruptum}),
senza naturalmente indicare alcuna lezione nel terzo sottocampo e inserendo
una \verb"*" nel primo.

\end{itemize}

Non proviamo nemmeno a costruire una casistica dettagliata
delle situazioni complesse in cui ci si potrebbe trovare (presenza di
varianti, interventi di altri editori, e altro ancora). Ci{\`o}
perch{\'e} si potr{\`a} sempre e comunque combinare l'uso di
\verb"\CRUX{}"\index{\bs{}CRUX} e di \verb"\ED{}"\index{\bs{}ED} in modo analogo a quanto
descritto qui sopra, inserendo cio{\`e} la lezione
incomprensibile come argomento di \verb"\CRUX"\index{\bs{}CRUX} e
ci{\`o} che si vuole che venga scritto in nota come argomento di
\verb"\ED"\index{\bs{}ED}.

%-----------------------------------------------------------------------
\subsection{Espunzioni}

\label{ref-6.2.2}
\index{espunzioni}

Intendiamo che si debba fare un'\textit{espunzione} quando
tutta la tradizione del testo presenta un luogo che l'editore
ritiene assolutamente da respingere. Come nel caso delle
integrazioni ci{\`o} si applica anche al caso di autografi di
Maurolico; e anche in questa situazione l'editore agisce di
testa sua. Le espunzioni si evidenziano in TC usando le
parentesi quadre ([\hphantom{a}])

Supponiamo come prima che la tradizione sia costituita da A e B e
che essi leggano ``primum et secundumque'' e che l'editore sia
certo che si debba espungere ``et'' e si debba leggere ``primum
secundumque''. In questo caso si dovr{\`a} ottenere

%\stelle
%primum [et\textsuperscript{1}] secundumque
%\hrule width 1cm
%\textsuperscript{1}\qquad {\rmnot et} {\slnot seclusi}
%\stelle

\begin{maurotex}
primum \EXPU{
             \VV{
                {*:\ED{seclusi}:et}
                }
            } secundumque
\end{maurotex}

\noindent dove ``et \textsl{seclusi}'' significa ``ho espunto
io \textit{et}''. Dato che si tratta di un intervento dell'editore
{\`e} opportuno specificarlo in apparato come nel caso delle
integrazioni, anche se l'espunzione {\`e} gi{\`a} segnalata dalle
[~] in TC. Non c'{\`e}, ovviamente,
necessit{\`a} di dare indicazioni sui testimoni giacch{\'e},
per definizione, si suppone che tutta la tradizione riporti
la lezione ``primum et secundumque''.

Ci{\`o} si otterr{\`a} in modo analogo a quanto fatto per i \textit{loci
desperati} utilizzando la nuova macro \verb"\EXPU{}"\index{\bs{}EXPU} che
provvede a racchiudere fra parentesi quadre ci{\`o} che le si
fornisce come argomento:

\begin{quote}
\begin{verbatim}
primum \EXPU{
             \VV{
                {*:\ED{seclusi}:et}
                }
            } secundumque
\end{verbatim}
\end{quote}

Naturalmente l'espunzione di TC pu{\`o} essere stata gi{\`a} fatta dal
solito Clagett, e allora si dovr{\`a} segnalarlo:

\begin{maurotex}
primum \EXPU{
             \VV{
                {*:\ED{secl. Clagett}:et}
                }
            } secundumque
\end{maurotex}

%\stelle
%primum [et\textsuperscript{1}] secundumque
%\hrule width 1cm
%\textsuperscript{1}\qquad {\rmnot et} {\slnot secl. Clagett}
%\stelle

\noindent dove l'espressione ``et \textsl{secl. Clagett}'' sta per
``et \textsl{seclusit Clagett}''. Si scriver{\`a} allora in {\mtex}:

\begin{quote}
\begin{verbatim}
primum \EXPU{
             \VV{
                {*:\ED{secl. Clagett}:et}
                }
            } secundumque
\end{verbatim}
\end{quote}

Potrebbero
presentarsi casi pi{\'u} complessi (ad esempio che Napoli
~---~accidenti a lui~---~avesse espunto ``que'' invece di ``et''),
ma non riteniamo necessario affrontarli \textit{a priori}.
Il presente \textit{Manuale} cerca infatti di prevedere molti
casi, ma i suoi autori~---~non essendo dotati di facolt{\`a} di
veggenza~---~hanno evitato di inoltrarsi in una casistica
troppo raffinata che complicherebbe il linguaggio fino
all'inverosimile. Su come trattare i casi eccezionali, si
veda il capitolo \ref{ref-9},{} in particolare il \S\,\ref{ref-9.2}.

%-----------------------------------------------------------------------
\subsection{E se si devono espungere cinque righe?}

\label{ref-6.2.3}
\index{espunzioni lunghe}

Come abbiamo gi{\`a} varie volte segnalato nel corso dei due
capitoli precedenti, ci{\`o} che {\`e} detto qui per le correzioni e
le espunzioni sarebbe difficilmente utilizzabile nel caso di
correzioni o di espunzioni lunghe. Per questo tipo di
situazioni rinviamo al
capitolo \ref{ref-7}, parte~C.

%-----------------------------------------------------------------------
\section{Integrazioni e lacune congetturali}

\label{ref-6.3}

\subsection{Integrazioni}
\index{integrazioni congetturali}

\label{ref-6.3.1}

Potr{\`a} accadere che l'editore si trovi davanti a una lacuna
condivisa da tutta la tradizione (in tal caso gli sar{\`a}
stata gi{\`a} segnalata dal trascrittore: cfr. \S\,\ref{ref-4.4.2}), oppure
che si renda conto (ad esempio in base al senso del
discorso) che tutta la tradizione omette una o pi{\'u} parole,
pur non presentando alcuna lacuna materiale e alcuno spazio
bianco. In questi casi sar{\`a} allora possibile procedere a
un'\textit{integrazione}, cio{\`e} si potranno supplire
congetturalmente le parole omesse, siano esse in lacuna o
meno.

Si noti bene che il caso {\`e}
diverso da quello delle correzioni (sia pure combinate con
omissioni e omissioni in lacuna) discusse nei
\S\S\,\ref{ref-6.1.1}--3,
perch{\'e} se si ``corregge'' il testo tr{\`a}dito
vuol dire che in almeno uno dei testimoni {\`e} attestata una
qualche lezione, sebbene sbagliata.  Nel caso
dell'integrazione, {\`e} invece l'intera tradizione a omettere
parole che pure~---~l'editore ne {\`e} certo~---~dovevano esistere
nell'originale.

Supponiamo ad esempio che la tradizione sia costituita da A
e B e che essi leggano ``per propositionem 17 erit''. Come
si vede non {\`e} specificato di quale proposizione 17  si stia
parlando, ma il nostro sagacissimo editore {\`e} certo che si
tratti della 17 del libro precedente e quindi che Maurolico
dovesse aver scritto ``per propositionem 17 praecedentis libri erit''. In
questo caso si dovr{\`a} ottenere in TC

%\stelle
%per propositionem 17 $\langle$praecedentis
%libri$^1\rangle$ erit
%\hrule width 1cm
%\textsuperscript{1}\qquad {\rmnot praecedentis libri} {\slnot supplevi}
%\stelle

\begin{maurotex}
per propositionem
17 \INTE{
         \VV{
            {*:\ED{supplevi}:praecedentis libri}
            }
        } erit
\end{maurotex}

\noindent dove ``praecedentis libri \textsl{supplevi}''
significa ``ho aggiunto io \textit{praecedentis libri}''. Le
parentesi uncinate $\langle$~$\rangle$ servono a rendere
evidente l'integrazione dell'editore gi{\`a} in TC, senza
bisogno di consultare l'apparato. Non si riporta poi
l'indicazione del fatto che ``praecedentis libri'' manca in
A e manca in B, dato che un'\textit{integrazione} suppone
appunto che la lezione sia stata omessa dall'intera
tradizione. In apparato, tuttavia, sar{\`a} opportuno
specificare ``supplevi'' (o ``supplevimus'' nel caso di
un'edizione a quattro mani), dato che qui {\`e} bene avvertire
che si tratta di una congettura dell'editore.

Le integrazioni vengono trattate in modo analogo a quanto visto per i
\textit{loci desperati} e le espunzioni, utilizzando \verb"\ED"\index{\bs{}ED} e una nuova
macro, \verb"\INTE{}"\index{\bs{}INTE}, nel seguente modo:

\begin{quote}
\begin{verbatim}
per propositionem
17 \INTE{
         \VV{
            {*:\ED{supplevi}:praecedentis libri}
            }
        } erit
\end{verbatim}
\end{quote}

Come \verb"\CRUX"\index{\bs{}CRUX} provvvedeva ad apporre una {\dag} all'inizio e
alla fine del testo che costituisce il suo argomento, cos{\'\i}
\verb"\INTE"\index{\bs{}INTE} racchiude il suo argomento fra parentesi uncinate.
E, come nel caso di \verb"\CRUX"\index{\bs{}CRUX}, si daranno dei casi in cui
risulter{\`a} necessario fornire maggiori indicazioni
in apparato.  Supponiamo che il testo di
B, ad esempio, sia:

\begin{itemize}

\item[B:] per propositionem 17 $\circ$ erit

\end{itemize}

\noindent dove il $\circ$ indica il fatto che B ha lasciato
un breve spazio vuoto fra ``17'' e ``erit''. In un caso del
genere dovremmo ottenere

%\stelle
%per propositionem 17 $\langle$praecedentis
%libri$^1\rangle$ erit
%\hrule width 1cm
%\textsuperscript{1}\qquad {\rmnot praecedentis libri} {\slnot %supplevi}~~{\slnot
%spatium aliquot literarum rel. B}
%\stelle

\begin{maurotex}
per propositionem
17 \INTE{
         \VV{
            {*:\ED{supplevi}:praecedentis libri}
            {B:\DES{spatium aliquot literarum rel.}:\LACm}
            }
        } erit
\end{maurotex}

\noindent dove ``spatium aliquot literarum rel. B''
significa ``B ha lasciato uno
spazio'' e vuole indicare al lettore che nel testimone B
manca s{\'\i} la lezione``praecedentis libri'' ma {\`e} stato
comunque lasciato uno spazio bianco. Per questo motivo nel
testo di B dovranno essere stampati, al posto dello spazio
bianco originario, tre \textbf{*}, come nel caso delle
omissioni in lacuna. Questo caso {\`e}
concettualmente diverso da quello delle omissioni e
omissioni in lacuna~---~e dovr{\`a} essere trattato
anche dall'{\mtex} in modo diverso. Infatti non
si pu{\`o} dire propriamente che B ometta, dato che l'intera
tradizione non riporta alcuna lezione fra ``17'' e ``erit''.
Non si utilizzer{\`a} quindi \verb"\OMLAC"\index{\bs{}OMLAC}, ma le macro \verb"\LACm"\index{\bs{}LACm} e
\verb"\DES"\index{\bs{}DES} (cfr. \S\,\ref{ref-4.4}) e i comandi da impartire
saranno\footnote{Si osservi che il trascrittore di B
dovrebbe aver gi{\`a} segnalato la cosa, in questo modo:
\begin{quote}\texttt{
... 17 $\backslash$VV\{ \\
          \{B:$\backslash$DES\{spatium aliquot literarum rel.\}:$\backslash$LACm\} \\
          \}
}\end{quote}
L'editore non dovr{\`a} quindi far altro che
aggiungere la sua integrazione.}:

\begin{quote}
\begin{verbatim}
per propositionem
17 \INTE{
         \VV{
            {*:\ED{supplevi}:praecedentis libri}
            {B:\DES{spatium aliquot literarum rel.}:\LACm}
            }
        } erit
\end{verbatim}
\end{quote}

Naturalmente l'integrazione che si accoglie nel testo pu{\`o}
essere stata gi{\`a} fatta dal solito Clagett e dal solito
Napoli. Se ad esempio Clagett ha integrato ``praecedentis
libri'' ma Napoli
``huius'' e noi accettiamo 
l'integrazione di Clagett, scriveremo:

%\stelle
%per propositionem 17 $\langle$praecedentis libri$^1\rangle$ erit
%\hrule width 1cm
%\textsuperscript{1}\qquad {\rmnot praecedentis libri} {\slnot
%suppl. Clagett}~~{\slnot
%spatium rel. B}~~{\rmnot huius} {\slnot suppl. Napoli}
%\stelle

\begin{maurotex}
per propositionem
17 \INTE{
         \VV{
            {*:\ED{suppl. Clagett}:praecedentis libri}
            {B:\DES{spatium aliquot literarum rel.}:\LACm}
            {*:\ED{suppl. Napoli}:huius}
            }
        } erit
\end{maurotex}

\noindent scrivendo:

\begin{quote}
\begin{verbatim}
per propositionem
17 \INTE{
         \VV{
            {*:\ED{suppl. Clagett}:praecedentis libri}
            {B:\DES{spatium aliquot literarum rel.}:\LACm}
            {*:\ED{suppl. Napoli}:huius}
            }
        } erit
\end{verbatim}
\end{quote}

%-----------------------------------------------------------------------
\subsection{Integrazioni in presenza di lacune materiali o di parole
indecifrabili}

\label{ref-6.3.2}
\index{integrazioni in presenza di lacune materiali}
\index{lacune materiali}

Abbiamo gi{\`a} specificato cosa si debba intendere per
lacuna materiale: intendiamo macchie di inchiostro, fori
nella carta, ecc. (\S\,\ref{ref-4.4}), a cui si aggiungono parole
indecifrabili dopo attentissimo esame o simili (\S\,\ref{ref-3.3.1}; ma 
si veda anche il \S\,\ref{ref-9.1}). Se
l'editore deve intervenire congetturalmente in un caso del
genere, si pu{\`o} supporre che, salvo casi veramente del
tutto eccezionali, il testo sia stato tramandato da un solo
testimone, A, che rechi ad esempio:

\begin{itemize}

\item[A:] a vertice $\bullet\bullet\bullet$ demittitur

\end{itemize}

\noindent dove $\bullet\bullet\bullet$ indica una macchia di inchiostro.
L'editore per suoi motivi ritiene che ci{\`o} che era scritto
sotto la macchia fosse ``trianguli'' e quindi scrive in TC:

%\stelle
%a vertice $\langle$trianguli$^1\rangle$ demittitur
%\hrule width 1cm
%\textsuperscript{1}\qquad {\rmnot trianguli} {\slnot supplevi~~aliquot
%litterae legi nequeunt in A}.
%\stelle

\begin{maurotex}
a vertice \INTE{
                \VV{
      {*:\ED{supplevi}:trianguli}
      {A:\DES{aliquot literae legi nequeunt in}:\LACm}
                   }
               } demittitur
\end{maurotex}

Tale testo viene prodotto dalla seguente
macro:

\begin{quote}
\begin{verbatim}
a vertice \INTE{
                \VV{
      {*:\ED{supplevi}:trianguli}
      {A:\DES{aliquot literae legi nequeunt in}:\LACm}
                   }
               } demittitur
\end{verbatim}
\end{quote}

\noindent dove si deve notare l'uso di \verb"\LACm"\index{\bs{}LACm} che, come si
ricorder{\`a}, produce una situazione simile a quella delle
omissioni in lacuna e delle lacune materiali, stampando i
tre *** anche nel testo di A.

Si pu{\`o} poi utilizzare una sorta di
``Principio di Sovrapposizione delle Situazioni'': se, per
esempio,
Clagett avesse congetturato ``trapezii'' si sarebbe inserito
un terzo campo con

\begin{quote}
\begin{verbatim}
{*:\ED{Clagett}:trapezii}
\end{verbatim}
\end{quote}

\noindent e cos{\'\i} via.

Si noti infine che nel caso di lacune materiali pi{\'u} estese
e, soprattutto, irrimediabili, il testo critico e
l'apparato continueranno a presentare la fisionomia discussa
nel \S\,\ref{ref-4.4.2}.

%-----------------------------------------------------------------------
\subsection{Manca un corollario? Le lacune
congetturali insanabili}

\label{ref-6.3.3}
\index{lacune congetturali insanabili}

Supponiamo che dopo una proposizione seguano vari
corollari; e che dal corollario~1 tutti i testimoni saltino
direttamente al corollario~3. Inoltre l'editore {\`e} certo (per
esempio a causa di rinvii interni) che un corollario 2 in origine
esisteva effettivamente e che non si tratta di un semplice errore
di numerazione. In una situazione del genere non {\`e}
evidentemente possibile ricostruire le esatte parole con cui
Maurolico esponeva il corollario~2 e quindi scriveremo:

\begin{maurotex}
\begin{Enunciatio}
Corollarium 1
\end{Enunciatio}
Constat ergo ratio sphaerae ad cylindrum esse ut duo
ad tria ... Quod erat demonstrandum.
\begin{center}
\VV{
   {*:\ED{lacunam statui}:\LACc}
   }
\end{center}
\begin{Enunciatio}
Corollarium 3
\end{Enunciatio}
\end{maurotex}

%\stelle
%\centerline{\sc Corollarium 1}
%Constat ergo ratio sphaerae ad cylindrum esse ut duo ad tria \dots Quod %erat demonstrandum.
%\centerline{***\textsuperscript{1}}
%\centerline{\sc Corollarium 3}
%\hrule width 1cm
%\textsuperscript{1}\qquad {\slnot lacunam statui}
%\stelle

\noindent dove ``lacunam statui'' significa ``ho stabilito
io l'esistenza di una lacuna''.

{\`E} ovvio che una lacuna di questo tipo {\`e} di genere assai
diverso da quelle che abbiamo incontrato nel \S\,\ref{ref-4.4}. Qui
infatti nessun segno materiale (la caduta di una carta, uno
spazio bianco lasciato dal copista o altri fatti del genere)
segnala il luogo non pi{\'u} ricostruibile: {\`e} l'editore che
stabilisce congetturalmente che il corollario 2 doveva
esistere originariamente grazie alla sua conoscenza del
testo. Quindi,  per ottenere il risultato qui sopra
descritto si utilizzer{\`a}, invece di \verb"\LACm"\index{\bs{}LACm} (riservata alle
lacune materiali, vedi \S\,\ref{ref-4.4.2}), la macro \verb"\ED"\index{\bs{}ED} in
combinazione con la nuova macro \verb"\LACc"\index{\bs{}LACc} (lacuna \textit{congetturale}) che provvede a
inserire  i tre *** nel testo critico, ma non nel testo dei
testimoni, dato che la lacuna non {\`e} attestata da essi. Si
scriver{\`a} allora:

\begin{quote}
\begin{verbatim}
\begin{Enunciatio}
Corollarium 1
\end{Enunciatio}
Constat ergo ratio sphaerae ad cylindrum esse ut duo
ad tria ... Quod erat demonstrandum.
\begin{center}
\VV{
   {*:\ED{lacunam statui}:\LACc}
   }
\end{center}
\begin{Enunciatio}
Corollarium 3
\end{Enunciatio}
\end{verbatim}
\end{quote}

Se  poi del salto si fosse gi{\`a} accorto l'onnipresente
Clagett, dovremmo scrivere:

\begin{maurotex}
Constat ergo ratio sphaerae ad cylindrum esse ut duo
ad tria ... Quod erat demonstrandum.
\begin{center}
\VV{
   {*:\ED{lacunam statuit Clagett}:\LACc}
   }
\end{center}
\begin{Enunciatio}
Corollarium 3
\end{Enunciatio}
\end{maurotex}

%\stelle
%Constat ergo \dots Quod erat demonstrandum.
%\centerline{***\textsuperscript{1}}
%\centerline{\sc Corollarium 3}
%\hrule width 1cm
%\textsuperscript{1}\qquad {\slnot lacunam statuit Clagett}
%\stelle

\noindent e la macro verrebbe usata cos{\'\i}:

\begin{quote}
\begin{verbatim}
Constat ergo ratio sphaerae ad cylindrum esse ut duo
ad tria ... Quod erat demonstrandum.
\begin{center}
\VV{
   {*:\ED{lacunam statuit Clagett}:\LACc}
   }
\end{center}
\begin{Enunciatio}
Corollarium 3
\end{Enunciatio}
\end{verbatim}
\end{quote}

%-----------------------------------------------------------------------
\subsection{Lacune in presenza di testi paralleli}

\label{ref-6.3.4}
\index{lacune in presenza di testi paralleli}

Cos{\'\i} come una corruttela pu{\`o} essere individuata e/o sanata
sulla base di un testo parallelo (\S\,\ref{ref-6.1.3}), anche una
lacuna pu{\`o} essere stabilita e/o parzialmente integrata allo
stesso modo.

Ad esempio l'editore della \textit{Cosmographia} si accorge
di un salto logico del discorso e, individuato il passo
parallelo nei \textit{Dialoghi tre della Cosmographia},
vi trova la conferma che nella tradizione del testo latino {\`e}
stato erroneamente omesso un passaggio relativo alla
dimensione dell'orbe terracqueo. Ora, pur non potendo ricostruire
le esatte parole con cui Maurolico affrontava il tema, non
si limiter{\`a} a segnalare la lacuna, ma indicher{\`a} in apparato
anche l'argomento che vi doveva essere trattato. Scriver{\`a}
pertanto:

%\stelle
%(TC) ***\textsuperscript{1} (TC)
%\hrule width 1cm
%\textsuperscript{1}\qquad {\slnot coll. {\itnot Dial. Cosm.} II 318 lacunam
%statui, ubi dimensio orbis pertractaretur}
%\stelle

\begin{maurotex}
(TC) \VV{
        {*:\ED{coll. {\it Cosm.} II 318 lacunam statui,
           ubi dimensio orbis pertractaretur}:\LACc}
        } (TC)
\end{maurotex}

\noindent ovvero: ``ho stabilito la lacuna confrontando il
testo dei \textit{Dialoghi della Cosmographia}, dialogo II,
\S\,318. In essa si trattava della dimensione della Terra''.
Tale risultato si otterr{\`a} sempre utilizzando la macro \verb"\ED"\index{\bs{}ED},
combinata con \verb"\LACc"\index{\bs{}LACc}:

\begin{quote}
\begin{verbatim}
(TC) \VV{
        {*:\ED{coll. {\it Cosm.} II 318 lacunam statui,
           ubi dimensio orbis pertractaretur}:\LACc}
        } (TC)
\end{verbatim}
\end{quote}

%-----------------------------------------------------------------------
\section{Un commento su $\protect\backslash$ED}

\label{ref-6.4}

Come il lettore avr{\`a} notato, \textit{tutti} gli interventi di
tipo congetturale che l'editore compie sul suo testo si
devono effettuare inserendo la macro \verb"\ED"\index{\bs{}ED} all'interno di
\verb"\VV"\index{\bs{}VV}. Essa {\`e} dunque una macro riservata all'editore, anzi
all'ultima fase del suo lavoro. 

Teniamo per{\`o} anche a
segnalare che, per il fatto di essere esplicitamente
dedicata a tale scopo, permetter{\`a} all'editore di recuperare
facilmente i suoi interventi congetturali per valutarli e
controllarli nel corso del suo lavoro e, a lavoro ultimato,
sar{\`a} pi{\'u} agevole valutare complessivamente la portata e
l'estensione delle congetture presenti nell'edizione con
semplici ricerche testuali che possono essere facilmente
svolte da un qualsiasi programma di \textit{text editing}.

Oltre a questo aspetto di \verb"\ED"\index{\bs{}ED}, {\`e} bene far osservare che
questa macro lascia completa libert{\`a} all'editore di scrivere
quello che vuole all'interno del suo argomento. Il che da
un lato {\`e} un bene, perch{\'e} in situazioni cos{\'\i} delicate quali
gli interventi congetturali, {\`e} difficile riuscire a 
prevedere tutte le possibilit{\`a} concrete che ci si potr{\`a}
trovare dinanzi nel lavoro di edizione e quindi ci {\`e} parso
sconsigliabile far imporre dall'{\mtex} formule precostituite;
dall'altro per{\`o} {\`e} chiaro che l'editore si trova molto pi{\'u}
facilmente esposto alla possibilit{\`a} di sbagliare e di
commettere errori di battitura nell'inserimento dei suoi
commenti come argomento di \verb"\ED"\index{\bs{}ED}.

A ci{\`o} pu{\`o} in parte ovviare la caratteristica che dicevamo
sopra: si potranno facilmente ritrovare tutti gli interventi
congetturali facendo ricercare la stringa \verb"\ED"\index{\bs{}ED} e
controllarli con estrema cura.

Va inoltre sottolineato un altro elemento: le differenze di
comportamento fra \verb"\ED{}"\index{\bs{}ED} e \verb"\DES{}"\index{\bs{}DES}. Il lettore attento si sar{\`a}
accorto che entrambe sembrano produrre la stessa cosa, cio{\`e}
una parte ``libera'' di una nota dell'apparato. La
differenza di nome per{\`o} rispecchia non solo il fatto che {\`e}
bene che gli interventi dell'editore siano distinguibili da
quelli del trascrittore per i motivi appena detti qui sopra.
Fra \verb"\ED"\index{\bs{}ED} e \verb"\DES"\index{\bs{}DES} c'{\`e} anche una differenza di comportamento
pi{\'u} sottile. Quando si scrive ad esempio:

\begin{quote}
\begin{verbatim}
\VV{
   {A:\DES{duo versus legi nequeunt in}:\LACm}
   }
\end{verbatim}
\end{quote}

\noindent tale annotazione del trascrittore di A verr{\`a}
riportata, oltre che in TC, anche nel testo di A estraibile
dal file {\mtex} dell'edizione, insieme con gli ***
che indicano la lacuna. Quando si usa \verb"\ED"\index{\bs{}ED}, invece, le
annotazioni dell'editore vengono riportate solo
nell'apparato di TC, ma non in quello dei testimoni
estratti. Sarebbe quindi un grave errore~---~e non solo
un'impropriet{\`a} di ``linguaggio''~---~ da parte di un
editore che coincida con s{\'e} stesso come trascrittore,
scrivere:

\begin{quote}
\begin{verbatim}
\VV{
   {A:\ED{duo versus legi nequeunt in}:\LACm}
   }
\end{verbatim}
\end{quote}

La sua annotazione, infatti andrebbe persa per la
costruzione del testo del testimone A, che si troverebbe ad
avere gli ***, senza per{\`o} che in apparato ci sia una
spiegazione della loro presenza.

La regola da ricordare, per distinguere gli effetti di \verb"\ED"\index{\bs{}ED}
e di \verb"\DES"\index{\bs{}DES} {\`e} la seguente:

\begin{itemize}

\item Tutto ci{\`o} che {\`e} scritto come argomento di \verb"\DES{}"\index{\bs{}DES}
viene riportato nell'apparato del testo del testimone cui si
riferisce; tutto ci{\`o} che {\`e} riportato come argomento di
\verb"\ED{}"\index{\bs{}ED} viene scritto solo nell'apparato del testo critico.

\end{itemize}

%-----------------------------------------------------------------------
\chapter{Cose lunghe o meglio longae}

\label{ref-7}

\section*{Varianti lunghe, note corte}
\index{varianti lunghe}

\label{ref-7}

Nei precedenti capitoli abbiamo pi{\'u} volte accennato al
fatto che quanto si andava dicendo era valido solo nel caso
di varianti e correzioni, per cos{\'\i} dire, ``puntuali'': in
situazioni cio{\`e} in cui sono coinvolte o una o comunque poche parole
del testo. Succede spesso,  per{\`o},  che i testimoni
differiscano per brani di una certa consistenza. Il sistema
fin qui descritto permetterebbe di trattare questi casi, ma
solo a patto di riportare in nota tutto il testo variato o
corretto gi{\`a} accolto nel testo critico creando cos{\'\i} una
situazione decisamente pesante per il lettore dell'edizione.
Nella nota dovr{\`a} essere ovviamente dato un richiamo al testo
critico, ma non il testo nella sua totalit{\`a}, e inoltre
andranno introdotti opportuni segnali di inizio e fine del
testo coinvolto.  Proponiamo subito un esempio per maggiore
chiarezza (che d'ora in poi chiameremo Esempio $\Lambda$).

\textbf{Esempio \textbf{$\Lambda$}}

Supponiamo che i testimoni siano tre, A, B e C  e che
ci sia la seguente situazione

\begin{itemize}

\item[A:] Si duae rectae ...  erunt aequales. Sint \textit{ab} et \textit{cd} ... ... Rursus, cum sint ... a vertice
coni. [\textit{il passo {\`e} complessivamente lungo 8 righe}]

\item[C:] \textit{stesso testo di A}

\item[B:] \textit{omette il brano completamente}

\end{itemize}

Si vuole ottenere allora il seguente testo
critico (i numeri in neretto indicano il numero di paragrafo
secondo quanto detto nel \S\,\ref{ref-3.7.2}), in cui in apparato si
segnali \textit{all'inizio del brano} che ci{\`o} segue {\`e} riportato
solo da A e C e che B lo omette:

%\stelle
%\textbf{10} Si\textsuperscript{1} duae rectae ...  erunt aequales.
%\textbf{11} Sint \textit{ab} et \textit{cd} ...
%... \textbf{12} Rursus, cum sint ... a vertice coni.
%\hrule width 1cm
%\textsuperscript{1}\qquad { Si~$\sim$~\textbf{12} coni}{\slnot : om. B}
%\stelle

\begin{maurotex}
\Unit \VV[longa]{
                 {*:\CR{gattocracra}:Si}
                 {B:\OM}
                }
                 duae rectae ... erunt aequales.
\Unit Sint \(ab\) et \(cd\) ... \Unit Rursus, cum sint ... a vertice
\LB{gattocracra}{coni}.
\end{maurotex}

\noindent dove la $\sim$ in nota indica la
sospensione del testo e il numero in neretto indica il
paragrafo in cui testo omesso da B termina\footnote{ In
questo caso ci {\`e} sembrato preferibile non indicare in nota
``Si $\sim$ \textbf{12} coni \textsl{A C\/}''perch{\'e}
questa nota non riguarda tanto ci{\`o} che dicono A e C quanto
il fatto che B {\`e} diverso da loro. Nell'esempio B omette ma
potrebbe avere un testo diverso, o omettere in lacuna, o
mancare due fogli di testo, ecc. L'editore, tuttavia, che
ritenesse opportuno non discostarsi dall'apparato positivo
(cfr. \S\,\ref{ref-4.1}), soprattutto nel caso in cui disponesse di due
soli testimoni, lo potr{\`a} fare, come si vedr{\`a} tra breve.}.

Quanto detto ora, vale evidentemente non solo per le
varianti fra i testimoni, ma anche per le aggiunte marginali o
(pi{\'u} raramente) interlineari e per le correzioni
fatte da una delle mani del testimone; e, infine,  per gli interventi
congetturali dell'editore che potrebbe, ad esempio, trovarsi
nella necessit{\`a} di espungere un brano consistente di testo.
Il presente capitolo
{\`e} quindi suddiviso in tre parti:

\begin{itemize}

\item[\textbf{1.}] Varianti, omissioni e lacune materiali lunghe

\item[\textbf{2.}] Integrazioni e correzioni lunghe (effettuate
dai copisti o dalle mani intervenute sui codici)

\item[\textbf{3.}] Interventi congetturali lunghi (effettuati
dall'editore)

\end{itemize}

%-----------------------------------------------------------------------
\section{Varianti, omissioni e lacune materiali lunghe}

\label{ref-7.1}

\subsection{Una struttura \texttt{[longa]}}

\label{ref-7.1.1}

\subsubsection{A cosa serve un campo}

\label{ref-7.1.1.1}

Per far in modo di spiegare al {\mtex} la situazione
descritta nell'esempio appena fatto
e per potere  poi ricostruire il testo dei vari testimoni
occorre introdurre delle nuove macro. Quella fondamentale
{\`e} \verb"\VV[longa]"\index{\bs{}VV[longa]} (Variante Lunga), che ha la seguente
struttura:

\begin{quote}
\begin{verbatim}
\VV[longa]{
          {.a.}{.b1.}{.b2.}...{.bn.}
          }
\end{verbatim}
\end{quote}

Il campo indicato con \verb".a." serve a far conoscere al {\mtex}
la lunghezza del brano e a produrre di conseguenza la prima
parte della nota. I campi indicati con
\verb"{.b1.}{.b2.}...{.bn.}"
servono invece a
indicare che cosa succede nei vari testimoni  non accolti
nel testo critico e nelle loro
mani: nell'Esempio $\Lambda$, B
omette semplicemente, ma potrebbe ad esempio, omettere in lacuna. Inoltre
potrebbe esserci un quarto testimone, D, che dove B omette,
presenta un testo completamente diverso da A e da C, ecc.

In altre parole, la struttura di \verb"\VV[longa]"\index{\bs{}VV[longa]} {\`e} la stessa di
quella di \verb"\VV"\index{\bs{}VV}, salvo per il campo \verb"{.a.}" che ha il ruolo
speciale di marcare l'inizio della variante lunga. Inoltre,
il campo \verb"{.a.}", che resta pur sempre il primo campo di
\verb"\VV[longa]"\index{\bs{}VV[longa]}, {\`e} quello che provvede a stabilire il testo
critico: la differenza rispetto a quanto normalmente si fa
con \verb"\VV"\index{\bs{}VV} {\`e} che in questa situazione il testo critico si
trova \textit{all'esterno} delle parentesi graffe che
delimitano \verb"\VV[longa]"\index{\bs{}VV[longa]}.

Per chiarezza cominceremo coll'illustrare l'Esempio
$\Lambda$ da cui
siamo partiti, che {\`e} relativamente semplice. In questo caso,
dunque, avremo bisogno solo di due campi, \verb"{.a.}" e \verb"{.b.}".

%-----------------------------------------------------------------------
\subsubsection{La struttura dei campi}

\label{ref-7.1.1.2}

Per
indicare all'{\mtex} la lunghezza del brano c'{\`e} ovviamente
bisogno di un segnale di inizio e di uno di fine brano. Il
segnale di inizio brano viene posto nel primo campo di
\verb"\VV[longa]"\index{\bs{}VV[longa]}; quello di fine pi{\'u} avanti, nel punto del testo
dove il brano finisce (lapalissianamente). Per ottenere il testo
dell'esempio su esposto, si dovr{\`a} scrivere:

\begin{quote}
\begin{verbatim}
\Unit \VV[longa]{
                {*:\CR{gatto}:Si}
                {B:\OM}
                }
                duae rectae ... erunt aequales.
\Unit Sint \(ab\) et \(cd\) ... \Unit Rursus, cum sint
... a vertice
\LB{gatto}{coni}.
\end{verbatim}
\end{quote}

Le due etichette che segnalano l'inizio e  la fine
del brano sono appunto le due nuove macro qui introdotte,
\verb"\CR"\index{\bs{}CR} e \verb"\LB"\index{\bs{}LB}: le analizzeremo tra un attimo con pi{\'u}
dettaglio.

Vediamo prima meglio il campo \verb".a." di \verb"\VV[longa]"\index{\bs{}VV[longa]},  quello cio{\`e}
che contiene \verb"{*:\CR{gatto}:Si}":

\begin{itemize}

\item[\texttt{*}~]  sta al posto di A e C, i testimoni su cui si
basa il testo critico che, come abbiamo osservato, sopra non
sono riportati in nota;

\item[\texttt{CR}] (come Cross Reference) indica
l'inizio del brano e tra poco vedremo perch{\'e} debba contenere
un \verb"gatto";

\item[\texttt{Si}]  {\`e} la prima parola dell'inizio del brano.
Ove lo si ritenga opportuno, nel terzo sottocampo si
potrebbero inserire pi{\'u} parole. Ad esempio, se si volesse che
in apparato comparisse: ``Si duae rectae~$\sim$~coni'',
basterebbe scrivere \verb"{*:\CR{gatto}: Si duae rectae}" e il
richiamo di nota verrebbe apposto dopo ``rectae''.

\end{itemize}

Si noti inoltre che, se l'editore volesse far esplicitamente
comparire in apparato l'informazione che il testo
``Si~$\sim$~coni''(indicato in modo abbreviato) {\`e} il testo
di A e di
C, dovrebbe semplicemente mettere \verb"A/C" al posto della
\verb"*", esattamente come abbiamo imparato a fare nel caso di
\verb"\VV"\index{\bs{}VV}. {\`E} bene anche osservare che l'inizio del brano
(``Si'')  compare all'interno di \verb"\VV[longa]"\index{\bs{}VV[longa]}, esattamente
come la lezione che deve andare nel testo critico compare
all'interno di \verb"\VV"\index{\bs{}VV}. Se ne tenga conto nel regolare
l'apposizione dei segni di interpunzione e degli spazi fra
le parole: si osservino bene gli esempi che verranno
presentati.

Passiamo ora al campo \verb".b." di \verb"\VV[longa]"\index{\bs{}VV[longa]}, quello in cui ora {\`e}
scritto \verb"{B:\OM}". Qui, come si {\`e} gi{\`a} detto, si pu{\`o}
scrivere qualunque cosa autorizzata dal presente manuale e
dalla situazione. Ad esempio se
B presentasse un testo completamente diverso da quello di A
e C si scriverebbe \verb"{B:`testo di B'}"; se B presentasse una
lacuna di due fogli \verb"{B:\DES{duo folia desunt}:\LACm}" e via di
seguito, secondo la sintassi di \verb"\VV"\index{\bs{}VV} descritta in
precedenza.

%-----------------------------------------------------------------------
\subsubsection{Le regole dell'etichetta}

\label{ref-7.1.1.3}
\index{etichette}

Veniamo infine a \verb"\CR"\index{\bs{}CR} e a \verb"\LB"\index{\bs{}LB} (per ``Label'', etichetta)
e al misterioso \verb"gatto" che compare nell'esempio. Come si
sar{\`a} notato la stessa stringa \verb"gatto" compare sia
all'interno del campo di \verb"\CR"\index{\bs{}CR}, sia all'interno di uno dei
due campi di \verb"\LB"\index{\bs{}LB}.

A cosa serve? Poich{\'e} nel testo potrebbero comparire
molte varianti lunghe e quindi l'{\mtex} non riuscirebbe pi{\'u} a
distinguere l'inizio dell'una e dell'altra, la stringa
\verb"gatto" {\`e} in effetti un'\textit{etichetta}, cio{\`e} un \textit{nome
convenzionale} che il trascrittore decide per fare in modo
che \textbf{quella} \verb"\CR"\index{\bs{}CR} corrisponda esattamente a \textbf{quella} \verb"\LB"\index{\bs{}LB}. Il nome, essendo convenzionale, pu{\`o} essere
scelto con una certa libert{\`a} dal trascrittore: la cosa
importante {\`e} che vengano rispettate le tre regole seguenti:

\begin{enumerate}

\item lo \textbf{stesso} nome convenzionale deve
comparire nel campo di \verb"\CR"\index{\bs{}CR} e nel primo campo di \verb"\LB"\index{\bs{}LB}. Se
in \verb"\CR"\index{\bs{}CR} c'{\`e} \verb"gatto" in \verb"\LB"\index{\bs{}LB} deve esserci \verb"gatto"; se in
\verb"\CR"\index{\bs{}CR} c'{\`e} \verb"coniglio" in \verb"\LB"\index{\bs{}LB} dovr{\`a} esserci \verb"coniglio";

\item Ogni volta che si usa \verb"\VV[longa]"\index{\bs{}VV[longa]} si deve
usare un nome convenzionale \textbf{diverso}. Se il
testo presenta due varianti lunghe e nella prima si {\`e} usato
\verb"leone", usate \verb"tigre" nella seconda;

\item Il nome convenzionale deve essere costituito da
caratteri alfabetici (maiuscoli o minuscoli) o numerici.
Ogni altro tipo di carattere {\`e} vietato. In particolare {\`e} \textbf{vietatissimo lo spazio}. Potrete usare \verb"cane1" se volete, ma
non \verb"cane_1" e mai e poi mai \verb"cane 1".

\end{enumerate}

Resta infine da esaminare meglio la struttura di
\verb"\LB"\index{\bs{}LB}. Nell'esempio era \verb"\LB{gatto}{coni}"\index{\bs{}LB}. Come si
{\`e} visto, \verb"\LB"\index{\bs{}LB} {\`e} una macro a due campi che deve
essere collocata in corrispondenza della fine del testo della
variante lunga. Del primo campo di \verb"\LB"\index{\bs{}LB} si {\`e} appena
detto; il secondo contiene \textbf{l'ultima parola} del testo
della variante lunga (in questo caso \verb"coni"). Cio{\`e} il
secondo campo di \verb"\LB"\index{\bs{}LB} {\`e} il corrispettivo del terzo
sottocampo del campo \verb".a."  di \verb"\VV[longa]"\index{\bs{}VV[longa]}. Come
nel caso del campo \verb".a.", se il
trascrittore ritenesse opportuno far riportare in nota
pi{\'u} di una parola,  lo pu{\`o} fare anche qui. Se, per
esempio,
pensasse che riportare in nota soltanto \verb"coni" rischi
di generare difficolt{\`a} al lettore, potrebbe scrivere:

\begin{quote}
\begin{verbatim}
\LB{gatto}{a vertice coni}
\end{verbatim}
\end{quote}

\noindent e ottenere in apparato:

%\stelle
%\textsuperscript{1}\qquad { Si~$\sim$~\textbf{12}
%a vertice coni}{\slnot : om. B}
%\stelle

\begin{maurotex}
\Unit \VV[longa]{
                {*:\CR{gattored}:Si}
                {B:\OM}
                }
                 duae rectae ... erunt aequales.
\Unit Sint \(ab\) et \(cd\) ... \Unit Rursus, cum sint ...
\LB{gattored}{a vertice coni}.
\end{maurotex}

%ici on triche a cause d'un probl\`eme de unit

Si noti poi che se la variante lunga fosse
tutta contenuta all'interno di un solo paragrafo, il
programma ometter{\`a} automaticamente il numero di paragrafo
producendo una nota del tipo ``Si~$\sim$~coni\textsl{: om. B}''.

Nel secondo campo di \verb"\LB{}{}"\index{\bs{}LB} {\`e} contenuta la lezione che andr{\`a}
nel testo critico, cio{\`e} l'ultima o le ultime parole del
brano in questione: tale lezione {\`e} \textit{interna} a \verb"\LB"\index{\bs{}LB} e
non esterna. Valgono al proposito le stesse avvertenze
che abbiamo fatto per l'inizio del brano, in particolar modo
per ci{\`o} che riguarda la punteggiatura. %

A questo proposito,  {\`e}
opportuno ricordare che la punteggiatura finale non viene riportata in
apparato (cfr. \S\,\ref{ref-4.2.3}). Di conseguenza i segni di punteggiatura si
devono trovare \textbf{fuori} dal secondo campo di \verb"\LB"\index{\bs{}LB}, come
nell'esempio che abbiamo riportato: ``\verb"\LB{gatto}{a vertice coni}."\index{\bs{}LB}'', e
\textbf{non si deve} invece scrivere ``\verb"\LB{gatto}{a vertice coni.}"\index{\bs{}LB}'' con il
segno di punteggiatura interno alla parentesi graffa.

%-----------------------------------------------------------------------
\subsubsection{$\backslash$Unit e le varianti lunghe. Compilazione delle varianti lunghe}

\label{ref-7.1.1.4}

Bisogna sottolineare che l'{\mtex} ha assolutamente
necessit{\`a} dei numeri di paragrafo per poter gestire
correttamente le varianti lunghe. Come si ricorder{\`a}
(\S\,\ref{ref-3.7.2}), i numeri di paragrafo sono generati dal comando
\verb"\Unit{}"\index{\bs{}Unit}, che genera un contatore automatico.

Anche se {\`e} compito dell'editore procedere alla suddivisione
del testo in paragrafi, il trascrittore che si imbattesse in
una variante lunga che abbracci pi{\'u} di una frase, potr{\`a} in
ogni caso gestirla collocando almeno una \verb"\Unit{}"\index{\bs{}Unit}
all'inizio del testo e un'altra subito prima dell'ultima
frase del testo che vuole compilare. In questo modo sar{\`a}
possibile far compilare all'{\mtex} il file; penser{\`a}
poi l'editore alla suddivisione in paragrafi vera e propria.

Sempre a proposito di compilazione, per produrre un testo
corretto nel caso sia stata utilizzata \verb"\VV[longa]"\index{\bs{}VV[longa]}, \textbf{l'{\mtex} ha bisogno che il \textit{file} venga
compilato due volte consecutive.}

%-----------------------------------------------------------------------
\subsubsection{Un esempio un po' pi{\'u} complesso: lacune
materiali lunghe}

\label{ref-7.1.1.5}
\index{lacune materiali lunghe}

Per chiarezza riteniamo opportuno aggiungere un esempio
diverso da quello su cui ci siamo basati fino ad ora. Oltre
ad illustrare le varie situazioni che abbiamo qui sopra
spiegato, mostra anche come comportarsi nel caso di una
lacuna materiale lunga, ma non irrimediabile (cfr. \S\,\ref{ref-4.4.1}).
Si abbia un testo tr{\`a}dito da tre testimoni che iniziano cos{\'\i}:

\begin{itemize}

\item[A:] Palam est quod in portione parabolica possibile est
inscribere polygonium rectilineum ita ut relictae portiones
sint minus omni proposito spatio. Nam intra portionem ...

\item[B:] Lascia due righe bianche poi ha: ``Nam intra
portionem~...''

\item[C:] Proculdubio possumus in parabolico segmento
inscribere iam dictam figuram rectilineam ita ut excessum
parabolici segmenti minorem sit quacunque proposita
magnitudine. Nam intra portionem~...

\end{itemize}

TC segue A e si vorr{\`a} quindi ottenere:

%\stelle
%\textbf{1} Palam est\textsuperscript{1} quod in
%portione parabolica possibile est inscribere polygonium rectilineum ita
%ut relictae portiones sint minus omni proposito spatio. \textbf{2} Nam %intra portionem~...
%\hrule width 1cm
%\textsuperscript{1}\qquad { Palam est~$\sim$~omni proposito
%spatio} {\slnot A} { Proculdubio possumus in
%parabolico segmento inscribere iam dictam figuram
%rectilineam ita ut excessum parabolici segmenti minorem sit
%quacunque proposita magnitudine} {\slnot C spatium duorum
%versuum relicto om. B}
%\stelle

\begin{maurotex}
\Unit \VV[longa]{
                {A:\CR{gufo}:Palam est}
                {C:Proculdubio possumus in
                 parabolico segmento
                 inscribere iam dictam figuram
                 rectilineam ita ut excessum
                 parabolici segmenti minorem
                 sit quacunque proposita
                 magnitudine}
                 {B:\DES{spatium duorum versuum
                 relicto om.}:\LACm}
                }
               quod in portione parabolica possibile
est inscribere polygonium rectilineum ita ut relictae
portiones sint minus \LB{gufo}{omni
proposito spatio}. \Unit  Nam intra portionem ...
\end{maurotex}

Per ottenere questo risultato bisogner{\`a} scrivere:

\begin{quote}
\begin{verbatim}
\Unit \VV[longa]{
                {A:\CR{gufo}:Palam est}
                {C:Proculdubio possumus in
                 parabolico segmento
                 inscribere iam dictam figuram
                 rectilineam ita ut excessum
                 parabolici segmenti minorem
                 sit quacunque proposita
                 magnitudine}
                 {B:\DES{spatium duorum versuum
                 relicto om.}:\LACm}
                 }
                  quod in portione parabolica possibile
est inscribere polygonium rectilineum ita ut relictae
portiones sint minus \LB{gufo}{omni
proposito spatio}. \Unit  Nam intra portionem ...
\end{verbatim}
\end{quote}

Si noti che questo esempio illustra anche come
nel caso di \textit{lacune materiali lunghe} in una parte
della tradizione occorra utilizzare la macro \verb"\LACm"\index{\bs{}LACm} in
combinazione con \verb"\DES{}"\index{\bs{}DES} (invece
di \verb"\NL"\index{\bs{}NL} o di \verb"\OMLAC"\index{\bs{}OMLAC}), dando a \verb"\DES"\index{\bs{}DES} come
argomento una delle formule elencate nel \S\,\ref{ref-4.4.2}.

%-----------------------------------------------------------------------
\subsection{Varianti puntuali all'interno di varianti lunghe.  Varianti
lunghe che si sovrappongono
parzialmente}

\label{ref-7.1.2}

\subsubsection{Varianti puntuali}

\label{ref-7.1.2.1}
\index{varianti puntuali}

Tutto il sistema descritto nel precedente \S\,\ref{ref-7.1.1} permette
di dar conto di eventuali varianti puntuali fra i
testimoni
all'interno del brano che {\`e} segnalato da \verb"\VV[longa]"\index{\bs{}VV[longa]}.

Riprendiamo l'Esempio $\Lambda$ e supponiamo che si
verifichi in pi{\'u} questa situazione:

\begin{quote}
A legga: ``Sint \textit{ab} \textbf{et} \textit{cd}''\\
C legga: ``Sint \textit{ab} \textbf{est} \textit{cd}''.

\end{quote}

\noindent mentre B ometta come prima tutto il passo. Si
vorr{\`a} allora ottenere:

%\stelle
%\textbf{10} Si\textsuperscript{1} duae rectae ...  erunt aequales.
%\textbf{11} Sint \textit{ab} et\textsuperscript{2} \textit{cd} ...
%... \textbf{12} Rursus, cum sint ... a vertice coni.
%
%\hrule width 1cm
%
%\textsuperscript{1}\qquad { Si~$\sim$~\textbf{12} coni}
%{\slnot om. B}
%
%\textsuperscript{2}\qquad { et} A { est} C
%
%\stelle

\begin{maurotex}
\Unit \VV[longa]{
                {*:\CR{gatto}:Si}{B:\OM}
                } duae rectae ... erunt aequales.
\Unit Sint \(ab\) \VV{
                     {A:et}{C:est}
                     } \(cd\) ... \Unit ... a vertice
\LB{gatto}{coni}.
\end{maurotex}

La situazione verr{\`a} trattata in questo modo:

\begin{quote}
\begin{verbatim}
\Unit \VV[longa]{
                {*:\CR{gatto}:Si}{B:\OM}
                } duae rectae ... erunt aequales.
\Unit Sint \(ab\) \VV{
                     {A:et}{C:est}
                     } \(cd\) ... \Unit ... a vertice
\LB{gatto}{coni}.
\end{verbatim}
\end{quote}

\noindent inserendo una \verb"\VV{}"\index{\bs{}VV} nel punto opportuno.

%-----------------------------------------------------------------------
\subsubsection{Varianti puntuali nelle ultime parole}

\label{ref-7.1.2.2}
\index{varianti puntuali nelle ultime parole}

Un caso particolare della situazione ora esposta {\`e} quello in
cui i testimoni si trovino a differire proprio nel punto cui
inizia o finisce la variante lunga. (Vi auguriamo che non vi
capiti, ma, nel caso, ecco i rimedi.)

Vediamo prima un esempio di varianti nell'ultima parola.
Alla fine del passo omesso da B (Esempio $\Lambda$) A e C
leggano:

\begin{itemize}

\item[A:] a vertice coni

\item[C:] a vertice cani

\end{itemize}

La cosa potr{\`a} essere trattata cos{\'\i}:

\begin{quote}
\begin{verbatim}
\Unit \VV[longa]{
                {A/C:\CR{gattoverde}:Si}
				{B:\OM}
                } duae rectae ... erunt aequales. \Unit
Sint \(ab\) et \(cd\) ... \Unit Rursus, cum sint... a
vertice \LB{gattoverde}{\VV{
                           {A:coni}{C:cani}
                           }
                       }.
\end{verbatim}
\end{quote}

Come si vede, per dar conto della variante puntuale
nell'ultima parola si {\`e} dovuto inserire una \verb"\VV{}"\index{\bs{}VV}
all'interno del secondo campo di \verb"\LB{}{}"\index{\bs{}LB}. Il risultato
sar{\`a} il seguente:

%\stelle
%\textbf{10} Si\textsuperscript{1} duae rectae ...  erunt aequales.
%\textbf{11} Sint \textit{ab} et \textit{cd} ...
%... \textbf{12} Rursus, cum sint ... a vertice coni\textsuperscript{2}.
%
%\hrule width 1cm
%
%\textsuperscript{1}\qquad { Si~$\sim$~\textbf{12}
%coni}{\slnot AC om. B}
%
%\textsuperscript{2}\qquad { coni} {\slnot A} { cani}
%{\slnot C}
%
%\stelle

\begin{maurotex}
\Unit \VV[longa]{
                {A/C:\CR{gattoverde}:Si}
				{B:\OM}
                } duae rectae ... erunt aequales. \Unit
Sint \(ab\) et \(cd\) ... \Unit Rursus, cum sint... a
vertice \LB{gattoverde}{\VV{
                           {A:coni}{C:cani}
                           }
                       }.
\end{maurotex}

%-----------------------------------------------------------------------
\subsubsection{Varianti puntuali nella prima parola}

\label{ref-7.1.2.3}
\index{varianti puntuali nella prima parola}

Si potrebbe pensare di usare un sistema analogo nel
disgraziatissimo caso in cui A e C presentino varianti proprio
nella prima parola di una variante lunga rispetto a B.
Bisogna per{\`o} aver cura di evitare di produrre due esponenti di
nota appiccicati, quello della variante lunga e quello della
variante puntuale. Questo implica che si debba, per cos{\'\i}
dire, scrivere una nota nella nota, ottenendo un apparato di
questo tipo: ``La vispa (vespa \textsl{C}\/)~$\sim$~farfalletta
\textsl{AC om. B\/}''. Vediamo un esempio pi{\'u} in dettaglio

Suppponiamo che all'inizio del brano dell'esempio \textbf{$\Lambda$} (brano che come al solito B omette) si abbia:

\begin{itemize}

\item[A:] Si duae rectae
\item[C:] Sunt duae rectae

\end{itemize}

Ci{\`o} che si vuole ottenere {\`e}:

%\stelle
%\textbf{10} Si duae rectae\textsuperscript{1} ...  erunt aequales.
%\textbf{11} Sint \textit{ab} et \textit{cd} ...
%... \textbf{12} Rursus, cum sint ... a vertice coni.
%
%\hrule width 1cm
%
%\textsuperscript{1}\qquad { Si (Sunt {\slnot C\/}) duae
%rectae~$\sim$~\textbf{12} coni} {\slnot AC~~om. B}
%
%\stelle

\begin{maurotex}
\Unit \VV[longa]{
                {A/C:\CR{gattogiallo}:\VV{
                                         {A:Si}
                                         {C:Sunt}
                                         } duae rectae
                }
                {B:\OM}
                } ... erunt aequales. \Unit Sint \(ab\)
et \(cd\) ... \Unit Rursus, cum sint ... a vertice
\LB{gattogiallo}{coni}.
\end{maurotex}

Abbiamo gi{\`a} trovato un esempio di questa tecnica
delle ``note nelle note'', e precisamente nel \S\,\ref{ref-5.4.2} quando
abbiamo spiegato come trattare glosse marginali che
presentino varianti. In quel caso si annidava
\verb"\VV{}"\index{\bs{}VV} all'interno di \verb"\NOTAMARG{}"\index{\bs{}NOTAMARG}, qui si deve annidare
\verb"\VV{}"\index{\bs{}VV} all'interno di \verb"\VV[longa]"\index{\bs{}VV[longa]}. Si proceder{\`a} in questo
modo:

\begin{quote}
\begin{verbatim}
\Unit \VV[longa]{
                {A/C:\CR{gatto}:\VV{
                                   {A:Si}
                                   {C:Sunt}
                                   } duae rectae
                }
                {B:\OM}
                } ... erunt aequales. \Unit Sint \(ab\)
et \(cd\) ... \Unit Rursus, cum sint ... a vertice
\LB{gatto}{coni}.
\end{verbatim}
\end{quote}

Per quanto riguarda poi le regole con cui annidare \verb"\VV"\index{\bs{}VV}
all'interno di \verb"\VV[longa]"\index{\bs{}VV[longa]} esse sono esattamente le stesse
di quelle enunciate per annidare \verb"\VV"\index{\bs{}VV} all'interno di
\verb"\NOTAMARG"\index{\bs{}NOTAMARG}.  Rinviamo il lettore che si imbatta
in questa disgraziata necessit{\`a} al \S\,\ref{ref-5.4.2}.

%-----------------------------------------------------------------------
\subsubsection{Varianti lunghe accavallate}

\label{ref-7.1.2.4}
\index{varianti lunghe accavallate}

Il lettore avr{\`a} capito che la casistica completa degli accavallamenti
di varianti lunghe e puntuali {\`e} alquanto complessa e non
intendiamo certo trattarla analiticamente. Segnaliamo per{\`o}
un caso che richiede un trattamento un po' diverso dal
solito. Si supponga che nella situazione
dell'Esempio~$\Lambda$ si abbiano, invece di 3, 4 testimoni:

\begin{itemize}
\item[A:] Si duae rectae ...  erunt aequales. Sint \textit{ab} et \textit{cd} ... ... Rursus, cum sint ... a vertice
coni. [\textit{il passo {\`e} complessivamente lungo 8 righe}]

\item[C:] \textit{stesso testo di A}

\item[B:] \textit{omette il brano fino a} ``erunt aequales''.
\textit{Poi prosegue come A e C:} Sint \textit{ab} et \textit{cd} ... ... Rursus, cum sint ... a vertice
coni.

\item[D:] \textit{omette tutto il brano, cio{\`e} fino a} ``a vertice
coni''

\end{itemize}


Ci troviamo dunque in presenza di due varianti
lunghe (per la precisione, di due omissioni) con lo stesso
punto iniziale ma diverso punto finale. Si vorrebbe
ottenere un apparato di questo tipo:

%\stelle
%\textbf{10} ... Si\textsuperscript{1} duae rectae ...  erunt aequales.
%\textbf{11} Sint \textit{ab} et \textit{cd} ...
%... \textbf{12} Rursus, cum sint ... a vertice coni.
%
%\hrule width 1cm
%
%\textsuperscript{1}\qquad { Si~$\sim$~\textbf{12} a vertice
%coni}{\slnot : om. D} { Si~$\sim$~aequales}{\slnot : om. B}
%
%\stelle

\begin{maurotex}
\Unit \VV[longa]{
                {*:\CR{gatto}:Si}
                {D:\OM}
                {*:\CR{cane}:Si}
                {B:\OM}
                }
            duae rectae ... erunt \LB{cane}{aequales}.
\Unit Sint \(ab\) et \(cd\) ... \Unit ... 
\LB{gatto}{a vertice coni}.
\end{maurotex}

Per ottenere questa nota bisogner{\`a} forzare la struttura di base di
\verb"\VV[longa]"\index{\bs{}VV[longa]} inserendo due
campi di tipo \verb".a." e due di tipo \verb".b.":

\begin{quote}
\begin{verbatim}
\VV[longa]{
          {.a1.}{.b1.}
          {.a2.}{.b2.}
          }
\end{verbatim}
\end{quote}

\noindent utilizzando due \verb"\CR"\index{\bs{}CR} (ovviamente con etichette diverse:
\verb"gatto" e \verb"cane", nell'esempio), in questo modo:

\begin{quote}
\begin{verbatim}
\Unit \VV[longa]{
                {*:\CR{gatto}:Si}
                {D:\OM}
                {*:\CR{cane}:Si}
                {B:\OM}
                }
                duae rectae ... erunt \LB{cane}{aequales}.
\Unit Sint \(ab\) et \(cd\) ... \Unit ... 
\LB{gatto}{a vertice coni}.
\end{verbatim}
\end{quote}

%-----------------------------------------------------------------------
\subsection{Ripetizioni}

\label{ref-7.1.3}
\index{ripetizioni}

Un particolare tipo di variante lunga {\`e} la ripetizione.
Supponiamo, per semplicit{\`a}, di avere due soli testimoni A e
B:

\begin{itemize}

\item[A:] tanta est inter curvum et rectum fortasse propter
dissimilitudinem, inimicitia.

\item[B:] tanta est inter curvum et rectum fortasse propter
dissimilitudinem inter curvum et rectum fortasse propter
dissimilitudinem, inimicitia.

\end{itemize}

Di fronte a questa situazione vorremmo ottenere la seguente
nota:

%\stelle
%tanta est inter\textsuperscript{1} curvum et rectum fortasse propter
%dissimilitudinem, inimicitia.
%
%\hrule width 1cm
%
%\textsuperscript{1}\qquad { inter~$\sim$~dissimilitudinem}{\slnot : bis B}
%\stelle

\begin{maurotex}
tanta est \VV[longa]{
                    {*:\CR{miaoverde}:inter}{B:\BIS:}
                    }
                    curvum et rectum fortasse propter
\LB{miaoverde}{dissimilitudinem}, inimicitia.
\end{maurotex}

\noindent cosa che si potr{\`a} ottenere usando \verb"\VV[longa]"\index{\bs{}VV[longa]} nel modo
descritto sopra e la  macro \verb"\BIS"\index{\bs{}BIS} che abbiamo gi{\`a}
incontrato nel \S\,\ref{ref-4.6}:

\begin{quote}
\begin{verbatim}
tanta est \VV[longa]{
                    {*:\CR{miao}:inter}{B:\BIS:}
                    }
                    curvum et rectum fortasse propter
\LB{miao}{dissimilitudinem}, inimicitia.
\end{verbatim}
\end{quote}

Si noti che potrebbe avvenire che il testo di B fosse
ripetuto, ma con varianti rispetto agli altri testimoni. Tali varianti
si potranno trovare  nella ripetizione del testo, come ad
esempio:

\begin{itemize}

\item[B:] tanta est inter \textit{curvum} et rectum fortasse
propter dissimilitudinem inter \textit{circunferentiam} et rectum fortasse propter
dissimilitudinem, inimicitia.

\end{itemize}

\noindent oppure nella prima occorrenza, come:

\begin{itemize}

\item[B:] tanta est inter  curvum et \textit{rectilineum} fortasse
propter dissimilitudinem inter
circunferentiam et \textit{rectum} fortasse propter
dissimilitudinem, inimicitia.

\end{itemize}

In tal caso si provveder{\`a} ad apporre una \verb"\VV"\index{\bs{}VV} accanto a
``curvum'' indicando cosa succede rispettivamente con l'espressione 
\textit{altero
loco B} o \textsl{priore loco B}. Per produrre tale risultato
si utilizzer{\`a} una nuova macro, \verb"\REP"\index{\bs{}REP}. Tale macro {\`e} dotata
di un'opzione e la sua sintassi completa {\`e} \verb"\REP[1]"\index{\bs{}REP} o
\verb"\REP[2]"\index{\bs{}REP} a seconda che si voglia ottenere \textsl{priore loco} o
\textsl{altero loco}. Di conseguenza la
trascrizione in {\mtex} del primo passo diventer{\`a}:

\begin{quote}
\begin{verbatim}
tanta est \VV[longa]{
                    {*:\CR{miao}:inter}
                    {B:\BIS:}
                    }
                    \VV{
                       {A/B:curvum}
                       {B:\REP[2]:circunferentiam}
                       } et rectum fortasse propter
\LB{miao}{dissimilitudinem}, inimicitia.
\end{verbatim}
\end{quote}

\noindent producendo cos{\'\i}

%\stelle
%tanta est inter\textsuperscript{1} curvum\textsuperscript{2} et rectum fortasse propter
%dissimilitudinem, inimicitia.
%
%\hrule width 1cm
%
%\textsuperscript{1}\qquad {
%inter~$\sim$~dissimilitudinem}{\slnot : bis B}
%
%\textsuperscript{2}\qquad { curvum} {\slnot AB}
%{ circunferentiam} {\slnot altero loco B}
%\stelle

\begin{maurotex}
tanta est \VV[longa]{
                    {*:\CR{miaorosso}:inter}
                    {B:\BIS:}
                    }
                    \VV{
                       {A/B:curvum}
                       {B:\REP[2]:circunferentiam}
                       } et rectum fortasse propter
\LB{miaorosso}{dissimilitudinem}, inimicitia.
\end{maurotex}

Nel secondo caso si scriver{\`a} in {\mtex}:

\begin{quote}
\begin{verbatim}
tanta est \VV[longa]{
                    {*:\CR{miao}:inter}
                    {B:\BIS:}
                    }
                    curvum et \VV{
                                 {A/B:rectum}
                                 {B:\REP[1]:rectilineum}
                                 } fortasse propter
\LB{miao}{dissimilitudinem}, inimicitia.
\end{verbatim}
\end{quote}

\noindent ottenendo come risultato

%\stelle
%tanta est inter\textsuperscript{1} curvum et rectum\textsuperscript{2} fortasse propter
%dissimilitudinem, inimicitia.
%
%\hrule width 1cm
%
%\textsuperscript{1}\qquad {
%inter~$\sim$~dissimilitudinem}{\slnot : bis B}
%
%\textsuperscript{2}\qquad { rectum} {\slnot AB}
%{ rectilineum} {\slnot priore loco B}
%\stelle

\begin{maurotex}
tanta est \VV[longa]{
                    {*:\CR{miaomarone}:inter}
                    {B:\BIS:}
                    }
                    curvum et \VV{
                                 {A/B:rectum}
                                 {B:\REP[1]:rectilineum}
                                 } fortasse propter
\LB{miaomarone}{dissimilitudinem}, inimicitia.
\end{maurotex}

Se poi A fosse un testimone unico che presentasse una
ripetizione con varianti
nella ripetizione, ci si comporter{\`a} allo stesso modo, a
seconda della lezione che si sceglie per TC: se la lezione
di A che viene scelta {\`e} la prima si segnaler{\`a} che per{\`o},
\textit{altero loco} (e usando quindi \verb"\REP[2]"\index{\bs{}REP}), ha una lezione
diversa; se la lezione di A che viene accolta {\`e} la seconda
si registrer{\`a} in apparato quella che ha \textit{priore loco},
utilizzando \verb"\REP[1]"\index{\bs{}REP}. Se A legge:

\begin{itemize}

\item[A:] Hic Archimedis de quadratura parabolae libellus ex
corruptissimo, \textit{quod} circumfertur, exemplari ex
corruptissimo, \textit{qui} circumfertur, exemplari, labore et
industria Francisci Maurolyci restitutus est.

\end{itemize}

\noindent e scegliessimo la lezione \textit{quod}, diremo in
apparato:

%\stelle
%Hic Archimedis de quadratura parabolae libellus
%ex\textsuperscript{1} corruptissimo, quod\textsuperscript{2} circumfertur, exemplari,
%labore et industria Francisci Maurolyci restitutus est.
%
%\hrule width 1cm
%
%\textsuperscript{1}\qquad { ex~$\sim$~exemplari}{\slnot : bis A}
%
%\textsuperscript{2}\qquad { quod}{\slnot:} { qui}
%{\slnot altero loco A}
%\stelle

\begin{maurotex}
Hic Archimedis de quadratura parabolae libellus
\VV[longa]{
          {*:\CR{miaonero}:ex}
          {A:\BIS:}
          } corruptissimo, \VV{
                              {*:quod}
                              {A:\REP[2]:qui}
                              }
                              circumfertur
\LB{miaonero}{exemplari}, labore  et industria
Francisci Maurolyci restitutus est.
\end{maurotex}

\noindent scrivendo in {\mtex}

\begin{quote}
\begin{verbatim}
Hic Archimedis de quadratura parabolae libellus
\VV[longa]{
          {*:\CR{miao}:ex}
          {A:\BIS:}
          } corruptissimo, \VV{
                              {*:quod}
                              {A:\REP[2]:qui}
                              }
                              circumfertur
\LB{miao}{exemplari}, labore  et industria
Francisci Maurolyci restitutus est.
\end{verbatim}
\end{quote}

\noindent dove {\`e} da notare il passaggio ad apparato negativo
nell'uso di \verb"\VV"\index{\bs{}VV} (\verb"{*:quod}").

Pi{\'u} semplice sar{\`a} la situazione se, in presenza di
due testimoni, quello che ripete il brano (B) diverge
dall'altro (A; 
o dal testo critico, ma in tal caso si tratter{\`a} di un
intervento congetturale e bisogner{\`a} utilizzare la macro
\verb"\ED"\index{\bs{}ED})
in entrambe le occorrenze nello stesso modo. Ad esempio
se  B, nel brano in cui si parla dell'inimicizia fra retto e
curvo, leggesse sia la prima che la seconda volta
``cirunferentiam''. In tal caso baster{\`a} 
registrare la variante:

%\stelle
%tanta est inter\textsuperscript{1} curvum\textsuperscript{2} et rectum fortasse propter
%dissimilitudinem, inimicitia
%
%\hrule width 1cm
%
%\textsuperscript{1}\qquad {
%inter~$\sim$~dissimilitudinem}{\slnot : bis B}
%
%\textsuperscript{2}\qquad { curvum} {\slnot A}
%{ circunferentiam} {\slnot  B}
%
%\stelle

\begin{maurotex}
tanta est \VV[longa]{
                    {*:\CR{miaobianco}:inter}
                    {B:\BIS:}
                    } \VV{
                         {A:curvum}
                         {B:circunferentiam}
                         } et rectum fortasse propter
\LB{miaobianco}{dissimilitudinem}, inimicitia.
\end{maurotex}

\noindent scrivendo in {\mtex}:

\begin{quote}
\begin{verbatim}
tanta est \VV[longa]{
                    {*:\CR{miao}:inter}
                    {B:\BIS:}
                    } \VV{
                         {A:curvum}
                         {B:circunferentiam}
                         } et rectum fortasse propter
\LB{miao}{dissimilitudinem}, inimicitia.
\end{verbatim}
\end{quote}

Se, rifacendosi all'esempio della quadratura della
parabola, si avesse:

\begin{itemize}

\item[A:] ... libellus ex corsicano quod circumfertur
exemplari ex corsicano quod circumfertur
exemplari labore et industria ...

\end{itemize}

\noindent e l'editore congetturasse ``corrupto'' per
correggere l'insensato ``corsicano'', scriveremmo in {\mtex}

\begin{quote}
\begin{verbatim}
libellus \VV[longa]{
                   {*:\CR{miao}:ex}
                   {A:\BIS:}
                   } \VV{
                        {*:\ED{conieci}:corrupto}
                        {A:corsicano}
                        } quod circumfertur
\LB{miao}{exemplari}, labore et industria Francisci
Maurolyci restitutus est.
\end{verbatim}
\end{quote}

ottenendo:

%\stelle
%libellus
%ex\textsuperscript{1} corrupto\textsuperscript{2}, quod circumfertur, exemplari,
%labore et industria Francisci Maurolyci restitutus est.
%
%\hrule width 1cm
%
%\textsuperscript{1}\qquad { ex~$\sim$~exemplari}{\slnot : bis A}
%
%\textsuperscript{2}\qquad { corrupto} {\slnot conieci}
%{ corsicano} {\slnot A}
%\stelle

\begin{maurotex}
libellus \VV[longa]{
                   {*:\CR{miaorosa}:ex}
                   {A:\BIS:}
                   } \VV{
                        {*:\ED{conieci}:corrupto}
                        {A:corsicano}
                        } quod circumfertur
\LB{miaorosa}{exemplari}, labore et industria Francisci
Maurolyci restitutus est.
\end{maurotex}

%-----------------------------------------------------------------------
\section{Integrazioni e correzioni lunghe dei copisti}

\label{ref-7.2}

\subsection{Uso di $\protect\backslash$VV[longa] per le integrazioni del copista}

\label{ref-7.2.1}
\index{integrazioni del copista}

Spesso le aggiunte del copista (specialmente nel caso che
esso sia Maurolico in persona) sono molto lunghe, a volte
mezze pagine intere. Si pone quindi per esse il problema descritto
all'inizio di questo capitolo, e lo si pu{\`o} risolvere nello
stesso modo.

Supponiamo ad esempio che Maurolico  nel testimone A
(l'unico a nostra disposizione) abbia
aggiunto in margine con un segno di richiamo
dieci
righe di testo (l'aggiunta andr{\`a} quindi marcata con \verb"A+", cfr. \S\,\ref{ref-5.1}):

\begin{itemize}

\item[A\textsuperscript{1}:] Hic est notandum quod si recta $ab$ parallelus
non erit rectae $cd$ ... Quod erat propositum, meliori modo explicatum.

\end{itemize}

\noindent indicando che questo passo doveva andare subito
dopo le parole di A ``ut dicebamus''. Si provveder{\`a} allora
nel seguente modo:

\begin{quote}
\begin{verbatim}
ut dicebamus.
\Unit \VV[longa]{
                {A+:\CR{zebra}\MARGSIGN:Hic}
                }
est notandum quod si recta \(ab\) parallelus
non erit rectae $cd$ ...
\Unit \LB{zebra}{Quod erat propositum},
meliori modo explicatum.
\end{verbatim}
\end{quote}

ottenendo:

%\stelle
%ut dicebamus. \textbf{58} Hic\textsuperscript{1} est notandum quod si
%recta $ab$ parallelus non erit rectae $cd$ ... \textbf{62}
%Quod erat propositum, meliori modo explicatum.
%
%\hrule width 1cm
%
%\textsuperscript{1}\qquad { Hic~$\sim$~\textbf{62} Quod erat
%propositum} {\slnot signo posito in marg. A}
%
%\stelle

\begin{maurotex}
ut dicebamus.
\Unit \VV[longa]{
                {A+:\CR{zebra}\MARGSIGN:Hic}
                }
est notandum quod si recta \(ab\) parallelus
non erit rectae $cd$ ...
\Unit \LB{zebra}{Quod erat propositum},
meliori modo explicatum.
\end{maurotex}

Naturalmente, se ci fossero da indicare altri
interventi di Maurolico o del copista nel passo aggiunto in
margine, lo si potrebbe fare usando \verb"\VV"\index{\bs{}VV}, come spiegato nel
\S\,\ref{ref-7.1.1.4}.

%-----------------------------------------------------------------------
\subsection{Correzioni lunghe del copista o di altre mani}

\label{ref-7.2.2}
\index{correzioni lunghe del copista}

Anche le correzioni di cui si {\`e} trattato nel \S\,\ref{ref-5.3.2},
come le integrazioni del copista di cui si {\`e} appena
detto, potranno essere di un'estensione tale da rendere
sconsigliabile far riportare in nota l'intero passo.

Come si {\`e} visto (\S\,\ref{ref-5.1}), tali correzioni nel testimone
A sono da imputare al copista che corregge s{\'e} stesso (e in
tal caso si utilizzer{\`a} A\textsuperscript{1}), o a una seconda mano (A\textsuperscript{2}),
ecc. Quando si utilizza \verb"\VV"\index{\bs{}VV} tali mani vengono trattate come se fossero un
testimone diverso; cos{\'\i}, allo stesso modo, si
potr{\`a} usare \verb"\VV[longa]"\index{\bs{}VV[longa]} per le correzioni pi{\'u}
ingombranti che non si vogliono ospitare in apparato per
intero. Se il testo di A fosse:

\begin{itemize}

\item[A\hphantom{\textsuperscript{1}}:] Et rursus, per praecedentem,
erit cubus sesquialterus dicti cylindri~...~ut demonstrari
potest. [\textit{5 righe di lunghezza}]

\end{itemize}

\noindent e il copista (o una seconda mano, o anche Maurolico
intervenuto personalmente sul testimone) l'avesse cancellato
con un tratto di penna, scrivendo in margine:

\begin{itemize}

\item[A\textsuperscript{1}:] Et rursus, cum sit cubus sesquitertius dicti
prismatis (demonstratio tota pendet ex
antepraemissa),~{...}~Et sic demonstrabitur: cum
sit~{...}~Quod erat propositum. [\textit{17 righe di
lunghezza}]

\end{itemize}

\noindent daremo in TC il testo della correzione del copista, indicando
in apparato che originariamente A aveva un testo diverso
(che ovviamente riporteremo
per intero) utilizzando come al solito per indicare la correzione del
copista il \textit{siglum} A\textsuperscript{1} (o A\textsuperscript{2} o A\textsuperscript{m}, ecc.):

%\stelle
% ... ut dicebamus. \textbf{112} Et\textsuperscript{1} rursus, cum
%sit cubus sesquitertius dicti prismatis (demonstratio tota
%pendet ex antepraemissa),~{...}~\textbf{117} Et sic
%demonstrabitur: cum
%sit~{...}~\textbf{120} Quod erat propositum.
%
%\hrule width 1cm
%
%\textsuperscript{1}\qquad { Et~$\sim$~\textbf{120} Quod erat
%propositum} {\slnot in marg. A$^{\hbox{\slnotmm 1}}$} {
%Et rursus, per praecedentem, erit cubus sesquialterus dicti
%cylindri~...~ut demonstrari potest.} {\slnot A}
%
%\stelle

\begin{maurotex}
... ut dicebamus.
\Unit \VV[longa]{
                {A1:\CR{lince}\MARG:Et}
                {A:Et rursus, per praecedentem,
                  erit cubus sesquialterus dicti
                  cylindri ... ut demonstrari
                  potest.}
                }
                rursus, cum sit cubus sesquitertius
dicti prismatis (demonstratio tota pendet ex
antepraemissa), ... \Unit Et sic demonstrabitur: cum
sit: ... \Unit \LB{lince}{Quod erat propositum}.
\end{maurotex}

Visto che A\textsuperscript{1} (e cos{\'\i} A\textsuperscript{2}, A\textsuperscript{m}, ecc.) viene
trattato come se fosse un altro testimone, viene subito in
mente  che per segnalarne le correzioni baster{\`a}  usare
\verb"\VV[longa]"\index{\bs{}VV[longa]} secondo le regole descritte nei paragrafi
precedenti. Tuttavia, bisogna anche indicare che il passo
riportato in TC si trova in margine. Per questo occorre
aggiungere una specificazione nel secondo sottocampo del
campo \verb"{.a.}" di  \verb"\VV[longa]"\index{\bs{}VV[longa]}, apponendo \verb"\MARG"\index{\bs{}MARG} subito
dopo \verb"\CR"\index{\bs{}CR}:

\begin{quote}
\begin{verbatim}
... ut dicebamus.
\Unit \VV[longa]{
                {A1:\CR{lince}\MARG:Et}
                {A:Et rursus, per praecedentem,
                  erit cubus sesquialterus dicti
                  cylindri ... ut demonstrari
                  potest.}
                }
                rursus, cum sit cubus sesquitertius
dicti prismatis (demonstratio tota pendet ex
antepraemissa), ... \Unit Et sic demonstrabitur: cum
sit: ... \Unit \LB{lince}{Quod erat propositum}.
\end{verbatim}
\end{quote}

Si segnala cos{\'\i} il fatto che la
correzione {\`e} fatta in margine. Analogamente, se B e C non avessero
affatto tale  passo, si sarebbe aggiunto un altro campo,
scrivendo:

\begin{quote}
\begin{verbatim}
(TC) ... \Unit \VV[longa]{
                {A1:\CR{lince}\MARG:Et}
                {A:Et rursus, per praecedentem,
                  erit cubus sesquialterus dicti
                  cylindri ... ut demonstrari
                  potest.}
                {B/C:\OM}
                }
                rursus, cum
sit cubus sesquitertius dicti prismatis (demonstratio tota
pendet ex antepraemissa), ... \Unit Et sic demonstrabitur:
cum sit ... \Unit \LB{lince}{Quod erat propositum}.
\end{verbatim}
\end{quote}

\noindent e ottenendo il seguuente risultato:

%\stelle
%(TC) ... \textbf{112} Et\textsuperscript{1} rursus, cum
%sit cubus sesquitertius dicti prismatis (demonstratio tota
%pendet ex antepraemissa),~{...}~\textbf{117} Et sic
%demonstrabitur: cum
%sit~{...}~\textbf{120} Quod erat propositum.
%
%\hrule width 1cm
%
%\textsuperscript{1}\qquad { Et~$\sim$~\textbf{120} Quod erat
%propositum} {\slnot in marg. A$^{\hbox{\slnotmm 1}}$} {
%Et rursus, per praecedentem, erit cubus sesquialterus dicti
%cylindri~...~ut demonstrari potest.} {\slnot A; om. B C}
%
%\stelle

\begin{maurotex}
(TC) ... \Unit \VV[longa]{
                {A1:\CR{lince}\MARG:Et}
                {A:Et rursus, per praecedentem,
                  erit cubus sesquialterus dicti
                  cylindri ... ut demonstrari
                  potest.}
                {B/C:\OM}
                }
                rursus, cum
sit cubus sesquitertius dicti prismatis (demonstratio tota
pendet ex antepraemissa), ... \Unit Et sic demonstrabitur:
cum sit ... \Unit \LB{lince}{Quod erat propositum}.
\end{maurotex}

Si osservi che in questo caso si {\`e} inserita un'altra
macro, \verb"\MARG"\index{\bs{}MARG}, nel secondo sottocampo del campo \verb"{.a.}", subito
dopo \verb"\CR{lince}"\index{\bs{}CR}. Questa {\`e} una cosa che si pu{\`o} sempre
realizzare: tutte le macro inseribili nel secondo sottocampo
di \verb"\VV"\index{\bs{}VV} sono inseribili anche qui (ovviamente con dosi
generose di \textit{granu salis}, se non volete ritrovarvi con
testi assai bizzarri.)

%-----------------------------------------------------------------------
\section{Interventi congetturali lunghi}

\label{ref-7.3}

\subsection{Correzioni}

\label{ref-7.3.1}
\index{correzioni congetturali lunghi}

Nel caso di interventi congetturali, si utilizza
sistematicamente l'accoppiamento di due
macro nel secondo sottocampo del campo \verb"{.a.}", dato che
l'editore dovr{\`a} sempre dichiarare esplicitamente la natura
del suo intervento. Cominciamo col vedere un esempio di
correzione congetturale lunga. La tradizione sia costituita
da A e B che leggono:

\begin{itemize}

\item[A:] Ibit ergo paradoxa sesquitertia tripodi eandem
altitudinem et eandem bastionem habentis

\item[B:] Ibit ergo paradoxa per centrum tripodi eandem
basim habentis.

\end{itemize}

Si tratta di due testi evidentemente molto
corrotti. \textit{Ibit, paradoxa, tripodi} non hanno alcun
senso, ma le varianti aggiuntive fra i due testi (\textit{sequitertia}/\textit{per centrum}, \textit{bastionem}/\textit{basim},
l'omissione di \textit{altitudinem} in B, \textit{eandem} che non
concorda con \textit{bastionem} in A) rendono possibile
proporre il seguente testo critico:

\begin{quote}

Erit ergo parabola sesquitertia
trigoni eandem altitudinem et eandem basim habentis.

\end{quote}

Se si volessero in apparato indicare una per una
tutte le varianti fra A e B il brano ricostruito diverrebbe
crivellato di esponenti di nota, rendendo assai difficile al
lettore cogliere il lavoro compiuto per stabilire il testo.
Sar{\`a} forse meglio allora ottenere un apparato di questo
tipo:

%\stelle
%\textbf{75} Erit\textsuperscript{1} ergo parabola sesquitertia
%trigoni eandem altitudine et eandem basim habentis.
%
%\hrule width 1cm
%
%\textsuperscript{1}\qquad { Erit~$\sim$~habentis} {\slnot conieci} { Ibit ergo paradoxa
%sesquitertia tripodi eandem altitudinem et eandem bastionem
%habentis} {\slnot A} { Ibit ergo paradoxa per centrum
%tripodi eandem basim habentis} {\slnot B}
%
%\stelle

\begin{maurotex}
\Unit \VV[longa]{
                {*:\CR{lupo}\ED{conieci}:Erit}
                {A:Ibit ergo paradoxa sesquitertia
                  tripodi eandem altitudinem et
                  eandem bastionem habentis}
                {B:Ibit ergo paradoxa per centrum
                   tripodi eandem basim habentis}
                } ergo parabola sesquitertia trigoni
eandem altitudine et eandem basim \LB{lupo}{habentis}.
\end{maurotex}

Per ottenere questa soluzione si scriver{\`a} in {\mtex}:

\begin{quote}
\begin{verbatim}
\Unit \VV[longa]{
                {*:\CR{lupo}\ED{conieci}:Erit}
                {A:Ibit ergo paradoxa sesquitertia
                  tripodi eandem altitudinem et
                  eandem bastionem habentis}
                {B:Ibit ergo paradoxa per centrum
                   tripodi eandem basim habentis}
                } ergo parabola sesquitertia trigoni
eandem altitudine et eandem basim \LB{lupo}{habentis}.
\end{verbatim}
\end{quote}

Naturalmente, se la restituzione di questo testo
fosse stata fatta da Clagett, nel campo di
\verb"\ED{}"\index{\bs{}ED} si scriverebbe \verb"Clagett", come gi{\`a} detto nel \S\,\ref{ref-6.1}
e suoi sottoparagrafi. Cos{\'\i} se gli editori del testo in
questione fossero pi{\'u} di uno, si scriverebbe \verb"coniecimus"
invece di \verb"conieci", ecc..

Tutti gli esempi e le situazioni del \S\,\ref{ref-6.1} si possono
adattare anche alle correzioni congetturali lunghe,
aggiungendo quanti campi si vogliono al'interno di
\verb"\VV[longa]"\index{\bs{}VV[longa]} dopo il campo \verb"{.a.}"

%-----------------------------------------------------------------------
\subsection{Cruces, integrazioni e espunzioni}

\label{ref-7.3.2}
\index{croci}
\index{integrazioni}
\index{espunzioni}

Come il lettore ricorder{\`a}, queste tre situazioni sono
accomunate dal fatto che per segnalare un passo privo di
senso, un passo che l'editore aggiunge secondo il suo
giudizio critico, o che espunge, si utilizzano,
rispettivamente i segni diacritici \dag~~\dag,
$\langle\,\,\rangle$ e [~~], che si ottengono con le macro
grafiche:

\begin{center}
\begin{tabular}{lll}

\dag\thinspace cane\thinspace\dag & $=$ &\verb"\CRUX{cane}"\index{\bs{}CRUX} \\

$\langle$~cane~$\rangle$ & $=$ &\verb"\INTE{cane}"\index{\bs{}INTE} \\

$\lbrack$~cane~$\rbrack$ & $=$ &\verb"\EXPU{cane}"\index{\bs{}EXPU}

\end{tabular}
\end{center}

Nel caso ci si trovasse a dovere mettere fra \textit{cruces} un passo piuttosto lungo, o integrarlo, o
espungerlo,  si
proceder{\`a} come ora descriveremo.

%-----------------------------------------------------------------------
\subsubsection{Croci lunghe}

\label{ref-7.3.2.1}
\index{croci lunghe}

Se il passo ritenuto incomprensibile non necessita di
particolari note (mancanza di varianti, nessun altro editore
lo ha trattato, ecc.) baster{\`a} inserirlo come argomento di
\verb"\CRUX{}"\index{\bs{}CRUX} senza bisogno di altro (cfr. \S\,\ref{ref-6.2.1}).

Supponiamo per{\`o} che il passo sia stato emendato da
Clagett, ma che l'editore non ritenga accettabile tale
emendazione pur non sapendo proporne una migliore. Ad
esempio la tradizione sia concorde nel riportare:

\begin{quote}

Demittatur a puncto sumpto in parabola super diametrum
ordinem $uK$ et similiter punctum $A$ sit ad praesentem
paradoxam $Z$ ita ut recta $AL$ ducatur per $A$ $uad$ punctum
supremum. Per 33\textsuperscript{am} primi
Conicorum tanget talis recta parabolam.

\end{quote}

Il passo {\`e} evidentemente corrotto, ma il
riferimento a una proposizione di Apollonio e l'inizio e la
fine del discorso, l'uso delle lettere pi{\'u} o meno coerente
con la presenza di una figura e altre
considerazioni  hanno convinto Clagett di poter
ricostruire il brano ``ordinem~...~supremum'':

\begin{quote}

Demittatur a puncto sumpto in parabola super
diametrum \textit{ordinatam usque ad $K$ et sumatur punctum $A$
in diametro ab altera parte verticis parabolae $Z$ ita ut
$AZ$ aequalis sit $ZK$. Et recta $AL$ ducatur per $A$ usque
ad punctum sumptum}. Per 33\textsuperscript{am} primi
Conicorum tanget talis recta parabolam.

\end{quote}

\noindent dove il brano in corsivo indica la congettura di
Clagett\footnote{ Per amore di verit{\`a}, precisiamo che
l'esempio che stiamo facendo {\`e} del tutto fittizio, e che
Clagett non si {\`e} mai sognato di fare una correzione del
genere!}. Siccome il nostro editore, pur concordando con il
senso dell'operazione, non  ritiene accettabile una
correzione di questa portata vorr{\`a} ottenere:

%\stelle
%\textbf{276} Demittatur a puncto sumpto in
%parabola super diametrum {\dag}\thinspace ordinem\textsuperscript{1}
%$uK$ et similiter punctum $A$ sit ad praesentem
%paradoxam $Z$ ita ut recta AL ducatur per $A$ $uad$ punctum
%supremum.{\dag}
%\textbf{277} Per 33\textsuperscript{am} primi Conicorum tanget talis
%recta parabolam.
%
%\hrule width 1cm
%
%\textsuperscript{1}\qquad { ordinem~$\sim$~supremum}
%{\slnot locum corruptum} { ordinatam usque ad $K$ et
%sumatur punctum $A$ in diametro ab altera parte verticis
%parabolae $Z$ ita ut $AZ$ aequalis sit $ZK$. Et recta $AL$
%ducatur per $A$ usque ad punctum sumptum} {\slnot coniecit Clagett}
%
%\stelle

\begin{maurotex}
\Unit Demittatur a puncto sumpto in parabola super diametrum
\CRUX{
      \VV[longa]{
                {*:\CR{volpe}\ED{locum corruptum}:ordinem}
                {*:\ED{coniecit Clagett}: ordinatam
                 usque ad \(K\) et sumatur punctum
                 \(A\) in diametro ab altera parte
                 verticis parabolae \(Z\) ita ut \(AZ\)
                 aequalis sit \(ZK\). Et recta \(AL\)
                 ducatur per \(A\) usque ad punctum
                 sumptum}
                }
                 \(uK\) et similiter punctum \(A\) sit ad
       praesentem paradoxam \(Z\) ita ut recta \(AL\) ducatur
       per \(A\) \(uad\) punctum \LB{volpe}{supremum}.
     }
       \Unit Per 33\Sup{am} primi Conicorum tanget
talis recta parabolam.
\end{maurotex}

Per ottenere tale risultato bisogner{\`a} utilizzare
\verb"\VV[longa]"\index{\bs{}VV[longa]} e \verb"\CRUX"\index{\bs{}CRUX}:

\begin{quote}
\begin{verbatim}
\Unit Demittatur a puncto sumpto in parabola super diametrum
\CRUX{
      \VV[longa]{
                {*:\CR{volpe}\ED{locum corruptum}:ordinem}
                {*:\ED{coniecit Clagett}: ordinatam
                 usque ad \(K\) et sumatur punctum
                 \(A\) in diametro ab altera parte
                 verticis parabolae \(Z\) ita ut \(AZ\)
                 aequalis sit \(ZK\). Et recta \(AL\)
                 ducatur per \(A\) usque ad punctum
                 sumptum}
                }
                 \(uK\) et similiter punctum \(A\) sit ad
       praesentem paradoxam \(Z\) ita ut recta \(AL\) ducatur
       per \(A\) \(uad\) punctum \LB{volpe}{supremum}.
     }
       \Unit Per 33\Sup{am} primi Conicorum tanget
talis recta parabolam.
\end{verbatim}
\end{quote}

Come si vede, anche in questa situazione la macro
\verb"\CRUX"\index{\bs{}CRUX} deve essere \textit{esterna} alla macro \verb"\VV[longa]"\index{\bs{}VV[longa]},
come si faceva nel caso di varianti puntuali.

%-----------------------------------------------------------------------
\subsubsection{Integrazioni lunghe}

\label{ref-7.3.2.2}
\index{integrazioni lunghe}

Anche le integrazioni si potrebbero trattare nello stesso
modo,
utilizzando \verb"\INTE"\index{\bs{}INTE} all'esterno di \verb"\VV[longa]"\index{\bs{}VV[longa]}. Non ci
dilunghiamo su questo caso, che ci sembra del tutto teorico:
per definizione l'integrazione prevede che l'intera
tradizione taccia e  l'editore difficilmente potr{\`a} integrare
pi{\'u} di una o comunque poche parole. E quindi difficilmente si dovr{\`a}
trovare a ricorrere a \verb"\VV[longa]"\index{\bs{}VV[longa]} per un'integrazione.

%-----------------------------------------------------------------------
\subsubsection{Espunzioni lunghe}

\label{ref-7.3.2.3}
\index{espunzioni lunghe}

Pi{\'u} probabile sar{\`a} invece che l'editore debba ricorrere ad
espungere un brano di una certa consistenza. Supponiamo ad
esempio che la tradizione legga concordemente:

\begin{quote}

Erit quadratum $ab$ aequalis duobus quadratis dictis
per doctrinam Euclidis. Erit quadratum $ab$
aequalis duobus quadratis $bc$, $ca$ simul sumpta per
47\textsuperscript{am} primi Elementorum.

\end{quote}

{\`E} ovvio che qui una delle due frasi {\`e} superflua,
dato che dicono la stessa cosa, ma non si tratta di un pura
ripetizione. La seconda formulazione {\`e} una precisazione
della prima, che Maurolico avr{\`a} dimenticato di cancellare
(per esempio) e che sar{\`a} cos{\'\i} filtrata nella tradizione.
Convinto di questo il nostro editore proceder{\`a} cos{\'\i}:

%\stelle
%(TC) \textbf{22} [Erit\textsuperscript{1} quadratum $ab$ aequalis duobus
%quadratis dictis per doctrinam Euclidis.] Erit quadratum $ab$
%aequalis duobus quadratis $bc$, $ca$ simul sumpta per
%47\textsuperscript{am} primi Elementorum.
%
%\hrule width 1cm
%
%\textsuperscript{1}\qquad { Erit~$\sim$~Euclidis}
%{\slnot seclusi}
%
%\stelle

\begin{maurotex}
\Unit \EXPU{
            \VV[longa]{
                      {*:\CR{orso}\ED{seclusi}:Erit}
                      }
             quadratum \(ab\) aequalis duobus quadratis
             dictis per doctrinam \LB{orso}{Euclidis}.
           } Erit quadratum \(ab\) aequalis duobus
quadratis \(bc\), \(ca\) simul sumpta per 47\Sup{am} 
primi Elementorum.
\end{maurotex}

\noindent scrivendo in {\mtex}:

\begin{quote}
\begin{verbatim}
\Unit \EXPU{
            \VV[longa]{
                      {*:\CR{orso}\ED{seclusi}:Erit}
                      }
             quadratum \(ab\) aequalis duobus quadratis
             dictis per doctrinam \LB{orso}{Euclidis}.
           } Erit quadratum \(ab\) aequalis duobus
quadratis \(bc\), \(ca\) simul sumpta per 47\Sup{am} 
primi Elementorum.
\end{verbatim}
\end{quote}

Se avesse espunto Napoli, si
sarebbe scritto \verb"\ED{secl. Napoli}"\index{\bs{}ED}, ecc. Come si vede,
anche nel
caso di espunzioni lunghe \verb"\EXPU"\index{\bs{}EXPU} deve essere utilizzato
all'esterno di \verb"\VV[longa]"\index{\bs{}VV[longa]}, proprio come \verb"\CRUX"\index{\bs{}CRUX}.

%-----------------------------------------------------------------------
\chapter{Trasposizioni}

\label{ref-8}

\section{Una situazione complessa}

\label{ref-8.1}
\index{trasposizioni}

Le trasposizioni sono un caso particolare di varianti, che {\`e}
pi{\'u} opportuno trattare separatamente perch{\'e} presenta una
molteplicit{\`a} di situazioni.

Una trasposizione, infatti, pu{\`o} essere:

\begin{itemize}

\item[(\textit{i})] \textit{puntuale\/}: ovvero su due o poche parole (``il
cieco povero''~/``il povero cieco'');

\item[(\textit{ii})] \textit{lunga\/}: coinvolgere cio{\`e} porzioni
consistenti di testo;

\item[(\textit{iii})] \textit{su due elementi\/}: scambiare cio{\`e} di posto
due soli elementi testuali, siano essi parole o frasi
intere;

\item[(\textit{iv})] \textit{ su molti elementi\/}: permutare il posto di
pi{\'u} di due elementi testuali (il che pu{\`o} avvenire ovviamente
in molti modi diversi); e ci{\`o} potr{\`a} accadere

\begin{itemize}

\item[(\textit{iv.a})] all'interno di una frase;

\item[(\textit{iv.b})] all'interno di una serie di
frasi.

\end{itemize}

\end{itemize}

Le trasposizioni inoltre possono essere:

\begin{itemize}

\item[(\textit{a})] \textit{accidentali\/}: un brano si
trova collocato in punti diversi
di uno o pi{\'u} testimoni (e in questo caso non si pu{\`o}
parlare propriamente di trasposizione);

\item[(\textit{b})] \textit{volontarie}: degli elementi
testuali sono stati volontariamente trasposti da qualcuno. E
questo pu{\`o} avvenire ad opera:

\begin{itemize}

\item[($b_1$)] \textit{del copista.\/}: o, pi{\'u} in generale di una
delle mani che sono intervenute sul testo;

\item[($b_2$)] \textit{dell'editore stesso\/}: che traspone
congetturalmente elementi testuali.

\end{itemize}

\end{itemize}

Questa rapida panoramica fa capire l'opportunit{\`a}
di distinguere il caso delle trasposizioni da quello di tutte
le altre varianti e di trattarlo quindi con una macro
distinta da \verb"\VV"\index{\bs{}VV} e dalla sua sottospecie \verb"\VV[longa]"\index{\bs{}VV[longa]}.
Abbiamo deciso pertanto che la madre di tutte le macro aveva
bisogno di una sorella (che, come si vedr{\`a}, le somiglia
molto da vicino): \verb"\TV"\index{\bs{}TV}.

Non ci {\`e} sembrato inoltre praticabile
un trattamento esaustivo di tutte le possibili
combinazioni di questi casi e sottocasi fra di loro,
soprattutto se si tiene presente che nel caso di
trasposizioni lunghe le cose possono ulteriormente
complicarsi per la presenza di varianti puntuali o lunghe
negli elementi testuali coinvolti nella trasposizione.

Questo capitolo ha lo scopo pi{\'u} limitato di fornire ai
trascrittori e agli editori alcuni strumenti che speriamo
possano essere sufficienti ad affrontare
le trasposizioni che si troveranno davanti. Star{\`a} al loro
acume scegliere di volta in volta quello che riterranno pi{\'u}
opportuno nel caso concreto che dovranno trattare.

%-----------------------------------------------------------------------
\section{Le trasposizioni puntuali}

\label{ref-8.2}
\index{trasposizioni puntuali}

La macro per affrontare le trasposizioni puntuali {\`e}

\begin{quote}
\begin{verbatim}
\TV{}
\end{verbatim}
\end{quote}

\noindent che, nonostante il nome diverso, possiede la stessa
sintassi di \verb"\VV{}"\index{\bs{}VV}. Distingueremo tre situazioni:
trasposizioni puntuali \textit{nei testimoni}; trasposizioni
\textit{dei
copisti} e trasposizioni \textit{congetturali}.

%-----------------------------------------------------------------------
\subsection{Inversioni del testo}

\label{ref-8.2.1}
\index{inversioni del testo nei testimoni}

Supponiamo di avere due testimoni, A e B, che leggano:

\begin{itemize}

\item[A:] Vertex coni sit $N$
\item[B:] Sit coni vertex $N$

\end{itemize}

Ci{\`o} che si vorrebbe ottenere, supponendo che TC
segua B, {\`e}:

%\stelle
%Sit coni vertex$^1 N$
%
%\hrule width 1cm
%
%\noindent\textsuperscript{1} { Sit coni vertex}
%\textsl{B} { Vertex coni sit} \textsl{A}
%\stelle

\begin{maurotex}
\TV{
   {B:Sit coni vertex}
   {A:Vertex coni sit}
   } $N$
\end{maurotex}

Esattamente come nel caso di \verb"\VV"\index{\bs{}VV} si scriver{\`a}:

\begin{quote}
\begin{verbatim}
\TV{
   {B:Sit coni vertex}
   {A:Vertex coni sit}
   } $N$
\end{verbatim}
\end{quote}

Tutte le regole valide per \verb"\VV"\index{\bs{}VV} continuano a
valere anche in questo caso: in particolare se la lezione
contenesse il segno di interpunzione  `\textbf{:}' dovr{\`a}
essere posta fra \verb"{}" (cfr. \S\,\ref{ref-4.2.1}). Inoltre se l'editore
ritenesse che questa trasposizione non {\`e} degna di
comparire in apparato, baster{\`a} che trasformi \verb"\TV{}"\index{\bs{}TV} in
\verb"\TB{}"\index{\bs{}TB} (``trasposizione banale'': cfr. \S\,\ref{ref-4.2.3}).

%-----------------------------------------------------------------------
\subsection{Trasposizioni del copista o di altre mani}

\label{ref-8.2.2}
\index{trasposizioni del copista}

Riprendendo l'esempio precedente suppponiamo che il copista
del testimone A abbia apposto dei segni nel seguente modo:

\begin{itemize}

%\item[A\hphantom{\textsuperscript{1}}:] ${\hbox{\rm Vertex}}^{2>}\,\,\,
%{\rm coni}\,\,\,{\hbox{\rm sit}}_{<1}\,\,\,N$.

\item[A\hphantom{\textsuperscript{1}}:] Vertex\textsuperscript{2$>$}~coni~sit$_{<1}$~$N$.

\end{itemize}

\noindent con il chiaro intento di ottenere ``Sit coni
vertex $N$''

Si tratta quindi di uno di quei casi in cui il
copista di A interviene correggendo s{\'e} stesso  e produce
un testo alternativo: tale tipo di intervento, come si
ricorder{\`a} (\S\,\ref{ref-5.1}), si denota con A\textsuperscript{1} e questo \textit{siglum} viene trattato \textit{come se} fosse quello di un
altro testimone. In un caso del genere si potr{\`a}
segnalare chiaramente l'intervento del copista e ottenere
qualcosa del tipo:

%\stelle
%Sit coni vertex$^1 N$
%
%\hrule width 1cm
%
%\noindent\textsuperscript{1} { Sit coni vertex}
%\textsl{BA$^{\hbox{\slnotmm 1}}$} { Vertex
%coni sit} \textsl{ante corr. A}
%\stelle

\begin{maurotex}
\TV{
   {B/A1:Sit coni vertex}
   {A:\DES{ante corr.}:Vertex coni sit}
   } $N$
\end{maurotex}

\noindent ovvero: ``B legge `Sit coni vertex', A\textsuperscript{1} ha
corretto disponendo le  parole cos{\'\i} mentre l'ordine
originale (\textsl{ante corr. = ante correctionem} = prima
della correzione) di A era
`Vertex coni sit'\thinspace''. Si scriver{\`a} allora 

\begin{quote}
\begin{verbatim}
\TV{
   {B/A1:Sit coni vertex}
   {A:\DES{ante corr.}:Vertex coni sit}
   } $N$
\end{verbatim}
\end{quote}

\noindent dove si noti l'uso di \verb"\DES{}"\index{\bs{}DES} per fornire le
informazioni desiderate.

%-----------------------------------------------------------------------
\subsection{Trasposizioni dell'editore}

\label{ref-8.2.3}
\index{trasposizioni dell'editore}

Se {\`e} l'editore stesso che intende operare una trasposizione, user{\`a}
un sistema analogo a quello ora descritto ma impiegando la macro
\verb"\ED"\index{\bs{}ED}. Se ad esempio disponessimo solo del testimone A (senza
interventi del copista o di altre mani) e l'editore decidesse che la
lezione da accogliere in  TC debba essere:

\begin{itemize}

\item[TC:] Sit coni vertex $N$

\end{itemize}

\noindent si dovr{\`a} ottenere come apparato:

%\stelle
%Sit coni vertex$^1\,\,\,N$
%
%\hrule width 1cm
%
%\noindent\textsuperscript{1} { Sit coni vertex}
%\textsl{conieci} { Vertex
%coni sit} \textsl{A}
%
%\stelle

\begin{maurotex}
\TV{
   {*:\ED{conieci}:Sit coni vertex}
   {A:Vertex coni sit}
   } $N$
\end{maurotex}

\noindent come nel caso  delle correzioni
congetturali (\S\S\,\ref{ref-6.1.1} e sgg.). L'unica differenza {\`e} che,
per registrare il fatto che siamo davanti a una
trasposizione, si user{\`a} \verb"\TV"\index{\bs{}TV} e non \verb"\VV"\index{\bs{}VV}, scrivendo in 
{\mtex}:

\begin{quote}
\begin{verbatim}
\TV{
   {*:\ED{conieci}:Sit coni vertex}
   {A:Vertex coni sit}
   } $N$
\end{verbatim}
\end{quote}

Naturalmente se la trasposizione congetturale fosse stata
fatta da Clagett si scriverebbe \verb"\ED{Clagett}"\index{\bs{}ED}; nel caso di un'edizione
a quattro mani \verb"\ED{coniecimus}"\index{\bs{}ED}, ecc.

%-----------------------------------------------------------------------
\section{Trasposizioni lunghe}

\label{ref-8.3}
\index{trasposizioni lunghe}

Come si sar{\`a} capito, \verb"\TV{}"\index{\bs{}TV} non {\`e} altro che la sorella
gemella di \verb"\VV"\index{\bs{}VV}. Allo stesso modo, anche \verb"\VV[longa]"\index{\bs{}VV[longa]} possiede una 
gemella~---~\verb"\TV[longa]{}"\index{\bs{}TV[longa]}~---~che ha esattamente la
stessa sintassi. Essa si utilizzer{\`a} in tutti quei casi 
in cui non si ritenga opportuno riportare in nota per
intero il testo critico coinvolto nella trasposizione. 
Come nel caso delle trasposizioni
puntuali, potremo avere trasposizioni lunghe congetturali, del
copista o di altre mani, o brani collocati in luoghi diversi
nei diversi testimoni. 

Ci limiteremo a trattare  a un solo esempio
dell'uso di \verb"\TV[longa]"\index{\bs{}TV[longa]}, relativo a una trasposizione congetturale.
Supponiamo per semplicit{\`a} che ci sia un unico
testimone, A,  che legga:

\begin{itemize}

\item[A:\hphantom{C}] Namque superficies ex
similibus conicis
superficiebus \textit{talium solidorum componuntur}.

\end{itemize}

\noindent e che l'editore, sulla base di sue considerazioni
sugli \textit{usus scribendi} mauroliciani o altro, voglia
invece proporre:

\begin{itemize}

\item[TC:] Namque superficies
\textit{talium solidorum componuntur}  ex similibus conicis
superficiebus.

\end{itemize}

In tal caso occorrer{\`a} indicare in nota il testo di
A e l'operazione di trasposizione:

%\stelle
%Namque\textsuperscript{1} superficies
%talium solidorum componuntur ex similibus conicis
%superficiebus.
%
%\hrule width 1cm
%
%\noindent\textsuperscript{1} {
%Namque~$\sim$~superficiebus}
%\textsl{correxi, verborum ordine mutato} { Namque superficies ex
%similibus conicis
%superficiebus talium solidorum componuntur} \textsl{A}
%\stelle

\begin{maurotex}
\Unit \TV[longa]{
 {*:\CR{acquablue}\ED{correxi, verborum ordine mutato}:Namque}
 {A:Namque superficies ex similibus
  conicis superficiebus talium
  solidorum componuntur
 }
}
superficies talium solidorum componuntur 
ex similibus conicis \LB{acquablue}{superficiebus}.
\end{maurotex}

\noindent che {\`e} l'esatto analogo della situazione precedente
(\S\,\ref{ref-8.2.3}), salvo per il fatto che qui la porzione di testo
manipolata {\`e} assai pi{\'u} consistente. Si scriver{\`a} allora:

\begin{quote}
\begin{verbatim}
\Unit \TV[longa]{
 {*:\CR{acqua}\ED{correxi, verborum ordine mutato}:Namque}
 {A:Namque superficies ex similibus
  conicis superficiebus talium
  solidorum componuntur
 }
}
superficies talium solidorum componuntur 
ex similibus conicis \LB{acqua}{superficiebus}.
\end{verbatim}
\end{quote}

Sul modello di quanto detto per le trasposizioni
puntuali si utilizzer{\`a} la macro \verb"\DES{}"\index{\bs{}DES} se si volesse dar conto preciso
delle trasposizioni effetuate dal copista o da altre mani.

%-----------------------------------------------------------------------
\section{Uno tocco  di classe}

\label{ref-8.4}

\verb"\TV"\index{\bs{}TV} e \verb"\TV[longa]"\index{\bs{}TV[longa]} permettono di trattare praticamente
tutti i casi possibili, ma non brillano certo per eleganza:
allo scopo di migliorare le prestazioni delle macro da
trasposizione, ne introduciamo un'altra: \verb"\TV[duplex]"\index{\bs{}TV[duplex]}.
Essa per{\`o} {\`e} utilizzabile esclusivamente  nel caso in cui la
trasposizione consista in uno \textit{scambio fra due soli
elementi del testo}~---~da qui il nome. In compenso tali
elementi possono essere arbitrariamente lunghi. Distinguiamo
come al solito fra scambi nei testimoni, del copista e di
altre mani, congetturali.

%-----------------------------------------------------------------------
\subsection{Scambi nel testo nei testimoni}

\label{ref-8.4.1}
\index{scambi nel testo nei testimoni}

Supponiamo di avere due testimoni, A e B e che la situazione
sia la seguente:

\begin{itemize}

\item[A:] ... \textit{perfectus numerus producitur ex multiplicatione
ultimi in serie pariter parium ab unitate dispositorum}, in
totum aggregatum ipsorum, dum tamen tale aggregatum sit
numerus primus, hoc est a nullo, preterquam ab unitate,
numeratus.

\item[B:] ... in totum aggregatum ipsorum, dum tamen tale
aggregatum sit numerus primus, hoc est a nullo, preterquam
ab unitate, numeratus, \textit{perfectus numerus producitur ex multiplicatione
ultimi in serie pariter parium ab unitate
dispositorum}.

\end{itemize}

Come si vede, B tramanda il brano ``perfectus numerus
\dots dispositorum'' dopo ``numeratus'' e di questo vorremmo
dar conto in nota:

%\stelle
%perfectus\textsuperscript{1} numerus producitur ex multiplicatione
%ultimi in serie pariter parium ab unitate dispositorum, in
%totum aggregatum ipsorum, dum tamen tale aggregatum sit
%numerus primus, hoc est a nullo, preterquam ab unitate,
%numeratus.
%
%\hrule width 1cm
%
%\noindent\textsuperscript{1} { perfectus~$\sim$~dispositorum}
%\textsl{hoc loco A~~~post} { numeratus} \textsl{B}
%\stelle

\begin{maurotex}
\Unit ... \TV[duplex]{
                     {A:\CR{pane}\DES{hoc loco}:perfectus}
                     {B:\CR{burro}:}
                    } numerus producitur
ex multiplicatione ultimi in serie pariter parium
ab unitate \LB{pane}{dispositorum}, in totum
aggregatum ipsorum, dum tamen tale aggregatum
sit numerus primus, hoc est a nullo, preterquam
ab unitate, \LB{burro}{numeratus}. \Unit ...
\end{maurotex}

Vale a dire: ``in A il brano che in TC va da
`perfectus' a `dispositorum' si trova
in questo punto [\textit{hoc loco}]; in B si trova dopo
`numeratus'\thinspace''.

Come si vede questo tipo di nota {\`e} assai pi{\'u} elegante di
quanto si sarebbe potuto ottenere usando \verb"\TV[longa]"\index{\bs{}TV[longa]}. Ma
la classe ha un prezzo, anche per le trasposizioni.
Per ottenere questo risultato dovremo far uso di una
nuova macro,
\verb"\TV[duplex]"\index{\bs{}TV[duplex]}, che ha una sintassi pi{\'u} complessa di quella di
\verb"\TV[longa]"\index{\bs{}TV[longa]}.

Essa {\`e} stata strutturata in modo da sapere
dove si trovano l'inizio e la fine del primo brano [perfectus
... dispositorum] e la fine del secondo [ ... numeratus].
A questo scopo occorrer{\`a} indicare mediante una seconda
coppia di \verb"\CR"\index{\bs{}CR}\thinspace--\thinspace\verb"\LB"\index{\bs{}LB} la situazione che viene a
verificarsi, in questo modo:

\begin{quote}
\begin{verbatim}
\Unit ... \TV[duplex]{
                     {A:\CR{pane}\DES{hoc loco}:perfectus}
                     {B:\CR{burro}:}
                     } numerus producitur
ex multiplicatione ultimi in serie pariter parium
ab unitate \LB{pane}{dispositorum}, in totum
aggregatum ipsorum, dum tamen tale aggregatum
sit numerus primus, hoc est a nullo, preterquam
ab unitate, \LB{burro}{numeratus}. \Unit ...
\end{verbatim}
\end{quote}

La struttura di \verb"\TV[duplex]"\index{\bs{}TV[duplex]} {\`e} dunque praticamente
identica  a quella di \verb"\TV[longa]"\index{\bs{}TV[longa]}. Tuttavia, nel campo \verb"{.b.}" di
\verb"\TV[duplex]"\index{\bs{}TV[duplex]} va lasciato vuoto il terzo sottocampo (che
sarebbe destinato alla lezione, qui assente), inserendo nel
secondo sottocampo (invece di una descrizione di B
come nel caso di \verb"\TV[longa]"\index{\bs{}TV[longa]}) un segnale che dica dove B ha
trasposto il testo: cio{\`e} un secondo \verb"\CR"\index{\bs{}CR} (quello
etichettato con \verb"burro") cui corrisponder{\`a} un secondo \verb"\LB"\index{\bs{}LB}
(anch'esso etichettato con \verb"burro") che segnala il punto
(\verb"numeratus") dopo il quale si trova il brano in B.

Occorre anche osservare l'uso di \verb"\DES{}"\index{\bs{}DES} per introdurre la
locuzione ``hoc loco''

%-----------------------------------------------------------------------
\subsection{Scambi effettuati dal copista o da altre mani}

\label{ref-8.4.2}
\index{scambi effettuati dal copista o da altre mani}

Suppponiamo che il testimone A rechi il testo seguente:

\onlylatex{
\begin{itemize}

\item[A\hphantom{\textsuperscript{1}}:] Campanus mathematicus pro
libidine sua $\overbrace{\rm et\ multa\ addidit}^{2>}$,\\
$\underbrace{\rm et\ multa\ mutavit\ ex\ sententia\ Euclidis}_{<1}$.

\end{itemize}
}

\onlyhtml{
\begin{itemize}

\item[A:] Campanus mathematicus pro libidine sua figforhtml.addidit,\\ figforhtml.mutavit.

\end{itemize}
}
% ici on triche

\noindent dove i segni aggiunti (le parentesi, i numeri, gli
indicatori) sono stati apposti dal copista di A che voleva
evidentemente ottenere:

\begin{itemize}

\item[A\textsuperscript{1}:] Campanus mathematicus pro libidine
sua et multa mutavit ex sententia Euclidis, et multa
addidit.

\end{itemize}

Pur continuando a  trattare A\textsuperscript{1} (o A\textsuperscript{m} o
A\textsuperscript{2}, ecc.) come se si trattasse di testimoni distinti da
A, il tipo di nota da ottenere sar{\`a} diverso da quella descritta nel
paragrafo precedente, dato che si deve segnalare che il copista {\`e}
intervenuto sul testo, trasponendo il brano ``et multa mutavit ex sententia
Euclidis'':

%\stelle
%Campanus mathematicus pro libidine
%sua et\textsuperscript{1} multa mutavit ex sententia Euclidis, et multa
%addidit.
%
%\hrule width 1cm
%
%\noindent\textsuperscript{1} { et~$\sim$~Euclidis}
%\textsl{huc transp. A$^{\hbox{\slnotmm 1}}$; post} {
%addidit} \textsl{A}
%
%\stelle

\begin{maurotex}
\Unit Campanus matematicus pro libidine
sua \TV[duplex]{
               {A1:\CR{sole}\DES{huc transp.}:et}
               {A:\CR{luna}:}
               } multa mutavit ex sententia
\LB{sole}{Euclidis}, et multa
\LB{luna}{addidit}. \Unit ...
\end{maurotex}

Ovvero: ``il brano `et ... Euclidis'' {\`e} stato
spostato qui [\textit{huc transposuit}] dal copista di A [\textit{A\textsuperscript{1}}];
originariamente di trovava dopo `addidit'\thinspace''. Per
ottenere ci{\`o} si user{\`a} la stessa sintassi del \S\,\ref{ref-8.4.1},
inserendo per{\`o} nel campo di \verb"\DES{}"\index{\bs{}DES} l'espressione ``huc
transp.'':

\begin{quote}
\begin{verbatim}
\Unit Campanus matematicus pro libidine
sua \TV[duplex]{
               {A1:\CR{sole}\DES{huc transp.}:et}
               {A:\CR{luna}:}
               } multa mutavit ex sententia
\LB{sole}{Euclidis}, et multa
\LB{luna}{addidit}. \Unit ...
\end{verbatim}
\end{quote}

%-----------------------------------------------------------------------
\subsection{Scambi congetturali}
\index{scambi congetturali}

\label{ref-8.4.3}

\verb"\TV[duplex]"\index{\bs{}TV[duplex]} pu{\`o}, ovviamente, essere utilizzata anche nel
caso che l'editore,
per qualche sua ragione, voglia trasporre un brano.
Riprendiamo l'esempio trattato nel \S\,\ref{ref-8.3} utilizzando
\verb"\TV[longa]"\index{\bs{}TV[longa]}, anche per far meglio vedere la differenza di uso e di
risultato fra questa macro e \verb"\TV[duplex]"\index{\bs{}TV[duplex]}.

Disponiamo dunque di un unico
testimone, A,  che legge

\begin{itemize}

\item[A\hphantom{C}:] Namque superficies ex
similibus conicis
superficiebus \textit{talium solidorum componuntur}.

\end{itemize}

\noindent e l'editore vuole ottenere

\begin{itemize}

\item[TC:] Namque superficies
\textit{talium solidorum componuntur}  ex similibus conicis
superficiebus.

\end{itemize}

Nel caso precedente si era costretti a riportare
in nota l'intero passo di A. Usando \verb"\TV[duplex]"\index{\bs{}TV[duplex]} si potr{\`a}
invece ottenere:

%\stelle
%Namque superficies
%talium\textsuperscript{1} solidorum componuntur ex similibus conicis
%superficiebus.
%
%\hrule width 1cm
%
%\noindent\textsuperscript{1} {
%talium~$\sim$~componuntur}
%\textsl{huc transposui~~~post} { superficiebus}
%\textsl{A}
%\stelle

\begin{maurotex}
\Unit Namque superficies \TV[duplex]{
               {*:\CR{acqua}\ED{huc transposui}:talium}
               {A:\CR{fuoco}:}
                                     } solidorum
\LB{acqua}{componuntur} ex similibus conicis
\LB{fuoco}{superficiebus}.
\end{maurotex}

Cio{\`e}: ``ho spostato io qui [\textsl{huc
transposui}] il brano `talium ... componuntur'; in \textit{A} si
trovava dopo `superficiebus'\thinspace''. La nota si otterr{\`a} in modo
simile a quanto visto sopra, 
ma~---~si tratta di un intervento congetturale~---~utilizzando
\verb"\ED"\index{\bs{}ED} e non \verb"\DES"\index{\bs{}DES}\thinspace:

\begin{quote}
\begin{verbatim}
\Unit Namque superficies \TV[duplex]{
               {*:\CR{acqua}\ED{huc transposui}:talium}
               {A:\CR{fuoco}:}
                                     } solidorum
\LB{acqua}{componuntur} ex similibus conicis
\LB{fuoco}{superficiebus}.
\end{verbatim}
\end{quote}

Come si vede la situazione {\`e} praticamente la stessa
di quella delle trasposizioni dei copisti, salvo per il
fatto che l'editore segnala chi ha compiuto la trasposizione
utilizzando \verb"\ED{}"\index{\bs{}ED} invece di \verb"\DES{}"\index{\bs{}DES}. Nell'esempio abbiamo
scritto ``huc transposui'', se il testo critico fosse curato da
pi{\'u} di un editore si scriverebbe ``huc transposuimus'', se
la trasposizione l'avesse effettuata Clagett, ``huc transp.
Clagett'' eccetera.

%-----------------------------------------------------------------------
\section{Trasposizioni su pi{\'u} paragrafi}

\label{ref-8.5}
\index{trasposizioni su pi{\'u} paragrafi}

L'ultima arma che mettiamo a disposizione dell'editore~---~che, novello
Ercole, si appresta a combattere l'idra della trasposizione~---~{\`e} la
macro \verb"\TV[unit]"\index{\bs{}TV[unit]}. Essa {\`e} stata pensata per trattare quei casi in
cui la trasposizione permuta pi{\'u} di due elementi e coinvolge interi
paragrafi. Non essendo uno scambio, \textit{non pu{\`o} essere trattata con}
\verb"\TV[duplex]"\index{\bs{}TV[duplex]}; potrebbe ovviamente essere trattata con
\verb"\TV[longa]"\index{\bs{}TV[longa]}, ma la nota prodotta sarebbe molto lunga e poco chiara.
Lo scopo di \verb"\TV[unit]"\index{\bs{}TV[unit]} {\`e} quello di fornire una specie di mappa
della situazione dei testimoni, indicando in nota l'ordine in cui i vari
paragrafi si presentano nei testimoni semplicemente richiamandone il
numero.

%-----------------------------------------------------------------------
\subsection{L'esempio di riferimento}

\label{ref-8.5.1}

I testimoni siano tre, A, B e C; il
testo sia costituito dalle seguenti frasi:

\begin{itemize}

\item[$\alpha$:] Centrum uniformis figurae in puncto axis
medio constituitur.

\item[$\beta$:] Centrum trianguli rectilinei trientem axis
ad basim relinquit.

\item[$\gamma$:] Centrum totius interiacet centris partium
in eadem recta constitutum.

\item[$\delta$:] Centrum nunquam cadit extra rei gravis
ambitum.

\item[$\epsilon$:] Centrorum partialium distantiae a centro
totius reciprocae sunt partibus.

\end{itemize}

I testimoni diano queste frasi secondo questi
ordinamenti:

\begin{itemize}

\item[A:] $\delta$ $\epsilon$ $\gamma$ $\beta$ $\alpha$

\item[B:] $\beta$ $\alpha$ $\delta$ $\gamma$ $\epsilon$

\item[C:] $\alpha$ $\beta$ $\epsilon$ $\delta$

\end{itemize}

L'editore, osserva che B d{\`a} il brano
$\epsilon$ per ultimo, con notevoli varianti; che C omette in lacuna
$\gamma$; che l'ordine di C e di A 
{\`e} invertito; studia il contenuto e il significato di queste
asserzioni, lo confronta con altri passi dell'opera
mauroliciana e finalmente decide per il seguente ordine in TC:

\begin{quote}

$\alpha$ $\beta$ $\gamma$ $\epsilon$ $\delta$

\end{quote}

Supponendo che questa porzione di TC inizi con il paragrafo
21 si vorr{\`a} ottenere il seguente risultato

%\stelle
%
%\textbf{21}\textsuperscript{1} Centrum uniformis figurae in puncto axis %medio
%constituitur. \textbf{22} Centrum trianguli rectilinei trientem axis ad
%basim relinquit. \textbf{23} Centrum\textsuperscript{2} totius interiacet centris partium
%in eadem recta constitutum. \textbf{24} Centrorum$^3$ partialium distantiae
%a centro totius reciprocae sunt partibus. \textbf{25} Centrum nunquam
%cadit extra rei gravis ambitum.
%
%\hrule width 1cm
%
%\textsuperscript{1} { \S\,\textbf{21}--\S\,\textbf{25}{\slnot: hoc ordine
%disposui}~ \S\,\textbf{21}-\S\,\textbf{22} \S\,\textbf{24}--\S\,
%\textbf{25}~\textsl{C}~~\S\,\textbf{25} \S\,\textbf{24} \S\,\textbf{23}
%\S\,\textbf{22}  \S\,\textbf{21}~\textsl{A}~~\S\,\textbf{22}
%\S\,\textbf{25} \S\,\textbf{23}--\S\,\textbf{24}~\textsl{omisso
%\S\,\textbf{21} B}} 
%
%\textsuperscript{2} { Centrum~$\sim$~constitutum} {\slnot
%spatio relicto om. C}
%
%$^3$ { Centrorum~$\sim$~partibus {\slnot
%AC} Centrorum distantiae a centro corporis gravis respectivae sunt
%partibus \textsl{B}}
%
%\stelle

\begin{maurotex}
\Unit[cervo1]\TV[unit]{
  {*:\ED{hoc ordine disposui}:\UN{cervo1}-\UN{cervo5}}
  {C:\UN{cervo1}-\UN{cervo2} \UN{cervo4}-\UN{cervo5}}
  {A:\UN{cervo5} \UN{cervo4} \UN{cervo3} 
     \UN{cervo2} \UN{cervo1}}
  {B:\DES{omisso \UN{cervo1}}:\UN{cervo2}
      \UN{cervo5} \UN{cervo3}-\UN{cervo4}}
                       } Centrum uniformis figurae in
puncto axis medio constituitur. \Unit[cervo2] Centrum
trianguli rectilinei trientem axis ad
basim relinquit. \Unit[cervo3] \VV[longa]{
                                   {*:\CR{foca}:Centrum}
                                   {C:\OMLAC}
                                          } totius
interiacet centris partium in eadem recta
\LB{foca}{constitutum}. \Unit[cervo4]
\VV[longa]{
        {A/C:\CR{delfino}:Centrorum}
        {B:Centrorum distantiae a centro 
          corporis gravis respectivae sunt
         partibus}
            } partialium distantiae a centro
totius reciprocae sunt \LB{delfino}{partibus}.
\Unit[cervo5] Centrum nunquam cadit extra rei gravis
ambitum. \Unit ...
\end{maurotex}

Come si vede la nota 1 d{\`a} un quadro della
situazione da cui si possono ricavare le informazioni
essenziali circa l'ordine e le omissioni delle frasi in A, B e
C; le note 2 e 3 specificano nel punto opportuno che
il tale e il tale passo hanno determinate varianti in certi testimoni.
Vedremo ora come si pu{\`o} ottenere questa mappa. Si noti che il
fatto che B ometta il passo $\alpha$ si ricava direttamente dalla nota
1, in cui, oltre che a dare l'ordine dei paragrafi in B si {\`e}
specificato ``omisso \S\,%2
1'';  sembra pi{\'u} opportuno invece specificare
in una nota separata il fatto che C omette $\gamma$, perch{\'e}
l'omette in lacuna.

%-----------------------------------------------------------------------
\subsection{Il segreto dell'Unit{\`a}}

\label{ref-8.5.2}

Nella descrizione dell'esempio si sar{\`a} notato che abbiamo
dovuto stabilire una corrispondenza fra le lettere
$\alpha\,\,\beta\,\,\gamma\,\, \epsilon\,\,\delta$ e i numeri
di paragrafo %2
1, %2
2, %2
3, 2%
4, 2%
5. Ci{\`o} ovviamente per noi {\`e}
facile da fare, visto che abbiamo deciso che la frase $\alpha$
corrisponde al numero %2
1. Ma quando si trascrive o si lavora
all'edizione del testo non si pu{\`o} sapere  \textit{a priori}
quale sar{\`a} il numero definitivo delle \verb"\Unit"\index{\bs{}Unit} che si vanno
introducendo. Il compito di \verb"\Unit"\index{\bs{}Unit} {\`e} quello appunto di
generare i numeri di paragrafo: se l'editore aggiunge o
toglie un paragrafo, tutte gli altri paragrafi del testo
saranno automaticamene rinumerati.

Come fare allora a dire all'{\mtex} che la tale \verb"\Unit"\index{\bs{}Unit}
corrisponde alla frase $\alpha$ e la tal altra alla frase
$\delta$? {\`E} giunto il momento di svelare un aspetto di \verb"\Unit"\index{\bs{}Unit}
che fin qui abbiamo tenuto accuratamene celato: la sintassi
completa di \verb"\Unit"\index{\bs{}Unit} {\`e}:

\begin{quote}
\begin{verbatim}
\Unit[ ]
\end{verbatim}
\end{quote}

Le \verb"[ ]" (a differenza delle \verb"{}") rappresentano
un campo opzionale, che pu{\`o} essere utilizzato o meno. Se non
viene utilizzato, si pu{\`o} tranquillamente evitare di apporre
tali parentesi. Anzi, \textit{si deve}: non avrebbe alcun senso
infarcire il testo di parentesi quadre destinate a contenere
solo la stringa vuota. Tuttavia il fatto che \verb"\Unit[ ]"\index{\bs{}Unit}
possieda un campo opzionale {\`e} ci{\`o} che ci permette di
risolvere il problema di far ricordare al {\mtex}
qual'{\`e} la prima \verb"\Unit"\index{\bs{}Unit} del testo che stiamo trattando,
quale la seconda, ecc. Baster{\`a} inserire nelle \verb"["\,\verb"]"
un'etichetta (cos{\'\i} come si {\`e} fatto per far capire a
\verb"\VV[longa]"\index{\bs{}VV[longa]} dove
inizia e dove finisce una variante lunga). Le regole per
assegnare i nomi alle etichette sono le stesse di quelle che
abbiamo elencato nel \S\,\ref{ref-7.1.3}:

\begin{itemize}

\item l'etichetta che si assegna a una \verb"\Unit"\index{\bs{}Unit} deve essere
univoca: due \verb"\Unit"\index{\bs{}Unit} diverse non possono avere la stessa
eichetta;

\item si possono usare caratteri alfabetici (maiuscoli e
minuscoli) e numerici (0--9), ma
non di altro tipo; in particolare non si pu{\`o} assolutamente
usare lo spazio;

\item le etichette che si scelgono per le \verb"\Unit"\index{\bs{}Unit} non
possono coincidere con etichette usate per le coppia (\verb"\CR"\index{\bs{}CR},
\verb"\LB"\index{\bs{}LB}).

\end{itemize}

Svelato questo segreto di \verb"\Unit"\index{\bs{}Unit}, passiamo a
vedere come potr{\`a} essere utilizzato.

%-----------------------------------------------------------------------
\subsection{L'uso di $\protect\backslash$TV[unit]}

\label{ref-8.5.3}

Per semplicit{\`a} piuttosto che utilizzare il testo
intero di $\alpha$, $\beta$, ecc. ci riferiremo per il
momento ai loro nomi. Alla fine della descrizione scriveremo
tutto per esteso, in modo che si possa avere ben presente il
da farsi.

L'editore comincer{\`a} con l'assegnare un'etichetta alle
\verb"\Unit"\index{\bs{}Unit} di cui {\`e} composto il brano incriminato:

\verb"\Unit[cervo1]"\index{\bs{}Unit[]} $\beta$
\verb"\Unit[cervo3]"\index{\bs{}Unit[]} $\gamma$ 

\verb"\Unit[cervo4]"\index{\bs{}Unit[]} $\epsilon$
\verb"\Unit[cervo5]"\index{\bs{}Unit[]} $\delta$

\noindent e a questo punto, subito dopo la prima \verb"\Unit[]"\index{\bs{}Unit[]}
(\verb"\Unit[cervo1]"\index{\bs{}Unit[]})
inserir{\`a} la macro \verb"\TV[unit]{}"\index{\bs{}TV[unit]}. Essa ha una sintassi
analoga a quella di \verb"\VV{}"\index{\bs{}VV} e di \verb"\TV{}"\index{\bs{}TV}, con i campi
interni divisi nei soliti tre sottocampi:
testimone/ informazioni/ lezione. Procediamo:

\begin{quote}
\begin{verbatim}
\Unit[cervo1]\TV[unit]{
  {*:\ED{hoc ordine disposui}:\UN{cervo1}-\UN{cervo5}}
  {C:\UN{cervo1}-\UN{cervo2} \UN{cervo4}-\UN{cervo5}
  {A:\UN{cervo5} \UN{cervo4} \UN{cervo3} 
     \UN{cervo2} \UN{cervo1}}
  {B:\DES{omisso \UN{cervo1}}:\UN{cervo2} 
     \UN{cervo5} \UN{cervo3}-\UN{cervo4}}
                       }
\end{verbatim}
\end{quote}

%\hphantom{UUU $\alpha$ Unit[cervo2] $\beta$ Unit}
$\alpha$ \verb"\Unit[cervo2]"\index{\bs{}Unit[]} $\beta$
\verb"\Unit[cervo3]"\index{\bs{}Unit[]} $\epsilon$
\verb"\Unit[cervo5]"\index{\bs{}Unit[]} $\delta$

In questo modo si {\`e} fatto sapere al {\mtex} quali numeri di paragrafo dovr{\`a} scrivere in nota
inserendo nel terzo sottocampo le etichette (i \verb"cervi"). L'{\mtex}
provveder{\`a} ad assegnare ai vari \verb"cervi" il numero di paragrafo
della loro \verb"\Unit"\index{\bs{}Unit}. Naturalmente per fare questo si {\`e} dovuta
introdurre un'ulteriore complicazione (coraggio!, {\`e} l'ultima): per
evitare che il {\mtex} scambiasse i richiami delle etichette
per parole da scrivere si {\`e} dovuta intodurre una nuova macro, la macro
\verb"\UN{}\index{\bs{}UN}" che gli spieghi che \verb"cervo1" non {\`e} una lezione o un
comando ma l'etichetta che richiama il paragrafo %2
1. Insomma il
rapporto fra \verb"\Unit[cervo1]"\index{\bs{}Unit[]} e \verb"\UN{cervo1}"\index{\bs{}UN} {\`e} lo stesso che inercorre
fra \verb"\CR{gatto}"\index{\bs{}CR} e \verb"\LB{gatto}{...}"\index{\bs{}LB}.

Con questo sistema si {\`e} potuto anche
dire esplicitamente che B ha omesso il \S\,%2
1, inserendo una macro
\verb"\DES{}"\index{\bs{}DES} appropriata nel campo riservato alla descrizione dei
paragrafi di B.

Nell'esempio, \verb"cervo1" corrisponde al paragrafo %2
1; tuttavia se
l'editore modificasse la suddivisione in paragrafi che precede questo
passo (aggiugendone ad esempio quattro), la \verb"\Unit[cervo1]"\index{\bs{}Unit[]} verr{\`a}
automaticamente rinumerata come paragrafo %2
5: ma l'{\mtex} lo sapr{\`a} e in nota scriver{\`a} sempre e comunque il
numero corretto.

Naturalmente l'editore dovr{\`a} stare molto attento se dovesse
modificare la suddivisione delle \verb"\Unit"\index{\bs{}Unit} proprio all'interno
del passo. Se decidesse ad esempio di unificare le
\verb"\Unit[cervo1]"\index{\bs{}Unit[]}, dovr{\`a} ricordare di
modificare di conseguenza anche le etichette che ha inserito
all'interno dei terzi sottocampi di \verb"\TV[unit]"\index{\bs{}TV[unit]}.

{\`E} importante osservare una particolarit{\`a} della sintassi di
\verb"\TV[unit]"\index{\bs{}TV[unit]}. Se si vuole che in nota i numeri di paragrafo compaiano
nella forma \textbf{%2
1--%2
5}, le etichette dovranno essere inserite nella
forma:

\begin{quote}
\begin{verbatim}
\UN{cervo1}-\UN{cervo5}
\end{verbatim}
\end{quote}

\noindent se si volesse ottenere \textbf{%2
1--%2
2,
%2
5, %2
3--%2
4} dovrebbe scrivere:

\begin{quote}
\begin{verbatim} 
\UN{cervo1}-\UN{cervo2}, \UN{cervo5},
\UN{cervo3}-\UN{cervo4}.
\end{verbatim}
\end{quote}

La regola {\`e} la seguente:

\begin{itemize}

\item fra i nomi delle etichette deve essere
lasciato uno spazio bianco; si pu{\`o} inserire una virgola o un trattino
(\verb"-"); se si inserisce una virgola in nota si produrr{\`a} una
virgola; se il trattino, il trattino. Si badi bene che, nel caso si
inserisca la virgola, esse deve essere seguita da uno spazio bianco.

\end{itemize}

L'{\mtex} provvede poi a estrarre il testo dei
testimoni secondo l'ordine che gli viene detto da
\verb"\TV[unit]"\index{\bs{}TV[unit]}: questo rende inutile specificare nel seguito del
testo gli scambi e le trasposizioni ``locali''.

%-----------------------------------------------------------------------
\subsection{L'esempio al completo}

\label{ref-8.5.4}

Come promesso, diamo qui la trascrizione completa in {\mtex}
dell'esempio:

\begin{quote}
\begin{verbatim}
\Unit[cervo1]\TV[unit]{
  {*:\ED{hoc ordine disposui}:\UN{cervo1}-\UN{cervo5}}
  {C:\UN{cervo1}-\UN{cervo2} \UN{cervo4}-\UN{cervo5}}
  {A:\UN{cervo5} \UN{cervo4} \UN{cervo3} 
     \UN{cervo2} \UN{cervo1}}
  {B:\DES{omisso \UN{cervo1}}:\UN{cervo2}
      \UN{cervo5} \UN{cervo3}-\UN{cervo4}}
                       } Centrum uniformis figurae in
puncto axis medio constituitur. \Unit[cervo2] Centrum
trianguli rectilinei trientem axis ad
basim relinquit. \Unit[cervo3] \VV[longa]{
                                   {*:\CR{foca}:Centrum}
                                   {C:\OMLAC}
                                          } totius
interiacet centris partium in eadem recta
\LB{foca}{constitutum}. \Unit[cervo4]
\VV[longa]{
        {A/C:\CR{delfino}:Centrorum}
        {B:Centrorum distantiae a centro 
          corporis gravis respectivae sunt
         partibus}
            } partialium distantiae a centro
totius reciprocae sunt \LB{delfino}{partibus}.
\Unit[cervo5] Centrum nunquam cadit extra rei gravis
ambitum. \Unit ...
\end{verbatim}
\end{quote}

Come si pu{\`o} osservare, per dar conto dell'omissione
di B e di C si sono ovviamente introdotte due \verb"\VV[longa]{}"\index{\bs{}VV[longa]}.
Inoltre si noti che alla prima \verb"\Unit"\index{\bs{}Unit} inserita dopo il brano
incriminato (quella che segue le
parole \verb"extra rei gravis ambitum") \textit{non} sono state
aggiunte le \verb"[ ]" per inserire l'argomento facoltativo, dato
che sarebbe completamente inutile.

%-----------------------------------------------------------------------
\subsection{Alcune avvertenze}

\label{ref-8.5.5}

Non riteniamo necessario dare altri esempi dell'uso di
\verb"\TV[unit]"\index{\bs{}TV[unit]}: basti osservare che se invece di una
trasposizione congetturale si fosse scelto l'ordine di uno
dei testimoni (ad esempio A) si sarebbe dovuto scrivere nel
primo campo di \verb"\TV[unit]"\index{\bs{}TV[unit]}:

\begin{quote}
\begin{verbatim}
{A:\DES{hoc ordine}:\UN{L1}-\UN{L3}}
\end{verbatim}
\end{quote}

\noindent dove \verb"L1 L2 L3" sono le etichette che si saranno apposte
alle \verb"\Unit[ ]"\index{\bs{}Unit}. E sempre usando \verb"\DES"\index{\bs{}DES} si dar{\`a} conto
di trasposizioni volontarie del copista o di altre mani.

Bisogna inoltre avvertire che se ci si trovasse in presenza
di una tradizione testuale particolarmente complessa in cui,
oltre a una trasposizione ``globale'' di parti del testo nei
vari testimoni, ci fossero varianti e trasposizioni che si
accavallassero fra le varie \verb"\Unit"\index{\bs{}Unit}, il sistema esposto in
questo \S\,\ref{ref-8.5}, diventa praticamente
inutilizzabile perch{\'e} rende assai problematica l'estrazione
del testo dei testimoni. Una situazione del genere, d'altra parte,
sar{\`a} opportuno che venga discussa attentamente
nell'introduzione all'edizione. Per darne conto in apparato
l'editore potr{\`a} o scegliere le sue \verb"\Unit"\index{\bs{}Unit} in modo che non
ci siano varianti che ne abbraccino due alla volta, o
rinunciare a usare \verb"\TV[unit]"\index{\bs{}TV[unit]} o ripiegare sulla meno
elegante, ma pi{\'u} sicura, \verb"\TV[longa]"\index{\bs{}TV[longa]}.

Avvertiamo infine che, come sempre quando si utilizzano
etichette, bisogna compilare due volte il \textit{file} (cfr.
\S\,\ref{ref-7.1.1.4}).


%-----------------------------------------------------------------------
\chapter{Casi eccezionali}

\label{ref-9}

\section{Le lacune soggettive}

\label{ref-9.1}
\index{casi eccezionali}
\index{lacune soggettive}

Come si ricorder{\`a}, nel \S\,\ref{ref-3.3.1} si {\`e} accennato al fatto che
il trascrittore potrebbe, per difficolt{\`a} soggettive, essere
incapace di leggere una o pi{\'u} parole del testo. L'editore
compir{\`a} ogni sforzo umanamente possibile per risolvere il
problema: ma non {\`e} detto che ci riesca.

Tale caso
potrebbe essere trattato nel seguente modo. Supponiamo di
avere due testimoni A e B; A abbia come lezione
``musica est $\widetilde{aia}$ mundi'' (anima mundi) e
in B si leggano le parole ``musica est'' seguite poi
da una ``a'' seguita da uno strano segno di
abbreviazione e poi da un simbolo altrettanto indecifrabile e
inusuale. Il trascrittore di B avr{\`a} registrato la situazione
in questo modo (\S\,\ref{ref-3.3.1}):

\begin{quote}
\begin{verbatim}
musica est \LACs{qui c'{\`e} un'abbreviazione
indecifrabile}
\end{verbatim}
\end{quote}

L'editore dopo aver consultato tutti i paleografi pi{\'u} noti,
studiato a fondo gli usi scrittorii del XVI secolo e aver
meditato a lungo sul problema, non
riesce a dare un senso alla misteriosa brachigrafia. {\`E} s{\'\i}
ben disposto a credere che significasse ``anima mundi'', ma
non {\`e} in grado di provarlo. Codificher{\`a} allora il suo
imbarazzo in questo modo:

\begin{quote}
\begin{verbatim}
musica est \VV{
              {A:\ABBR{anima} mundi}
              {B:\DES{signa mihi incognita}:\LACs}
              }
\end{verbatim}
\end{quote}

\verb"\LACs"\index{\bs{}LACs} provveder{\`a} poi a stampare  *** nel testo
di B e a marcare l'esistenza di una lacuna nel teso dovuta
non a cause oggettive, ma all'incapacit{\`a} soggettiva
dell'editore di leggere i segni che B riporta. Occorre
osservare che si deve usare \verb"\DES{}"\index{\bs{}DES} e non \verb"\ED{}"\index{\bs{}ED} (anche se
{\`e} l'editore che scrive) in modo che nel testo del testimone
B che verr{\`a} estratto compaia la nota ``signa mihi
incognita'' (cfr \S\,\ref{ref-6.4}).

Si noti anche
l'uso di \verb"\ABBR{}"\index{\bs{}ABBR} per marcare il fatto che ``anima'' {\`e} s{\'\i}
leggibile, ma solo dopo l'interpretazione di
un'abbreviazione (cfr. \S\,\ref{ref-3.4.5}).

%-----------------------------------------------------------------------
\section{Situazioni complicate}

\label{ref-9.2}

Specialmente quando ci si trova a dover descrivere lo stato di un
testimone (aggiunte in margine, cancellature, ecc.: cfr. capitolo
\ref{ref-5}), pu{\`o} avvenire che le macro fin qui trattate si rivelino
insufficienti. Introdurremo alcuni strumenti atti a trattare
situazioni del genere.

%-----------------------------------------------------------------------
\subsection{Cambi di carattere all'interno di
\texttt{$\protect\backslash$DES} e di \texttt{$\protect\backslash$ED}}

\label{ref-9.2.1}

Le macro \verb"\DES"\index{\bs{}DES} e \verb"\ED"\index{\bs{}ED} permettono di scrivere ci{\`o} che si vuole:
l'unica limitazione {\`e} che ci{\`o} che viene loro assegnato come argomento
viene scritto in tondo inclinato: nel carattere cio{\`e} utilizzato per 
distinguere in apparato fra ci{\`o} che viene deto dall'editore  e le
lezioni che vengono riportate, che vengono stampate in tondo. In
alcune situazioni ci{\`o} pu{\`o} rivelarsi piuttosto scomodo, e la difficolt{\`a}
deve poter essere aggirata. Supponiamo che il copista di A abbia
inizialmente scritto ``tringulum'', correggendosi poi aggiungendo la
lettera ``a'' mancante in interlinea. Il nostro trascrittore, molto
scrupoloso, vuole dare conto di questa situazione: e cio{\`e} non solo del
fatto che la lezione ``triangulum'' {\`e} corretta da ``tringulum'', ma
anche del come {\`e} stata corretta. Vorrebbe ottenere una nota di questo
tipo:

%\stelle
%triangulum\textsuperscript{1}
%\hrule width 1cm
%\textsuperscript{1}\qquad { triangulum \textsl{ex} tringulum} {\slnot
%addito in interl. { a} A}
%\stelle
%

\begin{maurotex}
\VV{
   {A:\EX{tringulum}\DES{addito in interl.}\LEC{a}:triangulum}
   }
\end{maurotex}
% ici je corrige le \Lec en \LEC et je mets le LEC end dehors de DES

Ci{\`o} pu{\`o} essere ottenuto nel seguente modo, combinando l'uso
di \verb"\DES"\index{\bs{}DES} e di \verb"\EX"\index{\bs{}EX} e con una nuova macro, \verb"\LEC"\index{\bs{}LEC}\new:

\begin{quote}
\begin{verbatim}
\VV{
   {A:\EX{tringulum}\DES{addito in interl.}\LEC{a}:triangulum}
   }
\end{verbatim}
\end{quote}

\noindent dove la nuova macro \verb"\LEC{}"\index{\bs{}LEC} serve a marcare
il fatto che il suo argomento {\`e} una lezione e non un commento
dell'editore  e provvede a scriverlo nel carattere opportuno. 

{\new} Attention: la macro \verb"\LEC" et son argument doivent se trouver
en dehors de l'argument de la macro \verb"\DES". De plus, il ne faut pas
mettre d'espace entre les diff\'erentes macros.

Ecco un altro esempio dell'uso di \verb"\LEC"\index{\bs{}LEC}, combinato con
un annidamento di \verb"\VV"\index{\bs{}VV}. Supponiamo che l'editore sia
incerto se in un certo passo di A si debba leggere ``litera'' (lettera) o
``litura'' (sgorbio). Tuttavia il testimone B reca chiaramente la lezione
``litera''. Decide quindi di porre in TC ``litera'', ma vuole segnalare
comunque la sua esitazione. Potr{\`a} optare per un apparato di questo
tipo:

%\stelle
%litera\textsuperscript{1}
%\hrule width 1cm
%\textsuperscript{1}\qquad { litera \textsl{(}litera \textsl{vel}
%litura} \textsl{B) AB}
%\stelle

\begin{maurotex}
\VV{
   {A/B:\VV{
           {A:litera}
           {B:\DES{vel}\LEC{litura}:litera}
           }
   }
   }
\end{maurotex}
% ici j'ai corrig\'e le \Lec en LEC et j'ai enlev\'e le LEC du DES
% Ok. Cela fonctionne sans m2lv en latex !

Per ottenerlo dovr{\`a} battere:

\begin{quote}
\begin{verbatim}
\VV{
   {A/B:\VV{
           {A:litera}
           {B:\DES{vel}\LEC{litura}:litera}
           }
   }
   }
\end{verbatim}
\end{quote}

Tutto quello che {\`e} stato detto per \verb"\DES"\index{\bs{}DES} vale
ovviamene anche per \verb"\ED"\index{\bs{}ED}: si ricordi tuttavia che la
macro \verb"\ED"\index{\bs{}ED} serve per annotazioni dell'editore che
verranno registrate solo nel testo critico e non in quello dei testimoni
(cfr. \S\,\ref{ref-6.4}).

%-----------------------------------------------------------------------
\subsection{Descrizioni complesse}

\label{ref-9.2.2} 
\index{descrizioni complesse}

In questo paragrafo introdurremo una nuova macro,
\verb"\DESCOMPL"\index{\bs{}DESCOMPL}, che serve a trattare situazioni
complesse.

Si sar{\`a} osservato (speriamo) che le macro \verb"\MARG"\index{\bs{}MARG},
\verb"\INTERL"\index{\bs{}INTERL}, \verb"\MARGSIGN"\index{\bs{}MARGSIGN} e
alcune altre non sono altro che abbreviazioni. Ad esempio 

\begin{quote}
\begin{verbatim}
\VV{{A:\MARG:conus}} 
\end{verbatim}
\end{quote}

{\`e} equivalente\footnote{ L'equivalenza tuttavia
riguarda solo l'output che si ottiene: dal punto di vista del \textit{markup} del testo usare \texttt{$\backslash$MARG} permette di recuperare l'informazione
relativa a tutte le aggiunte marginali, ecc.} a

\begin{quote}
\begin{verbatim}
\VV{{A:\DES{in marg.}:conus}}
\end{verbatim}
\end{quote}

Come si {\`e} appena visto,
tuttavia, \texttt{$\backslash$DES} permette di fare cose che sono difficilmente
codificabili una volta per tutte. 

Ora, lo stesso rapporto che c'{\`e} fra \verb"\DES"\index{\bs{}DES} e queste macro c'{\`e} anche
fra la macro \verb"\DESCOMPL{}{}{}"\index{\bs{}DESCOMPL} e le macro \verb"\POSTDEL{}"\index{\bs{}POSTDEL}, \verb"\ANTEDEL{}"\index{\bs{}ANTEDEL}, \verb"\EX{}"\index{\bs{}EX}.
Chiariamo subito il punto con un esempio.

Se vogliamo ottenere:

%\stelle
%triangulum secundum\textsuperscript{1} erit
%\hrule width 1cm
%\textsuperscript{1}\qquad \textsl{ante} { secundum} {\slnot
%del.} { primum} \textsl{A}
%\stelle

\begin{maurotex}
triangulum \VV{
               {A+:\ANTEDEL{primum}:secundum}
              } erit
\end{maurotex}

\noindent possiamo fare (cfr. \S\,\ref{ref-5.3.1}.a.):

\begin{quote}
\begin{verbatim}
triangulum \VV{
              {A+:\ANTEDEL{primum}:secundum}
              } erit
\end{verbatim}
\end{quote}

Gli elementi in gioco qui sono tre: la variante (\textit{primum}) e la descrizione. 

Perch{\'e} diciamo tre e non due? Perch{\'e} la
descrizione viene a essere ripartita in due parti: il fatto che {\`e}
stato cancellato qualcosa (\textsl{del.}) e il luogo dove questo {\`e} avvenuto
(\textsl{ante} \textit{secundum}). Di conseguenza una macro ``generale'' che
possa sostituire \verb"\ANTEDEL{}"\index{\bs{}ANTEDEL} dovr{\`a} avere tre campi: i primo e il
secondo li destineremo alla descrizione, il terzo alla
variante. In pratica \verb"\DESCOMPL{}{}{}"\index{\bs{}DESCOMPL} funziona cos{\'\i}: 

\begin{quote}
\begin{verbatim}
triangulum \VV{
              {A+:\DESCOMPL{ante}{del.}{primum}:secundum}
              } erit
\end{verbatim}
\end{quote}

Vediamo ora alcuni esempi che non
potrebbero essere trattati con le macro fin qui introdotte.

%-----------------------------------------------------------------------
\subsubsection{Esempio a}

\label{ref-9.2.2.1}

Il copista di A ha fatto un segno prima di ``pristinum'' e in margine
ha riportato il segno, scrivendo ``vel secundum''. L'editore non
accoglie in TC questo intervento del copista, ma lo vuole segnalare in
apparato:

%\stelle
%primum et pristinum\textsuperscript{1} erit
%\hrule width 1cm
%\textsuperscript{1}\qquad \textsl{signo posito ante} { pristinum} {\slnot
%in marg. add.} { vel secundum} \textsl{A}
%\stelle

\begin{maurotex}
primum et \VV{
{A+:\DESCOMPL{signo posito ante}{in marg add.}{vel secundum}:pristinum}
} erit
\end{maurotex}

\noindent potremo scrivere, usando \verb"\DESCOMPL"\index{\bs{}DESCOMPL}:

\begin{quote}
\begin{verbatim}
primum et \VV{
{A+:\DESCOMPL{signo posito ante}{in marg add.}{vel secundum}:pristinum}
} erit
\end{verbatim}
\end{quote}

{\new} Attention: toute la parenth\`ese de ``\{A+'' jusqu'\`a ``pristinum\}''
doit \^etre sur une m\^eme ligne.
%XXX \'ecrire \`a pm pour cela.

%-----------------------------------------------------------------------
\subsubsection{Esempio b}

\label{ref-9.2.2.2}

Il copista di A scrive ``.II.''; l'editore vuole invece 
``secundum'' e darne conto in apparato cos{\'\i}:

%\stelle
%Theorema secundum\textsuperscript{1} est
%\hrule width 1cm
%\textsuperscript{1}\qquad \textsl{pro} { secundum} {\slnot
%scripsit} { .II.} \textsl{A}
%\stelle

\begin{maurotex}
Theorema \VV{
{A+:\DESCOMPL{pro}{scripsit}{.II.}:secundum}
} est
\end{maurotex}

\noindent battendo:

\begin{quote}
\begin{verbatim}
Theorema \VV{
{A+:\DESCOMPL{pro}{scripsit}{.II.}:secundum}
} est
\end{verbatim}
\end{quote}

Come si vede, la regola per l'uso di \verb"\DESCOMPL{}{}{}"\index{\bs{}DESCOMPL} {\`e} la seguente:

\begin{itemize}

\item Ci{\`o} che {\`e} contenuto nel primo campo di \verb"\DESCOMPL"\index{\bs{}DESCOMPL}
viene scritto in apparato per primo (in tondo inclinato), seguito dalla
lezione contenuta nel terzo sottocampo di \verb"\VV"\index{\bs{}VV} (in tondo), seguito
da ci{\`o} che c'{\`e} nel secondo campo di \verb"\DESCOMPL"\index{\bs{}DESCOMPL} (in tondo
inclinato), seguito dalla 
variante contenuta nel 
terzo campo di \verb"\DESCOMPL"\index{\bs{}DESCOMPL} (in tondo). Se si {\`e} scritto.

\begin{quote}
\begin{verbatim}
\VV{{A:\DESCOMPL{D}{d}{V}:L}}
\end{verbatim}
\end{quote}

\noindent si otterr{\`a} una nota del tipo:

\begin{center}
\textsl{D}~L~\textsl{d}~V''.
\end{center}

\end{itemize}

Vale per{\`o} la pena  di fare ancora un paio di esempi. 

%-----------------------------------------------------------------------
\subsubsection{Esempio c}

\label{ref-9.2.2.3}

Riprendiamo l'esempio trattato nel paragrafo \S\,\ref{ref-9.2.1}. Per ottenere: 

%\stelle
%litera\textsuperscript{1}
%\hrule width 1cm
%\textsuperscript{1}\qquad { litera \textsl{(}litera \textsl{vel}
%litura} \textsl{B) AB}
%\stelle

\begin{maurotex}
\VV{
    {A/B:\VV{
            {A:litera}
            {B:\DESCOMPL{}{vel}{litura}:litera}
            }
	}
    }
\end{maurotex}

\noindent si potr{\`a} usare \verb"\DESCOMPL"\index{\bs{}DESCOMPL} lasciando il primo campo vuoto,
scrivendo:

\begin{quote}
\begin{verbatim}
\VV{
    {A/B:\VV{
            {A:litera}
            {B:\DESCOMPL{}{vel}{litura}:litera}
            }
	}
    }
\end{verbatim}
\end{quote}

In altre parole \verb"\DESCOMPL"\index{\bs{}DESCOMPL} codifica automaticamente,
secondo la regola qui sopra esposta, la differenza fra lezioni e
descrizioni. Si noti che l'esempio mostra anche che uno o due campi di
\verb"\DESCOMPL"\index{\bs{}DESCOMPL} possono essere lasciati vuoti, a seconda delle necessit{\`a}
dell'editore o del trascrittore. Un altro esempio del genere {\`e} il
seguente.

 
Supponiamo che l'editore voglia il seguente apparato:

%\stelle
%triangulum et quadratum\textsuperscript{1}
%\hrule width 1cm
%\textsuperscript{1}\qquad { quadratum\textsl{:} trigonum quadratum
%\textsl{deleto}
%trigonum} \textsl{A}
%\stelle

\begin{maurotex}
triangulum et
\VV{
   {*:quadratum}
   {A:\DESCOMPL{}{deleto}{trigonum}:trigonum quadratum}
   }
\end{maurotex}

\noindent per indicare che il copista di A aveva scritto `trigonum
quadratum'' ma ha poi cancellato ``trigonum''. Usando \verb"\DESCOMPL"\index{\bs{}DESCOMPL} si scriver{\`a}:

\begin{quote}
\begin{verbatim}
triangulum et
\VV{
   {*:quadratum}
   {A:\DESCOMPL{}{deleto}{trigonum}:trigonum quadratum}
   }
\end{verbatim}
\end{quote}

 
%-----------------------------------------------------------------------
\subsubsection{Esempio d}

\label{ref-9.2.2.4}

Segnaliamo infine che \verb"\DESCOMPL"\index{\bs{}DESCOMPL} pu{\`o} essere usata nel secondo
sottocampo insieme ad altre macro quali \verb"\DES"\index{\bs{}DES}, \verb"\MARG"\index{\bs{}MARG}, \verb"\INTERL"\index{\bs{}INTERL},
ecc. Si immagini che il copista di A dopo aver scritto ``Vide enim meum
astrolabium'' abbia aggiunto in interlinea ``quod non facile est'' e
che l'editore (che non vuole accogliere in TC l'aggiunta dello scriba)
si proponga di ottenere il seguente risultato:

%\stelle
%Vide enim meum astrolabium\textsuperscript{1}
%\hrule width 1cm
%\textsuperscript{1}\qquad \textsl{post} { astrolabium \textsl{add.}
%quod non facile est}
%\textsl{in interl. A$^{\hbox{\slnotm 1}}$}  
%\stelle

\begin{maurotex}
Vide enim meum \VV{
{A1:\DESCOMPL{post}{add.}{quod non facile est}\INTERL:astrolabium}
}
\end{maurotex}

\noindent cio{\`e}: ``dopo \textit{astrolabium} A\textsuperscript{1} ha
aggiunto in interlinea \textit{quod non facile est}''. 
Usando \verb"\DESCOMPL"\index{\bs{}DESCOMPL}:

\begin{quote}
\begin{verbatim}
Vide enim meum \VV{
{A1:\DESCOMPL{post}{add.}{quod non facile est}\INTERL:astrolabium}
}
\end{verbatim}
\end{quote}

%-----------------------------------------------------------------------
\subsection{Anche gli editori hanno i loro diritti}

\label{ref-9.2.3}

Abbiamo gi{\`a} insistito sul parallelismo fra \verb"\DES"\index{\bs{}DES} e \verb"\DESCOMPL"\index{\bs{}DESCOMPL}. Da
questo segue che anche \verb"\ED"\index{\bs{}ED} debba avere una sua parente che permetta
all'editore di trattare situazioni complesse per cui \verb"\ED"\index{\bs{}ED} non
basterebbe.

Tale macro {\`e} \verb"\EDCOMPL{}{}{}"\index{\bs{}EDCOMPL}, e la sua sintassi {\`e} identica a quella
di \verb"\DESCOMPL"\index{\bs{}DESCOMPL}, per cui non riteniamo opportuno riportare altri
esempi. La differenza fra \verb"\DESCOMPL"\index{\bs{}DESCOMPL} e \verb"\EDCOMPL"\index{\bs{}EDCOMPL} {\`e} una sola, ed {\`e} la
stessa che intercorre fra \verb"\ED"\index{\bs{}ED} e \verb"\DES"\index{\bs{}DES}: ci{\`o} che viene marcato con
\verb"\EDCOMPL"\index{\bs{}EDCOMPL} risulter{\`a} solo nell'apparato del testo critico, mentre ci{\`o}
che {\`e} marcato con \verb"\DESCOMPL"\index{\bs{}DESCOMPL} servir{\`a} a costruire il testo e
l'apparato critico dei singoli testimoni (cfr. \S\,\ref{ref-6.4}).

%-----------------------------------------------------------------------
\subsection{Post Scriptum}

\label{ref-9.2.4}

Pu{\`o} capitare di trovarsi in situazioni che nemmeno
\verb"\LEC"\index{\bs{}LEC} e \verb"\DESCOMPL"\index{\bs{}DESCOMPL}, da
sole, riescono a trattare. Tipico {\`e} il caso in cui la nota dell'editore
(o del trascrittore) debba venire \textit{dopo} la lezione di TC che si
riporta in apparato. Si consideri il seguente esempio. La tradizione sia
costituita dai testimoni A e H. H omette un brano (omissione che
probabilmente era gi{\`a} presente nel suo antigrafo), ma, per far correre
la frase, aggiunge una parola:

\begin{itemize}

\item[A:] Sit recta $ab$ parallela ipsae $cd$ et recta $mn$ parallela
ipsae $kl$ (tangens, ut supra dictum est, circulum $rsv$) 
quae est perpendicularis $cd$ et describatur ...

\item[H:] Sit recta $ab$ parallela ipsae $cd$ et \textit{non\/}
perpendicularis $cd$ et describatur ...

\end{itemize}

Si vorrebbe ottenere  un apparato di questo tipo:

\onlyhtml{
\begin{maurotex}
Sit recta $ab$ parallela ipsae $cd$ et \VV[longa]{
{*:\CR{unicorno}:recta}
{H:\POSTSCRIPT\DES{qui}\LEC{non}\DES{add. post }\LEC{et}:\OM}
}
$mn$ parallela ipsae $kl$ (tangens, ut
supra dictum est, circulum $rsv$) 
\LB{unicorno}{quae est} perpendicularis
$cd$ et describatur ...
\end{maurotex}
}

\onlylatex{
\begin{quote}
\par
\stelle
\par
Sit recta $ab$ parallela ipsae $cd$ et recta $mn$ parallela ipsae $kl$ (tangens, ut supra dictum est, circulum
$rsv$) quae est perpendicularis $cd$ et describatur ...\\
\rule{1.5cm}{1pt} \\
\begin{footnotesize}
recta\textsl{\textsf{ $\sim$ }}quae est\textsl{\textsf{:}}\hspace{3mm}\textsl{\textsf{ om.}} \textsl{\textsf{H}} \textsl{\textsf{qui}} non \textsl{\textsf{add. post
}} et
\end{footnotesize}
\par
\stelle
\par
\end{quote}
}
% ici on triche

Come si vede la descrizione ``qui \textit{non\/} add. post \textit{et\/}'' deve essere inserita dopo la ``lezione'' di H (che in questo
caso {\`e} assente, dato che omette. Per far questo introduciamo una nuova
macro, \verb"\POSTSCRIPT"\index{\bs{}POSTSCRIPT}, che permette di trattare la cosa:

\begin{quote}
\begin{verbatim}
Sit recta $ab$ parallela ipsae $cd$ et \VV[longa]{
{*:\CR{unicorno}:recta}
{H:\POSTSCRIPT\DES{qui}\LEC{non}\DES{add. post }\LEC{et}:\OM}
}
$mn$ parallela ipsae $kl$ (tangens, ut
supra dictum est, circulum $rsv$) 
\LB{unicorno}{quae est} perpendicularis
$cd$ et describatur ...
\end{verbatim}
\end{quote}

\noindent dove si osservi l'uso di \verb"\LEC"\index{\bs{}LEC} per marcare
le lezioni. {\new} La macro \verb"\POSTSCRIPT"\index{\bs{}POSTSCRIPT} sert de
marqueur pour indiquer que tout ce qui la suit doit \^etre \'ecrit apr\`es
le sigle du t\'emoin.

Un altro esempio. I testimoni siano F e P:

\begin{itemize}

\item[F:]  Sit recta $ab$ diameter $abe$ circuli rectae
$cd$ perpendicularis

\item[P:]  Sit recta $ab$ diameter $abe$
\onlylatex{{\d s}p{\d h}{\d a}{\d e}{\d r}{\d a}{\d e}}
\onlyhtml{\includegraphics{manicons/sphaerae.gif}}
rectae $cd$ perpendicularis

\end{itemize}
% ici on triche

\noindent e i puntini sotto ``sphaerae'' sono in inchiostro rosso. Si
vuole ottenere questo apparato:

\onlyhtml{
\begin{maurotex}
Sit recta $ab$ diameter $abe$ \VV{
{F:circuli}
{P:\POSTSCRIPT\DES{qui interpunxit rubro atramento}:sphaerae}
} rectae $cd$ perpendicularis
\end{maurotex}
}

\onlylatex{
\begin{quote}
\par
\stelle
\par
Sit recta
$ab$ diameter $abe$ circuli rectae $cd$ perpendicularis 
\\
\rule{1.5cm}{1pt} \\
\begin{footnotesize}
circuli \textsl{\textsf{F}}\hspace{3mm}sphaerae \textsl{\textsf{P}} \textsl{\textsf{qui interpunxit rubro atramento}}
\end{footnotesize}
\par
\stelle
\par
\end{quote}
}

Usando \verb"\POSTSCRIPT"\index{\bs{}POSTSCRIPT} si batter{\`a}:

\begin{quote}
\begin{verbatim}
Sit recta $ab$ diameter $abe$ \VV{
{F:circuli}
{P:\POSTSCRIPT\DES{qui interpunxit rubro atramento}:sphaerae}
} rectae $cd$ perpendicularis
\end{verbatim}
\end{quote}

%Insomma: \verb"\POSTSCRIPT"\index{\bs{}POSTSCRIPT} si usa come %\verb"\DES"\index{\bs{}DES} e ha esattamente
%tutte le sue caratteristiche. L'unica differenza {\`e} che %\verb"\POSTSCRIPT"\index{\bs{}POSTSCRIPT}
%colloca l'annotazione del trascrittore  o dell'editore \textit{dopo} la
%lezione o la pseudo lezione riportata in apparato, mentre %\verb"\DES"\index{\bs{}DES} la
%colloca \textit{prima}.

{\new} L'unique diff\'erence de l'emploi de \verb"\DES" avec ou sans
\verb"\POSTSCRIPT", est que cette derni\`ere macro place la note du
transcripteur ou de l'\'editeur \textit{apr\`es} la le\c{c}on ou la pseudo
le\c{c}on report\'ee en apparat, alors que sans \verb"\POSTSCRIPT", elle
est plac\'ee \textit{avant}.

%-----------------------------------------------------------------------
\section{A mali estremi}

\label{ref-9.3}

Come abbiamo gi{\`a} detto, questo manuale non pu{\`o} ovviamente
prevedere tutti i casi possibili che possano presentarsi al
trascrittore e all'editore di un testo mauroliciano; a parte
ogni considerazione di fattibilit{\`a}, ci{\`o} produrrebbe sintassi
mostruosamente complicate (assai pi{\'u} delle presenti\dots).

Abbiamo quindi previsto una macro
(\verb"\note"\index{\bs{}note})\footnote{\texttt{$\backslash$note} non va confusa con gli
altri tipi di ``note'' a disposizione (cfr. \S\,\ref{ref-3.8}). Gli altri tipi
servono per generare commenti al testo o agli interventi dell'editore, e
vengono gestiti in modo del tutto indipendente dalla numerazione delle note
dell'apparato testuale. \texttt{$\backslash$note}, invece, produce un testo
e un numero di nota coerente con l'uso di \texttt{$\backslash$VV} e di
\texttt{$\backslash$VV[longa]}.} da usarsi nei casi in cui tutte le risorse
del manuale siano esaurite e occorra intervenire ``a mano''. Essa andr{\`a}
usata con molta parsimonia, sia perch{\'e} impone all'editore e al
trascrittore una discreta mole di lavoro in pi{\'u}, sia perch{\'e} pu{\`o}
essere causa di difformit{\`a} procedurali anche gravi che potrebbero
rendere poi complesso l'ulteriore trattamento del testo per la sua messa in
rete o per l'estrazione del testo dei singoli testimoni.

Per esempio, il caso trattato qui sopra nel \S\,\ref{ref-8.4.3}, avrebbe
potuto essere risolto in questo modo con \verb"\note"\index{\bs{}note}:

\begin{quote}
\begin{verbatim}
Namque superficies talium\note{
talium~$\sim$~componuntur
{\sl: huc transposui~~post}
superficiebus {\sl A}
} solidorum
componuntur ex similibus conicis superficiebus.
\end{verbatim}
\end{quote}

Come si vede qui l'editore deve dichiarare esplicitamente
all'{\mtex} tutto quello che deve fare per scrivere
la nota: cambi di carattere (\verb"\sl"\index{\bs{}sl}); segni speciali come \verb"~"
che produce lo spazio bianco codificato da lasciare  prima e
dopo la $\sim$ che deve comparire in nota, in modo che essa
rimanga ``attaccata'' alle parole che la precedono e la
seguono; e la strana sequenza \verb"$\sim$", che produce appunto
la $\sim$. Si rifletta anche sul fatto che  questo {\`e} un
caso relativamente semplice.

Oppure, pensate al primo esempio del capitolo \ref{ref-4}:

%\stelle
%(TC) Sit data ratio\textsuperscript{1}, sit datus cubus.
%
%\hrule width 1cm
%
%\textsuperscript{1}\qquad { ratio} \textsl{A:} {
%gratia} {\slnot
%B;} { latio} \textsl{C}
%\stelle

\begin{maurotex}
(TC) Sit data
ratio\note{
           ratio {\sl A}~~gratia {\sl B}~~latio {\sl C}
          }, sit datus cubus.
\end{maurotex}

Per ottenerlo con \verb"\note"\index{\bs{}note} avreste dovuto scrivere:

\begin{quote}
\begin{verbatim}
(TC) Sit data
ratio\note{
           ratio {\sl A}~~gratia {\sl B}~~latio {\sl C}
          }, sit datus cubus.
\end{verbatim}
\end{quote}

\noindent dichiarando sempre il cambio di carattere da tondo
a tondo inclinato e aggiungendo i doppi spazi codificati \verb"~~"
per distanziare le varie lezioni.

Come vedete, l'uso di \verb"\note"\index{\bs{}note} comporta molta attenzione  redazionale per 
far s{\'\i} che  ci{\`o} che otterrete sia omogeneo tipograficamente 
al resto della vostra edizione. Ma ci{\`o} che {\`e} peggio, quello che
scrivete in \verb"\note"\index{\bs{}note}
non viene codificato: risulter{\`a} difficile da trasferire
automaticamente in \textsc{html} (il linguaggio che si utilizza per
le pagine Web) e di conseguenza la vostra edizione rischier{\`a}
di presentarsi in modo bizzarro sugli schermi del
cyberspazio. La difficolt{\`a} {\`e} accresciuta dal fatto
che il {\TeX} permette una gran libert{\`a} di espressione
ai suoi utilizzatori, e non potendosi dare una regola fissa
con cui scrivere le note, la conversione in \textsc{html} dovr{\`a}
necessariamente farsi a mano, caso per caso. Con la
conseguenza che se l'editore decidesse di ritoccare la sua
nota, bisogner{\`a} intervenire ancora una volta
manualmente e cos{\'\i} via~{\dots}

Inoltre, non essendo il contenuto di \verb"\note"\index{\bs{}note} codificato,
l'estrazione del testo dei singoli testimoni dovr{\`a} anch'essa
essere compiuta manualmente, con complicazioni ancora
peggiori di quelle qui sopra accennate.

%-----------------------------------------------------------------------
\subsection{Un caso inevitabile}

\label{ref-9.3.1}

Dopo avervi cercato di scoraggiare in tutti i modi possibili
dall'usare \verb"\note"\index{\bs{}note}, vi proponiamo un esempio in cui il ricorso
ad essa diventa praticamente indispensabile.

Riprendiamo l'esempio del \S\,\ref{ref-6.2.2} e supponiamo che di
fronte a un testimone unico, A, che legga ``primum et
secundumque'' Clagett abbia espunto ``et'', mentre Napoli
abbia espunto ``que''. Se il nostro editore vuol seguire
Clagett potr{\`a} elaborare un apparato di questo genere:

%\stelle
%Erunt primum [et] secundumque\textsuperscript{1} triangula \dots
%\hrule width 1cm
%\textsuperscript{1}\qquad { primum [et] secundumque}{\slnot
%:} { et}
%\textsl{secl. Clagett;}~~{ que} \textsl{secl. Napoli}
%\stelle

\begin{maurotex}
Erunt primum
%[et] secundumque\note{
%primum [et] secundumque {\sl :} et {\sl secl. Clagett;}~~
%que {\sl secl. Napoli}
%}
\end{maurotex}
%XXX ne passe pas le m2lv ou m2hv \`a cause des [].

Apparato che potr{\`a} essere ottenuto scrivendo:

\begin{quote}
\begin{verbatim}
Erunt primum [et] secundumque\note{
primum [et] secundumque {\sl :} et {\sl secl. Clagett;}~~
que {\sl secl. Napoli}
}
triangula ...
\end{verbatim}
\end{quote}

\centerline{\textbf{*~*~*}}

In conclusione: l'uso di \verb"\note"\index{\bs{}note} non {\`e} vietato
(altrimenti che cosa ci starebbe a fare?), ma {\`e}
opportuno usare questa macro \textit{esclusivamente} nel caso
in cui sia proprio impossibile farne a meno. Si tratta,
insomma, di un'\textit{extrema ratio}. E nel caso
l'editore la usi, deve assumersi poi la responsabilit{\`a} di
seguire il destino delle sue \verb"\note"\index{\bs{}note} fino in fondo, cio{\`e} fino
all'ultimo stadio del suo lavoro, tenendo inoltre un
registro chiaro, e utilizzabile da altri, dei suoi
interventi.

%-----------------------------------------------------------------------
\chapter{Il ``Conspectus siglorum''}

\label{ref-10}

\section{I testimoni di Maurolico}

\label{ref-10.1}

Negli esempi che abbiamo fin qui fatto abbiamo indicato i testimoni con le
lettere A, B, C, ecc. Tuttavia, essendo questo \textit{Manuale} dedicato
specificamente all'edizione dell'opera matematica di Maurolico, occorre
stabilire un sistema di \textit{sigla} uniformi che tenga conto delle
specificit{\`a} della tradizione dei testi mauroliciani\footnote{Per la
costituzione di questo \textit{Conspectus siglorum} dobbiamo un particolare
ringraziamento a Ottavio Besomi per l'attenzione e l'acume con cui ci ha
consigliati.}.

 
I testimoni di Maurolico si suddividono in tre categorie:

\begin{enumerate}

\item manoscritti autografi, che designeremo con la lettera A;
\item manoscritti di altra mano (copie), designati con la C;
\item edizioni a stampa, designati con la lettera S.

\end{enumerate}

All'interno di ciascuna categoria i testimoni vengono contraddistinti
da un numero arabo che segue la rispettiva lettera (A1, A2 ... An; C1,
C2 ... Cn; S1, S2 ... Sn).
Di tali testimoni viene fornita, nell'introduzione generale
all'edizione, un'apposita lista (che troverete qui nel  \S\,\ref{ref-10.4}) in cui:

\begin{itemize}

\item[a.]  i manoscritti (prima gli autografi, poi quelli di altra mano) sono
elencati secondo l'ordine alfabetico delle citt{\`a} in cui essi sono
conservati (eventualmente, all'interno delle citt{\`a}, secondo
l'ordine alfabetico delle Biblioteche), 

\item[b.]  le edizioni a stampa sono elencate cronologicamente.


\end{itemize}

%-----------------------------------------------------------------------
\section{Designazione dei testimoni per ogni singolo testo}

\label{ref-10.2}

In capo a ogni testo si indicano quali testimoni vengono utilizzati e
con quali sigle vengono designati all'interno di quella particolare
edizione. Se per quel testo disponiamo di un solo testimone per ogni
categoria, l'eventuale autografo verr{\`a} designato con la semplice
lettera A, il manoscritto non autografo con la C, l'edizione a stampa
con la S; oltre a fornire i dati fondamentali per identificare i
testimoni (citt{\`a}, biblioteca e segnatura per i manoscritti; titolo
e indicazioni tipografiche per i testi a stampa) l'editore
provveder{\`a} tuttavia a indicarne tra parentesi anche la sigla
alfanumerica con cui sono indicati nel \textit{conspectus siglorum}
generale. Per esempio:

\begin{itemize}

\item[A=] Paris, Biblioth{\`e}que Nationale, Lat. 7463 (A6)
\item[C=] Lucca, Biblioteca governativa, 2080 (C6)
\item[S=] \textit{Quadrati horarii fabrica et eius usus, ut hoc solo
instrumento, coeteris praetermisssis, unusquisque mathematicus
contentus esse possit. Per Franciscum Maurolicum nuper
edita. Illustriss. D.D. Ioanni Vigintimillio Ieraciensium Marchioni
d.}, Venetiis, apud Nicolaum Bascarinum, MDXLVI (S3)

\end{itemize}

Se invece siamo in presenza di due autografi, chiaremo l'uno `A' e
l'altro `a', fornendo il seguente prospetto:

\begin{itemize}

\item[A=] Paris, Biblioth{\`e}que Nationale, Lat. 7463 (A6)
\item[a=] Roma, Biblioteca Nazionale, San Pantaleo 115/32 (A18)
\item[C=] Lucca, Biblioteca governativa, 2080 (C6)
\item[S=] \textit{Quadrati horarii fabrica et eius usus, ut hoc solo
instrumento, coeteris praetermisssis, unusquisque mathematicus
contentus esse possit. Per Franciscum Maurolicum nuper
edita. Illustriss. D.D. Ioanni Vigintimillio Ieraciensium Marchioni
d.}, Venetiis, apud Nicolaum Bascarinum, MDXLVI (S3)

\end{itemize}

Similmente, se abbiamo a che fare con due manoscritti non autografi o
con due stampe,  useremo rispettivamente le sigla `c' e
`s'.

%-----------------------------------------------------------------------
\section[Interventi del copista]{Interventi del copista o di altra mano su un testimone}

\label{ref-10.3}

L'intervento sul manoscritto di base (aggiunte, correzioni,
cancellature ecc.) viene sempre indicato con una cifra arabica o con
una lettera posta in esponente dopo il \textit{siglum}. Come spiegato nel
\S\,\ref{ref-5.1}, gli esponenti distinguono gli
interventi di Maurolico, indicato con la lettera `m', e di altre mani,
indicate con un numero:

\begin{itemize}
\item[A\textsuperscript{m}:] intervento di Maurolico sull'autografo A,
\item[a\textsuperscript{m}:] intervento di Maurolico sull'autografo a,
\item[A\textsuperscript{3:}] intervento di una mano successiva sull'autografo A,
\item[a\textsuperscript{3:}] intervento di una mano successiva
sull'autografo a, 

\item[C\textsuperscript{1}:] intervento correttivo del copista sul manoscritto non
autografo C,
\item[c\textsuperscript{1}:] intervento correttivo del copista sul manoscritto non
autografo c, 
\item[C\textsuperscript{m}:] intervento di Maurolico sul manoscritto non autografo C,
\item[c\textsuperscript{m}:] intervento di Maurolico sul manoscritto non autografo c,
\item[C\textsuperscript{3:}] intervento di altra mano sul manoscritto non autografo C,
\item[c\textsuperscript{m}:] intervento di altra mano sul manoscritto non autografo c,

\item[S\textsuperscript{1}:] correzioni indicate dal tipografo dell'edizione S nella
tabella degli \textit{Errata corrige},
\item[s\textsuperscript{1}:] correzioni indicate dal tipografo dell'edizione s nella
tabella degli \textit{Errata corrige},
\item[S\textsuperscript{1}:] correzioni indicate dal tipografo dell'edizione S nella
tabella degli \textit{Errata corrige},
\item[S\textsuperscript{m}:] intervento di Maurolico su un esemplare (opportunamente
specificato nell'introduzione) dell'edizione S, 
\item[s\textsuperscript{m}:] intervento di Maurolico su un esemplare (opportunamente
specificato nell'introduzione) dell'edizione s.

\end{itemize}

%-----------------------------------------------------------------------
\section{Il ``Conspectus Siglorum''}

\label{ref-10.4}
\index{conspectus siglorum}

Ed ecco l'elenco dei manoscritti che concorrono all'edizione e dei
loro \textit{sigla}:

%-----------------------------------------------------------------------
\subsection{Scripta manu propria Maurolyci exarata}

\label{ref-10.4.1}

\begin{itemize}
\item[A1:] El Escorial, Biblioteca Real de San Loren{\c c}o J.III.31 
\item[A2:] Firenze, Biblioteca Nazionale, Magl. CI. XIV.39
\item[A3:] Molfetta, Biblioteca del seminario vescovile, 5-7 H 15
\item[A4:] Paris, Biblioth{\`e}que Nationale de France, Par. Lat. 6177
\item[A5:] Paris, Biblioth{\`e}que Nationale de France, Par. Lat. 7249
\item[A6:] Paris, Biblioth{\`e}que Nationale de France, Par. Lat. 7462
\item[A7:] Paris, Biblioth{\`e}que Nationale de France, Par. Lat. 7463
\item[A8:] Paris, Biblioth{\`e}que Nationale de France, Par. Lat. 7464
\item[A9:] Paris, Biblioth{\`e}que Nationale de France, Par. Lat. 7465
\item[A10:] Paris, Biblioth{\`e}que Nationale de France, Par. Lat. 7466
\item[A11:] Paris, Biblioth{\`e}que Nationale de France, Par. Lat. 7467
\item[A12:] Paris, Biblioth{\`e}que Nationale de France, Par. Lat. 7468
\item[A13:] Paris, Biblioth{\`e}que Nationale de France, Par. Lat. 7471
\item[A14:] Paris, Biblioth{\`e}que Nationale de France, Par. Lat. 7472
\item[A15:] Paris, Biblioth{\`e}que Nationale de Rrance, Par. Lat. 7472A 
\item[A16:] Paris, Biblioth{\`e}que Nationale de France, Par. Lat. 7473
\item[A17:] Parma, Biblioteca palatina, 1023.6
\item[A18:] Roma, Archivum Romanum Societatis Iesu, Ital. 137
\item[A19:] Roma, Biblioteca Nazionale, San Pantaleo 115/32 
\item[A20:] Roma, Biblioteca Nazionale, San Pantaleo 116/33 
\item[A21:] Roma, Biblioteca Nazionale, San Pantaleo 117/34 
\item[A22:] Vaticano (Citt{\`a} del), Bibliotheca Apostolica Vaticana,
Barb. Lat. 2158 

\end{itemize}

%-----------------------------------------------------------------------
\subsection{Codices aliis manibus exarati}

\label{ref-10.4.2}

\begin{itemize}

\item[C1:] Catania, Biblioteca Universitaria, U.52
\item[C2:] El Escorial, Biblioteca Real de San Lorenzo, \& IV.22
\item[C3:] Erlangen, Universit{\"a}tsbibliothek, 831
\item[C4:] Erlangen, Universit{\"a}tsbibliothek, 832
\item[C5:] Erlangen, Universit{\"a}tsbibliothek, 833
\item[C6:] Hamburg, Stadtsbibliothek, Cod. Math. 483 (Quarto)
\item[C7:] Lucca, Biblioteca governativa, 2080
\item[C8:] Madrid, Biblioteca de la Real Academia de la Historia, Cortes 2787
\item[C9:] Napoli, Biblioteca nazionale, I.E.56
\item[C10:] Oxford, Bodleian Library, 6556
\item[C11:] Paris, Biblioth{\`e}que Nationale de France, Par. Lat. 7251
\item[C12:] Paris, Biblioth{\`e}que Nationale de France, Par. Lat. 17859
\item[C13:] Roma, Archivio Pontificia Universit{\`a} Gregoriana, Fondo
Curia, 2052
\item[C14:] Vaticano (Citt{\`a} del), Bibliotheca Apostolica Vaticana,
Vat. Lat. 3131 

\end{itemize}

%-----------------------------------------------------------------------
\subsection{Editiones typis impressae}

\label{ref-10.4.3}

\begin{itemize}

\item[S1:] \textit{Grammaticorum rudimentorum libelli sex Franscisco
Maurolycio authore. Donati Barbarismus, Marii Servii Centimetrum,
Horatii necnon et Boethi Metrorum Ratio Sipontino Authore. Theoria
Grammatices. Sphaerae et Cosmographiae primordia quaedam}, Messanae in
Freto Siculo impressit Petrutius Spira. Anno Domini MDXXVIII mense
augusto. 

\item[S2:] \textit{Cosmographia Francisci Maurolyci messanensis siculi,
in tres dialogos distincta. In quibus de  forma, situ, numeroque tam
caelorum quam elementorum, aliisque rebus ad astronomica rudimenta
spectantibus satis disseritur. Ad reverendiss. Cardinalem Bembum},
Venetiis apud haeredes Lucae Antonii Iuntae Florentini, mense Ianuario
anno MDXXXXIII. 

\item[S3:] \textit{Quadrati horarii fabrica et eius usus, ut hoc solo
instrumento, coeteris praetermisssis, unusquisque mathematicus
contentus esse possit. Per Franciscum Maurolicum nuper
edita. Illustriss. D.D. Ioanni Vigintimillio Ieraciensium Marchioni
d.}, Venetiis, apud Nicolaum Bascarinum, MDXLVI. 

\item[S4:] \textit{Theodosii Sphaericorum elementorum libri III. Ex
traditione Maurolyci messanensis mathematici. Menelai Sphaericorum
libri III ex traditione eiusdem. Maurolyci Sphaericoram libri IL
Autolici De sphaera quae movetur liber. Theodosii De habitationibus
Euclidis Phaenomena brevissime demonstrata. Demonstratio et praxis
trium Tabellarum scilicet Sinus recti, Foecundae et Beneficae ad
sphaeralia triangula pertinenetium. Compendium mathematicae mira
brevitate ex clarissimis authoribus. Maurolyci De sphaera sermo},
Messanae in freto siculo impressit Petrus Spira, mense augusto
MDLVIII. 

\item[S5:] \textit{Sicanicarum rerum compendium Maurolyco abbate siculo
authore. Privilegium cautum est ne quis libellum hunc intra decem
annos proximos excudat}, Messanae in freto siculo impressit Petrus
Spira mense octobri MDLXII. 

\item[S6:] \textit{Martyrologium reveren. Domini Francisci Maurolyci
abbatis messanensis multo quam antea purgatum et locupletatum. In quo
addita sunt civitatum ac locoram nomina in quibus sancti Martyres
passi sunt: atque eorum corpora in praesentiarum requiescunt. Cun
indice locupletissimo ad invenienda Sanctorum nomina et dies in quo
festa eorum celebrantur}, Venetiis in officina Lucae Antonii Iuntae,
MDLXVIII. 

\item[S7:] \textit{D. Francisci Maurolyci abbatis messanensis Opuscula
Mathematica nunc primum in luce edita cum rerum omnium notatu dignarum
indice locupletissimo Pagella huic contigua eorum catalogus est}, Cum
privilegio, Venetiis apud Franciscum Franciscium senensem, MDLXXV. 

\item[S8:] \textit{D. Francisci Maurolyci abbatis messanensis mathematici
celeberrimi Arithmeticorum libri duo nunc primum in luce editi cum
rerum omnium notabilium indice copiosissimo}, Cum privilegio. Venetiis
apud Franciscum Franciscium senensem, MDLXXV.  

\item[S9:] \textit{Introductio in dialecticam Aristotelis per Magistrum
Franciscum Toletum}, Mexici, In collegio Sanctorum Petri et Pauli. Apud
Antonium Ricardum. 1578  

\item[S10:] \textit{Abbatis Francisci Maurolyci messanensis Photismi de
lumine et umbra ad perspectivam et radiorum incidentiam
facientes. Diaphanorum partes seu libri tres: in quorum primo de
perspicuis corporibus, in secundo de iride, in tertio de organi
visualis structura et conspiciliorum formis agitur. Problemata ad
perspectivam et iridem pertinentia. Omnia nunc primum in lucem edita},
Neapoli, ex typographia Tarquinii Longi MDCXI. Superiorum permissu. 

\item[S11:] \textit{Theoremata de lumine et umbra}, apud Bartholomaeum
Vincentium, 
(L. Hurillion), Lugduni, 1613. 

\item[S12:] \textit{D. Francisci Maurolyci abbatis messanensis Problemata
mechanica cum appendice et ad magnetem et ad pixidem nauticam
pertinentia. Omnia nunc primum in lucem edita}, Messanae, ex
typographia Petri Breae, 1613. Superiorum permissu.  

\item[S13:] \textit{Francisci Maurolyci messanensis Emendatio et
restitutio Conicorum Apollonii Pergaei. Nunc primum excusae, ubi primi
quatuor eiusdem Apollonii libri mendis, quibus foede scatebant
expurgantur, novisque interdum demonstrationibus illustrantur; quintus
vero, sextusve liber quorum tituli dumtaxat habebantur maximo labore
summaque industria denuo restituuntur. Ad illustrissimum Senatum
messanensem}, Messanae, Typis haeredum Petri Breae,
MDCLIIII. Superiorum Permissu. 

\item[S14:] \textit{Admirandi Archimedis syracusani monumenta omnia mathematica
quae extant quorumque catalogum inversa pagina demonstrat ex
traditione doctissimi viri D. Francisci Maurolyci, nobilis siculi,
abbatis Sanctae Mariae a Partu. Opus praeclarissimum, non prius a
typis commissum, a matheseos vero studiosis enixe desideratum,
tandemque e fuligine temporum accurate excussum. Ad Illust. et
Religiosissimum virum Fr. Simonem Rondinelli}, ... Panormi, apud
D. Cyllenium Hesperium, cum licentia Superiorum,
MDLXXXV. Sumpt. Antonini Giardinae, bibliopolae Panorm. 

\end{itemize}

%-----------------------------------------------------------------------
\section{Il ``Conspectus siglorum'' e il \mtex}

\label{ref-10.5}

\subsection{Trascrizione e collazione}

\label{ref-10.5.1}

Come ci si deve comportare, in pratica, quando si procede alla
trascrizione o all'edizione di un testo afferente all'edizione
mauroliciana? Il primo passo {\`e} quello di individuare il testimone che
si sta utilizzando all'interno della lista fornita nel paragrafo
precedente. Se si tratta di un autografo, nel primo sottocampo di
\verb"\VV"\index{\bs{}VV}  si batter{\`a} una \verb"A", se di una copia una \verb"C", se di una stampa
una \verb"S". Per quanto riguarda le eventuali mani presenti nel testimone
e gli interventi correttori ci si atterr{\`a} al sistema di esponenti
descritto nel % 
capitolo \ref{ref-5} e in modo pi{\'u} specifico qui sopra. 

Le cose si complicano leggermente quando si debba procedere alla
collazione di piu' testimoni. Come gi{\`a} detto qui sopra, tuttavia, nel
caso ci si trovi di fronte a due autografi (o a due apografi, o a due
stampe) si batter{\`a} per il secondo autografo una \verb"a" (o una \verb"c" o una
\verb"s"). Allo stato attuale delle nostre conoscenze non esistono casi di
di testi mauroliciani in cui si debba far ricorso per costruire il
testo critico a pi{\'u} di due testimoni dello stesso tipo. Ma tutto {\`e}
possibile. In particolare, il trascrittore che procede a collazionare
i testimoni potrebbe trovarsi di fronte al caso di molte copie dello
stesso testo. {\`E} questa la situazione in cui ci si trova per l'edizione del
\textit{Teodosio} e del \textit{Menelao} mauroliciani: ma dagli studi
condotti fin qui gi{\`a} sappiamo che si tratta di \textit{codices
descripti}, e quindi da eliminare. Tuttavia per vari motivi potrebbe
essere opportuno procedere ad una loro collazione completa: o perch{\'e}
solo da tale collazione si pu{\`o} ottenere la dimostrazione del fatto che
essi sono da eliminare, o perch{\'e} si vuole comunque poter ricostruire
il testo di tali testimoni, o anche perch{\'e} potrebbe trattarsi di un
testimone non finora studiato e quindi portatatore di novit{\`a} rispetto
alle regole e alle procedure enunciate in questo capitolo.

In tal caso, lo sfortunato trascrittore utilizzer{\`a} nel primo
sottocampo  di \verb"\VV"\index{\bs{}VV} la lettera \verb"C" seguita da uno o pi{\'u} punti
esclamativi a secondo che si tratti della terza, della quarta copia
ecc. Ad esempio, nel caso del \textit{Teodosio}, disponiamo di S4, C8,
C4, C5, 
C11 e C12. Ove si voglia procedere a una collazione completa di tutta
la tradizione si user{\`a} il seguente schema:

\begin{quote}
\begin{verbatim}
S4 = S   C8 = C   C4 = c
C5 = C!  C6 = C!! C11= C!!! C12 = C!!!!
\end{verbatim}
\end{quote}

\noindent scrivendo, ad esempio:

\begin{quote}
\begin{verbatim}
polus arcus \VV{
               {S:$ab$}
               {C:$ac$}  
               {c/C!:\EX{$kl$}:$ab$}
               {C!!/C!!!/C!!!!:$kl$} 
               } erit $h$
\end{verbatim}
\end{quote}

\noindent trattando \verb"C!", \verb"C!!", ecc. come se fossero  \textit{sigla}
composti da un unico segno e quindi facendoli seguire da un numero per
indicare le eventuali mani. Nell'apparato del testo critico il \verb"C!"
verr{\`a} reso come \textsl{C}$_*$; il \verb"C!!" come \textsl{C}$_{**}$; \verb"C!2"
(intervento correttivo di una mano diversa da quella del copista di
C5) come \textsl{C}$_*^2$, ecc.

%-----------------------------------------------------------------------
\subsection{Edizione e introduzione all'edizione}

\label{ref-10.5.2}
Quando poi l'editore decider{\`a} lo \textit{stemma} della sua edizione e
proceder{\`a} alla costruzione del testo critico, dovr{\`a} naturalmente
decidere se e quanti dei testimoni impiegati per la collazione
dovranno essere utilizzati. Nell'esempio di prima, supponendo che
l'editore voglia utilizzare solo S4 e C8, trasformer{\`a} la trascrizione
in {\mtex} cos{\'\i}:

\begin{quote}
\begin{verbatim}
polus arcus \VV{
               {S:$ab$}
               {C:$ac$}  
               \VB{ 
                  {c/C!:\EX{$kl$}:$ab$}
                  {C!!/C!!!/C!!!!:$kl$} 
                  } 			  
               } erit $h$
\end{verbatim}
\end{quote}

\noindent ovvero dichiarando che le lezioni di C4-C6 e C11-C12 non
sono da riportare in apparato.

Questo problema {\`e} ovviamente collegato al carattere \textit{in progress}
della nostra edizione.  E questro ci porta a discutere un ultimo
problema: {\`e} possibile (anzi, auspicabile) che il \textit{Conspectus
siglorum} che abbiamo ora fornito debba essere modificato pe rla
scoperta di nuovi testimoni. Saremmo tutti molto felici se, ad esempio
nella biblioteca comunale di Alessandria, si rinvenisse un autografo
mauroliciano. Ma, finiti i festeggiamenti (e trovato 
l'editore che se ne prenda cura), dovremmo affrontare il problema di
rinumerare tutti i testimoni autografi, dato che, purtroppo,  Alessandria comincia
per A. Ci{\`o} comporterebbe la correzione di tutte le edizioni condotte
fino a quel momento, nelle cui introduzioni i testimoni sarebbero
stati citati con il numero d'ordine attuale. Per ovviare a questo
problema, occorrer{\`a} che l'editore nello scrivere la sua introduzione~---~e ovunque ritenga necessario citare i suoi testimoni non solo con la
lettera A, C o S, ma con il relativo numero d'ordine~---~ utilizzi la
seguente macro:

\begin{quote}
\begin{verbatim}
\Wit{}
\end{verbatim}
\end{quote}

\noindent (da \textit{witness}, testimone). Se l'editore delle \textit{Coniche} nell'introduzione deve far
riferimento al manoscritto A9 (che contiene l'edizione
mauroliciana 
dei \textit{Sereni Cylindricorum libelli duo}), scriver{\`a} ad esempio:

\begin{quote}
\begin{verbatim}
 ... {\`e} da notare poi che  in \Wit{A9} si trova 
un'aggiunta marginale in inchiostro rosso identica 
alla correzione interlineare di c. 18v di \Wit{A1}. 
\end{verbatim}
\end{quote}

\verb"\Wit"\index{\bs{}Wit}, trattata da un opportuno programma, provveder{\`a} poi a
dare sempre il numero corretto secondo i criteri di ordinamento dei
testimoni stabiliti in questo capitolo: per questo motivo il
\textit{Conspectus} che abbiamo qui illustrato pu{\`o} essere considerato come
definitivo.

Ribadiamo infine che nei campi di \verb"\VV"\index{\bs{}VV} destinati ai \textit{sigla} \textbf{non} andr{\`a} utilizzata \verb"\Wit"\index{\bs{}Wit}, ma la sintassi usuale.

%-----------------------------------------------------------------------
\chapter[Come stampare un file {\mtex}]{Come stampare un file {\mtex} e
come prepararlo per la pubblicazione nel sito Web}

\label{ref-11}

\section{Compilare, analizzare, stampare \dots}

\label{ref-11.1}
\index{compilare}
\index{stampare}

E' ovvio che per poter sperimentare quello che avete letto in questo
manuale e poter vedere il vostro lavoro stampato secondo i criteri qui
enunicati (non foss'altro che per soddisfazione) e non solo come una serie
di pi\`u o meno comprensibili \verb"\VV"\index{\bs{}VV} dovete disporre sul
calcolatore che state utilizzando del {\LaTeX2e}. Per questo il presente
manuale non vi \`e di alcun aiuto: ma {\LaTeX2e} \`e un sistema
multipiattaforma e non vi dovrebbe essere troppo difficile trovare il modo
di installarlo, sia che utilizziate Windows in una delle sue varie
incarnazioni, che MacIntosh, che Linux. 

Compiuto queso primo passo dovrete provvedere a installare i
\textit{package} specifici del {\mtex}. A questo scopo baster\`a che
provvediate a copiare \textbf{nella directory in cui si trova il file che
volete stampare} i \textit{file} \texttt{mauro.sty}, \texttt{adn.sty},
\texttt{mauro.mf}. Nel caso che intendiate stampare anche parole greche
dovrete copiarvi anche i \textit{file} del \textit{package} \verb"ibycus".
Tutti questi file li troverete nel discheto allegato a questo manuale. Come
tutti i programmi, per�, anche il {\mtex} \`e in continua evoluzione: sul
sito Web \url{http://elabor.homelinux.org/mtex}{\new} dovreste trovare
sempre la versione pi\`u aggiornata da scaricare. Occorre anche notare che
l'installazine di \verb"ibycus" \`e un po' pi\`u complessa delle altre
(dovrete scompattare un file): nel caso non vi riuscisse (cos� come per
ogni alro problema che potreste trovare) potete contattare la direzione del
``Progetto Maurolico'' all'indirizzo
\verb"maurolico@dm.unipi.it"\footnote{In effetti la soluzione migliore per
l'installazione del {\mtex} sarebbe installare tutti questi
\textit{package} nel {\LaTeX} stesso: ma la cosa non \`e altrettanto facile
da descrivere in poche righe. Se volete seguire questa strada e trovate
difficolt\`a non esitate a contattare il ``Progetto Maurolico''.}.

Una volta che vi siate dotati di questi strumenti (niente paura: la
cosa \`e assai pi\`u semplice di quanto possiate temere!) per stampare il
vostro \textit{file} (diciamo \verb"edizione.tex") occorre che lanciate il
comando

\begin{quote}
\begin{verbatim}
latex edizione.tex
\end{verbatim}
\end{quote}

A questo punto lo schermo comincer\`a ad animarsi: vi verr\`a
detto che state usando

\begin{quote}
\begin{verbatim}
mauro.sty
stylefile per Maurolico, ver.4.0b5, 15/12/2000 Paolo Mascellani
basato su ver.3.4,  27/07/2000 Paolo Mascellani
basato su ver.2.74, 16/05/1999 Ken SAITO
\end{verbatim}
\end{quote}

\noindent e, ahivoi, con grande probabilit\`a, dopo poco il {\LaTeX}
emetter\`a un messaggio di errore che, sempre con grande probabilit\`a, vi
risulter\`a del tutto incomprensibile (anche nel caso che siate dei
{\TeX}-nici). Ci� dipende in gran parte da come il {\TeX} elabora i
\textit{file} e non ci si pu� fare nulla. Quasi sempre si tratta di una
parentesi graffa aperta ma non chiusa, di un campo di \verb"\VV"\index{\bs{}VV} lasciato con uno spazio prima dei due punti e altre sciocchezze del
genere. Per aiutarvi nella compilazione abbiamo elaborato un analizzatore
sintattico che, se fatto girare prima di compilare il file con {\LaTeX},
dovrebbe risolvere la stragrande maggioranza di questi problemi.

{\new} Depuis 2002, il a \'et\'e cr\'e\'e un
pre-processeur\index{preprocessore} n\'ecessaire au bon fonctionnement du
{\mtex}. Il s'agit d'un programme qui doit \^etre lanc\'e avant la commande
\texttt{latex}, dont le nom est \texttt{m2lv}\index{m2lv} et qui est
t\'el\'echargeable sur le site web \url{http://elabor.homelinux.org/mtex}.
La proc\'edure expliqu\'ee plus haut reste exacte avec les quelques
modifications suivantes:

\begin{enumerate}

\item lancer le pr\'eprocesseur: \texttt{m2lv edizione.tex}

\item on obtient un fichier \texttt{edizione.m.tex}. C'est sur ce fichier qu'il faut aujourd'hui lancer \texttt{latex}:

\item \texttt{latex edizione.m.tex}

\item on obtient alors le fichier \texttt{edizione.m.dvi} qui peut \^etre visualis\'e et imprim\'e avec un visualisateur \textsc{dvi}.

\end{enumerate}

%-----------------------------------------------------------------------
\subsection{Uso dell'analizzatore sintattico e del convertitore \textsc{html}}

\label{ref-11.1.1}
\index{analizzatore sintattico}
\index{convertitore \textsc{html}}

L'analizzatore sintattico non \`e altro che la prima parte del programma di
traduzione in \textsc{html} di un'edizione {\mtex}. {\`E} costituito da un programma
scritto in linguaggio \textit{ANSI C}, generato utilizzando il generatore
di compilatori \textit{Bison} ed il generatore di analizzatori lessicali
\textit{Flex}, e compilato utilizzando il compilatore \textit{GCC}. Tutti
questi strumenti sono liberamente disponibili (fanno tutti parte delle
distribuzioni standard di Linux) e possono essere trovati all'indirizzo:
\url{http://www.gnu.org}.

Per utilizzare l'analizzatore sintattico, bisogna installare il programma,
che si chiama \texttt{mcheck} e che {\`e} disponibile in versione gi{\`a}
compilata per Linux, Windows e MacOS (non {\`e} necessario installare anche
tutti gli strumenti citati sopra), o in versione sorgente, sul sito di
Maurolico (\url{http://elabor.homelinux.org/mtex})\new. Sotto Linux, se si
usa una shell di tipo \texttt{bash}, basta mettere il programma in una
delle directory indicate nella variabile d'ambiente \texttt{PATH}; sotto
Windows o MacOS, invece, si pu{\`o} utilizzare una cartella qualsiasi
(eventualmente aggiungendo dei collegamenti da altre cartelle, se {\`e}
comodo).

Una volta installato il programma, per eseguirlo basta dare
il comando \texttt{mchech <nome file>} (sotto Windows e MacOS, bisogna
fare un ``doppio click'' o un ``click'', dipende dalle impostazioni,
sull'icona del programma o su quella un collegamento ad esso, e, alla
richiesta del programma, inserire il nome del file), dove il nome del
file NON deve contenere l'estensione (\texttt{.tex}). A questo punto,
il programma stampa una riga di saluto, la lista degli errori o delle
situazioni ``strane'' che ha trovato, ed una riga di commiato.

Se siete stati (molto) bravi, le righe di errore o
avvertimento mancano del tutto e potete benissimo passare alle fasi
successive del vostro lavoro. Siccome per{\`o} succede spesso di
commettere qualche svista, probabilmente, avrete anche qualche riga di
errore; ciascuna di queste righe inizia col nome del file che {\`e}
stato analizzato, seguito dal numero della riga nella quale {\`e}
stato rilevato l'errore e da un messaggio di spiegazione (esistono
editor che possono essere istruiti in modo da posizionarsi
automaticamente sul punto indicato del file, a partire da messaggi di
questo tipo).

C'{\`e} un'avvertenza importante per chi usa l'analizzatore
sintattico: l'analizzatore sintattico {\`e} un po'
pi{\`u} ``cattivo'' del {\LaTeX} e segnala errori anche dove
quest'ultimo compila senza errori e produce una stampa impeccabile;
questo {\`e} in generale dovuto all'utilizzazione di sintassi
leggermente diverse da quelle specificate in questo manuale e che il
{\LaTeX} ammette, ma che rendono molto pi{\`u} difficile rintracciare
gli errori veri. La nostra scelta {\`e} stata, in questi casi, di
essere un pochino pi{\`u} ``pedanti'', ma di dare un aiuto concreto
nel rintracciare e correggere gli errori.

Anche se, in alcuni casi, ci sono dei messaggi di
avvertimento che non sembrano avere alcuna implicazione pratica, vi
consigliamo ``caldamente'' di cercare di eliminarli dai vostri
sorgenti {\mtex}. I benefici di questo piccolo sforzo sono tutti per
voi: in primo luogo, togliere di mezzo i messaggi poco significativi
{\`e} un modo di individuare con ragionevole certezza quelli che,
invece, significativi sono; in secondo luogo, l'analizzatore
sintattico {\`e} la prima parte di tutta la catena di programmi che
permettono l'analisi dell'edizione, come il programma che estrae i
testimoni, quello che compila la lista delle varianti, quelli che
permetteranno di effettuare ricerche per soggetto, ... eccetera (senza
contare il programma di traduzione in \textsc{html}): se esso ha delle
incertezze, queste non possono che riflettersi negativamente su tutte
queste operazioni, che costituiscono parte non secondaria di avere
un'edizione elettronica.

A proposito del traduttore \textsc{html}, il cui nome {\`e}
\texttt{m2hv}\index{m2hv}, il suo uso {\`e} assolutamente identico a quello
dell'analizzatore sintattico, che ne costituisce parte integrante (in
effetti, se si vuole, si pu{\`o} benissimo utilizzare il traduttore
\textsc{html} come analizzatore sintattico, tanto produce esattamente lo
stesso output, oltre ai file \textsc{html}).

%-----------------------------------------------------------------------
\subsection{Come ottenere una stampa}

\label{ref-11.1.2}
\index{stampare}

Per ottenere la stampa dell'edizione, bisogna prima di tutto compilare
il sorgente {\mtex} con il compilatore {\LaTeX}. Sotto Linux, questo
si ottiene dando il comando \texttt{latex <nome file>}; sotto Windows o
MacOS, si pu{\`o}, in generale, ma dipende anche dalla distribuzione
che si usa, trascinare l'icona del file da compilare sopra quella del
programma, oppure richiedere la compilazione dall'interno del
programma, tramite i suoi ``men{\`u}'', dopo averlo avviato con un
``doppio click'' o un ``click'', a seconda delle impostazioni del
sistema operativo.

La compilazione del file \texttt{edizione.tex}, produce, oltre
ad alcuni altri file, il file \texttt{edizione.dvi}, che {\`e} quello che ci
interessa; per stamparlo, sotto Linux, basta dare il comando \texttt{dvips <nome file>}, cio{\`e}, nel nostro esempio, \texttt{dvips edizione}
(il comando \texttt{dvips} permette anche di stampare solo alcune pagine,
o di stamparle in ordine inverso, oppure tante altre opzioni: per
conoscerle meglio bisogna dare il comando \texttt{man dvips}). Per
Windows e MacOS, la cosa migliore {\`e} utilizzare i comandi di stampa
disponibili nei ``men{\`u}'' dei rispettivi programmi.

%-----------------------------------------------------------------------
%\subsection{Quale stile di stampa scegliere}
%
%\label{ref-11.1.3}
%\index{stile di stampa}
%
%Siete finalmente riusciti a stampare la vostra edizione (un
%consiglio: non aspettate di averla finita per fare questi tentativi: se
%state facendo un errore ve ne accorgerete all'inizio e dovrete fare
%una o poche correzioni, e non diverse centinaia!). Assai probabilmete
%non ne sarete soddisfatti. Vorrete correggere gli errori di battitura,
%alcuen luoghi che vi sembravano incomprensibili ora li leggete
%perfettamente, ecc.  Per aiutarvi in questo lavoro di correzione e
%aggiustamento abbiamo pensato a tre macro che vi permettono stili
%diversi di stampa: 
%
%\begin{quote}
%\begin{verbatim}
%\CommentiNelTesto
%\MarkupNelTesto
%\ApparatoNelTesto
%\end{verbatim}
%\end{quote}
%
%Tutte e tre queste macro devono essere inserite nel
%preambolo del \textit{file} (prima di %\verb"\begin{document}"\index{\bs{}begin\{document\}}).
%
%\verb"\CommentiNelTesto"\index{\bs{}CommentiNelTesto} vi permette di %ottenere che i commenti del
%trascrittore e/o dell'editore inseriti con la macro %\verb"\Comm{}"\index{\bs{}Comm} invece
%di essere stampati in fondo al 
%\textit{file} (cfr. \S\,\ref{ref-3.3}) o di non essere stampati affatto %vengano
%stampati esattamente nel luogo cui il commento si riferisce, con un
%evidente risparmio di tempo e attenzione nello studiare il problema
%che un dato commento solleva. La stampa del commento avvien nel testo,
%ponendo il commento stesso fra parentesi quadre ben evidenziate, e
%viene abolito il richiamo di nota. Naturalmente l'opzione di %\textit{default} resta quella per cui i commenti vengono stampati a fine %testo
%o nel luogo dove \`e stata inserita la macro %\verb"\Commenti"\index{\bs{}Commenti}. Se nel
%premabolo non inserite %\verb"\CommentiNelTesto"\index{\bs{}CommentiNelTesto}, la stampa avverr\`a %in
%questo modo.
%
%\verb"\MarkupNelTesto"\index{\bs{}MarkupNelTesto} fa s� che vi venga %segnalato se certe
%macro che servono a marcare il testo sono state o no utilizzate in un
%determianto luogo. Ad esempio vi potrebbe essere capitato di scordarvi
%di usare \verb"\Cit"\index{\bs{}Cit} nel trascrivere questo passo ``erit %quadratum $ab$
%aequale duobus quadratis $bc$, $ac$ per penultimam primi
%\textit{Elementorum}''. Lo stile di \textit{default} non vi segnalerebbe
%nulla. Ma se scegliete \verb"\MarkupNelTesto"\index{\bs{}MarkupNelTesto} il %passo verr\`a stampato nel
%seguente modo:
%
%\begin{quote}
%
%erit quadratum $ab$ aequale duobus quadratis $bc$, $ac$
%[Cit]per penultimam primi \textit{Elementorum}
%
%\end{quote}
%
%\noindent se avete correttamente scritto (cfr. \S\,\ref{ref-3.5.2})
%
%\begin{quote}
%\begin{verbatim}
%erit quadratum $ab$
%aequale duobus quadratis $bc$, $ac$
%\Cit{
%    {per penultimam primi \Tit{Elementorum}}
%    {Elem.47.I}
%    }
%\end{verbatim}
%\end{quote}
%
%\noindent mentre non vi segnaler\`a nulla se vi foste scordati di
%marcare la citazione con \verb"\Cit"\index{\bs{}Cit}. Lo stesso vale anche %per tutte le
%altre macro di \textit{mark-up} che servono a distinguere semanticamene
%certi elemnti del testo.
%
%\verb"\ApparatoNelTesto"\index{\bs{}ApparatoNelTesto} vi permette di avere %l'apparato accanto
%alla lezione accolta nel testo critico o nel testo base su cui state
%operando la collazione, invece di doverla ricercare nelle note a pi\'e
%di pagina. Nel caso dell'esempio del \S\,\ref{ref-4.2.1}, invece di otenere %un
%apparato a pi\`e di pagina ottereste invece:
%
%\begin{quote}
%
%Sit data ratio [ratio \textsl{A}~~gratia \textsl{B}~~latio 
%\textsl{C}], sit datus cubus.
%
%\end{quote}

%-----------------------------------------------------------------------
\section{Preparare i file per la rete}

\label{ref-11.2}

\subsection{Quali comandi non usare}
\index{preparare i file per la rete}

\label{ref-11.2.1}

Come {\`e} stato varie volte ripetuto, specie nei primi tre
capitoli, {\`e} essenziale che i trascrittori e gli editori si
attengano il pi{\'u} possibile ai comandi che in questo manuale
sono esplicitamente descritti.

La conversione da {\mtex} a \textsc{html} infatti non {\`e} una
cosa semplice. Il linguaggio \textsc{html} non {\`e} fatto per
impaginazioni complesse, cambi di corpi, cambi di carattere,
e simili raffinatezze che il {\LaTeX} invece permette \textit{ad abundantiam}. Adattare un file derivato dal {\TeX}
all'\textsc{html} spesso significa impoverirlo da un punto di vista
grafico. Ci si guadagna~---~e non {\`e} poco!~---~il vantaggio
dell'editoria elettonica: la possibilit{\`a} di navigare da un
\textit{link} a un altro, effettuare ricerche sui testi, l'uso
dei colori. Altro vantaggio {\`e} il fatto che, essendo l'\textsc{html}
il linguaggio del \textit{web}, la conversione in \textsc{html} rende le
nostre edizioni visibili senza alcuna difficolt{\`a} a tutti
coloro che lo vogliano.

La conversione da {\mtex} a \textsc{html} {\`e} gestita da un
programma. Tale programma {\`e} piuttosto complesso ma, come
tutti i programmi, non {\`e} dotato di facolt{\`a} divinatorie: {\`e} in
grado di trasformare in comandi \textsc{html} solo i comandi del
{\TeX} o dei suoi dialetti che gli siano stati
esplicitamente ``insegnati'' e cio{\`e} quelli elencati
nell'indice analitico di questo \textit{Manuale}.

Di conseguenza, se un editore, per rendere pi{\'u} carina la sua
edizione (o, meglio, la stampa che ottiene a casa sua sulla
sua stampante) introduce un comando che non {\`e} stato inserito
nel programma di conversione, creer{\`a} problemi, lavoro
complicato, e il risultato finale sar{\`a} di ritardare
l'immissione della sua edizione nel sito, o~---~pi{\'u}
probabilmente~---~la sua edizione finir{\`a} nel sito con qualche
orrore grafico. Il desiderio di rendere pi{\'u} bella la sua
edizione avr{\`a} quindi come paradossale risultato quella di
renderla pi{\'u} brutta.

Che fare dunque se ci si trova davanti a situazioni che
richiedono assolutamente dei comandi non previsti
nell'elenco fornito in questo \textit{Manuale}? La soluzione {\`e}
semplice:

\begin{itemize}

\item occorre assolutamente che consulti il \textit{webmaster} e il direttore del progetto.

\end{itemize}

Verr{\`a} studiato il problema insieme e si trover{\`a}
una soluzione adeguata. Inoltre, in questo modo, sar{\`a}
possibile capire come e in che direzioni migliorare
l'{\mtex} e il programma di conversione.

%-----------------------------------------------------------------------
\subsection{La suddivisione di un'edizione in file}

\label{ref-11.2.2}
\index{suddivisione per la rete}

L'idea fondamentale {\`e} che ad un'opera corrisponda un \textit{file}.

Se l'opera {\`e} troppo lunga (le \textit{Coniche}, ad esempio) e
contiene delle divisioni interne (le \textit{Coniche} sono
divise in libri) ad ogni divisione interna dovr{\`a}
corrispondere un \textit{file}.

Internamente al \textit{file} c'{\`e} una suddivisione da
effettuare per la creazione delle pagine \textsc{html}. A questa
suddivisione pensa la macro \verb"\htmlcut"\index{\bs{}htmlcut}\new.
%(cfr. \S\,\ref{ref-3.7})
% comment\'e jps le 31-01-05

Il est absolument indispensable que le document commence par la commande
\verb"\htmlcut"\index{\bs{}htmlcut}:

\begin{quote}
\begin{verbatim}
\begin{document}
\htmlcut
\end{verbatim}
\end{quote}

Il faut ensuite distinguer deux cas.

%-----------------------------------------------------------------------
\subsubsection{Un testo matematico diviso in proposizioni}

\label{ref-11.2.2.1}
\index{divisioni per la rete}

Nel caso di un testo gi{\`a} suddiviso in proposizioni la suddisione
pi{\'u} naturale da seguire per inserire i vari \verb"\htmlcut"\index{\bs{}htmlcut}{\new} sono, ovviamente, le proposizioni del testo.

Con alcune accortezze, per{\`o}:

\begin{enumerate}

\item Introduzioni, prefazioni, lettere di dedica varie, devono essere
segnate con una \verb"\htmlcut"\index{\bs{}htmlcut} ciascuna.

\item Le definizioni (tutte insieme) che precedono un testo devono
anch'esse essere marcate da una \verb"\htmlcut"\index{\bs{}htmlcut} che le
raccolga in un'unica pagina. Ci{\`o} vale anche nel caso che nel corso
della trattazione si incontrino delle ``definizioni seconde'' o simili.

\item I lemmi devono stare insieme alla proposizione che li segue, i
corollari e gli scolii devono stare insieme a quella che li precede. Se la
prop. 15 ha due lemmi e tre corollari, mettete un \verb"\htmlcut"\index{\bs{}htmlcut} prima del lemma 1 e un \verb"\htmlcut"\index{\bs{}htmlcut} prima
della proposzione 16, in modo che tutto il materiale realtivo alla 15 resti
unito insieme.

\end{enumerate}

%-----------------------------------------------------------------------
\subsubsection{E se il testo non {\`e} diviso in proposizioni?}

\label{ref-11.2.2.2}

In questo caso star{\`a} all'editore valutare dove effettuare la spezzatura
delle pagine. Il comando da utilizzare {\`e} sempre
\verb"\htmlcut"\index{\bs{}htmlcut}{\new} e l'editore dovr{\`a} cercare di
operare suddivisioni non troppo lunghe e che abbiano un senso compiuto.
Deve tenere presente che  a una \verb"\htmlcut"\index{\bs{}htmlcut}
corrisponder{\`a} una pagina di testo \textsc{html}, e che una pagina
siffatta non deve sorpassare i 40 Kilobytes come dimensione. Il che {\`e}
come dire che la sua \verb"\htmlcut"\index{\bs{}htmlcut} sar{\`a} bene che
non sorpassi i 20.

In ogni caso, sar{\`a} bene che l'editore di un'opera siffatta contatti
preventivamente il \textit{webmaster} prima di procedere alla divisione in
``proposizioni''.

%-----------------------------------------------------------------------
\subsection{Figure, diagrammi, tabelle e materiale 
vario non testuale}

\label{ref-11.2.3}
\index{figure per la rete}
\index{diagrammi per la rete}
\index{tabelle per la rete}

La regola aurea {\`e}: ``evitate le ambiguit{\`a}!'' {\`E} bene tenere
presente che, al momento in cui devono essere costruite le pagine
\textsc{html}, non sempre si ha disposizione il tempo, la pazienza o anche solo
la competenza necessaria per sciogliere ambiguit{\`a} nei
riferimenti. {\`E} pur vero che tutti quelli che partecipano
all'edizione conoscono l'opera di Maurolico, ma l'editore deve
ricordarsi che \textit{solo lui} conosce \textit{bene} l'opera di cui ha
curato l'edizione critica.

Sar{\`a} bene che apponiate a una copia del vostro manoscritto o
della vostra stampa un numero per ogni figura, o tabella, o
altro. E che controlliate attentamente che ci sia una
corrispondenza biunivoca fra il numero che assegnate e il
numero che avete inserito accanto a \verb"\Figskip"\index{\bs{}Figskip} (cfr.
\S\,\ref{ref-3.6.4}).

Se associate o separate figure indicatelo chiaramente con un
\verb"\Comm"\index{\bs{}Comm}. In casi complessi come quello di un manoscritto con
figure che si accavallano, o di figure diverse nei diversi
testimoni, la cosa andr{\`a} discussa caso per caso.

Occorre tener conto che non {\`e} ancora stato stabilito uno
\textit{standard} per l'edizione delle figure, e che per ora si
forniscono solo gli originali dei manoscritti e/o delle
stampe.

L'editore che si voglia o si debba cimentare con la
ricostruzione delle figure o delle tabelle {\`e} benvenuto, ma
come al solito, occorre evitare l'anarchia. {\`E} opportuno che
si sappia cosa ha fatto, che programmi ha utilizzato ecc.

%-----------------------------------------------------------------------
\section{Aiutateci a trovare i bachi!}

\label{ref-11.3}
\index{bachi ?}

Come tutti i programmi anche l'{\mtex} ha i suoi \textit{bug}. Una
caratteristica dei \textit{bug} {\`e} quella di starsene zitti e
chiotti finch{\'e} non si verifica una combinazione particolare
che li fa risvegliare. Per rendere sempre pi{\'u} efficace l'{\mtex}
la collaborazione di trascrittori e editori {\`e} essenziale!

Comunicando che la tal macro nella tale situazione non
funziona bene o non funziona affatto ci aiuterete a
migliorare l'architettura generale del linguaggio e, di
conseguenza, la qualit{\`a} della pubblicazione del testo, sia in
\textsc{html} che, in futuro, in edizione su CD e in forma cartacea.

Tuttavia {\`e} essenziale che tali comunicazioni avvengano nella dovuta
forma. Per capire cos'{\`e} che non va, non basta che sapere che ``quando
compilo c'{\`e} \verb"\Prop" che mi d{\`a} errore!'' Occorre avere il
pezzetto di \textit{file} che si rifiuta di funzionare, in modo che si
possa capire di preciso qual'{\`e} il problema: se {\`e} un problema di
come {\`e} stato scritto il \textit{file} o se invece {\`e} un problema di
come {\`e} stato scritto l'{\mtex}.

%-----------------------------------------------------------------------
\chapter{Aggiunte}
\label{ref-12}

Questo capitolo descrive funzionalit{\`a} aggiuntive di {\mtex}. 
La sezione \ref{ref-12.1} descrive nuove macro. 
La sezione \ref{ref-12.2} descrive nuovi traduttori.

%-----------------------------------------------------------------------
\section{Macro}
\label{ref-12.1}

La sottosezione \ref{ref-12.1.1} descrive la macro \verb"ElencoTestimoni"\index{\bs{}ElencoTestimoni}, introdotta per riconoscere le sigle dei testimoni inserite inavvertitamente.
La sottosezione \ref{ref-12.1.2} descrive la macro \verb"textcircled"\index{\bs{}textcircles}, introdotta per gestire la stampa - verso \textsc{html} - di caratteri alfanumerici.
La sottosezione \ref{ref-12.1.3} descrive la macro \verb"banale"\index{\bs{}banale}, introdotta per gestire la stampa di varianti banali codificate come ``parte di'' varianti effettive.

%-----------------------------------------------------------------------
\subsection{ElencoTestimoni}
\label{ref-12.1.1}

\verb"ElencoTestimoni"\index{\bs{}ElencoTestimoni} rappresenta un insieme di ``testimoni ammissibili''. Accetta un argomento la cui sintassi {\`e} equivalente a quella del primo sottocampo di \verb"VV"\index{\bs{}VV}, \verb"VB"\index{\bs{}VB}, \verb"TV"\index{\bs{}TV}, \verb"TB"\index{\bs{}TB} (cfr. \S\,\ref{ref-10.5.1}) o \verb"Folium"\index{\bs{}Folium} (cfr. \S\,\ref{ref-3.5.1}). 

Utilizzare questa macro {\`e} opzionale: pu{\`o} essere inserita nel prologo - ovvero prima di \verb"\begin{document}"\index{\bs{}begin\{document\}} - eventualmente seguita da spazi. Se presente, definisce un controllo automatico sulle etichette utilizzate in \verb"VV"\index{\bs{}VV}, \verb"VB"\index{\bs{}VB}, \verb"TV"\index{\bs{}TV}, \verb"TB"\index{\bs{}TB} o \verb"Folium"\index{\bs{}Folium}: ogni occorrenza di un testimone \textbf{non} corrispondente a uno dei testimoni ammissibili genera un ``messaggio di avvertimento'' (cfr. \S\,\ref{ref-11.1.1}).

\noindent Pu{\`o} essere utilizzata pi{\`u} volte (sempre nel prologo): in tal caso l'insieme dei testimoni ammissibili corrisponde all'unione degli insiemi definiti in ciascuna occorrenza della macro.

Ciascun testimone ammissibile dovrebbe essere definito da una sigla univoca. Nel caso in cui appaia una sigla ripetuta in una stessa o in diverse occorrenze di \verb"ElencoTestimoni"\index{\bs{}ElencoTestimoni} viene generato un corrispondente messaggio di avvertimento.

%-----------------------------------------------------------------------
\subsubsection{Esempi}
Segue una lista di esempi diversi a seconda della tipologia di \textit{\textbf{argomento}}

\begin{description}
\item[\textit{vuoto}:] \verb"\ElencoTestimoni{}"\index{\bs{}ElencoTestimoni\{\}}. Un messaggio di errore sintattico informa circa la linea di codice in cui {\`e} inserita questa occorrenza della macro.
\item[\textit{unico}:] ad esempio \verb"\ElencoTestimoni{A}"\index{\bs{}ElencoTestimoni\{*\}}. Ogni linea di codice in cui appare un testimone diverso da \verb"A"\index{A} genera un messaggio di avvertimento.
\item[\textit{duplicato}:] ad esempio \verb"\ElencoTestimoni{A/A}"\index{\bs{}ElencoTestimoni\{A/A\}}. Un messaggio di avvertimento informa circa l'esistenza di un duplicato in corrispondenza della linea di codice in cui {\`e} stata inserita questa occorrenza della macro; inoltre ogni linea di codice in cui appare un testimone diverso da \verb"A"\index{A} genera un messaggio di avvertimento.
\item[\textit{multiplo}:] ad esempio \verb"\ElencoTestimoni{A/B}"\index{\bs{}ElencoTestimoni\{A/B\}}. Ogni linea di codice in cui appare un testimone diverso da \verb"A"\index{A} o \verb"B"\index{B} genera un messaggio di avvertimento.
\end{description}

\noindent \textbf{Nota}: la macro ignora il significato ``speciale'' (cfr. \S\,\ref{ref-3.2.1}) di caratteri che corrispondono a elementi propri del linguaggio {\mtex} quali ad esempio \verb"*"\index{*} e \verb"+"\index{+}. In altre parole, \verb"*"\index{*} {\`e} ``diverso'' - ad esempio - da \verb"A"\index{A}. Cosa succede se scriviamo \verb"\ElencoTestimoni{*}"\index{\bs{}ElencoTestimoni\{*\}}?

%-----------------------------------------------------------------------
\subsection{textcircled}
\label{ref-12.1.2}

\verb"textcircled"\index{\bs{}textcircled} gestisce caratteri alfanumerici chiusi. 
Accetta un argomento e va inserita all'interno del documento. 

La tabella seguente mostra il risultato della traduzione in codice \textsc{html} di \verb"\textcircled{"\index{\bs{}textcircled{}}$N$\verb"}"\index{\bs{}} operata da \verb"m2hv"\index{m2hv} (traduzione equivalente a quella di \verb"m2web"\index{m2web}) per i casi di $N$ compreso tra $1$ e $20$. I traduttori verso dialetti {\LaTeX} mantengono invariata la sintassi della macro.

\begin{center}
\footnotesize
\begin{tabular}{c|c}
                                               \mtex & \textsc{html}                     \\
\verb"\textcircled{1} "\index{\bs{}textcircled{1 }} & \verb"&#9312;"\index{\&\#9312;}\\
\verb"\textcircled{2} "\index{\bs{}textcircled{2 }} & \verb"&#9313;"\index{\&\#9313;}\\
\verb"\textcircled{3} "\index{\bs{}textcircled{3 }} & \verb"&#9314;"\index{\&\#9314;}\\
\verb"\textcircled{4} "\index{\bs{}textcircled{4 }} & \verb"&#9315;"\index{\&\#9315;}\\
\verb"\textcircled{5} "\index{\bs{}textcircled{5 }} & \verb"&#9316;"\index{\&\#9316;}\\
\verb"\textcircled{6} "\index{\bs{}textcircled{6 }} & \verb"&#9317;"\index{\&\#9317;}\\
\verb"\textcircled{7} "\index{\bs{}textcircled{7 }} & \verb"&#9318;"\index{\&\#9318;}\\
\verb"\textcircled{8} "\index{\bs{}textcircled{8 }} & \verb"&#9319;"\index{\&\#9319;}\\
\verb"\textcircled{9} "\index{\bs{}textcircled{9 }} & \verb"&#9320;"\index{\&\#9320;}\\
\verb"\textcircled{10}"\index{\bs{}textcircled{10}} & \verb"&#9321;"\index{\&\#9321;}\\
\verb"\textcircled{11}"\index{\bs{}textcircled{11}} & \verb"&#9322;"\index{\&\#9322;}\\
\verb"\textcircled{12}"\index{\bs{}textcircled{12}} & \verb"&#9323;"\index{\&\#9323;}\\
\verb"\textcircled{13}"\index{\bs{}textcircled{13}} & \verb"&#9324;"\index{\&\#9324;}\\
\verb"\textcircled{14}"\index{\bs{}textcircled{14}} & \verb"&#9325;"\index{\&\#9325;}\\
\verb"\textcircled{15}"\index{\bs{}textcircled{15}} & \verb"&#9326;"\index{\&\#9326;}\\
\verb"\textcircled{16}"\index{\bs{}textcircled{16}} & \verb"&#9327;"\index{\&\#9327;}\\
\verb"\textcircled{17}"\index{\bs{}textcircled{17}} & \verb"&#9328;"\index{\&\#9328;}\\
\verb"\textcircled{18}"\index{\bs{}textcircled{18}} & \verb"&#9329;"\index{\&\#9329;}\\
\verb"\textcircled{19}"\index{\bs{}textcircled{19}} & \verb"&#9330;"\index{\&\#9330;}\\
\verb"\textcircled{20}"\index{\bs{}textcircled{20}} & \verb"&#9331;"\index{\&\#9331;}\\
\end{tabular}
\end{center}

\noindent Per $N$ \textbf{non} compreso tra 1 e 20, il traduttore ``sfrutta'' la sintassi \textsc{css}. Ad esempio, \verb"\textcircled{21}"\index{\bs{}textcircled{21}} viene codificato tramite \verb"m2hv"\index{m2hv} (e \verb"m2web"\index{m2web}) come segue.

\begin{center}
\footnotesize
\verb"<span style=``border:1px solid black;border-radius:50\%\%;''>21</span>"\index{\bs{}}
\end{center}

\noindent Maggiori informazioni sui caratteri alfanumerici chiusi all'indirizzo \url{http://unicode.org/charts/beta/nameslist/c_2460.html}
\noindent Maggiori informazioni sulla sintassi \textsc{css} all'indirizzo \url{https://www.w3.org/Style/CSS}

%-----------------------------------------------------------------------
\subsection{banale}
\label{ref-12.1.3}

\verb"banale"\index{\bs{}banale} rappresenta un'indicazione aggiuntiva riferibile a uno o pi{\`u} testimoni.

Utilizzare questa macro {\`e} opzionale: pu{\`o} essere postposta - una sola vola e senza introdurre spaziatura - alla scrittura del primo sottocampo di uno o pi{\`u} argomenti di \verb"VV"\index{\bs{}VV}, \verb"VB"\index{\bs{}VB}, \verb"TV"\index{\bs{}TV} o \verb"TB"\index{\bs{}TB} (cfr. \S\,\ref{ref-10.5.1}). Se presente, inibisce l'uscita a stampa della porzione di macro a cui si riferisce (attenzione quindi a non postpostporla al primo sottocampo di \textbf{ogni} argomento, altrimenti comparir{\`a} in apparato una nota \textbf{vuota}).

%-----------------------------------------------------------------------
\subsubsection{Esempi}
Seguono tre esempi diversi che producono lo stesso risultato, mostrato in fondo.

{\footnotesize
\begin{verbatim}
Sit data \VV{{A:ratio}{B\banale:gratia}{C:latio}}, sit datus cubus.

Sit data \VV{{A:ratio}{B/D\banale:gratia}{C:latio}}, sit datus cubus.

Sit data \VV{{A:ratio}{B\banale:gratia}{C:latio}{D\banale:ratjo}}, sit datus cubus.
\end{verbatim}
}
\begin{maurotex}
Sit data \VV{{A:ratio}{C:latio}                                 }, sit datus cubus.
\end{maurotex}

%-----------------------------------------------------------------------
\section{Traduttori}
\label{ref-12.2}

Ogni traduttore {\`e} denominato ``m2'' seguito da una stringa diversa a seconda del linguaggio di destinazione (cfr. \S\,{ref-12.2.1}, \S\,{ref-12.2.2}).

%-----------------------------------------------------------------------
\subsection{m2ledmac}
\label{ref-12.2.1}

Introdotto nel 2020, \verb"m2ledmac"\index{\bs{}m2ledmac} produce un output in formato \textit{reledmac}.\\ \textbf{Nota}: potrebbe generare occorrenze di testo che 
\begin{enumerate}
\item iniziano con \verb"\pstart"\index{\bs{}pstart}
\item sono seguite \textbf{esclusivamente} caratteri di spaziatura o di nuovo capoverso
\item terminano con \verb"\pend"\index{\bs{}pend}
\end{enumerate}

\noindent Maggiori informazioni sul linguaggio di destinazione all'indirizzo \url{https://ctan.org/pkg/reledmac}.

\printindex

\end{document}
