\documentclass[italian, latin]{book}

\usepackage{mauro,latexsym,adn,endnotes,babel}
\ElencoTestimoni{1/A/A+/Bas/G/L/L2/N/N2}
\FoliumInMargine

\begin{document}
\htmlcut

\Title{\INTE{Varia de centris}}{}
\Capitolo{De centro solidi parabolae demonstratio acutissima cum collatione aliorum centrorum}
\par
Testo \VV{{L2/Bas:\MARG: prodiorthosis}{G/L/N/Bas:prodiorthosis}}, praecedens correctio. Haec figura, ubi aliquid necessarium dictu,
\VV{{*:set}{G/L/N/Bas:et}} insuave audientibus aut odiosum
nobis dicturi sumus, praemunit. \VV{{N2/Bas:Exemplum}{G/L/N:exemplo}} apud Ciceronem
frequens: ‘Quamquam sentio \VV{{N/Bas:quanta}{G/L:quanto}} hoc cum offensione \VV{{Bas:dicturus sim}{G/L:dicturus sum}{N:sum dicturus}}, dicendum \VV{{*:est}{G:est tamen}}’.
%%% Quando \par è seguito immediatamente da una macro, \pstart viene posizionato in maniera scorretta, all'interno della macro invece che fuori
\par
\VV{{L2/Bas:\MARG: prodiorthosis}{G/L/N/Bas:prodiorthosis}}, praecedens correctio. Haec figura, ubi aliquid necessarium dictu,
\VV{{*:set}{G/L/N/Bas:et}} insuave audientibus aut odiosum
nobis dicturi sumus, praemunit. \VV{{N2/Bas:Exemplum}{G/L/N:exemplo}} apud Ciceronem
frequens: ‘Quamquam sentio \VV{{N/Bas:quanta}{G/L:quanto}} hoc cum offensione \VV{{Bas:dicturus sim}{G/L:dicturus sum}{N:sum dicturus}}, dicendum \VV{{*:est}{G:est tamen}}’.
%%% Succede con tutte le macro, a parte quelle di sezione (\Capitolo, ecc.) e \Unit. Un solo esempio per tutti:
\par
\INTE{* * *
leptoogia * * *} tale pro Gallio de convivio luxurioso:
‘\VV{{*:Fit}{G/L/N/Bas:ut}} clamor, \VV{{*:fit}{G/L/N/Bas:ut}} \VV{{Bas:convicium}{L/N:convivium}{G:comitum}} mulierum, \VV{{*:fit}{G/L/N/Bas:ut}} symphoniae cantus.
\par
%%% Succede anche con l'ambiente matematico $...$. Notare che la prima espressione va bene, perché preceduta da \Unit
\Unit[collatioInizio]
$abc$ parabola\par 
$abc$ \ABBR{triangulum}

\Capitolo{De centro solidi parabolae demonstratio acutissima cum collatione aliorum centrorum}

%%% \Folium, dentro una macro di sezione, prende un \pstart/\pend che non dovrebbe esserci.
\Capitolo{\Folium{A:8r}De centro solidi parabolae demonstratio acutissima cum collatione aliorum centrorum}
\par
Testo

%%% \Lemma, \Aliter, \Additio vanno aggiunte alle macro che creano un titoletto, come \Capitolo, ecc. e che quindi non vogliono \pstart/\pend
\par
\Lemma{Titolo lemma}

%%% \NOTAMARG dà un output scorretto. Dovrebbe inserire \edtext{testo critico}{apparato} come \VV, invece che {apparato} soltanto, come fa adesso.
%%% Però, vista la peculiarità di \NOTAMARG, non sono sicuro dell'output; forse è meglio che su questo si esprima anche Daniele.
\par
\Unit 
Curam \VV{{1:\POSTDEL{habenda}:corporis}} tibi non suadeo, claro medico et \VV{{*:\DES{add.}\INTERL:qui}}
\VV{{*:\EX{clarissimi}:clarissimum}} \VV{{*:\DES{add.}\INTERL:huius etatis}} \VV{{*:\EX{medici}:medicum}} \VV{{1:\ANTEDEL{huius etatis filio}:patrem}}
habes.
%\NOTAMARG{{*:\INTERL:attende aliquid de questione medicorum}} Puto tibi et causas morbi tui notas et remedia;
\par

%%% Qui viene creato un \pstart/\pend vuoto di troppo, dovuto al fatto che c'è l'ambiente center.
%%% In questo caso, però, non lo considererei un vero e proprio errore: l'ambiente center non dovrebbe essere proprio usato in una situazione come questa.
%\begin{center}
%\Figskip{1cm}
%\end{center}

%%% Due \Figskip che non sono separati da testo dovrebbero stare dentro una sola sezione \pausenumbering/resumenumbering.
%%% Non implementare però ancora questa cosa, perché forse non c'è proprio più bisogno di usare \pausenumbering/resumenumbering.
%%% Devo fare qualche test prima.
%\Figskip{1cm}

%\Figskip{1cm}

%%% Qui sono posizionati male sia \pstart, dopo la prima parentesi aperta da \Date, che \pend, dopo \clearpage.
%%% È anche vero che \clearpage non dovrebbe esserci... In ogni caso il primo \pstart è collocato male, dentro un gruppo che si chiude prima che possa essere trovato \pend.
\par
\Date{
    {{\SAT} \VV{
             {A+:\POSTDEL{apr.}:5\Sup{o}}
             } maii 1565}
   {05.05.1565}
     }

\Capitolo{Prove novembre 2023}

\Sottocapitolo{Problema tabular}
\par

\begin{tabula}{lll}
 56.  \RTT ~$ac$~$ce$ &  & \\
                      &  & aequum \QDR\Sup{to} ~$bc$, 225, quod est propositum. \\
 169. \QDR ~$ab$      &  &  \\
\end{tabula}
\par

\Sottocapitolo{Problema tabular center}
\par

\begin{center}
Questo è per vedere se il center normale funziona.
\end{center}
\par

\begin{center}
\begin{tabula}{llll}
& 4 & 12 & \\
3 & & &  27\\
\linea \\
& 5 & 15 & \\
\end{tabula}
\end{center}
\par

\Sottocapitolo{Problema tagid}
\par

\VV{{A:\EX{ex}:ex}}
\VV{{A:\INTERL:interl}}
\VV{{A:\PC:pc}}
\VV{{A:\MARG:marg}}
\VV{{A:\BIS:bis}}
\VV{{A:\MARGSIGN:margsign}}

\end{document}

%%% PS: sarebbe utile che \pstart e \pend fossero su una riga isolata, per una questione di leggibilità.
