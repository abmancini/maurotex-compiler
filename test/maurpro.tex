\documentclass[12pt]{book}

%  File di prova per il package MauroTeX
%  allineato alla versione 4 di mauro.sty
%  allineato alla versione 1.14 di mvisit e programmi derivati
%
%  eLabor sc
%  via Ponye a Piglieri 8, 56121 Pisa - Italia
%  <https://elabor.biz>
%  <info@elabor.biz>
%
%  responsabile sviluppo:
%  Paolo Mascellani (paolo@elabor.homelinux.org)
%
%  coordinatore del progetto "Maurolico":
%  Pier Daniele Napolitani (napolita@dm.unipi.it)
%  
%  Questo programma e` software libero. Lei puo` redistribuirlo
%  e/o modificarlo nei termini della "GNU General Public License",
%  come pubblicata dalla "Free Software Foundation", versione 2 del
%  giugno 1991.
%  
%  Questo programma e` distribuito nella speranza che possa essere
%  utile, ma SENZA ALCUNA GARANZIA, implicita o espressa. Vedere la 
%  "GNU General Public License" per maggiori dettagli.
%  
%  Lei, assieme al programma, dovrebbe aver ricevuto anche una copia
%  della "GNU General Public License", contenuta nel file "LICENZA";
%  se no, scriva alla "Free Software Foundation, Inc., 675 Mass Ave,
%  Cambridge, MA 02139, USA".
%

%%%%%%%%%%%%%%%%%%%%%%%%%%%%%%%%%%%%%%%%%%%%%%%%%%%%%%%%%%%%%%%%%%%%
% COMMENTI
%
% l'innestamento di macro dentro \LB crea dei problemi.
%
% \OM, quando utilizzata come lezione nel primo campo di \VV, scrive
% ``om.'' nel testo critico.
%
% \MarginaliaInNota non fa nulla
%
% La macro \Folium viene trattata all'interno delle note come se fosse
% stata impostata \FoulimInTesto anche se e` stata impostata
% \FoliumInMargine
%
% Ci sono dei problemi di compatibilita` con il package babel:
% non che questo package sia fondamentale per Maurolico, ma gli errori
% che ne nascono sono piuttosto subdoli, quindi e` meglio evitarli.
%   1 - il package babel va obbligatoriamente caricato DOPO il mauro;
%   2 - la macro \SS (e, per coerenza anche la \DS) e` stata definita
%       come \SB (\DS diventa \DB)
%
%%%%%%%%%%%%%%%%%%%%%%%%%%%%%%%%%%%%%%%%%%%%%%%%%%%%%%%%%%%%%%%%%%
\usepackage{times}
\usepackage{geometry}
%\geometry{a4paper,tmargin=2cm,bmargin=2cm,lmargin=2cm,rmargin=2cm}
\setlength\parskip{\medskipamount}
\setlength\parindent{0pt}
\usepackage{ibycus4, latexsym, mauro, endnotes, adn}
\usepackage[greek,italian]{babel}
\usepackage[utf8x]{inputenc}
\usepackage{xr}

%\FoliumInMargine
\FoliumInTesto
%\MarginaliaInNota
%\CommentiNelTesto
%\MarkupNelTesto
%\ApparatoNelTesto

\externaldocument[Photismi_]{pluto.txt}
\Manus{paolo}{PaoloStart}{PaoloStop}{itshape}
\Manus{daniele}{\lbracket}{\rbracket}
\Manus{A55}{\lbracket}{\rbracket}{bfseries}
\Manus{50}{\ensuremath{\ulcorner}}{\ensuremath{\lrcorner}}{}

\ElencoTestimoni{A/Am/B/A+/A2/A1/B1/B2/Clavius/D/C/F/G/H/J/K/L/O/P/R/S/V/W/alpha/phi/omega}

\begin{document}
\htmlcut[inizio]
\par
\frontespizio{X}{MaurPro}{Paolo Mascellani#Pier Daniele Napolitani}
\title{Esempi e prove \MauroTeX}
\author{Paolo Mascellani#Ivan Ricotti}
\par
\Title{Prova Titolo}{Prova sottotitolo}
\par
\Title{Prova titolo senza sottotitolo}{}
\par
\author{Paolo Mascellani ({\tt p.mascellani@dm.unipi.it})}
\par
\date{4 dicembre 2011}
\par
\maketitle
\par
Documento di esempio delle macro 
\MTeX.
\label{gatto}
\section{Accipicchi{\`a}, però ... per{\'o} ... per{\"u} ... per{\^a}}
\par
\section{Elementi \textit{di base} (\textbf{capitolo 3})}\label{cane}
\par
Gli {\it Aritmeticorum libri duo} furono pubblicati nel 1575.
\par
\subsection{Elementi \LaTeX}
\par
Pier~Paolo non pu\`o essere staccato.
\par
Le seguenti lettere accentate vengono ottenute con comandi appositi del 
\LaTeX:
\`a, \`{e}, \`{i}, \`{o}, \`{u},
\'{a}, \'e, \'{i}, \'{o}, \'{u}, 
\'{A}, \'{E}, \'{I}, \'{O}, \'{U}, 
\`{A}, \`{E}, \`{I}, \`{O}, \`{U},
\^{a}, \^{e}, \^{i}, \^{o}, \^{u},\^{A}, \^{E}, \^{I}, \^{O}, \^{U}, 
\"{a}, \"{e}, \"{i}, \"{o}, \"{u},\"{A}, \"{E}, \"{I}, \"{O}, \"{U}, 
{\`\i}
\par
E poi ci sono i \dots ed i $\times$ (moltiplicazione).
\par
\cite{piripacchio}
\par
\maurocite{label}{testo della citazione}
\par
\begin{center}
Esempio di testo centrato orizzontalmente.
\end{center}
\par
{\it Questo {\`e} il corsivo.}
\par
{\sl E questo il tondo inclinato.}
\par
{\bf Questo {\`e} il nero.}
\par
{\it E questo {\rm {\`e} un po'} corsivo, un po'
{\sl  tondo inclinato} e {\rm un po' tondo normale}.}
\par
{\it E questo {\rm {\`e} un po'} corsivo
e un po' {\rm tondo}}.
\par
Nuove macro di stile: \textbf{questo è grassetto} e \textit{questo è italic} e \textsc{questo è maiuscoletto}.
\par
Questo\,{\`e} uno spazio fine non secabile.
\par
Questo $\flat$ {\`e} un bemolle.
\par
Questo $\natural$ {\`e} un bequadro.
\par
Questo $\cdot$ {\`e} un puntino centrato.
\par
Questa $\vert$ {\`e} una barra verticale.
\par
Questo -- {\`e} un trattino lungo (1567 -- 1672) - (1567-1672).
\par
Questo --- ancora di più.
\par
Ora ``virgolettiamo'' un po'.
%\par
%Questa \lbracket {\`e} una parentesi quadra sinistra.
%\par
%Questa \rbracket {\`e} una parentesi quadra destra.
\par
Questo \S {\`e} un paragrafo.
\par
\Teorema{Questo {\`e} un teorema}
\Aliter{Questo {\`e} un aliter}
\Lemma{Questo {\`e} un lemma}
\Corollario{Questo {\`e} un corollario}
\Scolio{Questo {\`e} un scolio}
\Additio{Questo {\`e} un additio}
\Capitolo{Questo {\`e} un capitolo}
\Sottocapitolo{Questo {\`e} un sotto capitolo}
\par
\subsection{Liste}
\par
\begin{itemize}
\item uno
\item due
\end{itemize}
\par
\begin{enumerate}
\item uno
\item due
\end{enumerate}
\par
\begin{description}
\item[descrizione uno] uno
\item[descrizione due] due
\end{description}
\par
\subsection{Quote}
\par
\begin{quote}
Questa è una quote.
\end{quote}
\par
\subsection{Footnote}
\par
Qui deve essere inserita\footnote{una nota a piè di pagina} ed in seguito ne va
inserita\footnote{un'altra, magari anche un po' più lunghettina, così da verificare cosa
succede quando si va a capo}.
\par
Alla fine, ne mettiamo\footnote{un'altra ancora}. E poi basta.
\par
\subsection{Macro da ignorare}
\par
Iniziamo da \noindent e
proseguiamo con \newpage e
poi

\vspace{1cm}

ed
infine una bella sil\-la\-ba\-zio\-ne.
\par
\subsection{Titolo \textit{fantasioso} ed anche \textbf{peggio} di prima}
\par
\subsection{Simboli astronomici e altri caratteri speciali}
\ref{gatto}
\par
%paolo: le definizioni di radix sono da rifare
\par Radix:           \RDX 
\par radix:           \rdx
\par
\par Sole:            \SOL 
\par Luna:            \LUN
\par Terra:           \TER 
\par Mercurio:        \MER 
\par Venere:          \VEN
\par Marte:           \MAR 
\par Giove:           \GIO 
\par Saturno:         \SAT
\par
\par Ariete:          \ARS 
\par Toro:            \TRS 
\par Gemelli:         \GMN
\par Cancro:          \CNC 
\par Leone:           \LEO
\par Vergine:         \VRG 
\par Bilancia:        \LBR 
\par Scorpione:       \SCR 
\par Sagittario:      \SGT
\par Capricorno:      \CPR 
\par Acquario:        \AQR
\par Pesci:           \PSC
\par
\par Congiunzione:    \CNJ
\par Opposizione:     \OPP
\par
\ref{Photismi_peppa}

\subsection{date}
\label{topo} 
\Date{
        {Completum Messanae in Freto siculo Dominicae
         Incarnationis 1554 Indictione XII}
}
\par
\Date{
        {Completum Messanae in Freto siculo Dominicae
         Incarnationis 1554 Indictione XII}
         {15.06.1554}
     }
\par
\Date{
        {Completum Messanae in Freto siculo Dominicae
         Incarnationis 1554 Indictione XII}
         {15.06.1554}
        {L'indizione del 1554 non era la XII. O {\`e}
        sbagliato l'anno o {\`e} sbagliata l'indizione}
     }
\par
\subsection{Ambiente matematico}
\par
... erit $a$---$b$ ita $c$---$d$
\par 
Rectae $ab$, $cd$ erunt aequales per 22\Sup{am} 5\Sup{i}

\par
Proviamo ora con una formula un po' complicata nel testo
$\left(\frac{n}{\int_{a}^{b}{\VV{{A:\frac{1}{x^{2}}}{B:\frac{2}{x^{3}}}}
\VV{{A:dx}{B:dy}}}}\right)^{2}$ seguita da altro testo. E poi con la
stessa come equazione
$$\left(\frac{n}{\int_{a}^{b}{\frac{\VV{{A:1}{B:2}}}{x^{2}}
\VV{{A:dx}{B:dy}}}}\right)^{2}$$ seguita, ovviamente, da altro testo
ancora.
\par
\ref{topo} 

\subsection{Enunciatio e Protasis}
\begin{Enunciatio}
Apollonius Eudemo salutem
\end{Enunciatio}
Si corpore bene vales ...
\par
\begin{Enunciatio}
Propositio XVII
\end{Enunciatio}
\begin{Protasis}
Sphaerarum superficies sunt quadratis diametrorum
proportionales.
\end{Protasis}
\par
Sint duae sphaerae ...
\par
\subsection{Altri ambienti}
\par
\begin{minipage}{5cm}
Questo {\`e} un esempio di utilizzazione di un ambiente (minipage) che
non {\`e} esplicitamente gestito.
\end{minipage}
\par
\subsection{Scansione ragionamento}
\par
Prima parte {\DB} seconda parte {\SB} terza parte.
\par
\subsection{Greco}
\par
\GG{Eidos} \GG{Lo'gos} \GG{a)rxai=a gra'mmata}
\par
In latino \begin{textgreek}{
Ειδοσ Λόγοσ ἀρχαῖα γράμματα
}\end{textgreek} ancora in latino.
\par
{\c p}, 
{\.o}, 
{\v o}, 
{\"u}, 
{\&}, 
{\it\&}, 
{\ss}.
\par
\ABBR{proportio}
\subsection{Folia}
\par
tangat recta \Folium{A:35r} circulum iam descriptum \par
xxxx xxxxxxxx xxxxxxxxxx xxxxxxxxxxxx xxxxxxxxxx xxxxxx tangat recta \par \Folium{35*r}
circulum iam descriptum xxxxxxxxxxxxx xxxxxxxxxxx xxxxxxxx xxxxxxxxx xxx   xxxxxxxx xxxxxxxxxxx
xxxxxxxxx xxxxxxxxxxxx xxxxxxxxxxxxxxx xxxxxxxxxxxxxxxx xxxxxxxxxxxx xxxxxxxxxxxx xxxxxxxx\par
colla\Folium{A:35r}terali
\par
\subsection{Persone, Luoghi, Opere}
\par
In questa sezione vediamo come si marcano i \Nome{{nomi}{nome di fantasia}}, i
\Luogo{{luoghi}{luogo di fantasia}} e le \Opera{{opere}{opera di fantasia}}.
\par
\subsection{Citazioni}
\par
\Cit{{per 29\Sup{am} 3\Sup{ii}}}.
\par
\subsubsection{Senza riferimento}
\par
\par
\paragraph{paragrafo}
\par
\label{etichetta}
\par
erit, \Cit{
  {per primam sexti}
}, ut triangulus
\par
\subsubsection{Con riferimento certo}
\par
erit, \Cit{
        {per primam sexti}{E.6.1}
}, ut triangulus
\par
\subsubsection{Con riferimento incerto}
\par
erit \Cit[imp]{
        {per doctrinam Euclidis}
}, quadratus praedictus aequalis duobus quadratis
\par
erit \Cit[imp]{
        {per doctrinam Euclidis}{E.1.47}
}, quadratus praedictus aequalis duobus quadratis
\par
erit \Cit[imp]{
        {per doctrinam Euclidis}
        {E.1.47}
        {\`{e} senza dubbio il teorema di Pitagora}
}, quadratus praedictus aequalis duobus quadratis
\par
\subsubsection{Equivoche}
\par
tales gentes, \Cit[eqv]{
        {ut vult Ptolemaeus}
        {Ptol.Alm.1.7/Ptol.Geogr.3.5}
        {Tolomeo parla della
                distribuzione dei popoli
                in due luoghi diversi}
}, vivunt in germanicis sylvis
\par
\par
\ref{etichetta}
\ref{puppa}
\subsubsection{Nuova gestione}
\ref{cane}
\par
testo prima \Cit{{testo citazione}{EUC/DAT/Y/5}{commento}} testo dopo.
\subsection{Simboli grafici}
\par
\par parallelogramma:  \PRL
\par triangolo:        \TRN
\par quadrato:         \QDR
\par rettangolo:       \RTT
\par doppiorettangolo: \DRTT
\par cubo:             \CUB
\par piramide:         \PYR
\par parallelepipedo:  \PPD
\par pentagono:        \PEN
\par esagono:          \HEX
\par trapezio:         \TRP
\par esagono centrale: \HEXC
\par nove tagliato:    \CON
\par p tagliata:       \PER
\par
\par taglio:           \strike{xxx}
\par m con tilde:      \mtilde
\par p con tilde:      \ptilde
\par Prova di \RUBR{rubrica}.
\par
\htmlcut
\par
\subsection{Tabelle}
\par
\begin{center}
\begin{tabula}{rcl}
\inizio{Questo prima della tabella}
prima.1
&
\VV{{A:prima.2}{B:seconda.3}}
&
prima.3 \\
seconda.1 & seconda.2 & \regula \\
3.1 & 3.2 & 3.3
\fine{Questo dopo la tabella}
\end{tabula}
\end{center}
\par
\begin{center}
\begin{tabula}{rcl}
\lgra
&
prima.2
&
\rgra \\
seconda.1 & seconda.2 & seconda.3 \\
\lang & 3.2 & \rang
\end{tabula}
\end{center}
\par
\begin{center}
\begin{tabula}{|rcl|}
\linea \\
prima.1 & prima.2 & prima.3 \\
seconda.1 & \regula & seconda.3 \\
3.1 & 3.2 & 3.3 \\ \linea
\end{tabula}
\end{center}
\par
\begin{center}
\begin{tabula}{|rcl|}
\linea \\
prima.1 & prima.2 & prima.3 \\
seconda.1 & seconda.2 & seconda.3 \\
3.1 & 3.2 & 3.3 \\ \linea
\end{tabula}
\end{center}
\par
\begin{center}
\begin{tabula}{|rcl|}
\linea \\
\mcol{2} prima.1 e prima.2 & prima.3 \\
seconda.1 & \mcol{2} \regula \\
\mcol{2} \regula & terza.3 \\ \linea
\end{tabula}
\end{center}
\par
\begin{center}
\begin{tabula}{|rcl|}
\linea \\
\mrow{2} prima.1 e seconda.1 & prima.2 & prima.3 \\
\mcol{2} \regula \\
\mcol{2} \regula & terza.3 \\ \linea
\end{tabula}
\end{center}
\par
\subsection{Casistiche}
\par
\Rcases{{uno}{due}}
\par
\Lcases{{uno}{due}{tre}}
\par
\RLcases{{uno}{due}{tre}}
\par
\Rbracecases{{uno}{due}{tre}}
\par
\Lbracecases{{uno}{due}{tre}}
\par
\RLbracecases{{uno}{due}{tre}}
\par
\Voidcases{{uno}{due}{tre}}
\par
\Rbracecases{
  {uno} 
  {\Rbracecases{
    {\Voidcases{{2.1.uno}{2.1.due}{2.1.tre}{2.1.quattro}}} 
    {2.due} 
    {\Rbracecases{
      {2.3.uno}
      {\RLbracecases{{2.3.2.uno}{2.3.2.due}}}
    }}
  }} 
  {\Lbracecases{{3.uno}{3.due}}}
}
\par
\subsection{Formule, figure e tavole}

\par
\Formula{di c. 37v}
\par
\Formula[didascalia formula]{di c. 37v}
\par
prima della figura
\Figura{maurpro}
dopo la figura e prima dell'altra
\Figura[didascalia figura]{maurpro}
dopo la seconda figura
\Figskip[didascalia figskip]{5 cm}\Comm{qui va la figura 17}
\Figskip{3.5 cm}
\Figskip{3,5 cm}
\Figskip{35 mm}
\Figskip{6cm}\Comm{qui va la prima figura di c. 23v}
dopo la seconda figura.
\par
Anche questa \`e una figura, ma si trova \Marg{a margine} a margine
del testo principale. Questa, invece, \`e una frase riempitiva, che
serve solo per vedere se il {\TeX} smette di lamentarsi, una volta per
tutte e tutte per una.\par
\CitMarg{Ecco come viene CitMarg}\par
\CitMargSign{Ecco come viene CitMargSign}\par
\underline{Ecco come viene fatto l'underline} \par
\LACs
\par
Anche questa \`e una figura, ma si trova \Marg[didascalia]{a margine}
a margine del testo principale.
\par
prima della tavola \Tavola{5} dopo la tavola
\par
prima della tavola \Tavola[didascalia tavola]{5} dopo la tavola
\par
\subsection{Scansione testo}
\par
\Prop{}
\Unit[piripicchio] Apollonius Eudemo Salutem ... valemus
et nos. \Unit Tempore quo eramus ...
a nobis compositorum ...\Unit ...
\par
\Prop{}
\par
\subsection{annotazioni}
\par
Bisogna poter aggiungere\Adnotatio{questo \`e quello che ho fatto; per\`o,
voglio aggiungere una nota un poco lunghetta per vedere se funziona il posizionamento per
la nota successiva (che, naturalmente, andr\`o ad aggiungere tra poco)} delle
annotazioni indispensabili alla comprensione del testo\Adnotatio{Non \`e il caso di questa}.
\par
\section{intestazione lettere}
\par
\numero{I}\label{I}\mittente{Francesco Maurolico} \destinatario{Ettore
Pignatelli} \luogomitt{Messina} \data{14 agosto 1528}
\commento{Lettera di dedica dei \Tit{Grammaticorum rudimentorum
libri sex}
(\maurocite{001}{{\it S1}},
cc. 1v--2r) al duca di Monteleone e vicer{\'e} di Sicilia.}
\intestazione 
\par
\section{Varianti (capitolo 4)}
\par
\AliaManus{paolo}{peppa}
\par
\AliaManus{daniele}{puppa!}
\par
\AliaManus{A55}{Quoniam (\Cit{{ut ait Euclides}{EUC/CAT/Supp. 6}})}
\par
Sit data \VV{
    {A:ratio}
    {B\banale:gratia}
    {C:latio}
}, sit datus cubus.
\par
Sit data \VV{
    {A:ratio}
    {B/D\banale:gratia}
    {C:latio}
}, sit datus cubus.
\par
\VV{
  {A:{Manifestum ergo est: triangulum aequale quadrato \(ab\)}}
  {B:Patet igitur \Folium{Z:15} quadratum \(ab\) aequalem esse trigono \(mno\)}
}
\par
\VV{
    {*:multitudinis}
    {A:\CONTR{mul.nis}}
    {B:multiplicationis}
    }
\par
sit data \VV{
    {*:ratio}
    {B:gratia}
    {C:latio}
    }, sit datus cubus.
\par
\VV{
    {A:aequales}
    {C:aequates}
    {B:equites}
    }
\par
\VV{
    {A/B:aequales}
    {C/D:equites}
    }
\par
a vertice \VV{
    {A:coni}
    {B:canis}
    {C:\OM}
    } demittitur
\par
a vertice \VV{
    {A:coni}
    {B:canis}
    {C:\OMLAC}
    } demittitur
\par
erit \VV{
          {B:triangulum}
          {A:\NL}
        }\Comm{In A al posto di ``triangulum'' c'{\`e}
               una macchia d'inchiostro}
aequalis quadrato
\par
erit \VV{
    {A:\DES{duo versus legi nequeunt in}:\LACm}
    {D:\DES{spatium duorum versuum rel.}:}
    {C:\DES{spatium aliquot verborum rel.}:}
   } aequalis quadrato
\par
erunt quatuor \VV{
                  {B:\BIS:triangula}
                  } maius quam dimidio portionum.
\par
erunt quatuor \VV{
                  {A:triangula}
                  {B:quadrata quadrata}
                  } maius quam dimidio portionum.
\par
\TV{
    {A:il povero cieco}
    {B:il cieco povero}
   }
\par
primus \VV{
           {Clavius:secundusque}
           {A:et secundus}
           {B:vel secundus}
           }
\par
primus \VV{
           {A:et secundus}
           {Clavius:secundusque}
           {B:vel secundus}
           }
\par
primus \VV{
           {A:et secundus}
           {B:vel secundus}
           {Clavius:\OM}
           }
primus \VV{
           {A:et secundus}
           {B:vel secundus}
           {Clavius:\NIHIL}
           }
\section{Integrazioni (capitolo 5)}
primum \VV{
    {A+:\INTERL:et secundum}
}
\par
primum \VV{
    {A+:\SUPRA:et secundum}
}
\par
primum \VV{
   {A+:\INTERL:et secundum}
   {B:\OM}
   }
\par
primum \VV{
    {A+:\INTERL:et secundum}
    {B:\OMLAC}
     }
\par
primum\VV{
    {B:et secundum}
    {A+:\INTERL\DES{diverso atramento}:}
     }
\par
primum \VV{
    {B:et secundum}
    {A1:\INTERL:}
    {A:\OM}
    }
\par
primum \VV{
    {B:et secundum}
    {A:\OM}
    {A2:\INTERL:}
    }
\par
primum \VV{
    {B:et secundum}
    {A:\OM}
    {A2:\INTERL:et tertium}
    }
\par
primum \VV{
    {A/B:\OM}
    {A2:\INTERL:et secundum}
}
\par
primum \VV{
    {A2:\INTERL:et secundum}
    {A/B:\OM}
}
\par
... $ab$ et $bc$ \VV{
                     {A:\MARG:per 47 primi}
                    }
\par
... $ab$ et $bc$ \VV{
   {B:\INTERL:per 47 primi}
   {A:\OM}
   {A2:\MARG:}
   }
\par
... $ab$ et $bc$ \VV{
   {A:\MARGSIGN:per 47 primi}
   }
\par
triangulum \VV{
               {A:\ANTEDEL{primum}:secundum}
              } erit
\par
... \VV{
   {A:\POSTDEL{primum}:triangulum}
   } secundum erit
\par
... \VV{
    {A+:\POSTDEL{\PL{aliquot literas}}:Propositio}
    } 13
\par
... \VV{
    {A+:\EX{primum}:pristinum}
    }
\par
... \VV{
    {A+:\EX{pre stinum}:pristinum}
    }
\par
... \VV{
   {A+:\PC:pristinum}
   }
\par
... \VV{
   {A+:\EX{\PL{aliquot literas}}:pristinum}
   }
\par
... \VV{
    {A+:\EX{primum}:pristinum}
    {B/C:\INTERL:priorem}
    }
\par
... \VV{
   {C:pristinum}
   {A+:\PC:}
   {B:primum}
   }
\par
... \VV{
   {A1:secundum}
   {A:primum}
   }
\par
est aequalis \VV{
                 {A/B2/C:primae}
                 {B:primae et secundae}
                  }
\par
est aequalis \VV{
                {A/B2/C:primae}
                {B:primae et secundae}
                }, \VV{
                       {B2:\MARGSIGN:per 5 quinti Euclidis}
                       {A/B/C:\OM}
                        }, ergo ...
\par
... \VV{
    {B1:Euclides}
    {B:\NL}
    {A:Archimedes}
   }
\par
\BeginNM
Et talis est praestantia sphaericorum doctrina\NOTAMARG{
{Am:\MARG:Vide etiam quod dicit Martianus de astrologia}
}.
\EndNM
Sed ad definitiones accedamus.
\par
\BeginNM
Et talis est praestantia sphaericorum doctrina\NOTAMARG{
{Am:\DES{in rubro atramento}:Vide etiam quod dicit Martianus de astrologia}
}.
\EndNM
Sed ad definitiones accedamus.
\par
\BeginNM
Et talis est praestantia sphaericorum doctrina\NOTAMARG{
{Am/B:\MARG:Vide
  \VV{{*:etiam}{B:\OM}}
  quod dicit
  \VV{{Am:Martianus}{B:Capella}} de astrologia}
}.
\EndNM
Sed ad definitiones accedamus.
\par
\section{Congetture (capitolo 6)}
a vertice \VV{
    {*:\ED{conieci}:coni}
    {C:comico}
    {B:cunni}
    {A:canis}
    } 
\par
... \VV{
    {*:\ED{correxi}:$2+2=4$}
    {A:$2+2=5$}
    }
\par
... \VV{
        {*:\ED{conieci}:ch'\`e}
        {A:che}
        {B/C:qui rectus est}
} ...
\par
... \VV{
    {A:\POSTDEL{secundum}:primum}
   } \VV{
         {*:\ED{correxi}:est triangulum}
         {A:sunt triangula}
        }
\par
... \VV{
    {*:\ED{Clagett}:132}
    {A:123}
    {B:112}
    }
\par
... \VV{
    {*:\ED{Clagett}:132}
    {A:123}
    {B:112}
    {*:\ED{Napoli}:125}
    }
\par
... \VV{
    {*:\ED{conieci}:132}
    {A:123}
    {B:112}
    {*:\ED{Napoli}:125}
    {*:\ED{Clagett}:113}
    }
\par
... \VV{
    {A:132}
    {B:112}
    {*:\ED{Clagett}:518}
    }
\par
... \VV{
    {*:\ED{Clagett}:recta}
    {B:secta}
    {A:sancta}
    }
\par
a vertice \VV{
    {*:\ED{conieci}:coni}
    {A:cani}
    {C:comico}
    {B:\OM}
    {*:\ED{Clagett}:conico}
    } demittitur 
\par
la \VV{
    {*:\ED{conieci coll. {\it Cosm.}, I, 101}:terra}
    {A:torre}
    } \`e immobile al centro del mondo
\par
la \VV{
    {*:\ED{conieci coll. {\it Cosm. 1543}, I, f. 17v, 34}:terra}
    {A:torre}
    } \`e immobile al centro del mondo
\par
... \VV{
    {A:\Cit{
        {per 21\Sup{am} sexti Euclidis}
        {E.6.21}
        {Qui avrebbe dovuto citare E.5.15}
        }
    }
    {*:\ED{intelligendum est}:per 15\Sup{am} quinti Euclidis}
   }
\par
parallelus $cd$ per doctrinam \CRUX{de piscibus siculis}
\par
parallelus $cd$ per doctrinam \CRUX{\VV{
                        {A:de piscibus siculis}
                        {B:de arcanis antiquis}
                        {*:\ED{locum valde corruptum}:}}}
\par
... primum \EXPU{\VV{
          {*:\ED{seclusi}:et}
}} secundumque
\par
... primum \EXPU{\VV{
          {*:\ED{secl. Clagett}:et}
}} secundumque
\par
propositionem 17 \INTE{\VV{
            {*:\ED{supplevi}:praecedentis libri}
}} erit
\par
... 17 \INTE{\VV{       
   {*:\ED{supplevi}:praecedentis libri}
   {B:\DES{spatium aliquot literarum rel.}:\LACm}
}} erit
\par
... 17 \INTE{\VV{       
         {*:\ED{suppl. Clagett}:praecedentis libri}
         {B:\DES{spatium rel.}:\LACm}
         {*:\ED{suppl. Napoli}:huius}
}} erit
\par
a vertice \INTE{\VV{    
          {*:\ED{supplevi}:trianguli}
          {A:\DES{aliquot literae legi nequent in}:\LACm}
}} demittitur
\par
a vertice \INTE{\VV{    
          {*:\ED{supplevi}:trianguli}
          {A:\DES{aliquot literae legi nequent in}:\LACm}
          {*:\ED{Clagett}:trapezii}
}} demittitur
\par
Constat ergo ... Quod erat demonstrandum.
\par
\VV{{*:\ED{lacunam statui}:\LACc}}
\begin{Enunciatio}
Corollarium 3
\end{Enunciatio}
\par
... \VV{
    {*:\ED{lacunam statuit Clagett}:\LACc}
    }
\par
... \VV{
    {*:\ED{coll. {\it Cosm.} II 318 lacunam statui, ubi
           dimensio orbis pertractaretur}:\LACc}
    }
\par
\section{Varianti lunghe - Capitolo 7}
\Unit ... \VV[longa]{
                  {*:\CR{gatto1}:Si}
                  {B:\OM}
                } duae rectae ... erunt aequales.
Sint \(ab\) et \(cd\) ... a vertice \LB{gatto1}{coni}.
\par
\Unit ... \VV[longa]{
                  {*:\CR{gatto2}:Si}
                  {B:\OM}
                } duae rectae ... erunt aequales.
\Unit Sint \(ab\) et \(cd\) ... \Unit ... \LB{gatto2}{a vertice coni}.
\par
\Unit ...
\VV[longa]{
  {A:\CR{gufo}:Palam est}
  {C:Proculdubio possumus in parabolico segmento inscribere
     iam dictam figuram
     rectilineam ita ut excessum parabolici segmenti minorem
     sit quacunque proposita magnitudine}
  {B:\DES{spatium duorum versuum relicto om.}:\LACm}
}
quod in portione parabolica possibile est inscribere polygonium rectilineum 
ita ut relictae portiones sint minus\LB{gufo}{omni
proposito spatio}. \Unit  Nam intra portionem ...
\par
\Unit ... \VV[longa]{
                {*:\CR{gatto3}:Si}{B:\OM}
                } duae rectae ... erunt aequales.
\Unit Sint \(ab\) \VV{
                   {A:et}{C:est}
                    } \(cd\) ... \Unit ... a vertice
\LB{gatto3}{coni}.
\par
\Unit ... \VV[longa]{
                   {*:\CR{gatto4}:Si}{B:\OM}
                   } duae rectae ... erunt aequales. \Unit 
Sint \(ab\) et \(cd\) ... \Unit Rursus, cum sint... a
vertice \LB{gatto4}{\VV{
   {A:coni}{C:cani}
}}.
\par
\Unit ... \VV[longa]{
        {A/C:\CR{gatto5}:\VV{
                {A:Si}
                {C:Sunt}}
                duae rectae}
        {B:\OM}}
... erunt aequales
\Unit Sint \(ab\) et \(cd\) ...
\Unit ... a vertice \LB{gatto5}{coni}.
\par
\Unit \VV[longa]{
                 {*:\CR{gatto6}:Si}
                 {D:\OM}
                 {*:\CR{cane}:Si}
                 {B:\OM}
                }
              duae rectae ... erunt \LB{cane}{aequales}.
\Unit Sint \(ab\) et \(cd\) ...
\Unit ... \LB{gatto6}{a vertice coni}.
\par
\Unit ... tanta est \VV[longa]{
                        {*:\CR{miao1}:inter}
                        {B:\BIS:}
                          }
                   curvum et rectum fortasse propter
\LB{miao1}{dissimilitudinem}, inimicitia.
\par
\Unit ... tanta est \VV[longa]{
                          {*:\CR{miao2}:inter}
                          {B:\BIS:}
                         }
                      \VV{
                          {A/B:curvum}
                          {B:\REP[2]:circunferentiam}
                          } et rectum fortasse propter
\LB{miao2}{dissimilitudinem}, inimicitia.
\par
\Unit ... tanta est \VV[longa]{
                          {*:\CR{miao3}:inter}
                          {B:\BIS:}
                         }
                         curvum et \VV{
                          {A/B:rectum}
                          {B:\REP[1]:rectilineum}
                          } fortasse propter
\LB{miao3}{dissimilitudinem}, inimicitia.
\par
\Unit Hic Archimedis de quadratura parabolae libellus \VV[longa]{
                          {*:\CR{miao4}:ex}
                          {A:\BIS:}
                         } corruptissimo,
                      \VV{
                          {*:quod}
                          {A:\REP[2]:qui}
                          }
circumfertur, \LB{miao4}{exemplari}, labore et industria
Francisci Maurolyci restitutus est.
\par
\Unit ... tanta est \VV[longa]{
                          {*:\CR{miao5}:inter}
                          {B:\BIS:}
                         }
                      \VV{
                          {A:curvum}
                          {B:circunferentiam}
                          } et rectum fortasse propter
\LB{miao5}{dissimilitudinem}, inimicitia.
\par
\Unit Hic Archimedis de quadratura parabolae libellus \VV[longa]{
                      {*:\CR{miao6}:ex}
                      {A:\BIS:}
                     }
                      \VV{
                          {*:\ED{conieci}:corrupto}
                          {A:corsicano}
                          } quod circumfertur,
\LB{miao6}{exemplari}, labore et industria
Francisci Maurolyci restitutus est.
\par
\Unit ... ut dicebamus. \Unit \VV[longa]{
                                {A+:\CR{zebra}\MARGSIGN:Hic}
                              }
                                 est notandum quod
si recta \(ab\) parallelus non erit rectae \(cd\) ...
\Unit \LB{zebra}{Quod erat propositum},
meliori modo explicatum.
\par
\Unit ... ut dicebamus. \Unit \VV[longa]{
                                {A+:\CR{zebra1}\MARG:Hic}
                             } est notandum quod
si recta \(ab\) parallelus non erit rectae \(cd\) ...
\Unit \LB{zebra1}{Quod erat propositum}, meliori modo explicatum.
\par
\Unit 
... ut dicebamus.
\Unit \VV[longa]{
  {A1:\CR{lince}\MARG:Et}
  {A:Et rursus, per praecedentem, erit cubus sesquialterus dicti
     cylindri ... ut demonstrari potest}
}
rursus, cum sit cubus sesquitertius dicti prismatis (demonstratio tota 
pendet ex antepraemissa), ... \Unit Et sic demonstrabitur: cum sit: ...
\Unit \LB{lince}{Quod erat propositum}.
\par
\Unit 
... ut dicebamus.
\Unit \VV[longa]{
  {A1:\CR{lince1}\MARG:Et}
  {A:Et rursus, per praecedentem, erit cubus sesquialterus dicti
     cylindri ... ut demonstrari potest}
  {B/C:\OM}
}
rursus, cum sit cubus sesquitertius dicti prismatis (demonstratio tota 
pendet ex antepraemissa), ... \Unit Et sic demonstrabitur: cum sit: ...
\Unit \LB{lince1}{Quod erat propositum}.
\par
\Unit \VV[longa]{
      {*:\CR{lupo}\ED{conieci}:Erit}
      {A:Ibit ergo paradoxa sesquitertia tripodi eandem altitudinem et
         eandem bastionem habentis}
      {B:Ibit ergo paradoxa per centrum tripodi eandem basim habentis}
} ergo parabola sesquitertia trigoni eandem
altitudine et eandem basim \LB{lupo}{habentis}.
\par
\Unit Demittatur a puncto sumpto in parabola super diametrum
\CRUX{\VV[longa]{
    {*:\CR{volpe}\ED{locum corruptum}:ordinem}
    {*:\ED{coniecit Clagett}: ordinatam usque ad \(K\) et sumatur punctum
       \(A\) in diametro ab altera parte verticis parabolae \(Z\) ita ut
       \(AZ\) aequalis sit \(ZK\). Et recta \(AL\) ducatur per \(A\) usque 
       ad punctum sumptum}
} \(uK\) et similiter punctum \(A\) sit ad praesentem paradoxam \(Z\) ita ut 
recta \(AL\) ducatur per \(A\) \(uad\) punctum \LB{volpe}{supremum}.}
\Unit Per 33\Sup{am} primi Conicorum tanget
 talis recta parabolam.
\par
\Unit
\EXPU{\VV[longa]{
    {*:\CR{orso}\ED{seclusi}:Erit}
  }
  quadratum \(ab\) aequalis duobus quadratis dictis per doctrinam
  \LB{orso}{Euclidis}.}
Erit quadratum \(ab\) aequalis duobus quadratis \(bc\), \(ca\) simul sumpta 
per 47\Sup{am} primi Elementorum.
%\par
%\Unit ... 
%\VV[trans]{
 % {A:\CR{pane}\DES{hoc loco}:perfectus}
 % {B:\CR{burro}:}
%} 
%numerus producitur ex multiplicatione ultimi in serie pariter parium ab 
%unitate dispositorum,\LB{pane}{dispositorum} in totum aggregatum
%ipsorum, dum tamen tale aggregatum sit numerus primus, hoc
%est a nullo, preterquam ab unitate,
%numeratus.\LB{burro}{numeratus} \Unit ...
%\par
%\Unit Campanus mathematicus erat sed pro libidine sua
%\VV[trans]{
 % {A1:\CR{sole}\DES{huc transp.}:et}
 % {A:\CR{luna}:}
%} 
%multa mutavit ex sententia Euclidis,\LB{sole}{Euclidis} et multa
%addidit.\LB{luna}{addidit} \Unit ...
%\par
%\Unit Namque superficies
%\VV[trans]{
 % {*:\CR{acqua}\ED{huc transposui}:talium}
 % {A:\CR{fuoco}:}
%} 
%solidorum componuntur\LB{acqua}{componuntur} ex similibus conicis
%superficiebus.\LB{fuoco}{superficiebus}
%\par
%\Unit Questo \`e un caso 
%inevitabile\note{visto che l'esempio non c'\`e, 
%me lo sono inventato}.
\section{Trasposizioni - Capitolo 8}
\par
\TV{
    {B:Sit coni vertex}
    {A:Vertex coni sit}
   }
\par
\TV{
    {B:Sit coni vertex}
    {A1:\DES{sic disposuit}:}
    {A:\DES{priore ordine}:Vertex coni sit}
   }    
\par
\TV{
    {*:\ED{sic disposui}:Sit coni vertex}
    {A:Vertex coni sit}
   }    
\par
\Unit \TV[longa]{
                {*:\CR{acqua}\ED{sic disposui}:Namque}
                {A:Namque superficies ex similibus
                   conicis superficiebus talium
                   solidorum componuntur}
                } superficies talium solidorum componuntur ex similibus 
conicis \LB{acqua}{superficiebus}.
\par
\Unit ...\TV[duplex]{
                    {A:\CR{pane}\DES{hoc loco}:perfectus}
                    {B:\CR{burro}:}
                    } numerus producitur
ex multiplicatione ultimi in serie pariter parium
ab unitate \LB{pane}{dispositorum}, in totum
aggregatum ipsorum, dum tamen tale aggregatum
sit numerus primus, hoc est a nullo, preterquam
ab unitate, \LB{burro}{numeratus}. \Unit ...
\par
\Unit Campanus matematicus pro libidine 
sua \TV[duplex]{
               {A1:\CR{sole}\DES{huc transp.}:et}
               {A:\CR{luna}:}
               } multa mutavit ex sententia
\LB{sole}{Euclidis}, et multa
\LB{luna}{addidit}. \Unit ...
\par
\Unit Namque superficies \TV[duplex]{
                        {*:\CR{acqua1}\ED{huc transposui}:talium}
                        {A:\CR{fuoco}:}
                                     } solidorum
\LB{acqua1}{componuntur} ex similibus conicis
\LB{fuoco}{superficiebus}. 
\par
{\Unit[cervo1]}\TV[unit]{
             {*:\ED{hoc ordine disposui}:\UN{cervo1}-\UN{cervo5}}
             {C:\UN{cervo1}-\UN{cervo2} \UN{cervo4}-\UN{cervo5}}
             {A:\UN{cervo5} \UN{cervo4} \UN{cervo3} \UN{cervo2} \UN{cervo1}}
             {B:\UN{cervo2} \UN{cervo1} \UN{cervo5} \UN{cervo3}}
} Centrum uniformis figurae in puncto axis medio constituitur.
{\Unit[cervo2]} Centrum trianguli rectilinei trientem axis ad basim relinquit. 
{\Unit[cervo3]} \VV[longa]{
                     {*:\CR{foca}:Centrum}
                     {C:\OM}
} totius interiacet centris partium in eadem recta
\LB{foca}{constitutum}.
{\Unit[cervo4]} \VV[longa]{
       {*:\CR{delfino}:Centrum}
       {C:\OM}
} partialium distantiae a centro
totius reciprocae sunt \LB{delfino}{partibus}.
{\Unit[cervo5]} Centrum nunquam cadit extra rei gravis
ambitum. \Unit ...
\section{Casi eccezionali - Capitolo 9}
\par
musica est \VV{
              {A:\ABBR{anima} mundi}
              {B:\ED{signa mihi incognita}:\LACs}
              }.
\par
\section{Casi problematici}
\par
\VV{
   {A/B:\VV{
           {A:litera}
           {B:\DES{vel}\LEC{litura}:litera}
           }
   }
}
\par
\VV{{A:\DES{litteris}\LEC{-ris}\DES{ex alio correptis}:decembris}}
\par
\VV{{A:\DES{litteris}\LEC{-ris}\EX{\PL{alio}}\DES{correptis}:decembris}}
\par
decemb\VV{{A:\EX{\PL{alio}}:-ris}}
\par
\VV{
  {*:semicirculum}
  {A:\DESCOMPL{ex}{addito in interl.}{semi}:circulum}
  {U:\DES{supra}\LEC{recta}\DES{erasit}\LEC{\PL{quoddam verbum}}:recta}
  {V:\DESCOMPL{supra}{erasit}{\PL{quoddam verbum}}:recta}
}
\par
\VV{
  {A:\EX{circulum}:semicirculum}
}
\par
\VV{
  {A:\EX{circulum}\INTERL:semicirculum}
}
\par
\VV{
  {A:\EX{circulum}\INTERL\LEC{semi}:semicirculum}
}
\par
\VV{
  {A:\EX{circulum}\DES{addito}\INTERL\LEC{semi}:semicirculum}
}
\par
\VV{
{A:\DES{ex}\LEC{circulum}\DES{addito in interl.}\LEC{semi}:semicirculum}
}
\par
\Unit \VV[longa]{
    {*:\CR{focax}:Item}
    {C:\BIS:}
          }
\VV{
  {S:2\Sup{us} et 3\Sup{us}}
  {C:\VV{
    {*:2\Sup{us}}
    {*:\REP[2]:\OM}
  } et 3\Sup{us} et 4\Sup{us}}
} trianguli, scilicet 3 et 6 faciunt
\VB{{C:tertium}{S:3\Sup{um}}} \QDR\Sup{tum}
\LB{focax}{scilicet 9}.
\par
\Unit \VV[longa]{
          {S:\CR{ape}\BIS:Omnis}
         } columna quadrata secunda cum duplo
\VV{
    {*:\ED{correxi}:collateralis}
    {S:coll.}
    {C:collaterali}
   } quadrati primi, facit triplum suae \LB{ape}{pyramidis}.
\par
\Unit \VV[longa]{
          {*:\CR{ape1}:Omnis}
          {S:\BIS:}
         } columna quadrata secunda cum duplo
\VV{
    {*:\ED{correxi}:collateralis}
    {S:coll.}
    {C:collaterali}
   } quadrati primi, facit triplum suae \LB{ape1}{pyramidis}.
\par
\Unit \Cit{
     {postremo problematum \VB{{S:mechanicorum}{C:meccanicorum}}}
     }.
\par
\Prop{}
\begin{Enunciatio}
Propositio 5\Sup{a}
\end{Enunciatio}
%
\Unit His praelibatis, ponatur unitas \(a\), quilibet autem
numerus \(b\), ipse autem \(c\) unitate maior quam \(b\).
\Unit Deinde \(b\) in se faciat \(d\), \(b\) in \(c\) faciat
\(e\) et \(c\) in 
\VV[longa]{
       {S:\CR{yak}:se faciat \(f\)}
       {C:\OM}
}
se faciat \(f\). 
\Unit Post haec
\(b\) in \(d\) faciat \(g\). 
\LB{yak}{Item \Math{b}\Sup{2} in}
\(e\) faciat \(h\).
\VB{{S:Adhuc}{C:Ad huc}} 
\(b\) in \Math{f} faciat \(k\). Demum
\(c\) in \(f\) faciat \(l\). 
\Unit Tandem \(b\) in \(g\)
faciat \(m\). Item \(b\) in \(h\) faciat \(n\).
\VB{{S:Necnon}{C:Nec non}} 
\(b\) in \(k\) faciat \(o\). Sic
\(b\) in \(l\) faciat \(p\). Denique \(c\) in \(l\) faciat
\(q\). Quibus dispositis.
\par
\Unit Omnis pyramis pentagona conflatur
\TV[duplex]{
    {S:\CR{cane9}\DES{hoc loco}:ex}
    {C:\CR{orsa}:}
} pyramide quadrata
\LB{cane9}{collaterali} et ex pyramide
\LB{orsa}{\VB{{C:triangula}{S:\TRN\Sup{la}}} praecedenti}.
\par
\Unit
\TV[duplex]{
      {*:\CR{circe}\ED{huc transposui}:quare pyramis}
      {S/C:\CR{arturo}:}
}
quinta \TRN\Sup{la} scilicet 35 cum praecedenti pyramide
\TRN\Sup{la} scilicet 20 construunt 55 pyramidem
\LB{circe}{quadratam quintam}.
\Unit Idemque similiter, in omni exemplo, cuiuslibet pyramidis
\VV{{*:\ED{conieci}:\TRN\Sup{lae}}{S/C:\TRN\Sup{li}}} et
\VV{{C:praecedentis}{S:praecedenti}} demonstrabo per
{\Cit{{\VB{{C:undecimam}{S:11\Sup{am}}} huius}}} arguendo
toties, quoties
\LB{arturo}{combinantur trianguli}.
\par
\VV{
    {A+:\POSTDEL{\PL{aliquot litteras}}:mexxanica}
   }
\par
\VV{
    {A+:\ANTEDEL{\PL{aliquot litteras}}:mexxanica}
   }
\par
prima \VV{
  {*:cane}
  {*:\MARG:}
} dopo
\par
\VV{
    {A:existimatur}
    {B/C:\VV{
      {B:habetur}
      {C:abetur}
    }}
}     
\par
\VV{
    {A:existimatur}
    {B/C:\VB{{B:habetur}
             {C:abetur}
    }}
}
\par
la macro \Comm{ceci est un commentaire} me donne:
\par
\Unit Quoniam per \Cit{{quintam huius}{}}, quantitatis $ab$ ad quantitatem $ef$ ratio componitur  
 \VV[longa]{
{*:\CR{circe6}:\VV{{S:ex}{C:et ex}}}{C:\BIS:}
}
 ratione \VV{{S/C:numeri}{C:\REP[2]:numerus}} 
$a$ ad numerum \LB{circe6}{$e$}, et ex ratione \VV{{S:numeri}{C:numerus}} $f$ ad numerum $b$. \Unit Ac 
\par
\Date{{Hic complet numeros
fertilis ENNA meos {\DB} 24 dec. 1553.}}
o anche
\Date{{Hic complet numeros
fertilis ENNA meos 24\Sup{o} dec. 1553.}}
danno degli errori.
\par
in eodem plano \VV{{A:\OM}{B:cum}} lineis, cuius summitas sit punctum,
\par
questo \`e un \Tit{titolo di un'opera} codificato secondo il manuale.
\par
\begin{titmarg}
questo \`e un titolo a margine.
\end{titmarg}
\par
\Personaggio{Paolo} ha detto che funziona.
\par
Prova \VV{{A:\DES{omisso \UN[OPERETTA]{cervo1}}:questa}} e stai felice.
\par
Prova esponente\Sup{sup}, deponente\Sub{sub}
ed entrambi\SupSub{sup}{sub}.
\par
Prove di markup opera\Opera{{prova}},
proposizione\Prop{prova}, 
argomento\Arg{prova},
libro\Libro{prova},
beginnm\BeginNM{},
endnm\EndNM{}
e generico\Markup{markup: }{generico}.
Compare solo se viene richiesto il markup nel testo.
\par
Prove di congetture \Cong{Serve per formulare delle congetture}.
\par
(\Cit{{per 2.XI}})
\par
Bisogna che accetti commenti\Comm{prima riga \par seconda riga}
su pi{\`u} paragrafi.
\begin{Enunciatio}
Ci sono dei problemi \INTE{\VB{{*:\ED{correxi}:quando}{B:quanto}} si inserisce} un commento dell'editore
in una variante banale.
\end{Enunciatio}
\par
... prima \VV{{A:\POSTDEL{\TRN\RTT\QDR}:21}} dopo ...
\par
... prima \VV{{A:\DESCOMPL{miao}{bau}{\TRN\RTT\QDR}:21}} dopo ...
\par
esempio di frazione: $\frac{2}{3}$
\par
%%%
\begin{Enunciatio}
MUSICAE TRADITIONES CARPTIM COLLECTAE
\par
Vel Musica elementa Maurolyci studio congesta
\end{Enunciatio}
%%%%
\par Prova con tabula. Dovrebbe andare.

\par Prova adnotatio in una
\VV{{A:variante\Adnotatio{annotazione}}{B:variazione}}. Prova
adnotazio finita.  \par
\section{Macro di semplificazione}
\par
\NomeTestimone{A} Per semplificare: \Smarg{questa lezione} \`e a
margine, mentre \Smarg[A1]{questa} \`e stata aggiunta da un'altra
mano.  \par
 Per semplificare: \Smargsign{questa lezione} \`e a margine
(con segnodi richiamo), mentre \Smargsign[A1]{questa} \`e stata
aggiunta da un'altra mano.  \par
 Per semplificare: \Sinterl{questa
lezione} \`e in interlinea, mentre \Sinterl[A1]{questa} \`e stata
aggiunta da un'altra mano.  \par
 Per semplificare: \Sinterl{questa
lezione} \`e in interlinea, mentre \Ssupra[A1]{questa} \`e stata
aggiunta da un'altra mano.  \par
 Per semplificare: \Spc{questa
lezione} \`e dopo una correzione, mentre \Spc[A1]{questa} \`e stata
aggiunta da un'altra mano.  \par
 Per semplificare: \Sbis{questa
lezione} \`e ripetuta, mentre \Sbis[A1]{questa} \`e stata aggiunta da
un'altra mano.  \par
 \VV{{A+:\MARG:questa lezione}{A:\DES{ante
corr.}:sta roba}} Per semplificare: \Smargcorr{sta roba}{questa
lezione} \`e corretta in margine, mentre \Smargcorr[A1]{sta
qui}{questa} \`e stata aggiunta da un'altra mano.  \par
 Per
semplificare: \Smargsigncorr{sta roba}{questa lezione} \`e corretta in
margine con segno di richiamo, mentre \Smargsigncorr[A1]{sta
qui}{questa} \`e stata aggiunta da un'altra mano.  \par
 Per
semplificare: \Sintcorr{sta roba}{questa lezione} \`e corretta in
interlinea, mentre \Sintcorr[A1]{sta qui}{questa} \`e stata aggiunta
da un'altra mano.  Questa, invece, \Sintcorr[A+]{premier}{deuxieme}
non ha senso.  \par
 Per semplificare: \Santedel{sta roba}{questa
lezione} \`e stata corretta, mentre \Santedel[A1]{sta qui}{questa} \`e
stata aggiunta da un'altra mano.  \par
 Per semplificare: \Spostdel{sta
roba}{questa lezione} \`e stata corretta, mentre \Spostdel[A1]{sta
qui}{questa} \`e stata aggiunta da un'altra mano.  \par
 Per
semplificare: \Sex{sta roba}{questa lezione} \`e stata corretta,
mentre \Sex[A1]{sta qui}{questa} \`e stata aggiunta da un'altra mano.
\par
 \NomeTestimone{Clavius} Per semplificare: \Smarg{questa lezione}
\`e a margine, mentre \Smarg[A1]{questa} \`e stata aggiunta da
un'altra mano.  \par
 triangulum \VV{{A:\DESCOMPL{prima di}{Clavio
ha cancellato}{primum}:secundum} } erit \par
 triangulum \VV{
{A:\EDCOMPL{prima di}{ho cancellato}{primum}:secundum} } erit \par
questa {\`e} un'\VV{ {*:abbreviazione} {A:\CONTR{abb.ne}}
{B:\CONTR{a.ne}} {C:abbr.}  }

\section{Macro speciali}

\par Macro per \href{http://www.w3c.org}{link ipertestuali} in
\voidhref{HTML}.

\section{POSTSCRIPT}

\par prima della variante
\VV{{*:Vizinii}{A:\DES{contraxit}\POSTSCRIPT\DES{de quo loco vide
      \Tit{Fragm. Arith} x.\UN{cervo1}}:Viz.}} dopo la variante.
\par
Sit recta $ab$ parallela ipsae $cd$ et \VV[longa]{
{*:\CR{unicorno}:recta}
{H:\POSTSCRIPT\DES{qui}\LEC{non}\DES{add. post }\LEC{et}:\OM}
}
$mn$ parallela ipsae $kl$ (tangens, ut
supra dictum est, circulum $rsv$)
\LB{unicorno}{quae est} perpendicularis
$cd$ et describatur ...

\par

\textcircled{1 }
\textcircled{2 }
\textcircled{3 }
\textcircled{4 }
\textcircled{5 }
\textcircled{6 }
\textcircled{7 }
\textcircled{8 }
\textcircled{9 }
\textcircled{10}
\textcircled{11} 
\textcircled{12} 
\textcircled{13} 
\textcircled{14} 
\textcircled{15} 
\textcircled{16} 
\textcircled{17} 
\textcircled{18} 
\textcircled{19} 
\textcircled{20}
\textcircled{21}

\par

%\section{Messaggi dell'analizzatore}
%
%SOL
%BeginNM
%
%{
%1
%2
%3
%4
%5
%6
%7
%8
%9
%10
%11
%12
%13
%14
%15
%16
%17
%18
%19
%20
%}
%
%[
%1
%2
%3
%4
%5
%6
%7
%8
%9
%10
%11
%12
%13
%14
%15
%16
%17
%18
%19
%20
%]
%
%(
%1
%2
%3
%4
%5
%6
%7
%8
%9
%10
%11
%12
%13
%14
%15
%16
%17
%18
%19
%20
%)
%%
%\begin{pippo}
%  ciao ciao ciao
%\end{pluto}

%\VV[x]{
%}
%
%\Cit[y]{
%}
%
%[prova quadre]
%pippo[]
%pappa []
%
%Punto interrogativo?
%
%{
%[
%(

\section{bug}

\subsection{VV con lezione vuota}

\VV{{B/F/V:quod}{U:\INTERL:}}

\subsection{VV senza altri testimoni}

\VV{{*:cane}{A:gatto}}

\VV{{*:\DES{ut vid. An}\LEC{cane}\DES{?}:cene}}

\VV{{*:\DES{ut vid. An}\LEC{cane}\DES{?}:cene}{A:gatto}}

\VV{{A:\DES{ut vid. An}\LEC{cane}\DES{?}:cene}}

\VV{{A:\DES{ut vid. An}\LEC{cane}\DES{?}:cene}{B:cena}}

\VV{{*:\ED{conieci ex}\LEC{cene}:cane}}

\htmlcut Un'altra pagina numerata.
\Unit[lince]Pagina di prova della unit per lince.
\htmlcut[epilogo] Ed una con descrizione.

\Commenti

\Annotazioni

\end{document}
